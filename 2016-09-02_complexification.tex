\documentclass{article}

\usepackage[a5paper,margin=1.5cm]{geometry}
\usepackage{amsmath,amssymb,amsthm,microtype}
\DeclareMathOperator{\GL}{GL}
\DeclareMathOperator{\Lie}{Lie}
\DeclareMathOperator{\SL}{SL}
\DeclareMathOperator{\Sp}{Sp}
\DeclareMathOperator{\U}{U}
\newcommand{\bC}{{\mathbf C}}
\newcommand{\bR}{{\mathbf R}}
\newcommand{\fg}{{\mathfrak g}}
\newcommand{\fh}{{\mathfrak h}}
\newcommand{\fk}{{\mathfrak k}}
\newcommand{\dd}{\mathrm{d}}
\newcommand{\real}{\mathrm{real}}
\newtheorem{theorem}{Theorem}
\theoremstyle{definition}
\newtheorem{definition}{Definition}

\title{Complexification}
\author{Daniel Miller}
\date{2 September 2016}

\begin{document}
\maketitle





So far we have only considered real Lie groups, even though some of our 
examples, like $\GL(n,\bC)$, seem like they should be ``complex'' objects in 
some way. So, without further ado, we make a definition. 

\begin{definition}
A \emph{complex Lie group} is a group which is also a complex manifold, such 
that all the group operations are analytic maps.  
\end{definition}

With this definition, it is easy to see that $\GL(n,\bC)$ is a complex Lie 
group, as is $\Sp(2n,\bC)$. Clearly, every complex Lie group can be made into 
a real Lie group in a trivial way, by forgetting the complex structure. That 
is, we have a kind of map (a functor, for those in the know)
\[
	(-)^\real \colon \{\textnormal{complex Lie groups}\} \to \{\textnormal{real Lie groups}\} .
\]
It would be nice for there to be a kind of ``inverse'' map in the opposite 
direction (an adjoint functor, for those who care). That is, given a real Lie 
group $G$, we would like there to be a complex Lie group $G_\bC$, together 
with a homomorphism $G\to G_\bC$, such that for any complex Lie group $H$ and 
a homomorphism $f\colon G\to H$, there exists a unique extension 
$\widetilde f\colon G_\bC \to H$. We call $G_\bC$ the \emph{complexification} 
of $G$. 

(Brief interlude on universal properties and uniqueness.)

\begin{theorem}
Let $G\to H$ be a map from a real Lie group to a complex one, both connected, 
that induces an isomorphism $\fg_\bC \to \fh$. If the induced map 
$\pi_1(G) \to \pi_1(H)$ is an isomorphism, then $H$ is a complexification of 
$G$. 
\end{theorem}
\begin{proof}
We only need to check that $H$ satisfies the universal property. Let 
$H'$ be an arbitrary complex Lie group together with a homomorphism 
$f\colon G\to H'$. This gives us a real Lie algebra map 
$\dd f\colon \fg \to \fh'$, which uniquely extends to a complex Lie algebra 
map $(\dd f)_\bC\colon \fg_\bC\to \fh'$. Equivalently, $\dd f$ extends uniquely 
to a complex Lie algebra map $\widetilde{\dd f}\colon \fh\to \fh'$. 

(Show that $\dd f$ can be extended to Lie group map. Use universal cover.)

Since 
$H$ is simply connected, we get a unique homomorphism of real Lie groups 
$\widetilde f\colon H\to H'$, extending $f$. Since 
$\dd \widetilde f = \widetilde{\dd f}$, a $\bC$-linear map, $\widetilde f$ is 
in fact complex analytic. 
\end{proof}

As an example, since $\pi_1(\SL(n,\bC))=1$, the inclusion 
$\SL(n,\bR)\hookrightarrow \SL(n,\bC)$ makes $\SL(n,\bC)$ the complexification 
of $\SL(n,\bR)$. 

\begin{theorem}
Let $K$ be a compact connected Lie group. Then the complexification of $K$ 
exists. 
\end{theorem}
\begin{proof}
We know there is an embedding $K\subset \U(n)$ for some $n\geqslant 1$. Put 
$\fk=\Lie(K)$, $P=\exp(i \fk)$, and $G=K\cdot P\subset \GL(n,\bC)$. Since 
$G$ is the product of a closed subgroup and a compact subgroup of $\GL(n,\bC)$, 
general nonsense tells us that $G$ is closed in $\GL(n,\bC)$. Moreover, 
$\Lie(G)=\Lie(K)+\Lie(P) = \fk+i\fk = \fk_\bC$, a complex vector space, so we 
can use the exponential map $\fk_\bC \to G$ to give $G$ a complex structure. 
Since $P$ is contractible (why?), $K\to G$ induces an isomorphism 
$\pi_1(K)\to \pi_1(G)$. Thus $G=K_\bC$. 
\end{proof}





\end{document}
