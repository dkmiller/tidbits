\documentclass{article}

\usepackage[a5paper,margin=1.5cm]{geometry}
\usepackage{amsmath,amssymb,amsthm,microtype}
\DeclareMathOperator{\GL}{GL}
\DeclareMathOperator{\Sp}{Sp}
\newcommand{\bC}{{\mathbf C}}
\newcommand{\bR}{{\mathbf R}}
\newcommand{\real}{\mathrm{real}}
\theoremstyle{definition}
\newtheorem{definition}{Definition}

\title{Complexification}
\author{Daniel Miller}
\date{2 September 2016}

\begin{document}
\maketitle





So far we have only considered real Lie groups, even though some of our 
examples, like $\GL(n,\bC)$, seem like they should be ``complex'' objects in 
some way. So, without further ado, we make a definition. 

\begin{definition}
A \emph{complex Lie group} is a group which is also a complex manifold, such 
that all the group operations are analytic maps.  
\end{definition}

With this definition, it is easy to see that $\GL(n,\bC)$ is a complex Lie 
group, as is $\Sp(2n,\bC)$. Clearly, every complex Lie group can be made into 
a real Lie group in a trivial way, by forgetting the complex structure. That 
is, we have a kind of map (a functor, for those in the know)
\[
	(-)^\real \colon \{\textnormal{complex Lie groups}\} \to \{\textnormal{real Lie groups}\} .
\]





\end{document}
