\documentclass{article}

\usepackage[a5paper,margin=1.5cm]{geometry}
\usepackage{amsmath,amssymb}
\DeclareMathOperator{\cdf}{cdf}
\DeclareMathOperator{\D}{D}
\DeclareMathOperator{\GL}{GL}
\DeclareMathOperator{\ram}{ram}
\DeclareMathOperator{\SU}{SU}
\DeclareMathOperator{\sym}{sym}
\DeclareMathOperator{\tr}{tr}
\newcommand{\bC}{\mathbf{C}}
\newcommand{\bF}{\mathbf{F}}
\newcommand{\bQ}{\mathbf{Q}}
\newcommand{\bR}{\mathbf{R}}
\newcommand{\bZ}{\mathbf{Z}}
\newcommand{\CM}{\mathrm{CM}}
\newcommand{\dd}{\mathrm{d}}
\newcommand{\frob}{\mathrm{fr}}
\newcommand{\nonCM}{\textnormal{non-CM}}
\newcommand{\ST}{\mathrm{ST}}

\title{Constructing Galois representations with specified Sato--Tate 
distributions\thanks{Notes for a talk given in Cornell's Number Theory 
Seminar.}}
\author{Daniel Miller}
\date{24 March 2017}

\begin{document}
\maketitle





\section{Introduction and motivation}

Let $E_{/\bQ}$ be an elliptic curve, and fix a rational prime $l$. A well-known 
construction of Tate yields a continuous homomorphism 
$\rho_l\colon G_\bQ \to \GL_2(\bZ_l)$ such that at each prime $p\ne l$ for 
which $E$ is unramified, $\rho_l$ is unramified at $p$ and moreover 
\[
	a_p = \tr\rho_l (\frob_p) = p + 1 - \# E(\bF_p) .
\]
It follows that $a_p\in \bZ$ satisfies the Hasse bound 
$|a_p| \leqslant 2\sqrt p$. Let 
$\theta_p = \cos^{-1}\left( \frac{a_p}{2\sqrt p}\right) \in [0,\pi]$, and let 
\begin{align*}
	\ST_\nonCM &= \frac{2}{\pi} \sin^2\theta\, \dd\theta \\
	\ST_\CM &= \frac 1 2 \left( \delta_{\pi/2} + \frac 1 \pi \dd \theta\right) .
\end{align*}
Then the Sato--Tate conjecture (now a theorem) states that the $\{\theta_p\}$ 
are equidistributed with respect to $\ST_\ast$, where $\ast\in \{\nonCM,\CM\}$ 
describes $E$. 

The Sato--Tate measures here arise because of deep modularity results. Aftab 
Pande's paper \emph{Deformations of Galois representations and the theorems of 
Sato--Tate and Lang--Trotter} considers the question of whether there might be 
a purely Galois-theoretic proof of these equidistribution results. He proves 
that for any $\epsilon>0$, there exist Galois representations 
$\rho\colon G_\bQ \to \GL_2(\bZ_l)$, ramified at an infinite (but density zero) 
set of primes, for which all $\theta_p\in B_\epsilon(\pi/2)$ at each unramified 
prime. Pande extensively uses the results and techniques from 
Khare--Larsen--Ramakrishna's paper \emph{Constructing semisimple $p$-adic 
Galois representations with prescribed properties}. It is natural to wonder: 
can Pande's results be strengthened to yield equidistribution? Can the ``rate 
of convergence'' of the $\theta_p$ to the given measure be specified? Can the 
density of the set of ramified primes be controlled? We will see that all these 
questions can be answered in the affirmative. 





\section{Discrepancy}

Let $\{\theta_p\}$ be a set of angles in $[0,\pi]$ indexed by a subset $U$ of 
the rational primes. Given a cutoff $x$, let 
$\mu_x = \frac{1}{\pi_U(x)}\sum_{p\leqslant x} \delta_{\theta_p}$ be the 
empirical measure capturing the set $\{\theta_p\}_{p\leqslant x}$. If $\mu$ is 
some other measure on $[0,\pi]$, the \emph{discrepancy} is 
\[
	D_x = \D(\mu_x,\mu) = \sup_{t\in [0,\pi]} \left| \frac{\#\{p\leqslant x : \theta_p \leqslant t\}}{\pi_U(x)} - \int_0^t \, \dd \mu\right| .
\]
In other words, $D_x = \|\cdf_{\mu_x} - \cdf_\mu\|_\infty$. Weak convergence 
$\mu_x \to^\ast \mu$ is equivalent to $D_x \to 0$. Heuristics suggest (and 
Akiyama--Tanigawa have conjectured) that for $E_{/\bQ}$ non-CM, we have 
$\D(\mu_x,\ST_\nonCM) \ll x^{-\frac 1 2+\epsilon}$. Their conjecture implies 
the Riemann Hypothesis for all $L(\sym^k E,s)$. 

Given $\alpha\in (0,1/2)$ and any $\mu = f(t)\, \dd t$ for $f$ bounded, there 
is a sequence of $\{\theta_p\}$ such that 
$|\D(\mu_x,\mu) - \pi(x)^{-\alpha}| \ll x^{-1+\epsilon}$; in particular, 
$D_x \sim \pi(x)^{-\alpha}$. We can even arrange that the $\theta_p$ come 
from integral $a_p$ (which also satisfy the Hasse bound), though this weakens 
the bound to $|D_x - \pi(x)^{-\alpha}| \ll x^{-\frac 1 2+\epsilon}$. Moreover, 
if $\{a_p^{(1)}\}$ is any collection of integers satisfying the Hasse bound, and 
$|a_p^{(1)} - a_p|$ is sufficiently close to $p^{-1/2}$, then 
$\D(\mu_x^{(1)},\mu) \sim \D(\mu_x,\mu)$. 





\section{Main result}

The main theorem involves a number of pieces.
\begin{enumerate}
\item
Fix a rational prime $l\geqslant 7$. 

\item
Fix an odd, absolutely irreducible, weight-$2$ representation 
$\bar\rho\colon G_\bQ \to \GL_2(\bF_l)$. 

\item
Fix a function $h\colon \bR^+ \to \bR^+$ which decreases rapidly to zero (for 
example, $h(x) = e^{-x}$ or $h(x) = e^{-e^x}$). 

\item
Fix a measure $\mu$ on $[0,\pi]$ of the form discussed above. 

\item
Fix $\alpha\in \left(0,\frac 1 2\right)$. 
\end{enumerate}
Then there exists $\rho\colon G_\bQ \to \GL_2(\bZ_l)$, also of weight $2$, 
such that 
\begin{enumerate}
\item
$\rho\equiv \bar\rho\pmod l$. 

\item
$\pi_{\ram(\rho)}(x) \ll h(x) \pi(x)$. 

\item
For each unramified prime $p$, $a_p = \tr \rho(\frob_p)\in \bZ$ and satisfies 
the Hasse bound. 

\item
$\D(\mu_x,\mu) \sim \pi(x)^{-\alpha}$. 

\item
If $(\theta\mapsto \pi - \theta)_\ast \mu = \mu$, then for each odd 
$k$, $L(\sym^k \rho,s)$ satisfies the Riemann Hypothesis. 
\end{enumerate}





\section{Some techniques in the proof}

The representation $\rho$ is build as a limit $\rho = \varprojlim \rho_n$, 
where $\rho_n\colon G_\bQ \to \GL_2(\bZ/l^n)$ is chosen so as to ensure the 
statement of the theorem. We have $\rho_1 = \bar\rho$, and further $\rho_n$ 
are constructed inductively. Enumerate the unramified primes as 
$\{p_{u_1},p_{u_2},\dots\}$. Then the goal is to force each 
$a_{p_{u_n}} \sim 2\sqrt{p_{u_n}} \cos(\widetilde \theta_{p_{u_n}})$, where 
$\{\widetilde\theta_p\}$ is a sequence with desired rate of decay of 
discrepancy. At any given stage, we'll have along with $\rho_n$, a large finite 
set $U$ of unramified primes, and choices of $a_p$ for each $p\in U$ such that 
$a_p\equiv \tr\rho(\frob_p)\pmod{l^n}$. The set of ramified primes $R$ will be 
very thin. Choose a new $U'\supset U$, large enough that we can enforce the 
statements of the theorem. Then there exist choices of $a_p$ for 
$p\in U'\smallsetminus U$ such that the statements about discrepancy continue 
to hold. The results of Khare--Larsen--Ramakrishna show that there is 
$R'\supset R$, sufficiently thin, along with a lift 
$\rho_{n+1}\colon G_{\bQ,R'}\to \GL_2(\bZ/l^{n+1})$, such that 
$a_p\equiv\tr\rho_p(\frob_p)\pmod{l^{n+1}}$ for all $p\in U'$. 

We've seen (very roughly) how to enforce the desired $\mu$ and discrepancy, but 
how can we get the Riemann Hypothesis for $L(\sym^k \rho,s)$, $k$ odd? Let 
$U_k(\theta) = \frac{\sin((k+1)\theta)}{\sin\theta}$; this is the trace of the 
$k$-th symmetric power of $\SU(2)\hookrightarrow \GL_2(\bC)$ in 
``theta-space.'' The Riemann Hypothesis for $L(\sym^k \rho,s)$ follows from 
bounds of the form 
\[
	\left| \sum_{p\leqslant x} U_k(\theta_p)\right| \ll x^{\frac 1 2+\epsilon} .
\]
Since $U_k(\pi - \theta) = - U_k(\theta)$ when $k$ is odd, 
we force $\theta_q \approx \pi - \theta_p$ fr $p<q$ successive 
unramified primes. We can get 
$\left|\theta_q - (\pi - \theta_p)\right| \ll p^{-1/2}$; 
since $U_k(\cos^{-1} t)$ is a polynomial in $t$ this gives the desired bound. 





\end{document}
