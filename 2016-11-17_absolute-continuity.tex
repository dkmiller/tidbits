\documentclass{article}

\usepackage{amsmath,amssymb,amsthm}
\DeclareMathOperator{\AC}{AC}
\DeclareMathOperator{\Lie}{Lie}
\DeclareMathOperator{\SU}{SU}
\newcommand{\bR}{\mathbf{R}}
\newcommand{\bZ}{\mathbf{Z}}
\newcommand{\dd}{\mathrm{d}}
\newcommand{\ft}{\mathfrak{t}}

\newtheorem{theorem}{Theorem}
\theoremstyle{definition}
\newtheorem{definition}{Definition}

\title{Absolute continuity and Fourier coefficients}
\author{Daniel Miller}

\begin{document}
\maketitle





Consider the compact Lie group $\SU(2)$. It has an obvious maximal torus, 
namely 
\[
	T = \left\{\begin{pmatrix} e^{i\theta} \\ & e^{-i\theta}\end{pmatrix} : \theta\in [0,2\pi)\right\} .
\]
The Weyl group is 
\[
	W = \left\{\begin{pmatrix} 1 \\ & 1 \end{pmatrix}, \begin{pmatrix} & -1 \\ 1 \end{pmatrix}\right\} ,
\]
whose non-trivial element acts on $T$ by $\theta\mapsto -\theta$. It is 
well-known that the map $T/W \to \SU(2)^\natural$ is a bijection. We use it to 
make a couple definitions. First, note that for any function on $T$, we will 
write 
\[
	f(\theta) = f\begin{pmatrix} e^{i\theta} \\ & e^{-i\theta} \end{pmatrix} .
\]
Moreover, for $f\in L^1(T)$, we have the Fourier coefficients:
\[
	\widehat f(m) = \frac{1}{2\pi} \int_0^{2\pi} f(\theta) e^{-i m \theta}\, \dd \theta .
\]

\begin{definition}
A function $f\in L^1(\SU(2)^\natural)$ is \emph{absolutely continuous} if it is 
the descent of a $W$-invariant absolutely continuous function on $T$. In 
other words, $\AC(T/W) = \AC(T)^W$. 
\end{definition}

Recall that $f\in C(T)$ is absolutely continuous if there exists $g\in L^1(T)$ 
for which 
\[
	f(\theta) = f(0) + \int_0^\theta g(t)\, \dd t, \qquad \theta\in [0,2\pi) .
\]
Note that if $f\in \AC(T/W)$, the corresponding $g$ may not descend to 
$T/W$. 

\begin{theorem}
If $f\in \AC(T/W)$ and $m\ne 0$, then 
$\widehat f(m) = -\frac{i}{m} \widehat g(m)$. 
\end{theorem}
\begin{proof}
We compute directly:
\begin{align*}
	\widehat f(m) 
		&= \frac{1}{2\pi} \int_0^{2\pi} \left(f(0) + \int_0^\theta g(t)\, \dd t\right) e^{-i m \theta}\, \dd \theta \\
		&= \frac{f(0)}{2\pi} \int_0^{2\pi} e^{-i m \theta}\, \dd\theta + \frac{1}{2\pi} \int_0^{2\pi} g(t)\int_t^{2\pi} e^{-i m \theta} \,\dd\theta\dd t \\
		&= \frac{i}{2m\pi} \int_0^{2\pi} g(t) (e^{-2\pi i m} - e^{-i m t})\, \dd t \\
		&= \frac{i e^{-2\pi i m}}{2\pi m}\int_0^{2\pi} g(t)\, \dd t - \frac{i}{2\pi m} g(t) e^{-i m t}\, \dd t \\
		&= -\frac{i}{m} \widehat g(m) .
\end{align*}
\end{proof}

Recall that for $k\geqslant 0$, we write $S_k f$ for the $k$-th partial Fourier 
series for $f$, namely 
\[
	S_k f(x) = \sum_{|m|\leqslant k} \widehat f(m) e^{i m x} .
\]
It is known that if $f$ is absolutely continuous, then $S_k f\to f$ uniformly. 
We give a quantitative bound on the rate of convergence. 

\begin{theorem}
Let $f\in \AC(T/W)$. Then 
\[
	\|S_k f - f\|_\infty \ll k^{-1/2}\cdot \|f'\|_2 .
\]
\end{theorem}
\begin{proof}
Let $g$ be such that $f(x)=f(0)+\int_0^x g(t)\, \dd t$. We then compute
\begin{align*}
	|S_k f(x) - f(x)| 
		&\leqslant \sum_{|m|>k} |\widehat f(m)| \\
		&\ll \sum_{|m|>k} \frac{1}{m} |\widehat g(m)| \\
		&\leqslant \sqrt{\sum_{|m|>k} m^{-2}} \sqrt{\sum_{|m|>k} |\widehat g(m)|^2} \\
		&\ll k^{-1/2}\cdot \|f'\|_2 ,
\end{align*}
using Cauchy--Schwartz for the third inequality and the Plancherel theorem for 
the fourth.
\end{proof}

\begin{theorem}
Fix $x\in T$ with $\omega_0(x)$ finite, and let $x_n = n x\mod\pi\in T/W$. 
Then if $f\in \AC(T/W)$ with $\int_0^\pi f(t)\, \dd t=0$, we have 
\[
	\left|\sum_{n\leqslant N} f(x_n)\right| \ll ?
\]
\end{theorem}
\begin{proof}
We begin by splitting the sum in the theorem into two parts. Let $k\geqslant 0$ 
be arbitrary. Then 
\[
	\left|\sum_{n\leqslant N} f(x_n)\right| 
		\leqslant \sum_{|m|\leqslant k} |\widehat f(m)| \left| \sum_{n\leqslant N} e^{i m x_n}\right| 
		+ \sum_{n\leqslant N} |S_k f(x_n) - f(x_n)| .
\]
Recall that $|\widehat f(m)| \leqslant \frac{1}{|m|} |\widehat g(m)|$ and the 
Fourier coefficients of $g$ are bounded. Moreover, we already know that 
\[
	\left|\sum_{n\leqslant N} e^{im x_n}\right| \ll_{\epsilon,x} |m|^{\omega_0(x)+\epsilon} ,
\]
which tells us that 
\[
	\sum_{|m|\leqslant k} |\widehat f(m)| \left| \sum_{n\leqslant N} e^{i m x_n}\right|
		\ll_f \sum_{|m|\leqslant k} |m|^{-1+\omega_0(x)+\epsilon} 
		\ll_f \frac{1}{\omega_0(x)+\epsilon} (k^{\omega_0(x)+\epsilon}-1)
\]

Combining everything with the previous bound on $\|S_k f-f\|_\infty$, we get 
\[
	\left|\sum_{n\leqslant N} f(x_n)\right| \ll_{f,x,\epsilon} \log(k) |m|^?
\]
\end{proof}

\ldots we only get $\ll N^{\alpha(\omega_0(x)+\epsilon)} + N^{1-\alpha/2}$. Best that 
can be done is 

$\max(aw, 1-a/2)$

$aw=1-a/2$

$(w+1/2) a = 1$

$a=1/(w+1/2)$

So, for $w\in (1/2,1)$, the best power of $N$ we can get as a bound for the sums 
is 
\[
	\frac{w}{w+1/2}
\]
As $w\to 1$, the power is $<2/3$. 





\end{document}
