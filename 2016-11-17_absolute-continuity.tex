\documentclass{article}

\usepackage{amsmath,amssymb,amsthm}
\DeclareMathOperator{\AC}{AC}
\DeclareMathOperator{\Lie}{Lie}
\DeclareMathOperator{\SU}{SU}
\newcommand{\bR}{\mathbf{R}}
\newcommand{\bZ}{\mathbf{Z}}
\newcommand{\dd}{\mathrm{d}}
\newcommand{\ft}{\mathfrak{t}}

\newtheorem{theorem}{Theorem}
\theoremstyle{definition}
\newtheorem{definition}{Definition}

\title{Absolute continuity and Fourier coefficients}
\author{Daniel Miller}

\begin{document}
\maketitle





Consider the compact Lie group $\SU(2)$. It has an obvious maximal torus, 
namely 
\[
	T = \left\{\begin{pmatrix} e^{i\theta} \\ & e^{-i\theta}\end{pmatrix} : \theta\in [0,2\pi)\right\} .
\]
The Weyl group is 
\[
	W = \left\{\begin{pmatrix} 1 \\ & 1 \end{pmatrix}, \begin{pmatrix} & -1 \\ 1 \end{pmatrix}\right\} ,
\]
whose non-trivial element acts on $T$ by $\theta\mapsto -\theta$. It is 
well-known that the map $T/W \to \SU(2)^\natural$ is a bijection. We use it to 
make a couple definitions. First, note that for any function on $T$, we will 
write 
\[
	f(\theta) = f\begin{pmatrix} e^{i\theta} \\ & e^{-i\theta} \end{pmatrix} .
\]
Moreover, for $f\in L^1(T)$, we have the Fourier coefficients:
\[
	\widehat f(m) = \frac{1}{2\pi} \int_0^{2\pi} f(\theta) e^{i m \theta}\, \dd \theta .
\]

\begin{definition}
A function $f\in L^1(\SU(2)^\natural)$ is \emph{absolutely continuous} if it is 
the descent of a $W$-invariant absolutely continuous function on $T$. In 
other words, $\AC(T/W) = \AC(T)^W$. 
\end{definition}

Recall that $f\in C(T)$ is absolutely continuous if there exists $g\in L^1(T)$ 
for which 
\[
	f(\theta) = f(0) + \int_0^\theta g(t)\, \dd t, \qquad \theta\in [0,2\pi) .
\]
Note that if $f\in \AC(T/W)$, the corresponding $g$ may not descend to 
$T/W$. 

\begin{theorem}
If $f\in \AC(T/W)$, then $\widehat f(m) = ? \widehat g(m)$. 
\end{theorem}
\begin{proof}
We compute directly:
\begin{align*}
	\widehat f(m) 
		&= \frac{1}{2\pi} \int_0^{2\pi} \left(f(0) + \int_0^\theta g(t)\, \dd t\right) e^{i m \theta}\, \dd \theta \\
		&= \frac{f(0)}{2\pi} \int_0^{2\pi} e^{i m \theta}\, \dd\theta + \frac{1}{2\pi} \int_0^{2\pi} g(t)\int_t^{2\pi} e^{i m \theta} \,\dd\theta\dd t \\
		&= f(0) \delta_{m=0} -\frac{i}{2m\pi} \int_0^{2\pi} g(t) (e^{2\pi i m} - e^{i m t})\, \dd t \\
		&= \begin{cases} f(0) & m=0 \\ \frac{i}{m} \widehat g(m) & \text{else} \end{cases} .
\end{align*}
\end{proof}

\begin{theorem}
Let $f\in \AC(T/W)$. Then 
\[
	\|S_k f - f\|_\infty \ll k^{-1/2}\cdot \|f'\|_2 .
\]
\end{theorem}
\begin{proof}
Recall that $S_k f(x) = \sum_{|m|\leqslant k} \widehat f(m) e^{i m x}$. Note 
that 
\begin{align*}
	|S_k f(x) - f(x)| 
		&\leqslant \sum_{|m|>k} |\widehat f(m)| \\
		&\ll \sum_{|m|>k} \frac{1}{m} |\widehat g(m)| \\
		&\leqslant \sqrt{\sum_{|m|>k} m^{-2}} \sqrt{\sum_{|m|>k} |\widehat g(m)|^2} \\
		&\ll k^{-1/2}\cdot \|f'\|_2.
\end{align*}
\end{proof}





\end{document}
