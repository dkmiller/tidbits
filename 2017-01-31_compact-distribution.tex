\documentclass{article}

\usepackage{amsmath,amssymb}
\DeclareMathOperator{\tr}{tr}
\newcommand{\dd}{\mathrm{d}}

\title{Equidistributed subgroups in compact Lie groups}
\author{Daniel Miller}

\begin{document}
\maketitle





Let $G$ be a compact connected Lie group, $\Gamma\subset G$ a dense free 
subgroup with two generators $\gamma_1,\gamma_2$. For $n\to \infty$, write 
$\Gamma_n$ for the set of all products of $n$ letters taken from 
$\{\gamma_1,\gamma_2\}$. Put 
\[
	\mu_n(f) = 2^{-n} \sum_{\gamma\in \Gamma_n} f(\gamma) .
\]
Claim: $\mu_n$ converge to the Haar measure of $G$. It is sufficient to prove 
that $\mu_n(\tr\rho) \to 0$ for all non-trivial irreducible $\rho$. 

Recall the left-translation operators 
\[
	L_\gamma f(x) = f(\gamma^{-1} x). 
\]
Fact: 
\[
	\mu_n = \left(\frac{L_{\gamma_1} + L_{\gamma_2}}{2}\right)^n .
\]
What do we have to prove: 
\[
	\left\| \left( \frac{L_{\gamma_1} + L_{\gamma_2}}{2}\right)^n\right\| < 1
\]
for all $n$, otherwise\ldots. 





\section{General perspective}

Let $G$ be a compact connected semisimple group. Then there is a dense subgroup 
$\Gamma\subset G$ generated by two elements. Claim: for ``almost all'' pairs 
$\gamma_1,\gamma_2$, the group $\Gamma=\langle \gamma_1,\gamma_2\rangle$ is 
free and dense in $G$. 

For any $n$, let $\Gamma_n$ be the ``ball'' in $\Gamma$ consisting of all 
products of $n$ elements from the set $\{\gamma_1^{\pm 1}, \gamma_2^{\pm 1}\}$. 
Consider 
\[
	\mu_n = \frac{1}{\# \Gamma_n} \sum_{\gamma\in \Gamma_n} \delta_\gamma .
\]
Claim: if $\rho\in \widehat G$ (so $\rho$ is an irreducible unitary 
representation of $G$) then $\mu_n(\tr\rho) \to 0$. 

Note that: 
\[
	\mu_1 = \left.\frac 1 4 \left( L_{\gamma_1}+ L_{\gamma_1^{-1}} + L_{\gamma_2} + L_{\gamma_2^{-1}}\right) \right|_{x = 0}
\]

What is $\delta_\gamma\ast f$?
\begin{align*}
	(\delta_\gamma\ast f)(S) 
		&= \iint 1_S(x y) \, \dd \delta_\gamma(x) f(y)\dd y \\
		&= \int 1_S(\gamma y) f(y)\, \dd y \\
		&= \int 1_S(y) f(\gamma^{-1} y)\, \dd y \\
		&= \int_S L_\gamma f .
\end{align*}
In other words, $\delta_\gamma\ast f = L_\gamma f$. Also, let's see what is 
\begin{align*}
	(\delta_\gamma \ast \delta_\eta)(S) 
		&= \iint 1_S(x y)\, \dd \delta_\gamma(x) \dd \delta_\eta(y) \\
		&= \int 1_S(\gamma y)\, \dd \delta_\eta(y) \\
		&= 1_S(\gamma\eta) \\
		&= \delta_{\gamma\eta}(S) .
\end{align*}
In other words, 
$\delta_{\gamma_1} \ast \delta_{\gamma_2} = \delta_{\gamma_1 \gamma_2}$. 

So, if $\Gamma=\langle \gamma_1,\gamma_2\rangle$ is free on two generators, then 
for 
\[
	\mu = \frac 1 4 \left(\delta_{\gamma_1} + \delta_{\gamma_2} + \delta_{\gamma_1^{-1}} + \delta_{\gamma_2^{-1}}\right) 
\]
the measure $\mu^{\ast n}$ is the $n$-th ``empirical measure'' $\mu_n$ above. 





\end{document}
