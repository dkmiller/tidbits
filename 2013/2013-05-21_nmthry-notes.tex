\documentclass{article}

\title{Notes on algebraic number theory}
\author{Daniel Miller} % created 5-21-2013


\usepackage{amsmath,amssymb,amsthm,fullpage}
\usepackage{hyperref}
\hypersetup{colorlinks=true}

\DeclareMathOperator{\aut}{Aut}
\DeclareMathOperator{\gal}{Gal}
\DeclareMathOperator{\rk}{rk}

\newcommand{\dQ}{\mathbb{Q}}
\newcommand{\dZ}{\mathbb{Z}}

\newtheorem{definition}{Definition}
\newtheorem{theorem}[definition]{Theorem}
\newtheorem{corollary}[definition]{Corollary}
\newtheorem{lemma}[definition]{Lemma}
\numberwithin{definition}{subsection}


\begin{document}
\maketitle










\section{Symmetric polynomials and resultants}





\subsection{Symmetric polynomials}

For $r\leqslant n$, the $r$th \emph{elementary symmetric polynomial} in $n$ 
variables is defined by 
\[
  s_r(X_1,\dotsc,X_n) = \sum_{i_1\leqslant \cdots \leqslant i_r} X_{i_1} \dotsm X_{i_r}
\]
It is easy to see that $s_r$ is invariant under permutation of the $X_i$. In 
fact, $s_r$ is (up to a factor of $\pm 1$) the coefficient of $X^r$ in the product 
\[
  (X-X_1) \dotsm (X-X_n) = X^n - s_1 X^{n-1} + \cdots + (-1)^n s_n
\]
Now let $A$ be a commutative ring, and let $S_n$ act on $A[X_1,\dotsc,X_n]$ by 
$\sigma X_i = X_{\sigma i}$. 

\begin{theorem}
The map $A[X_1,\dotsc,X_n] \to A[X_1,\dotsc,X_n]^{S_n}$ given by 
$X_i\mapsto s_i$ is a ring isomorphism. 
\end{theorem}





\subsection{Resultants}

\begin{definition}\label{def-resultant}
Let $A$ be a commutative ring, $f,g\in A[X]$. The \emph{resultant} of $f$ and 
$g$, written $R(f,g)$, is the determinant of the matrix 
\[
  \begin{pmatrix}
    a_n    & a_{n-1} & a_{n-2} & \cdots & 0      & 0      & 0      \\
    0      & a_n     & a_{n-1} & \cdots & 0      & 0      & 0      \\
    \vdots & \vdots  & \vdots  &        & \vdots & \vdots & \vdots \\
    0      & 0       & 0       & \cdots & a_1    & a_0    & 0      \\
    0      & 0       & 0       & \cdots & a_2    & a_1    & a_0    \\ 
    b_m    & b_{m-1} & b_{m-2} & \cdots & 0      & 0      & 0      \\
    0      & b_m     & b_{m-1} & \cdots & 0      & 0      & 0      \\
    \vdots & \vdots  & \vdots  &        & \vdots & \vdots & \vdots \\
    0      & 0       & 0       & \cdots & b_1    & b_0    & 0      \\
    0      & 0       & 0       & \cdots & b_2    & b_1    & b_0    \\ 
  \end{pmatrix}
  \in M_{m+n}(A)
\]
where $f = \sum a_i X^i$ and $g = \sum b_i X^i$. 
\end{definition}

The following theorem is fundamental.
\begin{theorem}\label{resultant-main-theorem}
Let $k$ be a field with $f,g\in k[X]$. Let $s_1,\dotsc,s_n$ be the roots of 
$f$, $t_1,\dotsc,t_m$ be the roots of $g$, with multiplicities. Then 
\[
  R(f,g) = a_n^m b_m^n \prod_{i,j} (s_i-t_j)
\]
\end{theorem}
\begin{proof}
This is Theorem 1.6 of \cite{Jan}. 
\end{proof}

As a corollary, we see that if $f$ is a polynomial, then $f$ is separable if 
and only if $R(f,f') \ne 0$, so that separable polynomials of fixed degree are 
dense (open, in fact) in the Zariski topology. 










\section{Field extensions}





If $k$ is a field, $\bar k = k^a$ denotes a fixed algebraic closure of $k$. We 
will write $G_k=\gal(\bar k/k)$ for the absolute Galois group of $k$. 





\subsection{Separability}

If $k$ is a field, $k^s$ denotes the separable closure of $k$. We will also 
write $G_k$ for $\gal(k^s/k)$, since this is canonically isomorphic to 
$\gal(\bar k/k)$. 

\begin{theorem}\label{separability-1}
Let $k$ be a field, $K/k$ a (not-necessarily algebraic) extension such that 
$K\cap k^s=k$. If $f\in k[X]$ is irreducible and separable, then $f$ is also 
irreducible over $K$. 
\end{theorem}
\begin{proof}
Suppose $f = g h$ over $K$. Since the roots of $f$ are all separable over $k$, 
we have $g,h\in k^s[X]$. But also $g,h\in K[X]$, so $g,h\in k[X]$, which 
forces one of $f,g$ to be a unit. 
\end{proof}










\section{Valuations}





\subsection{Definitions and notation}

\begin{definition}
Let $k$ be a field. A \emph{valuation} on $k$ is a homomorphism 
$v:k^\times \to \Gamma$, where $\Gamma$ is a totally-ordered abelian group, 
such that $v(x+y)\geqslant\inf\{v(x),v(y)\}$ for all $x,y\in k^\times$. 
\end{definition}

If $v:k^\times \to \Gamma$ is a valuation, we call $\Gamma_v=v(k^\times)$ the 
\emph{value group} of $v$. We say that two valuations $v$, $w$ are 
\emph{equivalent} if there is an isomorphism of ordered groups 
$f:\Gamma_v\to \Gamma_w$ such that $f\circ v=w$. We will often regard 
equivalent valuations as identical. If $k\subset K$ are fields with valuations 
$v,w$, we say that $w$ \emph{divides} $v$, and write $w\mid v$, if $w|_k=v$. 
The \emph{rank} of a valuation is defined to be 
\[
  \rk(v)=\rk_\dZ(\Gamma_v\otimes\dQ) \text{.}
\]
We will generally consider valuations of rank one, in which case we will 
tacitly assume the value group is a subgroup of $\dQ$. 

\begin{definition}
Let $v$ be a valuation on $k$. Set 
\begin{itemize}
  \item $k^\circ = \{x\in k:v(x)\geqslant 0\}$, the \emph{ring of integers}, 
  \item $k^+ = \{x\in k : v(x)>0\}$, the \emph{maximal ideal}, 
  \item $k^\natural = k^\circ/k^+$, the \emph{residue field}, 
  \item $k^\wedge$, the \emph{completion}. 
\end{itemize}
\end{definition}
If two of these operations are applied successively, no parentheses will be 
used -- one should interpret the leftmost as being applied first, the rest in 
order from left to right. For example, $k^{\wedge+}$ denotes the unique 
maximal ideal in the ring of integers of $k^\wedge$. 

The \emph{residue characteristic} of $k$ is the characteristic of 
$k^\natural$. We say that $k$ is \emph{mixed characteristic} if 
$k$ has characteristic zero and $k^\natural$ has positive characteristic. 





\subsection{Extensions of valuations}

\begin{theorem}\label{valuations-extend}
Let $k$ be a field with valuation $v$, $K/k$ a field extension. Then there is 
a valuation $w$ on $K$ with $w\mid v$. 
\end{theorem}
\begin{proof}
This is \cite[III.4.3 pr.5]{Bou}. 
\end{proof}

Throughout this section, $k$ will be a field with valuation $v$, and $K/k$ 
will be an extension, with a valuation $w\mid v$ on $K$. If 
$\sigma\in\gal(K/k)$, then $w^\sigma(x) = w(\sigma x)$. One readily verifies 
that this induces a right action of $\gal(K/k)$ on the valuations of $K$ above 
$v$. The stabilizer is the \emph{decomposition subgroup}:
\[
  D_w = \{\sigma\in \gal(K/k) : w^\sigma = w\}
\]
There is a natural map $D_w\to \gal(K^\natural/k^\natural)$. For 
$\sigma\in D_w$, define $\bar\sigma$ on $K^\natural$ by 
$\bar\sigma(\bar x) = \overline{\sigma x}$; this is well-defined because 
$\sigma\in D_w$. The kernel of this map is called the \emph{decomposition 
subgroup}
\[
  I_w = \ker\left(D_w \to \gal(K^\natural/k^\natural)\right)
\]
If $w$ is a ``canonical'' extension of $v$ to $K$, or if the choice of such an 
extension does not matter, we will sometimes write $D_v$ and $I_v$ instead of 
$D_w$ and $I_w$. If $K$ has not been given, $D_v$ and $I_v$ denote the 
subgroups of $G_k$ induced by some extension of $v$ to $k^s$. Such an 
extension exists by \ref{valuations-extend}. 





\subsection{Ramification and inertia}

\begin{definition}
Let $k$ be a field with valuation $v$, and $K$ an extension with $w\mid v$. 
Define the \emph{ramification index} by $f = f_{w/v} = [\Gamma_w:\Gamma_v]$. 
\end{definition} 

\begin{definition}
Let $k$ be a field with valuation $v$, and $K$ an extension of $k$ with 
valuation $w\mid v$. The \emph{inertia degree} of $K/k$ is 
$e = e_{w/v} = [K^\natural:k^\natural]$. 
\end{definition}





\subsection{Henselian fields and rings}

\begin{theorem}\label{henselian-defs}
For a field $k$ with valuation, the following are equivalent:
\begin{enumerate}
  \item Any finite $k^\circ$-algebra is a direct product of local rings. 
  \item If $f\in k^\circ[X]$ is monic, then for every factorization 
    $f=g_0 h_0$ where $g_0,h_0\in k^\natural[X]$ are relatively prime, there 
    exist monic $g,h\in k^\circ[X]$ with $f = g h$ and 
    $\bar g=g_0,\bar h=h_0$. 
  \item If $K/k$ is an algebraic extension, then the valuation on $k$ admits a 
    unique extension to $K$. 
\end{enumerate}
\end{theorem}
\begin{proof}
The equivalence $1\Leftrightarrow 2$ is \cite[III.4 ex.3]{Bou}, while 
$2\Leftrightarrow 3$ is \cite[II.6.6]{Neu}. 
\end{proof}

\begin{definition}
A valued field $k$ is \emph{henselian} if any of the conditions of the 
previous theorem hold. 
\end{definition}

We will often say ``let $k$ be a henselian field'' with the valuation assumed. 
This will note generally cause harm because by \cite[II.6 ex.3]{Neu}, a field 
that is henselian with respect to two inequivalent valuations is already 
separably closed. If $k$ is a henselian field and $K/k$ is an algebraic 
extension, we will generally assume that $K$ is equipped with the unique 
valuation extending that of $k$.

\begin{theorem}\label{complete-henselian}
Let $k$ be a field that is complete with respect to a valuation. Then $k$ is 
henselian. 
\end{theorem}
\begin{proof}
This is \cite[III.4.3 th.1]{Bou}. 
\end{proof}

\begin{theorem}
Let $\{A_\alpha\}$ be a directed system of Henselian rings and local 
homomorphisms. Then the direct limit $\varinjlim A_\alpha$ is also Henselian. 
\end{theorem}
\begin{proof}
This is \cite[III.4 ex.3(a)]{Bou}. 
\end{proof}

\begin{lemma}\label{valuation-ring-integral-closure}
Let $k$ be a henselian field, $K/k$ an algebraic extension. Then $K^\circ$ is 
the integral closure of $k^\circ$ in $K$. 
\end{lemma}
\begin{proof}
If $x\in K^\circ$, then all the conjugates of $x$ over $k$ are also in 
$K^\circ$, hence the minimal polynomial of $x$ is in $k^\circ[X]$, i.e. $x$ is 
integral over $k^\circ$. Conversely, if $x$ is integral over $k^\circ$, let 
$f=X^n+\cdots + a_0$ be the minimal polynomial of $x$. From the fact that 
$v(a_i)\geqslant 0$ for all $i$, we deduce that $v(x)\geqslant 0$, i.e. 
$x\in K^\circ$. 
\end{proof}

\begin{corollary}\label{ext-of-hensel-is-hensel}
Let $k$ be a henselian field, $K/k$ an algebraic extension. Then $K$ is also 
henselian. 
\end{corollary}
\begin{proof}
This follows easily from \cite[III.3 ex.3(c)]{Bou}, which states that a local 
integral extension of a henselian ring is henselian. Use 
\ref{valuation-ring-integral-closure} to note that $K^\circ$ is integral over 
$k^\circ$. 
\end{proof}





\subsection{Completion and algebraic closure}

Let $v$ be a valuation on a field $k$, and let $\sigma$ be an automorphism of 
$k$. We define $v^\sigma$ by $v^\sigma(x) = v(\sigma x)$. It is easy to see 
that this gives a right action of $\aut(k)$ on the valuations of $k$. 

We begin with a lemma.
\begin{lemma}[Krasner]\label{lemma-krasner}
Let $k$ be a henselian valued field, $K=k(x)$ a finite separable extension. If 
$y\in k^s$ satisfies $v(y-x) > v(y-\sigma x)$ for all $\sigma\in G_k$ with 
$\sigma x\ne x$, then $k(x)\subset k(y)$. 
\end{lemma}
\begin{proof}
It is equivalent to prove that $G_{k(y)}\subset G_{k(x)}$. If not, then there 
is some $\sigma\in G_k$ such that $\sigma y = y$ but $\sigma x\ne x$. One then 
computes 
\[
  v(y-\sigma x) = v(\sigma y - \sigma x) = v(y-x) > v(y-\sigma x)\text{,}
\]
a contradiction. We have $v(\sigma t) = v(t)$ for all $t\in k^s$ because 
$v^\sigma$ is also a valuation on $k^s$ extending $v$, and such valuations are 
unique by \ref{henselian-defs}. 
\end{proof}

\begin{corollary}\label{henselian-ext-induced-by-dense-subfield}
Let $k$ be a henselian field, $K/k$ a finite separable extension. If 
$k_0\subset k$ is dense, then $K = k(x)$ for some $x\in k_0^s$. 
\end{corollary}
\begin{proof}
Write $K=k(x)$ for some $x\in k^s$. Let $f\in k[X]$ be the minimal polynomial 
of $x$, $n=\deg f$. We interpret elements of affine $n$-space $k^n$ as 
degree $n$ monic polynomials via 
\[
  (a_0,\dotsc,a_{n-1}) \leftrightarrow X^n + \cdots + a_1 X + a_0 = a\in k[X]\text{.}
\]
Let $R$ denote the resultant (\ref{def-resultant}), and define $\phi:k^n\to k$ 
by 
\[
  (a_0,\dotsc,a_{n-1}) \mapsto R(X^n + a_{n-1} X^{n-1} + \cdots + a_0,f)
\]
This is a polynomial mapping, so it is continuous. Let 
$N = \sup\{v(x-\sigma x) : x\ne \sigma x\}$. Consider the open set 
\[
  U = \{a\in k^n : \text{$a$ separable and $v(\phi a)>n^2 N$}\}
\]
Since $k_0$ is dense in $k$, $U\cap k_0^n$ is nonempty, so there exists some 
separable $g\in k_0[X]$ with $v(R(f,g))>n^2 N$. By 
\ref{resultant-main-theorem}, $R(f,g) = \prod (x_i - y_j)$, where $x_i$ runs 
over the conjugates of $x$ and $y_i$ are the roots of $g$. Note further that 
\[
  n^2 \sup \{v(x_i - y_j)\}_{i,j} \geqslant v(R(f,g)) > n^2 N\text{,}
\]
so there exists $i,j$ with $v(x_i - y_j)>N$. After applying some 
$\sigma\in G_k$, we may assume $x_i = x$. An application of Krasner's lemma 
(\ref{lemma-krasner}) shows that $k(x)\subset k(y_j)$. Since 
$[k(y_j):k]\leqslant n$, we actually have equality. 
\end{proof}

\begin{corollary}\label{alg-closure-induced-by-subfield}
Let $k$ be a henselian field, $k_0\subset k$ a dense subfield. One has 
$k^s = k\cdot k_0^s$. 
\end{corollary}

\begin{corollary}\label{alg-closure-dense}
If $k$ is henselian, $k_0\subset k$ is dense, then $k_0^s$ is dense in 
$k^s$. 
\end{corollary}
\begin{proof}
Let $x\in k^s$. The field $K=k(x)$ has finite degree $n$ over $k$, so by 
\ref{henselian-ext-induced-by-dense-subfield}, $K = k(y)$ for some 
$y\in k_0^s$. It easily follows that $k_0(y)$ is dense in $K$. So, if 
$U\subset k^s$ is an arbitrary open set with $x\in U$, $U\cap K$ is open, so 
there exists $z\in k_0(y)\cap U$, i.e. $U\cap k_0^s\ne \varnothing$. 
\end{proof}

Let $k$ be a field with valuation $v$. The completion $k^\wedge$ of $k$ is 
henselian by \ref{complete-henselian}, so the induced valuation on $k^\wedge$ 
has a unique extension (also denoted $v$) to $k^{\wedge s}$. At the same time, 
the map $k\to k^\wedge$ extends to a non-canonical embedding 
$\iota:k^s\to k^{\wedge s}$. This yields a map $\iota_*:G_{k^\wedge}\to G_k$ 
given by $\iota_*\sigma = \iota^{-1}\sigma \iota$. Of course, $\iota^{-1}$ is 
not well-defined as a map $k^{\wedge s} \to k^s$, but it is well-defined on 
the image of $\iota$, which is preserved by $G_{k^\wedge}$. We set, for 
$x\in k^s$, $v(x) = v(\iota x)$. 

\begin{theorem}
Let $k$ be an arbitrary field with valuation $v$. The homomorphism 
$\iota_*:G_{k^\wedge} \to G_k$ is a continuous injection with image 
$G_v = \{x\in G_k : v^\sigma = v\}$. 
\end{theorem}
\begin{proof}
By the definition of $\iota_*$, its image is inside $G_v$. 

It is essentially trivial that $\iota_*$ is continuous. For, basic open sets 
in $G_k$ are translates of stabilizers of elements of $k^s$, and the preimage 
of such an open set is just the stabilizer in $G_{k^\wedge}$, which is also 
open. 

First, we prove that $\iota_*$ is injective. If $\iota_* \sigma = 1$, then 
``$\sigma|_{k^s}$'' is the identity map. By \ref{alg-closure-dense}, $k^s$ is 
dense in $k^{\wedge s}$, which forces $\sigma = 1$. 

Now we prove $\iota_*$ is surjective. If $\sigma\in G_v$, then define $\tau_0$ 
on $\iota k^s$ by $\tau_0 = \iota\sigma\iota^{-1}$. Then $\tau_0\in G_v$, so 
when restricted to each Galois $K/k$, $\tau_0$ extends by continuity to the 
completion $K^\wedge$. Since $k^{\wedge s}$ is the filtered union of the 
$K^\wedge$, $\tau_0$ extends by continuity to $\tau\in G_{k^\wedge}$, and 
clearly $\iota_* \tau = \sigma$. 
\end{proof}










\begin{thebibliography}{9}
  \bibitem{Bou} Bourbaki, N. \emph{Commutative algebra}. 
  \bibitem{Jan} Janson, S. \emph{Resultant and discriminant of polynomials}, 
    \texttt{http://www2.math.uu.se/\~{}svante/papers/sjN5.pdf}. 
  \bibitem{Neu} Neukirch, J. \emph{Algebraic number theory}. 
\end{thebibliography}





\end{document}
