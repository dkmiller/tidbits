\documentclass{article}

\usepackage[a5paper,margin=1.5cm]{geometry}
\usepackage{
	amsmath,
	amssymb,
	amsthm,
	bookmark,
	microtype,
	thmtools
}
\usepackage[
  hyperref = true,
  backend  = bibtex,
  sorting  = nyt,
  style    = alphabetic
]{biblatex}
\addbibresource{tidbit-sources.bib}
\hypersetup{colorlinks=true,linkcolor=green}

\DeclareMathOperator{\Ad}{Ad}
\DeclareMathOperator{\disc}{disc}
\DeclareMathOperator{\GL}{GL}
\DeclareMathOperator{\Li}{Li}
\DeclareMathOperator{\Lie}{Lie}
\DeclareMathOperator{\sgn}{sgn}
\DeclareMathOperator{\ST}{ST}
\DeclareMathOperator{\tr}{tr}
\newcommand{\bC}{\mathbf{C}}
\newcommand{\bQ}{\mathbf{Q}}
\newcommand{\bR}{\mathbf{R}}
\newcommand{\bx}{\boldsymbol{x}}
\newcommand{\by}{\boldsymbol{y}}
\newcommand{\bZ}{\mathbf{Z}}
\newcommand{\dd}{\mathrm{d}}
\newcommand{\ddiv}{\mathrm{div}}
\newcommand{\frob}{\mathrm{fr}}
\newcommand{\fg}{\mathfrak{g}}
\newcommand{\ft}{\mathfrak{t}}
\newtheorem{theorem}[subsection]{Theorem}
\newtheorem{lemma}[subsection]{Lemma}
\newtheorem{corollary}[subsection]{Corollary}
\theoremstyle{definition}
\newtheorem{definition}[subsection]{Definition}

\title{Analytic and arithmetic properties of a new class of \texorpdfstring{$L$}{L}-functions}
\author{Daniel Miller}

\begin{document}
\maketitle





\section{Introduction}

The work in this paper is inspired by the following example of Ramakrishna. Let 
$E_{/\bQ}$ be a non-CM elliptic curve. Let $l$ be an odd prime and 
$\rho_{E,l}\colon G_\bQ \to \GL_2(\bZ_l)$ the associated representation. Recall 
that $a_p = \tr(\rho_{E,l}(\frob_p))$, and these satisfy the Hasse bound 
$|a_p|<2\sqrt p$. Then we have the following curious $L$-function with only one 
Euler factor at each prime: 
\[
	L_{\sgn}(E,s) = \prod_p \frac{1}{1-\sgn(a_p) p^{-s}} .
\]
We are interested in the analytic and arithmetic properties of a class of 
$L$-functions generalized from this one. 

\begin{definition}
Let $\rho\colon G_\bQ \to \GL_n(\bZ_l)$ be geometric in the sense of 
\cite{fontain-mazur-1995}. Assume the Sato--Tate group of $\rho$ is 
well-defined; denote it by $\ST(\rho)$. Let 
$\eta\colon \ST(\rho)^\natural \to \bR$ be a function of bounded variation.  
\end{definition}

Some conventions. Let $X$ be a compact topological space and 
write $\bx = (x_2,x_3,\dots)$, $\by$, etc.~for sequences in $X$ 
indexed by the prime numbers. Given such a sequence, we write 
\[
	\bx^C(f) = \frac{1}{\pi(C)}\sum_{p\leqslant C} f(x_p) .
\]
So the $\bx^C$ are probability measures on $X$. 





\begin{lemma}[Abel summation]\label{lem:abel}
Let $\{x_p\}$ be a sequence of real numbers, $\phi\in C^1(\bR)$. Then 
\[
	\sum_{p\leqslant C} \phi(p) x_p = \phi(C) \sum_{p\leqslant C} x_p - \int_2^C \phi'(x) \sum_{p\leqslant x} x_p\, \dd x .
\]
\end{lemma}
\begin{proof}
Simply note that if $p_1,\dots,p_n$ is an enumeration of the primes 
$\leqslant X$, we have 
\begin{align*}
	\int_2^C \phi'(x) \sum_{p\leqslant x} x_p\, \dd x 
		&= \sum_{p\leqslant C} x_p\int_{p_n}^C \phi' + \sum_{i=1}^{n-1} \sum_{p\leqslant p_{i+1}}x_p\int_{p_i}^{p_{i+1}} \phi' \\
		&= (\phi(C)-\phi(p_n))\sum_{p\leqslant C} x_p + \sum_{i=1}^{n-1} (\phi(p_{i+1})-\phi(p_i)) \sum_{p\leqslant p_{i+1}} x_p \\
		&= \phi(C) \sum_{p\leqslant C} x_p - \sum_{p\leqslant X} \phi(p) x_p ,
\end{align*}
as desired.
\end{proof}





\begin{lemma}\label{debruijn-post}
Let $(X,\mu)$ be a separable metric space with Radon measure whose support is 
$X$. Let $f$ be a bounded function on $X$. Then the following condition holds:
\begin{quote}
$\lim_{C\to \infty} \bx^C(f) = \mu(f)$ for all $\mu$-equidistributed sequences $\bx$
\end{quote}
if and only if $f$ is continuous almost everywhere.
\end{lemma}
\begin{proof}
This follows from \cite{chersi-volcic-1992}, Corollary 1 and Remark 3. 

[Give my own proof.]
\end{proof}





\section{General setting}

Let $G$ be a compact, connected Lie group, $\bx=\{x_p\}$ a sequence in 
$G^\natural$. We write $\widehat G$ for the collection of irreducible unitary 
representations of $G$. 

\begin{definition}
Let $\eta\colon G^\natural \to \bC$ be bounded and continuous almost 
everywhere. Then the associated \emph{curious $L$-function} is 
\[
	L_\eta(s) = \prod_p \frac{1}{1-\eta(x_p) p^{-s}} ,
\]
wherever the product converges. 
\end{definition}

\begin{definition}
For $\rho\in \widehat G$, the associated $L$-function is defined following 
\cite{serre-1968}:
\[
	L(s,\rho) = \prod_p \frac{1}{\det(1-\rho(x_p) p^{-s})} .
\]
\end{definition}

\begin{lemma}
Assume $\|\eta\|_\infty \leqslant 1$. Then the product formula for 
$L_\eta(s)$ converges absolutely on $\{\Re s>1\}$. The function $L_\eta$ is 
holomorphic on that region. 
\end{lemma}
\begin{proof}
By \cite[\S3.7, Th.~5]{knopp-1956}, the product for $L_\eta(s)$ converges 
whenever $\Re s>1$. The rest is well-known ``general nonsense'' about Dirichlet 
series. 
\end{proof}

We are interested in the analytic continuation of $L_\eta(s)$ past $\Re s=1$, 
in particular to line $\Re s=\frac 1 2$. 

\begin{lemma}
Assume $\sum \frac{\eta(x_p)}{p^s}$ converges to a holomorphic function on  
$\{\Re s>s_0\}$, $s_0\in \left[\frac 1 2, 1\right]$. Then $L_\eta(s)$ can be 
analytically continued to a holomorphic function on $\{\Re s > s_0\}$. 
\end{lemma}
\begin{proof}
By \cite[11.9, Ex.~2]{apostol-1976}, on the domain of absolute convergence for 
$L_\eta$, we have 
\[
	L_\eta(s) = \exp\left(\sum_p \sum_{\nu\geqslant 1} \frac{\eta(x_p)^\nu}{\nu p^{\nu s}}\right) .
\]
So, it suffices to prove that the argument of $\exp$ converges on 
$\{\Re s>s_0\}$. Now note that 
\[
	\left|\sum_{n\geqslant 2} \frac{\eta(x_p)^\nu}{\nu p^{\nu s}}\right|
		\leqslant \sum_{\nu\geqslant 2} (p^{-\Re s})^\nu 
		= p^{-2 s} \frac{1}{1-p^{-s}} .
\]
Since $p\geqslant 2$ and $\Re s>1/2$, we have $1-2^{-1/2}<1-p^{-s}<1$, so the 
argument of $\exp$ converges if and only if 
$\sum_p \left(\frac{\eta(x_p)}{p^s} + p^{-2\Re s}\right)$ does. But 
$\sum p^{-2\Re s}$ converges absolutely, so we the desired result. 
\end{proof}

\ldots definition of star-discrepancy on $G^\natural$\ldots

\begin{theorem}
If $\disc^\ast(\bx^C) = O(C^{-\frac 1 2+\epsilon})$, then 
$\left|\int f - \bx^C(f)\right| = O_f(C^{-\frac 1 2+\epsilon})$
\end{theorem}

\begin{theorem}
Assume that $\left|\int_{G^\natural} f - \bx^C(f)\right| = O_f(C^{-\frac 1 2+\epsilon})$. 
If $\int \eta = 0$, then $L_\eta$ has analytic continuation to $\{\Re s=1/2\}$, and 
$\log L_\eta$ has no poles in that region. 

\end{theorem}
\begin{proof}
By \autoref{lem:abel} with $\phi(x)=x^{-s}$, we have 
\begin{align*}
	\sum_{p\leqslant C} \frac{\eta(x_p)}{p^s} 
		&= C^{-s} \sum_{p\leqslant C} \eta(x_p) + s \int_2^C \sum_{p\leqslant x} \eta(x_p)\, \frac{\dd x}{x^{s+1}} \\
		&= C^{-s}\Li(C) O(C^{-\frac 1 2+\epsilon}) + s \int_2^C \Li(x) O(x^{-\frac 1 2+\epsilon})\, \frac{\dd x}{x^{s+1}} .
\end{align*}
Since $\Li(x)=O(x/\log x)$, the first term is 
$O(C^{\frac 1 2 -s+\epsilon}/\log C) = o(1)$. We prove that the integral is 
absolutely convergent. Since $\Re s+\frac 1 2>1$ and $\epsilon$ is arbitrary, 
\[
	\int_2^C \frac{x^{\epsilon-\frac 1 2-\Re s}}{\log x}\, \dd x
\]
converges, and the proof is complete. 
\end{proof}





\begin{theorem}
Let $\eta\colon G^\natural\to \bC$ be bounded and have bounded variation, and 
moreover $\int \eta = 0$. Then 
\[
	\sum_p\frac{\eta(x_p)}{p^s}
	\qquad \textnormal{and}\qquad
	\sum_p \frac{\log p}{p^s} \eta(x_p)
\]
are holomorphic on the region $\{\Re s>\frac 1 2\}$. 
\end{theorem}
\begin{proof}
We use Abel summation:
\[
	\sum_{p\leqslant C} \frac{\log p}{p^s} \eta(x_p) 
		= \frac{\log C}{C^s} \sum_{p\leqslant C} \eta(x_p) - \int_2^C \frac{1-s\log x}{x^{s+1}} \sum_{p\leqslant x} \eta(x_p)\, \dd x .
\]
We show the the first term approaches zero and that the integral converges 
absolutely. We have:
\[
	\left|\frac{\log C}{C^s} \sum_{p\leqslant C} \eta(x_p)\right| \ll \frac{\log C}{C^s} \frac{C}{\log C} C^{-\frac 1 2+\epsilon} = C^{1-s-\frac 1 2+\epsilon} .
\]
Since $\epsilon$ is arbitrary, the exponent of $C$ is negative. Moreover:
\begin{align*}
	\int_2^C \frac{1}{x^{s+1}} \left|\sum_{p\leqslant x} \eta(x_p)\right|\, \dd x 
		&\ll \int_2^C \frac{1}{x^{s+1}} \frac{x}{\log x} x^{-\frac 1 2 +\epsilon}\, \dd x \\
	\int_2^C \frac{\log x}{x^{s+1}} \left|\sum_{p\leqslant x} \eta(x_p)\right|\, \dd x 
		&\ll \int_2^C \frac{\log x}{x^{s+1}} \frac{x}{\log x} x^{-\frac 1 2 +\epsilon}\, \dd x .
\end{align*}
Both of these integrals converge because $\epsilon$ is arbitrary. 
\end{proof}

\begin{corollary}
If $\rho\in \widehat G$, then RH + analytic continuation to 
$\{\Re s>\frac 1 2\}$ holds for $L(s,\rho)$. 
\end{corollary}
\begin{proof}
Take logarithmic derivative, reduce to the previous theorem. 
\end{proof}





\section{Discrepancy on compact Lie groups}

Let $G$ be a compact, connected Lie group. Let $G^\natural$ be the space of 
conjugacy classes of $G$. We will define \emph{star-discrepancy} for sequences 
on $G^\natural$. Let $T$ be a maximal torus in $G$. Then the exponential map 
$\exp\colon \ft\to T$ is surjective. Choose a basis $\{t_1,\dots,t_r\}$ for 
$\ft$. Then we can identify $G^\natural$ with 
\[
	\int_{G^\natural} f(x)\, \dd x = \frac{1}{\# W}\int_T \det(1-\Ad(t^{-1})|_{\fg/\ft}) f(t)\, \dd t .
\]

\section{Some examples}

Let $G$ be a compact connected abelian lie group, and $g\in G$ such that 
$g^\bZ\subset G$ is equidistributed. Then for any function $\eta$ on $G$ with 
$\|\eta\|_\infty\leqslant 1$, we have a curious $L$-function:
\[
	L_\eta(g,s) = \prod_p \frac{1}{1-\eta(g^p) p^{-s}} .
\]

[Bla bla general nonsense.]

For example, let $G=\bR/\bZ$ and $\alpha\in \bR$ be an algebraic irrational, 
for example $\alpha=\sqrt 2$. Then the corresponding $L$-function is:
\[
	L_{\exp(2\pi it)}(\alpha,s) = \prod_p \frac{1}{1-\exp(2\pi i \alpha p) p^{-s}}
\]





\printbibliography
\end{document}
