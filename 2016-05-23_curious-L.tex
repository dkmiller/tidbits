\documentclass{article}

\usepackage[a5paper,margin=1.5cm]{geometry}
\usepackage{
	amsmath,
	amssymb,
	amsthm,
	hyperref,
	microtype
}
\usepackage[
  hyperref = true,
  backend  = bibtex,
  sorting  = nyt,
  style    = alphabetic
]{biblatex}
\addbibresource{tidbit-sources.bib}
\hypersetup{colorlinks=true,linkcolor=green}

\DeclareMathOperator{\GL}{GL}
\DeclareMathOperator{\Lie}{Lie}
\DeclareMathOperator{\sgn}{sgn}
\DeclareMathOperator{\ST}{ST}
\DeclareMathOperator{\tr}{tr}
\newcommand{\bQ}{\mathbf{Q}}
\newcommand{\bR}{\mathbf{R}}
\newcommand{\bx}{\boldsymbol{x}}
\newcommand{\by}{\boldsymbol{y}}
\newcommand{\bZ}{\mathbf{Z}}
\newcommand{\dd}{\mathrm{d}}
\newcommand{\ddiv}{\mathrm{div}}
\newcommand{\frob}{\mathrm{fr}}
\newcommand{\ft}{\mathfrak{t}}
\newtheorem{theorem}{Theorem}
\newtheorem{lemma}[theorem]{Lemma}
\theoremstyle{definition}
\newtheorem{definition}[theorem]{Definition}
\numberwithin{theorem}{section}

\title{Analytic and arithmetic properties of a new class of \texorpdfstring{$L$}{L}-functions}
\author{Daniel Miller}

\begin{document}
\maketitle





\section{Introduction}

The work in this paper is inspired by the following example of Ramakrishna. Let 
$E_{/\bQ}$ be a non-CM elliptic curve. Let $l$ be an odd prime and 
$\rho_{E,l}\colon G_\bQ \to \GL_2(\bZ_l)$ the associated representation. Recall 
that $a_p = \tr(\rho_{E,l}(\frob_p))$, and these satisfy the Hasse bound 
$|a_p|<2\sqrt p$. Then we have the following curious $L$-function with only one 
Euler factor at each prime: 
\[
	L_{\sgn}(E,s) = \prod_p \frac{1}{1-\sgn(a_p) p^{-s}} .
\]
We are interested in the analytic and arithmetic properties of a class of 
$L$-functions generalized from this one. 

\begin{definition}
Let $\rho\colon G_\bQ \to \GL_n(\bZ_l)$ be geometric in the sense of 
\cite{fontain-mazur-1995}. Assume the Sato--Tate group of $\rho$ is 
well-defined; denote it by $\ST(\rho)$. Let 
$\eta\colon \ST(\rho)^\natural \to \bR$ be a function of bounded variation.  
\end{definition}

Some conventions. Let $X$ be a compact topological space and 
write $\bx = (x_2,x_3,\dots)$, $\by$, etc.~for sequences in $X$ 
indexed by the prime numbers. Given such a sequence, we write 
\[
	\bx^C(f) = \frac{1}{\pi(C)}\sum_{p\leqslant C} f(x_p) .
\]
So the $\bx^C$ are probability measures on $X$. 





\begin{lemma}[Abel summation]
Let $\{x_p\}$ be a sequence of real numbers, $\phi\in C^1(\bR)$. Then 
\[
	\sum_{p\leqslant C} \phi(p) x_p = \phi(C) \sum_{p\leqslant C} x_p - \int_2^C \phi'(x) \sum_{p\leqslant x} x_p\, \dd x .
\]
\end{lemma}
\begin{proof}
Simply note that if $p_1,\dots,p_n$ is an enumeration of the primes 
$\leqslant X$, we have 
\begin{align*}
	\int_2^C \phi'(x) \sum_{p\leqslant x} x_p\, \dd x 
		&= \sum_{p\leqslant C} x_p\int_{p_n}^C \phi' + \sum_{i=1}^{n-1} \sum_{p\leqslant p_{i+1}}x_p\int_{p_i}^{p_{i+1}} \phi' \\
		&= (\phi(C)-\phi(p_n))\sum_{p\leqslant C} x_p + \sum_{i=1}^{n-1} (\phi(p_{i+1})-\phi(p_i)) \sum_{p\leqslant p_{i+1}} x_p \\
		&= \phi(C) \sum_{p\leqslant C} x_p - \sum_{p\leqslant X} \phi(p) x_p ,
\end{align*}
as desired.
\end{proof}





\begin{lemma}
Let $(X,\mu)$ be a compact separable space with Radon measure whose support is 
$X$. Let $f$ be a bounded function on $X$. Then the following condition holds:
\begin{quote}
$\lim_{C\to \infty} \bx^C(f) = \mu(f)$ for all $\mu$-equidistributed sequences $\bx$
\end{quote}
if and only if $f$ is continuous almost everywhere.
\end{lemma}
\begin{proof}
Suppose $f$ is a.e.~continuous. Let $\epsilon>0$. Then there exists open 
$U\subset X$ such that $f$ is continuous away from $U$ and $\mu(U)<\epsilon$. 
We now note that 
\[
	\lim_{C\to \infty} \bx^C(f)
		= \frac{1}{\pi(C)} \sum_{\substack{p\leqslant C \\ x_p\in U}} f(x_p) + \frac{1}{\pi(C)} \sum_{\substack{p\leqslant C \\ x_p\notin U}} f(x_p) .
\]
The first term is bounded above in absolute value by $\epsilon  \|f\|_\infty$, 
and the second term converges to $\int_{X\smallsetminus U} f\, \dd\mu$. Since 
\[
	\left|\int f\, \dd\mu - \int_{X\smallsetminus U} f\, \dd\mu\right| \leqslant \mu(U)\|f\|_\infty ,
\]
we get that 
\[
	\left|\lim_{C\to \infty} \bx^C(f) - \mu(f)\right| < 2\epsilon \|f\|_\infty ,
\]
which proves the desired result.

Conversely, suppose $f$ satisfies the condition. Let the \emph{oscillation} of 
$f$ be 
\[
	\omega_f(x) = \inf_{U\ni x} \sup_{u,v\in U} |f(u)-f(v)| ,
\] 
where the infimum ranges over all open $U$ containing $x$. Clearly $f$ is 
continuous at $x$ if and only if $\omega_f(x)=0$. 

We need to show that each set $\{\omega_f > \epsilon\}$ has measure zero, since 
$\{\omega_f>0\}$ is a countable union of such sets with $\epsilon>0$. Suppose 
for contradiction that $\mu(\omega_f>\epsilon)>0$. Then $\{\omega_f>\epsilon\}$ 
contains a compact set $K$ with positive measure. For every point $k\in K$ and 
every neighborhood $U\ni k$, there are points $x,y\in U\cap K$ such that 
$f(x)-f(y) > \epsilon$. Choose a countable dense set of the $k$, and extend 
the corresponding $x$'s and $y$'s (by the same points) to a dense set in $X$. 
By \cite[Th.~2.4, Ch.~3]{kuipers-niederreiter-1974}, the (extended) $x$'s and 
$y$'s can be rearranged to be $\mu$-equidistributed. Call these rearranged 
sequences $\bx$ and $\by$. Note that 
$\bx \smallsetminus K = \by \smallsetminus K$. Thus 
\begin{align*}
	\lim_{C\to \infty} (\bx^C-\by^C)(f) 
		&= \sum_{\substack{p\leqslant X \\ x_p\in K}}
\end{align*}
\end{proof}





\section{General setting}

Let $\rho\colon G_\bQ\to \GL(V)$ be a $\bQ_l$-representation that is ``nice.'' Let 
$\ST(\rho)$ be its Sato--Tate group; this is a compact Lie group. Recall that if 
$T\subset \ST(\rho)$ is a maximal torus and $W$ the corresponding Weyl group, 
then the natural map $T/W\to \ST(\rho)^\natural$ is an isomorphism. Since $T/W$ 
is the quotient of a torus, it makes sense to talk about (images of) cubes in 
$\ST(\rho)^\natural$, so the usual notion of discrepancy makes sense:
\[
	D_C
\]

Let $\ft=\Lie(T)$. Then $T=\ft/\pi_1(T)$, and the exponential map gives us a 
surjection $\ft/\pi_1(T) \twoheadrightarrow \ST(\rho)^\natural$. 





\printbibliography
\end{document}
