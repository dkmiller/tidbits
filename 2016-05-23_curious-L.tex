\documentclass{article}

\usepackage[a5paper,margin=1.5cm]{geometry}
\usepackage{
	amsmath,
	amssymb,
	amsthm,
	bookmark,
	microtype,
	thmtools
}
\usepackage[
  hyperref = true,
  backend  = bibtex,
  sorting  = nyt,
  style    = alphabetic
]{biblatex}
\addbibresource{tidbit-sources.bib}
\hypersetup{colorlinks=true,linkcolor=green}

\DeclareMathOperator{\Ad}{Ad}
\DeclareMathOperator{\disc}{disc}
\DeclareMathOperator{\Ei}{Ei}
\DeclareMathOperator{\GL}{GL}
\DeclareMathOperator{\Li}{Li}
\DeclareMathOperator{\Lie}{Lie}
\DeclareMathOperator{\sgn}{sgn}
\DeclareMathOperator{\ST}{ST}
\DeclareMathOperator{\SU}{SU}
\DeclareMathOperator{\tr}{tr}
\DeclareMathOperator{\Var}{Var}
\newcommand{\bC}{\mathbf{C}}
\newcommand{\bN}{\mathbf{N}}
\newcommand{\bQ}{\mathbf{Q}}
\newcommand{\bR}{\mathbf{R}}
\newcommand{\bx}{\boldsymbol{x}}
\newcommand{\by}{\boldsymbol{y}}
\newcommand{\bZ}{\mathbf{Z}}
\newcommand{\cM}{\mathcal{M}}
\newcommand{\dd}{\mathrm{d}}
\newcommand{\ddiv}{\mathrm{div}}
\newcommand{\frob}{\mathrm{fr}}
\newcommand{\fg}{\mathfrak{g}}
\newcommand{\ft}{\mathfrak{t}}
\newtheorem{theorem}[subsection]{Theorem}
\newtheorem{lemma}[subsection]{Lemma}
\newtheorem{corollary}[subsection]{Corollary}
\theoremstyle{definition}
\newtheorem{definition}[subsection]{Definition}

\title{Analytic and arithmetic properties of a new class of \texorpdfstring{$L$}{L}-functions}
\author{Daniel Miller}

\begin{document}
\maketitle





\section{Introduction}

The work in this paper is inspired by the following example of Ramakrishna. Let 
$E_{/\bQ}$ be a non-CM elliptic curve. Let $l$ be an odd prime and 
$\rho_{E,l}\colon G_\bQ \to \GL_2(\bZ_l)$ the associated representation. Recall 
that $a_p = \tr(\rho_{E,l}(\frob_p))$, and these satisfy the Hasse bound 
$|a_p|<2\sqrt p$. Then we have the following curious $L$-function with only one 
Euler factor at each prime: 
\[
	L_{\sgn}(E,s) = \prod_p \frac{1}{1-\sgn(a_p) p^{-s}} .
\]
We are interested in the analytic and arithmetic properties of a class of 
$L$-functions generalized from this one. 

\begin{definition}
Let $\rho\colon G_\bQ \to \GL_n(\bZ_l)$ be geometric in the sense of 
\cite{fontain-mazur-1995}. Assume the Sato--Tate group of $\rho$ is 
well-defined; denote it by $\ST(\rho)$. Let 
$\eta\colon \ST(\rho)^\natural \to \bR$ be a function of bounded variation.  
\end{definition}

Some conventions. Let $X$ be a compact topological space and 
write $\bx = (x_2,x_3,\dots)$, $\by$, etc.~for sequences in $X$ 
indexed by the prime numbers. Given such a sequence, we write 
\[
	\bx^C(f) = \frac{1}{\pi(C)}\sum_{p\leqslant C} f(x_p) .
\]
So the $\bx^C$ are probability measures on $X$. 





\begin{lemma}[Abel summation]\label{lem:abel}
Let $\{x_p\}$ be a sequence of real numbers, $\phi\in C^1(\bR)$. Then 
\[
	\sum_{p\leqslant C} \phi(p) x_p = \phi(C) \sum_{p\leqslant C} x_p - \int_2^C \phi'(x) \sum_{p\leqslant x} x_p\, \dd x .
\]
\end{lemma}
\begin{proof}
Simply note that if $p_1,\dots,p_n$ is an enumeration of the primes 
$\leqslant X$, we have 
\begin{align*}
	\int_2^C \phi'(x) \sum_{p\leqslant x} x_p\, \dd x 
		&= \sum_{p\leqslant C} x_p\int_{p_n}^C \phi' + \sum_{i=1}^{n-1} \sum_{p\leqslant p_{i+1}}x_p\int_{p_i}^{p_{i+1}} \phi' \\
		&= (\phi(C)-\phi(p_n))\sum_{p\leqslant C} x_p + \sum_{i=1}^{n-1} (\phi(p_{i+1})-\phi(p_i)) \sum_{p\leqslant p_{i+1}} x_p \\
		&= \phi(C) \sum_{p\leqslant C} x_p - \sum_{p\leqslant X} \phi(p) x_p ,
\end{align*}
as desired.
\end{proof}





\begin{lemma}\label{debruijn-post}
Let $(X,\mu)$ be a separable metric space with Radon measure whose support is 
$X$. Let $f$ be a bounded function on $X$. Then the following condition holds:
\begin{quote}
$\lim_{C\to \infty} \bx^C(f) = \mu(f)$ for all $\mu$-equidistributed sequences $\bx$
\end{quote}
if and only if $f$ is continuous almost everywhere.
\end{lemma}
\begin{proof}
This follows from \cite{chersi-volcic-1992}, Corollary 1 and Remark 3. 

[Give my own proof.]
\end{proof}





\section{General setting}

** new **

If $X$ is a space, we write $X^\infty$ be the set of sequences 
in $X$ indexed by primes, i.e. $\bx\in X^\infty$ is of the form 
$\bx=(x_2,x_3,\dots)$. 

\begin{theorem}
Let $(X,\mu)$ be a compact topological space with positive Radon measure, 
$\eta\colon X\to \bC$ almost-everywhere continuous with $\|\eta\|_\infty<2$. 
Then 
\begin{equation}\label{def:curious-L}
	L_\eta(s,\bx) = \prod_p \frac{1}{1-\eta(x_p) p^{-s}} .
\end{equation}
defines a continuous function on the domain 
$\bC^{\Re>1}\times X^\infty$, which is holomorphic in the first variable. 
\end{theorem}
\begin{proof}
The assumption $\|\eta\|_\infty<2$ means that each denominator is non-zero, so 
each term in the product is finite. 
\end{proof}

** old **

Let $G$ be a compact, connected Lie group, $\bx=\{x_p\}$ a sequence in 
$G^\natural$. We write $\widehat G$ for the collection of irreducible unitary 
representations of $G$. 

\begin{definition}
Let $\eta\colon G^\natural \to \bC$ be bounded and continuous almost 
everywhere. Then the associated \emph{curious $L$-function} is 
\[
	L_\eta(s) = \prod_p \frac{1}{1-\eta(x_p) p^{-s}} ,
\]
wherever the product converges. 
\end{definition}

\begin{definition}
For $\rho\in \widehat G$, the associated $L$-function is defined following 
\cite{serre-1968}:
\[
	L(s,\rho) = \prod_p \frac{1}{\det(1-\rho(x_p) p^{-s})} .
\]
\end{definition}

\begin{lemma}
Assume $\|\eta\|_\infty \leqslant 1$. Then the product formula for 
$L_\eta(s)$ converges absolutely on $\{\Re s>1\}$. The function $L_\eta$ is 
holomorphic on that region. 
\end{lemma}
\begin{proof}
By \cite[\S3.7, Th.~5]{knopp-1956}, the product for $L_\eta(s)$ converges 
whenever $\Re s>1$. The rest is well-known ``general nonsense'' about Dirichlet 
series. 
\end{proof}

We are interested in the analytic continuation of $L_\eta(s)$ past $\Re s=1$, 
in particular to line $\Re s=\frac 1 2$. 

\begin{lemma}
Assume $\sum \frac{\eta(x_p)}{p^s}$ converges to a holomorphic function on  
$\{\Re s>s_0\}$, $s_0\in \left[\frac 1 2, 1\right]$. Then $L_\eta(s)$ can be 
analytically continued to a holomorphic function on $\{\Re s > s_0\}$. 
\end{lemma}
\begin{proof}
By \cite[11.9, Ex.~2]{apostol-1976}, on the domain of absolute convergence for 
$L_\eta$, we have 
\[
	L_\eta(s) = \exp\left(\sum_p \sum_{\nu\geqslant 1} \frac{\eta(x_p)^\nu}{\nu p^{\nu s}}\right) .
\]
So, it suffices to prove that the argument of $\exp$ converges on 
$\{\Re s>s_0\}$. Now note that 
\[
	\left|\sum_{n\geqslant 2} \frac{\eta(x_p)^\nu}{\nu p^{\nu s}}\right|
		\leqslant \sum_{\nu\geqslant 2} (p^{-\Re s})^\nu 
		= p^{-2 s} \frac{1}{1-p^{-s}} .
\]
Since $p\geqslant 2$ and $\Re s>1/2$, we have $1-2^{-1/2}<1-p^{-s}<1$, so the 
argument of $\exp$ converges if and only if 
$\sum_p \left(\frac{\eta(x_p)}{p^s} + p^{-2\Re s}\right)$ does. But 
$\sum p^{-2\Re s}$ converges absolutely, so we the desired result. 
\end{proof}

\ldots definition of star-discrepancy on $G^\natural$\ldots

\begin{theorem}
If $\disc^\ast(\bx^C) = O(C^{-\frac 1 2+\epsilon})$, then 
$\left|\int f - \bx^C(f)\right| = O_f(C^{-\frac 1 2+\epsilon})$
\end{theorem}

\begin{theorem}
Assume that $\left|\int_{G^\natural} f - \bx^C(f)\right| = O_f(C^{-\frac 1 2+\epsilon})$. 
If $\int \eta = 0$, then $L_\eta$ has analytic continuation to $\{\Re s=1/2\}$, and 
$\log L_\eta$ has no poles in that region. 

\end{theorem}
\begin{proof}
By \autoref{lem:abel} with $\phi(x)=x^{-s}$, we have 
\begin{align*}
	\sum_{p\leqslant C} \frac{\eta(x_p)}{p^s} 
		&= C^{-s} \sum_{p\leqslant C} \eta(x_p) + s \int_2^C \sum_{p\leqslant x} \eta(x_p)\, \frac{\dd x}{x^{s+1}} \\
		&= C^{-s}\Li(C) O(C^{-\frac 1 2+\epsilon}) + s \int_2^C \Li(x) O(x^{-\frac 1 2+\epsilon})\, \frac{\dd x}{x^{s+1}} .
\end{align*}
Since $\Li(x)=O(x/\log x)$, the first term is 
$O(C^{\frac 1 2 -s+\epsilon}/\log C) = o(1)$. We prove that the integral is 
absolutely convergent. Since $\Re s+\frac 1 2>1$ and $\epsilon$ is arbitrary, 
\[
	\int_2^C \frac{x^{\epsilon-\frac 1 2-\Re s}}{\log x}\, \dd x
\]
converges, and the proof is complete. 
\end{proof}





\begin{theorem}
Let $\eta\colon G^\natural\to \bC$ be bounded and have bounded variation, and 
moreover $\int \eta = 0$. Then 
\[
	\sum_p\frac{\eta(x_p)}{p^s}
	\qquad \textnormal{and}\qquad
	\sum_p \frac{\log p}{p^s} \eta(x_p)
\]
are holomorphic on the region $\{\Re s>\frac 1 2\}$. 
\end{theorem}
\begin{proof}
We use Abel summation:
\[
	\sum_{p\leqslant C} \frac{\log p}{p^s} \eta(x_p) 
		= \frac{\log C}{C^s} \sum_{p\leqslant C} \eta(x_p) - \int_2^C \frac{1-s\log x}{x^{s+1}} \sum_{p\leqslant x} \eta(x_p)\, \dd x .
\]
We show the the first term approaches zero and that the integral converges 
absolutely. We have:
\[
	\left|\frac{\log C}{C^s} \sum_{p\leqslant C} \eta(x_p)\right| \ll \frac{\log C}{C^s} \frac{C}{\log C} C^{-\frac 1 2+\epsilon} = C^{1-s-\frac 1 2+\epsilon} .
\]
Since $\epsilon$ is arbitrary, the exponent of $C$ is negative. Moreover:
\begin{align*}
	\int_2^C \frac{1}{x^{s+1}} \left|\sum_{p\leqslant x} \eta(x_p)\right|\, \dd x 
		&\ll \int_2^C \frac{1}{x^{s+1}} \frac{x}{\log x} x^{-\frac 1 2 +\epsilon}\, \dd x \\
	\int_2^C \frac{\log x}{x^{s+1}} \left|\sum_{p\leqslant x} \eta(x_p)\right|\, \dd x 
		&\ll \int_2^C \frac{\log x}{x^{s+1}} \frac{x}{\log x} x^{-\frac 1 2 +\epsilon}\, \dd x .
\end{align*}
Both of these integrals converge because $\epsilon$ is arbitrary. 
\end{proof}

\begin{corollary}
If $\rho\in \widehat G$, then RH + analytic continuation to 
$\{\Re s>\frac 1 2\}$ holds for $L(s,\rho)$. 
\end{corollary}
\begin{proof}
Take logarithmic derivative, reduce to the previous theorem. 
\end{proof}





\section{Discrepancy on compact Lie groups}

Let $G$ be a compact, connected Lie group. Let $G^\natural$ be the space of 
conjugacy classes of $G$. We will define \emph{star-discrepancy} for sequences 
on $G^\natural$. Let $T$ be a maximal torus in $G$. Then the exponential map 
$\exp\colon \ft\to T$ is surjective. Choose a basis $\{t_1,\dots,t_r\}$ for 
$\ft$. Then we can identify $G^\natural$ with 
\[
	\int_{G^\natural} f(x)\, \dd x = \frac{1}{\# W}\int_T \det(1-\Ad(t^{-1})|_{\fg/\ft}) f(t)\, \dd t .
\]

\section{Some examples}

Let $G$ be a compact connected abelian lie group, and $g\in G$ such that 
$g^\bZ\subset G$ is equidistributed. Then for any function $\eta$ on $G$ with 
$\|\eta\|_\infty\leqslant 1$, we have a curious $L$-function:
\[
	L_\eta(g,s) = \prod_p \frac{1}{1-\eta(g^p) p^{-s}} .
\]

[Bla bla general nonsense.]

For example, let $G=\bR/\bZ$ and $\alpha\in \bR$ be an algebraic irrational, 
for example $\alpha=\sqrt 2$. Then the corresponding $L$-function is:
\[
	L_{\exp(2\pi it)}(\alpha,s) = \prod_p \frac{1}{1-\exp(2\pi i \alpha p) p^{-s}}
\]





\section{Special Unitary group}

For $G=\SU(2)$, the space of conjugacy classes is $[0,\pi]$, with $\theta$ 
corresponding to the matrix $\begin{pmatrix} e^{i\theta} \\ & e^{-i\theta} \end{pmatrix}$. Note that the trace of the $n$-th symmetric power is 
$\sin((n+1)\theta)/\sin(\theta)$. 
\[
	f(\pi-x) = f(\pi) + \mu([0,x]) \Rightarrow \mu([0,x]) = f(\pi-x)-f(\pi)
\]
So
\[
	\int_0^x \frac{\dd}{\dd t}(f(\pi-t)) = f(\pi-x)-f(\pi)
\]
Moreover
\[
	\frac{\dd}{\dd t} f(\pi-t) = -f'(\pi-t) .
\]
So the variation is:
\[
	\int_0^\pi \left|\frac{\dd}{\dd \theta} \frac{\sin((n+1)\theta)}{\sin\theta}\right|\, \dd \theta 
\]

$n=1$,4

$n=2$, 8

$n=3$, $(8 (9 + 2 \sqrt 6))/9$

$n=4$, 17

Let $G$ be a compact connected Lie group, $T\subset G$ a maximal torus. Then 
we have the exponential map $\exp\colon \ft\twoheadrightarrow T$ with kernel 
$\Gamma$. 

E.g. $\bR^2/\bZ^2$. Here $\{(1,3),(0,1)\}$ is a basis, so we could get, as 
basic sets:
\[
	\{(\lambda+3\mu,\mu)\mod 1\}
\]

Suppose we have a lattice $\Gamma\subset V$ with basis 
$\gamma_1,\dots,\gamma_r$. Call a ``rectangle'' a set of the form:
\[
	I_{\boldsymbol\alpha} = \{t_1 \gamma_1 + \cdots + t_r \gamma_r : 0\leqslant t_1 < \alpha_1,\dots,0\leqslant t_r < \alpha_r\}
\]
Suppose $\Gamma=\bZ^r\subset \bR^r$. If $r=2$ and we choose a basis as above, 
we are looking at:
\[
	I_{\alpha_1,\alpha_2} = \{(t_1,3t_1+t_2) : t_1\in [0,\alpha_1),t_2\in [0,\alpha_2)\}
\]





\section{Koksma--Hlawka}

Let $[0,\pi]^r$ be our space. The $\mu$-star discrepancy is:
\[
	D^\ast(\bx^N) = \sup_{x\in [0,\pi]^r} \left| \frac{1}{N} \sum_{n\leqslant N} 1_{[0,x]}(x_n) - \int 1_{[0,x]} \, \dd\mu\right|
\]
Let $f$ be a function on $[0,\pi]^r$. We say $f$ has \emph{bounded 
variation} if there is a measure $\mu$ such that 
\[
	\int_{[0,x]} f' = \mu[0,x] = f(x)
\]
Thus, the variation of $f$ is $\int |f'|$. 

For $[0,\pi]^r$, we have 
\[
	\mu([0,\alpha_1]\times [0,\alpha_2]) = \int_0^{\alpha_2} \int_0^{\alpha_1} \frac{\dd}{\dd x_1} \frac{\dd}{\dd x_2} f = f(\alpha_1,\alpha_2) - f(0,0)
\]
Note:
\[
	f(x_1,\dots,x_n) = f(0) + \int_0^{x_n} \dots \int_0^{x_1} g(t_1,\dots,t_n)\,  \dd t_1\dotsm \dd t_n
\]
\begin{align*}
	\int_0^{x_2} \int_0^{x_1} \partial_{1,2} g(t_1,t_2)\,  \dd t_1\dd t_2 
		&= \int_0^{x_2} \partial_2g(x_1,t_2) - \partial_2 g(0,t_2)\, \dd t_2 \\
		&= g(x_1,x_2) - g(x_1,0) - g(0,x_2) + g(0,0)
\end{align*}





\section{Analytic continuation}

Consider our usual $L_\eta(s)$. Recall that $(-L'/L)_\eta(s)$ is governed by 
\[
	\log L_\eta(s) = \sum_{p^\nu} \frac{\eta(x_p)}{\nu p^{\nu s}} .
\]
This we split up as follows:
\begin{align*}
	\log L_\eta(s) = \sum_p \frac{\eta(x_p)}{p^s} + \sum_{\nu\geqslant 2}\frac{1}{\nu} \sum_p \frac{\eta(x_p)}{p^{\nu s}}
\end{align*}
Let $H(s) = \sum_p \eta(x_p) p^{-s}$; then 
\[
	\log L_\eta(s) = H(s) + \sum_{\nu\geqslant 2} \frac{1}{\nu} H(\nu s)
\]

\begin{theorem}
Suppose $|\bx^C(f)-\int f| = O(C^{-\frac 1 2+\epsilon})$. If $\int \eta\ne 0$, 
then $L_\eta$ has a pole of order $\int\eta$ at $s=1$. 
\end{theorem}
\begin{proof}
It suffices to prove that 
$\log L_\eta(1+\epsilon) = -(\int\eta) \log\epsilon+O(1)$. This is a simple 
computation:
\begin{align*}
	\log L_\eta(1+\epsilon) 
		&= \sum_p \frac{\eta(x_p)}{p^{1+\epsilon}} + O(1) \\
		&= \int_2^\infty \frac{x}{\log x} (\mu(\eta)+x^{-\frac 1 2+})\, \frac{\dd x}{x^{2+\epsilon}} + O(1) \\
		&= \mu(\eta) \int_2^\infty \frac{\dd x}{x^{1+\epsilon}\log x} + O(1) \\
		&= -\mu(\eta) \Ei(-\epsilon \log 2) + O(1) \\
		&= -\mu(\eta) \log \epsilon + O(1) .
\end{align*}
\end{proof}

Try analytically continuing 
\[
	(-L'/L)_\eta(s) = \sum_{p^\nu} \frac{\log(p)\eta(x_p)^\nu}{p^{\nu s}}
\]
We follow the easy proof. Start with:
\[
	\pi^{-s} \Gamma(s) \frac{1}{n^{2s}} = \int_0^\infty e^{-\pi n^2 y} y^s \frac{\dd y}{y}
\]
Make a sum:
\begin{align*}
	\sum_{p^\nu} \log(p) \eta(x_p)^\nu \pi^{-s} \Gamma(s) \frac{1}{(p^\nu)^{2s}}
		&= \sum_{p^\nu} \log(p) \eta(x_p)^\nu \int_0^\infty e^{-\pi n^2 y} y^s \frac{\dd y}{y} \\
	\pi^{-s} \Gamma(s) (-L'/L)_\eta(2s) 
		&=\int_0^\infty \sum_{p^\nu} \log(p) \eta(x_p)^\nu e^{-\pi (p^\nu)^2 y} y^s \frac{\dd y}{y}
\end{align*}
So we are interested in the ``series under the integral'':
\[
	\varphi(y) = \sum_{p^\nu} \log(p) \eta(x_p)^\nu e^{-\pi p^{2\nu} y} 
\]
The series for $\varphi$ converges absolutely on $\{\Re>0\}$. Better, 
\[
	\sum_p\log(p)\sum_{\nu\geqslant 1}  M^\nu (e^{-\pi})^{(p^2)^\nu y}
\]

Recall the \emph{Mellin transform}:
\[
	(\cM f)(s) = \int_0^\infty (f(t)-f(\infty))t^s\, \frac{\dd t}{t}
\]
Thus we have the identity:
\[
	\pi^{-s} \Gamma(s) (-L'/L)_\eta(2s) = (\cM \varphi)(s)
\]
First, we need good bounds for $\varphi(y)$, i.e. it needs to be a constant 
plus $O(e^{-c y^\alpha})$, i.e.~very fast decay. Better, write 
\[
	\varphi(y) = \sum_{\nu\geqslant 1}\sum_p  \log(p) \eta(x_p)^\nu \exp(-\pi p^{2\nu}y)
\]
Try Abel summation for some fixed $\nu$:
\begin{align*}
	& \sum_{p\leqslant C}\log(p) \eta(x_p)^\nu \exp(-\pi p^{2\nu}y) \\
		&= \log(C)\exp(-\pi C^{2\nu} y) \sum_{p\leqslant C} \eta^\nu(x_p) + \int_2^C (1-2\pi \nu y x^{2\nu}\log x) \exp(-\pi y x^{2\nu})\,(*) \frac{\dd x}{x} \\
		&= C \exp(-\pi C^{2\nu} y) (\mu(\eta^\nu)+O(C^{-\frac 1 2+\epsilon})) + \int_2^\infty \ldots \\
		&\approx \int_2^\infty (1-2\pi \nu y x^{2\nu}\log x) \exp(-\pi y x^{2\nu})\sum_{p\leqslant x} \eta^\nu(x_p)\, \frac{\dd x}{x} \\
		&\ll \int_2^\infty \nu y x^{2\nu}\log(x) \exp(-\pi y x^{2\nu}) \frac{x}{\log x}(\mu(\eta^\nu) + x^{-\frac 1 2+\epsilon})\, \frac{\dd x}{x} \\
		&\ll \int_2^\infty \nu y x^{2\nu} \exp(-\pi y x^{2\nu})(\mu(\eta^\nu)+x^{-\frac 1 2+\epsilon})\, \dd x \\
		&= \nu y \mu(\eta^\nu) \int_2^\infty x^{2\nu} (e^{-\pi y})^{ x^{2\nu}}\, \dd x + 
\end{align*}





\section{Different perspective}

Recall that our curious $L$-function can be written as a Dirichlet series:
\[
	L_\eta(s) = \sum_{n\geqslant 1} \frac{\prod_{p^\nu \| n} \eta(x_p)^\nu}{n^s}
\]
We can write $\eta(n) = \prod_{p^\nu\| n} \eta(x_p)^\nu$; then $\eta$ is a 
completely multiplicative function $\bN\to \bC$. Then 
$L_\eta(s) = \sum \eta(n) n^{-s} = \prod (1-\eta(p) p^{-s})^{-1}$. 

Again, we have 
\[
	\pi^{-s} \Gamma(s) \frac{1}{n^{2s}} = \int_0^\infty e^{-\pi n^2 y} y^{s} \frac{\dd y}{y} .
\]
Sum over $n$ and we get:
\[
	\pi^{-s} \Gamma(s) \sum_{n\geqslant 1} \frac{\eta(n)}{n^{2s}} =  \int_0^\infty\sum_{n\geqslant 1} \eta(n) e^{-\pi n^2 y} y^{s} \frac{\dd y}{y} .
\]
Thus $\pi^{-s} \Gamma(s) L_\eta(2s) = (\cM\varphi)(s)$, for 
\[
	\varphi(y) = \sum_{n\geqslant 1} \eta(n) e^{-\pi n^2 y}
\]
We are now interested in:
\[
	\varphi(1/y) = \sum_{n\geqslant 1} \eta(n) e^{-\pi n^2 / y}
\]
The main question is: can we approximate $\sum_{n\leqslant N} \eta(n)$ in terms 
of stuff we know? Start with something easier:
\begin{align*}
	\sum_{p^\nu\leqslant N} \eta(x_p)^\nu
		&= \sum_{\nu\leqslant \log_2 N} \sum_{p\leqslant N^{1/\nu}} \eta(x_p)^\nu \\
		&= \sum_{\nu\leqslant \log_2 N} \pi(N^{1/\nu}) \bx^{N^{1/\nu}}(\eta^\nu) \\
		&= \sum_{\nu\leqslant \log_2 N} \frac{\nu N^{1/\nu}}{\log N} (\mu(\eta^\nu)+O(N^{1/\nu})^{-\frac 1 2+\epsilon}) \\
		&= \sum_{\nu\leqslant \log_2 N} \frac{\nu N^{1/\nu}}{\log N} \mu(\eta^\nu) + O \sum_{\nu\geqslant 1} \frac{\nu N^{\frac{1+2\epsilon}{2\nu}}}{\log N} \\
		&=
\end{align*}





\section{Following Bucar--Kedlaya}

Can we sum:
\begin{align*}
	\sum_{p\leqslant C} \eta(x_p) \log p 
		&= \log C \sum_{p\leqslant C} \eta(x_p) - \int_2^C \frac{1}{x} \sum_{p\leqslant x} \eta(x_p) \, \dd x \\
		&= C (\mu(\eta)+O(C^{-\frac 1 2+\epsilon})) + O \int_2^\infty x^{-\frac 3 2+\epsilon}\, \dd x \\
		&= C \mu(\eta)+ O(C^{\frac 1 2+\epsilon})
\end{align*}





\printbibliography
\end{document}
