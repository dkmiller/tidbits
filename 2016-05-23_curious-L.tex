\documentclass{article}

\usepackage[a5paper,margin=1.5cm]{geometry}
\usepackage{
	amsmath,
	amssymb,
	amsthm,
	hyperref,
	microtype
}
\usepackage[
  hyperref = true,
  backend  = bibtex,
  sorting  = nyt,
  style    = alphabetic
]{biblatex}
\addbibresource{tidbit-sources.bib}
\hypersetup{colorlinks=true,linkcolor=green}

\DeclareMathOperator{\GL}{GL}
\DeclareMathOperator{\sgn}{sgn}
\DeclareMathOperator{\ST}{ST}
\DeclareMathOperator{\tr}{tr}
\newcommand{\bQ}{\mathbf{Q}}
\newcommand{\bR}{\mathbf{R}}
\newcommand{\bZ}{\mathbf{Z}}
\newcommand{\dd}{\mathrm{d}}
\newcommand{\ddiv}{\mathrm{div}}
\newcommand{\fr}{\mathrm{fr}}
\newtheorem{theorem}{Theorem}
\newtheorem{lemma}[theorem]{Lemma}
\theoremstyle{definition}
\newtheorem{definition}[theorem]{Definition}
\numberwithin{theorem}{section}

\title{Analytic and arithmetic properties of a new class of \texorpdfstring{$L$}{L}-functions}
\author{Daniel Miller}

\begin{document}
\maketitle





\section{Introduction}

The work in this paper is inspired by the following example of Ramakrishna. Let 
$E_{/\bQ}$ be a non-CM elliptic curve. Let $l$ be an odd prime and 
$\rho_{E,l}\colon G_\bQ \to \GL_2(\bZ_l)$ the associated representation. Recall 
that $a_p = \tr(\rho_{E,l}(\fr_p))$, and these satisfy the Hasse bound 
$|a_p|<2\sqrt p$. Then we have the following curious $L$-function with only one 
Euler factor at each prime: 
\[
	L_{\sgn}(E,s) = \prod_p \frac{1}{1-\sgn(a_p) p^{-s}} .
\]
We are interested in the analytic and arithmetic properties of a class of 
$L$-functions generalized from this one. 

\begin{definition}
Let $\rho\colon G_\bQ \to \GL_n(\bZ_l)$ be geometric in the sense of 
\cite{fontain-mazur-1995}. Assume the Sato--Tate group of $\rho$ is 
well-defined; denote it by $\ST(\rho)$. Let 
$\eta\colon \ST(\rho)^\natural \to \bR$ be a function of bounded variation.  
\end{definition}





\begin{lemma}
Let $X$ be a compact Riemannian manifold with associated probability measure 
$\dd x$, and $f\in L^1(X, \dd x)$. Then the following two conditions are 
equivalent:
\begin{enumerate}
\item
For every equidistributed sequence $\{x_p\}$ on $X$, we have 
\[
	\lim_{X\to \infty} \frac{1}{\pi(X)} \sum_{p\leqslant X} f(x_p) = \int f(x)\, \dd x .
\]
\item
There exists a finite Radon measure $D f$ on $X$ such that for all 
$\phi\in C^\infty(X)$, we have 
\[
	\int f\ddiv \phi\, \dd x = -\int \phi\, \dd Df .
\]
\end{enumerate}
\end{lemma}
\begin{proof}
We can write (if $f\in L^2$ and $X=G^\natural$), 
\[
	f = \sum_{\rho\in \widehat G} a_\rho \tr \rho .
\]
This series converges absolutely always. So we have:
\begin{align*}
	\lim_{X\to \infty} \frac{1}{\pi(X)} \sum_{p\leqslant X} \sum_\rho a_\rho \tr \rho(x_p)
		&= \sum_\rho a_\rho \lim_{X\to \infty} \frac{1}{\pi(X)}\sum_{p\leqslant X} \tr \rho(x_p)
\end{align*}
\end{proof}





\begin{lemma}[Abel summation]
Let $\{x_p\}$ be a sequence of real numbers, $\phi\in C^1(\bR)$. Then 
\[
	\sum_{p\leqslant X} \phi(p) x_p = \phi(X) \sum_{p\leqslant X} x_p - \int_2^X \phi'(x) \sum_{p\leqslant x} x_p\, \dd x .
\]
\end{lemma}
\begin{proof}
Simply note that if $p_1,\dots,p_n$ is an enumeration of the primes 
$\leqslant X$, we have 
\begin{align*}
	\int_2^X \phi'(x) \sum_{p\leqslant x} x_p\, \dd x 
		&= \sum_{p\leqslant X} x_p\int_{p_n}^X \phi' + \sum_{i=1}^{n-1} \sum_{p\leqslant p_{i+1}}x_p\int_{p_i}^{p_{i+1}} \phi' \\
		&= (\phi(X)-\phi(p_n))\sum_{p\leqslant X} x_p + \sum_{i=1}^{n-1} (\phi(p_{i+1})-\phi(p_i)) \sum_{p\leqslant p_{i+1}} x_p \\
		&= \phi(X) \sum_{p\leqslant X} x_p - \sum_{p\leqslant X} \phi(p) x_p ,
\end{align*}
as desired.
\end{proof}





\begin{lemma}
Let $(X,\mu)$ be a compact topological space with Radon measure. Let $f$ be a 
bounded function on $X$. Then the following condition holds:
\begin{quote}
For all $\mu$-equidistributed sequences $\{x_p\}$ on $X$, 
\[
	\lim_{X\to \infty} \frac{1}{\pi(X)} \sum_{p\leqslant X} f(x_p) = \int_X f\, \dd \mu .
\]
\end{quote}
if and only if $f$ is continuous almost everywhere.
\end{lemma}
\begin{proof}
Suppose $f$ is a.e.~continuous. Let $\epsilon>0$. Then there exists open 
$U\subset X$ such that $f$ is continuous away from $U$ and $\mu(U)<\epsilon$. 
We now note that 
\[
	\lim_{X\to \infty} \frac{1}{\pi(X)} \sum_{p\leqslant X} f(x_p)
		= \frac{1}{\pi(X)} \sum_{\substack{p\leqslant X \\ x_p\in U}} f(x_p) + \frac{1}{\pi(X)} \sum_{\substack{p\leqslant X \\ x_p\notin U}} f(x_p) .
\]
The first term is bounded above in absolute value by $\epsilon  \|f\|_\infty$, 
and the second term converges to $\int_{X\smallsetminus U} f\, \dd\mu$. Since 
\[
	\left|\int f\, \dd\mu - \int_{X\smallsetminus U} f\, \dd\mu\right| \leqslant \mu(U)\|f\|_\infty ,
\]
we get that 
\[
	\left|\lim_{X\to \infty} \frac{1}{\pi(X)} \sum_{p\leqslant X} f(x_p) - \int_X f\, \dd \mu\right| < 2\epsilon \|f\|_\infty ,
\]
which proves the desired result.

Conversely, suppose $f$ satisfies the condition. Let the \emph{oscillation} of 
$f$ be 
\[
	\omega_f(x) = \inf_{U\ni x} \sup_{u,v\in U} |f(u)-f(v)| ,
\] 
where the infimum ranges over all open $U$ containing $x$. Clearly $f$ is 
continuous at $x$ if and only if $\omega_f(x)=0$. 

We need to show that each set $\{\omega_f > \epsilon\}$ has measure zero, since 
$\{\omega_f>0\}$ is a countable union of such sets with $\epsilon>0$. 
\end{proof}





\printbibliography
\end{document}
