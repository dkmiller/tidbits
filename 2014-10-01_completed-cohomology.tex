\documentclass{article}

\usepackage{amsmath,amssymb,amsthm,bookmark,fullpage,mathrsfs,microtype}
\DeclareMathOperator{\chars}{X}
\DeclareMathOperator{\End}{End}
\DeclareMathOperator{\GL}{GL}
  \newcommand{\gl}{\mathfrak{gl}}
\DeclareMathOperator{\h}{H}
\DeclareMathOperator{\induce}{ind}
\DeclareMathOperator{\lie}{Lie}
\DeclareMathOperator{\SL}{SL}
\DeclareMathOperator{\symmetric}{sym}
\DeclareMathOperator{\witt}{W}
\newcommand{\dC}{\mathbf{C}}
\newcommand{\dH}{\mathbf{H}}
\newcommand{\dN}{\mathbf{N}}
\newcommand{\dQ}{\mathbf{Q}}
\newcommand{\dR}{\mathbf{R}}
\newcommand{\dZ}{\mathbf{Z}}
\newcommand{\eR}{\mathrm{R}}
\newcommand{\fH}{\mathfrak{H}}
\newcommand{\fm}{\mathfrak{m}}
\newcommand{\fp}{\mathfrak{p}}
\newcommand{\ft}{\mathfrak{t}}
\newcommand{\fX}{\mathfrak{X}}
\newcommand{\sV}{\mathscr{V}}
\newcommand{\adele}{\mathbf{A}}
\newcommand{\etale}{\textnormal{\'et}}
\newcommand{\finite}{\mathrm{f}}
\newcommand{\mat}[4]{\begin{pmatrix} #1 & #2 \\ #3 & #4 \end{pmatrix}}
\newcommand{\ordinary}{\mathrm{ord}}
\newtheorem{theorem}[subsection]{Theorem}

\usepackage[
  hyperref = true,      % links to online documents
  backend  = bibtex,    % use bibtex instead of biber
  sorting  = nyt,       % sorts by (name, year, title)
  style    = alphabetic % citations look like [Har77]
]{biblatex}
\addbibresource{tidbit-sources.bib}
\hypersetup{
  colorlinks = true,
  linkcolor  = blue,
  urlcolor   = cyan
}

\title{Completed cohomology, deformation theory, and a torsion local Langlands correspondence}
\author{Daniel Miller}

\begin{document}
\maketitle





\section{Definitions}

Consider the group $\GL(2)_{/\dQ}$. For each open compact 
$K\subset \GL_2(\adele_\finite)$, let $X_K$ be the variety over $\dQ$ 
underlying the (compactification of the quotient 
\[
  Y_K(\dC) = \GL_2(\dQ) \backslash\left(\dH^\pm \times \GL_2(\adele_\finite)\right)/ K .
\]
The projective system $\{X_K\}$ admits an action of $\GL_2(\adele_\finite)$. 
Moreover, if $\rho$ is a representation of $\GL(2)_{/\dQ}$, there is a 
canonical sheaf, also denoted $\rho$, on the projective system $\{X_K\}$. 

Let $k$ be a finite field, $\witt(k)$ its ring of Witt vectors. For any 
Artinian $\witt(k)$-algebra $A$, put 
\[
  \h^\bullet(\rho)_A = \varinjlim_{K\subset \GL_2(\adele_\finite)} \h^1_\etale\left((X_K)_{\overline\dQ},\rho_A\right) .
\]
This is an $A[\Gamma_\dQ\times \GL_2(\adele_\finite)]$-module. 

If $w\geqslant 0$ is an integer, we put 
$\h^\bullet(w)_A=\h^\bullet(\symmetric^{w-2})_A$. If $A=\varprojlim A_i$ is a 
pro-artinian $\witt(k)$-module, put 
$\h^\bullet(\rho)_A=\varprojlim \h^\bullet(\rho)_{A_i}$. 





\section{Some deformation theory}

For a residual representation $\bar\rho:\Gamma_\dQ\to \GL_2(k)$, we write 
$\fX=\fX(\bar\rho)$ for the deformation functor classifying lifts 
$\Gamma_{\dQ,S}\to \GL_2(k)$, for some unspecified $S$. To be precise, we are 
considering $\fX(\bar\rho)$ as an ind-(formal scheme). Assume $\bar\rho$ is 
odd and absolutely irreducible; then $\bar\rho$ is modular. By 
\cite[1.2.6]{emerton-2011}, there is a natural isomorphism 
\[
  \bar\pi(\bar\rho) \simeq \hom_{\Gamma_\dQ}(\bar\rho,\h^1_k)
\]
of $\GL_2(\adele_\finite)$-modules, assuming some technical conditions on 
$\bar\rho$. In particular, the hom-set is non-zero. 

We define a functor  
$\fH(\bar\rho)$ on local artinian $\witt(k)$-algebras with residue field $k$. 
For such an algebra $A$, we let $\fH(\bar\rho)(A)$ be the set of pairs 
$(\rho,f)$, where $\rho\in \fX(\bar\rho)(A)$ and $f:\rho\to \h^1_A$ is 
$A[\Gamma_\dQ]$-linear and reduces to some specified 
$\bar f:\bar\rho\hookrightarrow \h_k^1$. 





\section{Ordinary parts}

We work out \cite{emerton-2010-i,emerton-2010-ii} for the group $\GL_2(\dQ_p)$. 
Let $k$ be a finite field, $\witt(k)$ its ring of Witt vectors, and $A$ an 
artinian local $\witt(k)$-algebra. Let $M$ be a locally profinite abelian 
group, $M^+\subset M$ an open sub-semigroup. Let $\pi$ be a finitely generated 
$A$-module with smooth $M^+$-action. Put 
\[
  \pi^\ordinary = \hom_{M^+}(A[M],\pi) .
\]

\begin{theorem}
The natural map $\pi^\ordinary\to \pi$ given by evaluation at $1$ induces an 
isomorphism between $\pi^\ordinary$ and the maximal $A[M^+]$-submodule of 
$\pi$ on which $M^+$ acts invertibly. 
\end{theorem}
\begin{proof}
This is essentially the proof of \cite[3.1.5]{emerton-2010-i}. 
Let $B$ be the image of $A[M^+]$ in $\End_A(\pi)$. Then there is a 
$B=\prod B_\fp$, where each $B_\fp$ is local Artinian. This induces a 
decomposition $\pi=\prod \pi_\fp$. Call $\fp$ \emph{ordinary} if 
$M^+$ acts invertibly on $\pi_\fp$, and \emph{non-ordinary} otherwise. We claim 
that if $\fp$ is ordinary, then $(\pi_\fp)^\ordinary=\pi_\fp$, and that 
$(\pi_\fp)^\ordinary=0$ otherwise. The first claim is obvious: if 
$M^+$ acts invertibly on $\pi_\fp$, then its action extends uniquely to one of 
$M$. If some $m^+\in M^+$ does not act invertibly on $\pi_\fp$, it acts 
nilpotently, and we may as well assume that $m^+ \pi_\fp=0$. But then for 
$\phi:A[M]\to \pi$, we have 
\[
  \phi(m) = m^+\cdot \phi\left((m^+)^{-1} m\right) = 0,
\]
so $\phi=0$. 
\end{proof}

Now let $M\subset \GL_2(\dQ_p)$ be the subgroup 
$\mat{\ast}{}{}{\ast}$ of diagonal matrices, and $M^+$ be the sub-semigroup 
consisting of those $\mat{a}{}{}{b}$ with $|a|\geqslant |b|$. Note that if we 
put $K=\GL_2(\dZ_p)\subset \GL_2(\dQ_p)$, then by 
\cite[4.6.2]{bump-1997}, the natural map $M^+\to K\backslash G/K$ is 
surjective. In particular, if $\pi$ is a spherical representation of 
$\GL_2(\dQ_p)$, it should be determined by its restriction to $M^+$. In 
fact, it should be determined by the action of $\mat{1}{}{}{p}$. If 
$\pi$ is a smooth $G$-representation, we define 
\[
  \pi^\ordinary=\hom_{M^+}\left(A[M],\pi^{N_0}\right)_{M\mathrm{-finite}} ,
\]
where $N_0=\mat{1}{}{\dZ_p}{1}$. In the 
notation of \cite{emerton-2010-i}, the functor $(-)^\ordinary$ is 
$\mathrm{Ord}_{\overline B}$. By \cite[4.4.6]{emerton-2010-i}, if 
$\pi$ is a smooth $G$-representation and $\rho$ is a smooth $B$-representation, 
then there is a natural isomorphism 
\[
  \hom_G\left(\induce_B^G\rho,\pi\right) = \hom_M\left(\rho,\pi^\ordinary\right) .
\]
Consider $\eR^1 \pi^\ordinary$





\section{Representations of \texorpdfstring{$\GL(2)_{/\dQ}$}{GL(2)/Q}}

Consider the split reductive group $\GL(2)_{/\dQ}$. It has maximal torus 
\[
  T = \mat{\ast}{}{}{\ast} \subset \GL(2) .
\]
We identify $\chars^\ast(T)$ with $\dZ^2=\langle \chi_1,\chi_2\rangle$ via 
$\chi_i(g) = g_{ii}$. We have 
$\gl(2)=\ft\oplus \gl(2)_{\chi_1-\chi_2}\oplus \gl(2)_{\chi_2-\chi_1}$. In 
particular, if we put $\alpha=\chi_1-\chi_2$, we have 
$R=\{\pm \alpha\}$. We identify $\chars_\ast(T)$ with $\chars^\ast(T)$ in the 
obvious way, e.g.~$\chi_1(g)=\mat{g}{}{}{1}$. Under this identification, 
$(\pm \alpha)^\vee=\pm \alpha$, and the group $W\simeq S_2$ is generated by 
$(\chi_1,\chi_2)\mapsto (\chi_2,\chi_1)$. 

The root lattice $Q=\dZ\cdot R=\dZ\alpha$ consists of all $n\chi_1-n\chi_2$. 
Similarly, $X_0=\{n\chi_1+n\chi_2\}$. The weight lattice is 
$P=\dZ\lambda=\{\frac n 2 (\chi_1-\chi_2)\}$, where $\lambda=\frac 1 2 \alpha$. 
Thus $P^+=\dZ_{\geqslant 0} \lambda$. The space of dominant weights is 
$X^+=2\dN\cdot \lambda+X_0=\{m\chi_1+n\chi_2:m\geqslant n\}$. 

The standard representation $\symmetric^1$ of $\GL(2)$ has highest weight 
$\chi_1$. Similarly, $\symmetric^k$ has highest weight $k \chi_1$. So 
$\symmetric^k\otimes \det^d$ has highest weight 
$k\chi_1+d(\chi_1+\chi_2)=(d+k)\chi_1+\chi_2$. To sum it up, we have the 
following: 

\begin{theorem}
Up to isomorphism, every irreducible representation of $\GL(2)$ is of the 
form $\symmetric^k\otimes \det^d$ for $k\geqslant 0$, $d\in \dZ$. Such a 
representation has highest weight $(d+k,d)$. 
\end{theorem}





\section{Locally symmetric spaces}

We will continue to work with the group $\GL(2)_{/\dQ}$. Let $A=Z(G)$ be 
the maximal split central torus in $G$. Let 
$M=\cap_{\chi\in \chars^\ast(G)} \ker(\chi)=\SL(2)$, $\fm=\lie(M)$. Then 
$\lie(A)=\ft$ and $\gl(2)=\fm\oplus \ft$. For 
$K\subset \GL_2(\adele_\finite)$, put 
\[
  Y_K = \GL_2(\dQ) A(\dR)^\circ \backslash \GL_2(\adele) / K_\infty K .
\]
Since $A(\dR)^\circ\backslash \GL_2(\dR)/K_\infty=\dH^\pm$, this can be 
rewritten as 
\[
  Y_K=\GL_2(\dQ)\backslash \left(\dH^\pm\times \GL_2(\adele_\finite)\right)/K .
\]
The space of connected components of $Y_K$ is naturally isomorphic to  
$\widehat\dZ^\times/\det(K)$. It is known that $Y_K$ is a moduli space of 
elliptic curves with level-$K$ structure. As such, it has the canonical 
structure of a curve over $\dQ$. 





\printbibliography

\end{document}
