\documentclass{article}

\usepackage{amssymb,amsmath,fullpage,mathpazo,mathtools}
\usepackage[hidelinks]{hyperref}
\DeclareMathOperator{\aut}{Aut}
\DeclareMathOperator{\lie}{Lie}
\DeclareMathOperator{\out}{Out}
\newcommand{\conj}[2]{\prescript{#1}{}{#2}}
\newcommand{\fg}{\mathfrak{g}}

\title{Galois representations from nilpotent completions}
\author{Daniel Miller}

\begin{document}
\maketitle





Introductory sources for some of the following material are 
\cite{kkl98} and \cite{ih91}. 





\section{Generalities on nilpotent completion}

Let $\Gamma$ be a profinite group. There is a natural action of $\Gamma$ on 
itself by inner automorphisms: $\conj x y = x y x^{-1}$. The image of the 
induced homomorphism $\Gamma \to \aut\Gamma$ is a normal subgroup because 
\[
  \phi \conj{x}{(-)} \phi^{-1}(y) 
    = \phi\left( x \phi(y) x^{-1}\right) 
    = \phi(x) y \phi(x)^{-1} 
    = \conj{\phi(x)} y .
\]
In other words, the homomorphism $\Gamma \to \aut\Gamma$ is 
$\aut\Gamma$-equivariant. Write $\out\Gamma$ for the quotient of $\aut\Gamma$ 
by the subgroup of inner automorphisms. 

Write $[-,-]$ for the standard bracket operation: $[x,y] = x y x^{-1} y^{-1}$. 
We will frequently use the following identities relating $[-,-]$ and inner 
automorphisms (see \cite[10.2]{ha76}): 
\begin{align*}
  [x,y]^{-1} &= [y,x] \\
  [x,[y,z]] &= \conj{x}{[y,z]}\cdot [z,y] \\ \tag{1}\label{g}
  \conj x{[y,z]} &=  [x,[y,z]]\cdot [y,z] .
\end{align*}
We define the \emph{lower central series} of $\Gamma$ by the rule 
\begin{align*}
  \Gamma_1 &= \Gamma \\
  \Gamma_{n+1} &= [\Gamma,\Gamma_n] .
\end{align*}
Here $[\Gamma,\Gamma_n]$ is the closed subgroup of $\Gamma_n$ generated by 
all $[x,y]$ with $x\in \Gamma$, $y\in \Gamma_n$. The identity 
$\conj{x}{[y,z]} = [x,[y,z]]\cdot [y,z]$ implies that $\Gamma_{n+1}$ is normal 
in $\Gamma_n$, and that inner automorphisms act trivially on the quotient 
$\Gamma_n / \Gamma_{n+1}$. Thus, each quotient $\Gamma_n/\Gamma_{n+1}$ is 
abelian, so it we can define  
\[
  \lie\Gamma = \bigoplus_{n\geqslant 1} (\lie \Gamma)_n = \bigoplus_{n\geqslant 1} \Gamma_n / \Gamma_{n+1} .
\]

I claim that the bracket on $\Gamma$ induces the structure of a graded Lie 
algebra on $\fg = \lie\Gamma$. Tautologically, $[\fg_1,\fg_1]\subset \fg_2$. 
\begin{align*}
  [x_n, [y_m, z_m]] = \conj{x_n}{.}
\end{align*}





\bibliographystyle{alpha}
\bibliography{tidbit-sources}

\end{document}
