\documentclass{article}

\usepackage[a5paper,margin=1.5cm]{geometry}
\usepackage{amsmath,amssymb}
\DeclareMathOperator{\cdf}{cdf}
\DeclareMathOperator{\D}{D}
\DeclareMathOperator{\GL}{GL}
\DeclareMathOperator{\ram}{ram}
\DeclareMathOperator{\SU}{SU}
\DeclareMathOperator{\sym}{sym}
\DeclareMathOperator{\tr}{tr}
\newcommand{\bC}{\mathbf{C}}
\newcommand{\bF}{\mathbf{F}}
\newcommand{\bQ}{\mathbf{Q}}
\newcommand{\bR}{\mathbf{R}}
\newcommand{\bZ}{\mathbf{Z}}
\newcommand{\CM}{\mathrm{CM}}
\newcommand{\dd}{\mathrm{d}}
\newcommand{\frob}{\mathrm{fr}}
\newcommand{\nonCM}{\textnormal{non-CM}}
\newcommand{\ST}{\mathrm{ST}}

\title{Counterexamples related to the Sato--Tate conjecture for CM abelian 
varieties\thanks{Notes for a talk given in Cornell's Number Theory 
Seminar.}}
\author{Daniel Miller}
\date{31 March 2017}

\begin{document}
\maketitle





\section{Introduction and motivation}


%Abstract: The Akiyama-Tanigawa conjecture sets a bound on the rate of convergence of the Stake parameters of an elliptic curve to the Sato-Tate measure. Their conjecture implies the Riemann Hypothesis for all L-functions associated with the elliptic curve. I construct a range of examples, using Diophantine approximation, which show that GRH does not imply the Akiyama-Tanigawa conjecture for CM abelian varieties.





\end{document}
