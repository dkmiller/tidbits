\documentclass{article}

\usepackage[a5paper,margin=1.5cm]{geometry}
\usepackage{amsmath,amssymb}
\DeclareMathOperator{\cdf}{cdf}
\DeclareMathOperator{\D}{D}
\DeclareMathOperator{\End}{End}
\DeclareMathOperator{\GL}{GL}
\DeclareMathOperator{\im}{im}
\DeclareMathOperator{\Lie}{Lie}
\DeclareMathOperator{\N}{N}
\DeclareMathOperator{\R}{R}
\DeclareMathOperator{\ST}{ST}
\DeclareMathOperator{\SU}{SU}
\DeclareMathOperator{\sym}{sym}
\DeclareMathOperator{\tr}{tr}
\DeclareMathOperator{\Var}{Var}
\DeclareMathOperator{\X}{X}
\newcommand{\bC}{\mathbf{C}}
\newcommand{\bF}{\mathbf{F}}
\newcommand{\bQ}{\mathbf{Q}}
\newcommand{\bR}{\mathbf{R}}
\newcommand{\bZ}{\mathbf{Z}}
\newcommand{\dd}{\mathrm{d}}
\newcommand{\fa}{\mathfrak{a}}
\newcommand{\fp}{\mathfrak{p}}
\newcommand{\frob}{\mathrm{fr}}
\newcommand{\Gm}{\mathbf{G}_\mathrm{m}}

\title{Counterexamples related to the Sato--Tate conjecture for CM abelian 
varieties\thanks{Notes for a talk given in Cornell's Number Theory 
Seminar.}}
\author{Daniel Miller}
\date{31 March 2017}

\begin{document}
\maketitle





\section{Introduction and motivation}

%Abstract: The Akiyama-Tanigawa conjecture sets a bound on the rate of convergence of the Stake parameters of an elliptic curve to the Sato-Tate measure. Their conjecture implies the Riemann Hypothesis for all L-functions associated with the elliptic curve. I construct a range of examples, using Diophantine approximation, which show that GRH does not imply the Akiyama-Tanigawa conjecture for CM abelian varieties.

Let $K/\bQ$ be a finite Galois extension, $A_{/K}$ a $g$-dimensional abelian 
variety. Assume that $A$ has CM defined over $K$; then there is a CM field $F$ 
together with an isomorphism $F\simeq \End_K(A)_\bQ$. Let $\fa =\Lie(A)$. Then 
the determinant of the action of $K$ on $\fa$ (viewed as an $F$-vector space) 
gives a map $\det_\fa\colon \R_{K/\bQ}\Gm \to \R_{F/\bQ}\Gm$. Put 
$G_A = \im(\det_\fa)$ and $G_A^1 = \im(\det_\fa)^{\N_{F/\bQ} = 1}$. The group 
$G_A$ is the motivic Galois group of $A$, and $\ST(A)$, the Sato--Tate group of 
$A$, is a maximal compact subgroup of $G_A^1(\bC)$. It is obvious that the 
$l$-adic Galois representation associated with $A$, 
$\rho_l\colon G_\bQ \to \GL_{2g}(\bQ_l)$, has image in 
$(\R_{F/\bQ}\Gm)(\bQ_l)$; it actually is a map 
$\rho_l\colon G_\bQ \to G_A(\bQ_l)$. 

Unitary representations of $\ST(A)$ are just characters, and basic 
representation theory tells us that all such representations are induced by an 
(algebraic) character of $G_A$ defined over $\overline\bQ$. For 
$r\in \X^\ast(G_A)$, there is an $L$-function $L(r_\ast \rho_l,s)$ coming from 
the composite Galois representation 
$r\circ \rho_l\colon G_\bQ \to G_A(\bQ_l) \to \overline{\bQ_l}^\times$. The 
Sato--Tate conjecture for $A$ says that all $L(r_\ast \rho_l,s)$ have 
non-vanishing analytic continuation past $\Re = 1$, and the Generalized Riemann 
hypothesis for $A$ says that all $L(r_\ast \rho_l,s)$ satisfy the Riemann 
hypothesis. 

Choose an isomorphism $(\bR/\bZ)^d \simeq \ST(A)$, and put 
\[
	\D_x(A) = \sup_{t\in [0,1]^d} \left| \frac{1}{\pi_K(x)} \sum_{\N(\fp) \leqslant x} 1_{[0,t)}(\theta_\fp) - \int 1_{[0,t)} \right| .
\]
Akiyama and Tanigawa conjectured that for non-CM elliptic curves, 
$\D_x(E) \ll x^{-\frac 1 2+\epsilon}$. We call the ``Akiyama--Tanigawa 
conjecture'' for $A$ the discrepancy decays like 
$\D_x(A) \ll x^{-\frac 1 2+\epsilon}$. Via the Koksma--Hlawka inequality, the 
Akiyama--Tanigawa conjecture implies that for all bounded-variation functions 
$f$ on $\ST(A)$, the estimate 
\[
	\left| \sum_{\N(\fp) \leqslant x} f(\theta_\fp)\right| \ll \Var(f)x^{\frac 1 2+\epsilon} .
\]
For $r\in \X^\ast(G_A)$, this estimate implies the Riemann hypothesis for the 
$L$-function $L(r_\ast \rho_l,s)$. 

Analogy with Artin $L$-functions (go into detail here!) seems to suggest that 
if all $L(r_\ast \rho_l,s)$ satisfy the Riemann Hypothesis, then the 
Akiyama--Tanigawa conjecture for $A$ holds. We'll show: this converse is false, 
in a limited sense. 





\section{Diophantine approximation}

Let $x\in [0,1]$ be irrational. It is well known that the sequence 
$(x\mod 1,2 x\mod 1,3 x\mod 1,\dots)$ is equidistributed in $[0,1]$. What is 
less well known is that the rate of convergence of empirical measures from this 
sequence to the uniform measure is governed by the irrationality measure of 
$x$. The irrationality measure $\mu(x)$ is the supremum of the set of 
$w\geqslant 1$ such that there are infinitely many $p/q$ with 
$\left| \mu - \frac p q\right| \leqslant q^{-w}$. Let's generalize this to 
higher-dimensional space. 





\section{Fake Satake parameters}

d





\end{document}
