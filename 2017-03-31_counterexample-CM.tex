\documentclass{article}

\usepackage[a5paper,margin=1.5cm]{geometry}
\usepackage{amsmath,amssymb}
\DeclareMathOperator{\cdf}{cdf}
\DeclareMathOperator{\D}{D}
\DeclareMathOperator{\End}{End}
\DeclareMathOperator{\Gal}{Gal}
\DeclareMathOperator{\GL}{GL}
\DeclareMathOperator{\GSp}{GSp}
\DeclareMathOperator{\im}{im}
\DeclareMathOperator{\Lie}{Lie}
\DeclareMathOperator{\N}{N}
\DeclareMathOperator{\R}{R}
\DeclareMathOperator{\ST}{ST}
\DeclareMathOperator{\SU}{SU}
\DeclareMathOperator{\sym}{sym}
\DeclareMathOperator{\tr}{tr}
\DeclareMathOperator{\Var}{Var}
\DeclareMathOperator{\X}{X}
\newcommand{\bA}{\mathbf{A}}
\newcommand{\bC}{\mathbf{C}}
\newcommand{\bF}{\mathbf{F}}
\newcommand{\bQ}{\mathbf{Q}}
\newcommand{\bR}{\mathbf{R}}
\newcommand{\bZ}{\mathbf{Z}}
\newcommand{\dd}{\mathrm{d}}
\newcommand{\fa}{\mathfrak{a}}
\newcommand{\fp}{\mathfrak{p}}
\newcommand{\frob}{\mathrm{fr}}
\newcommand{\Gm}{\mathbf{G}_\mathrm{m}}

\title{Counterexamples related to the Sato--Tate conjecture for CM abelian 
varieties\thanks{Notes for a talk given in Cornell's Number Theory 
Seminar.}}
\author{Daniel Miller}
\date{31 March 2017}

\begin{document}
\maketitle





\section{Introduction and motivation}

Let $K/\bQ$ be a finite Galois extension, $A_{/K}$ a $g$-dimensional abelian 
variety. Fix a rational prime $l$; then the $l$-adic Tate module of $A$ gives 
a representation 
$\rho_l\colon G_\bQ = \Gal(\overline\bQ/\bQ) \to \GL_{2g}(\bQ_l)$. The image 
actually lies in the subgroup $\GSp_{2g}(\bQ_l)$ preserving the Weil pairing, 
but we won't worry about that. If we write $F = \End_K(A)_\bQ$, then since 
$\rho_l$  commutes with $F$, the representation actually takes values in 
$\GL_{2g/[F:\bQ]}(F\otimes \bQ_l)$. We are interested in the extreme case, when 
$[F:\bQ] = 2g$. When this occurs, we say that $A$ has complex multiplication 
defined over $K$. Write $\R_{F/\bQ} \Gm$ for the algebraic group whose functor 
of points is, for any $\bQ$-algebra $R$, given by $(R\otimes \bQ_l)^\times$. 
The representation $\rho_l$ is a map
$\rho_l\colon G_\bQ \to (\R_{F/\bQ} \Gm)(\bQ_l)$. The \emph{motivic Galois 
group} of $A$ is a $\bQ$-subgroup $G_A\subset \R_{F/\bQ} \Gm$ such that for 
all $l$, $\overline{\im(\rho_l)}^\mathrm{Zar} = G_A(\bQ_l)$. It has a canonical 
subgroup $G_A^1 = G_A^{\N_{F/\bQ} = 1}$, which we will not motivate here. There 
is a direct description of $G_A$. Let 
$\det_\fa \colon \R_{K/\bQ} \to \R_{F/\bQ}$ be induced by the determinant of 
the $K$-action on $\fa = \Lie(A)$, viewed as an $F$-vector space. Then 
$G_A = \im(\det_\fa)$. The \emph{Sato--Tate group} of $A$ is the maximal 
compact subgroup of the torus $G_A^1(\bC)$. So $\ST(A) = (\bR/\bZ)^d$ for some 
$1\leqslant d\leqslant g$. If $A$ has good reduction at $\fp\nmid l$, then 
$\rho_l(\frob_\fp)$ actually lives in $F^\times$ and is independent of $l$. 
Write $\pi_\fp\in F^\times$ for this quantity; it is a $\fp$-Weil number of 
weight $1$, i.e. $|\sigma(\pi_\fp)| = \N(\fp)^{1/2}$ for all 
$\sigma\colon F\hookrightarrow \bC$. Even better, Shimura--Taniyama--Weil have 
constructed a continuous homomorphism 
$\varepsilon\colon \bA_K^\times \to F^\times$ which agrees with $\det_\fa$ on 
$K^\times\subset \bA_K^\times$, and for almost all $\fp$, sends a uniformizer 
$\varpi_\fp$ for $\fp$ to the element $\pi_\fp$. 

For any $\sigma\colon F\hookrightarrow \bC$, let 
$\chi_\sigma\colon \bA_K^\times / K^\times \to \bC^\times$ be the 
quasicharacter $\chi_\sigma(x) = \sigma(\varepsilon(x)\psi(x_\infty)^{-1})$. 
Here, $\varepsilon(x) \psi(x_\infty)^{-1} \in (F\otimes\bR)^\times$, and we 
write $\sigma$ for the map $(F\otimes \bR)^\times \to \bC^\times$ induced by 
$\sigma$. If we write also $\sigma$ for the corresponding character of 
$\R_{F/\bQ} \Gm$, then there is equality 
\[
	L(\sigma_\ast \rho_l, s) = L(s,\chi_\sigma) .
\]
Given any $r = \sum m_\sigma \sigma \in \X^\ast(\R_{F/\bQ} \Gm)$, put 


, and 
$\theta_\fp = \frac{\rho_l(\frob_\fp)}{\N(\fp)^{1/2}}$ lies in $\ST(A)$. The 
\emph{Sato--Tate conjecture} for $A$ tells us that the $\theta_\fp$ are 
equidistributed in $\ST(A)$, i.e.~for all $f\in C(\ST(A))$, we have 
\[
	\int f(x)\, \dd x = \lim_{x\to \infty} \frac{1}{\pi_K(x)} \sum_{\N(\fp)\leqslant x} f(\theta_\fp)
\]
Serre outlined a way to prove the Sato--Tate conjecture. Note first that any 
character of $\ST(A)$ is induced by an algebraic character of $G_A$. Given 
$r\in \X^\ast(G_A)$, we associate $L(r_\ast \rho_l,s)$ with the composite 
Galois representation $r\circ \rho_l\colon G_\bQ \to \overline{\bQ_l}^\times$. 
To prove the Sato--Tate conjecture for $A$, it suffices to show that for all 
$f$, the function $L(r_\ast \rho_l,s)$ admits non-vanishing meromorphic 
continuation past $\Re = 1$, with at most a simple pole at $s = 1$. We will see 
later how this is done. For our purposes, note that it also suffices to show 
that for all $r\in \X^\ast(G_A)$ which induce a nontrivial character of 
$\ST(A)$, a bound of the form 
\[
	\left| \sum_{\N(\fp)\leqslant x} r(\theta_\fp)\right| = o\left(\pi_K(x)\right) .
\]
If we can replace $o(\pi_K(x))$ with $x^{-\frac 1 2+\epsilon}$, we will have 
established the Riemann hypothesis for $L(r_\ast \rho_l,s)$. 




Unitary representations of $\ST(A)$ are just characters, and basic 
representation theory tells us that all such representations are induced by an 
(algebraic) character of $G_A$ defined over $\overline\bQ$. For 
$r\in \X^\ast(G_A)$, there is an $L$-function $L(r_\ast \rho_l,s)$ coming from 
the composite Galois representation 
$r\circ \rho_l\colon G_\bQ \to G_A(\bQ_l) \to \overline{\bQ_l}^\times$. The 
Sato--Tate conjecture for $A$ says that all $L(r_\ast \rho_l,s)$ have 
non-vanishing analytic continuation past $\Re = 1$, and the Generalized Riemann 
hypothesis for $A$ says that all $L(r_\ast \rho_l,s)$ satisfy the Riemann 
hypothesis. 

Choose an isomorphism $(\bR/\bZ)^d \simeq \ST(A)$, and put 
\[
	\D_x(A) = \sup_{t\in [0,1]^d} \left| \frac{1}{\pi_K(x)} \sum_{\N(\fp) \leqslant x} 1_{[0,t)}(\theta_\fp) - \int 1_{[0,t)} \right| .
\]
Akiyama and Tanigawa conjectured that for non-CM elliptic curves, 
$\D_x(E) \ll x^{-\frac 1 2+\epsilon}$. We call the ``Akiyama--Tanigawa 
conjecture'' for $A$ the discrepancy decays like 
$\D_x(A) \ll x^{-\frac 1 2+\epsilon}$. Via the Koksma--Hlawka inequality, the 
Akiyama--Tanigawa conjecture implies that for all bounded-variation functions 
$f$ on $\ST(A)$, the estimate 
\[
	\left| \sum_{\N(\fp) \leqslant x} f(\theta_\fp)\right| \ll \Var(f)x^{\frac 1 2+\epsilon} .
\]
For $r\in \X^\ast(G_A)$, this estimate implies the Riemann hypothesis for the 
$L$-function $L(r_\ast \rho_l,s)$. 

Analogy with Artin $L$-functions (go into detail here!) seems to suggest that 
if all $L(r_\ast \rho_l,s)$ satisfy the Riemann Hypothesis, then the 
Akiyama--Tanigawa conjecture for $A$ holds. We'll show: this converse is false, 
in a limited sense. 





\section{Diophantine approximation}

Let $x\in [0,1]$ be irrational. It is well known that the sequence 
$(x\mod 1,2 x\mod 1,3 x\mod 1,\dots)$ is equidistributed in $[0,1]$. What is 
less well known is that the rate of convergence of empirical measures from this 
sequence to the uniform measure is governed by the irrationality measure of 
$x$. The irrationality measure $\mu(x)$ is the supremum of the set of 
$w\geqslant 1$ such that there are infinitely many $p/q$ with 
$\left| \mu - \frac p q\right| \leqslant q^{-w}$. Let's generalize this to 
higher-dimensional space. 





\section{Fake Satake parameters}

d





\end{document}
