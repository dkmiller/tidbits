\documentclass{article}

\usepackage{amsmath,amssymb}
\DeclareMathOperator{\GL}{GL}
\DeclareMathOperator{\tr}{tr}
\newcommand{\bC}{\mathbf{C}}
\newcommand{\bR}{\mathbf{R}}
\newcommand{\bZ}{\mathbf{Z}}
\newcommand{\dd}{\mathrm{d}}
\newcommand{\N}{\mathrm{N}}

\title{The method of Hadamard--de la Vall\'ee-Poussin}
\author{Daniel Miller}

\begin{document}
\maketitle





We follow Deligne's \emph{Weil II}. Let $\Gamma$ be a group isomorphic to 
either $\bR$ or $\bZ$, $\omega_1$ a non-trivial quasi-character 
$\omega_1\colon \Gamma\to \bR^+$. Let $G$ a locally compact group which is an 
extension of $\Gamma$ by a compact group $G^\circ$. Let $\Sigma$ be a 
countably infinite set, $(x_v)_{v\in \Sigma}\subset G^\natural$ a family of 
conjugacy classes in $G$. We assume: 
\begin{itemize}
\item[A'] If $\Gamma\simeq \bR$, the extension $G$ is trivial.
\end{itemize}
The following hypothesis is automatically true:
\begin{itemize}
\item[A''] If $\Gamma\simeq \bZ$, the center of $G$ maps to a subgroup of 
finite index in $\Gamma$. 
\end{itemize}

For $s\in \bC$, put $\omega_s = \omega_1^s$. Write also $\omega_s$ for the 
composite map $G\to \Gamma\to \bC^\times$, and put $\N v = \omega_{-1}(x_v)$. 
If $\Gamma\simeq \bZ$, write $q$ and $\deg$ the number $>1$ such that the 
isomorphism between $\Gamma$ and $\bZ$ such that 
$\omega_1(\gamma) = q^{-\deg(\gamma)}$. One has 
$\omega_s = \omega_{s+2\pi i\log q}$. Writing $\deg$ for the composite 
map $G\to \Gamma^{\deg}\to \bZ$, we have $\deg(v)=\deg(x_v)$. 

Let $g$ be an element of the center of $G$ with non-trivial image in $\Gamma$. 
It exists by hypothesis. A complex representation $\tau\colon G\to \GL(V)$ is 
unitarisable if and only if $\tau(g)$ is: a $g$-invariant Hermitian structure 
yields a $G$-invariant structure by integration on the compact group 
$G/g^\bZ$. If $\tau$ is irreducible, then $\tau(g)$ is a scalar, so there 
exists a unique real number $\sigma$ such that $|\tau(g)|=\omega_\sigma(g)$, 
and $\tau\cdot \omega_{-\sigma}$ is unitarizable. We call $\sigma$ the 
\emph{real part} $\Re\tau$ of $\tau$. We have 
$\Re(\tau\omega_s)=\Re(\tau)+\Re(s)$. 

Let $\widetilde G$ the set of isomorphism classes of irreducible 
representations of $G$, and $\widehat G$ the set of unitary such 
representations. The sets $\{\tau \omega_s : s\in \bC\}$ form a partition of 
$\widetilde G$, and the application $s\mapsto \tau \omega_s$ identifies 
$\{\tau\omega_s\}$ with the quotient of $\bC$ by a discrete subgroup of 
$i\bR$ if $\Gamma\simeq \bZ$, and with $\bC$ if $\Gamma\simeq \bR$. We give 
$\widetilde G$ the structure of a Riemann surface via the disjoint union of 
these quotients. 
\begin{itemize}
\item[B'] For each $v\in\Sigma$, $\N v>1$.
\item[B''] The infinite product $\prod_{v\in \Sigma} (1-\N v^{-s})^{-1}$ 
converges absolutely for $\Re s>1$. 
\end{itemize}

For $\Gamma\simeq \bZ$, these conditions tells us that $\deg(v)>0$, and, for 
each $\epsilon>0$:
\[
	\#\{v : \deg(v)=n\} = O(q^{(1+\epsilon)n}) .
\]

The hypothesis (B'') tells us that for $\tau\in \widetilde G$, the infinite 
product 
\[
	L(\tau) = \prod_{v\in \Sigma} \det(1-\tau(x_v))^{-1} 
\]
converges absolutely for $\Re(\tau)>1$. Each factor is holomorphic in $\tau$ 
for $\Re(\tau)>0$, and $L(\tau)$ is holomorphic for $\Re(\tau)>1$. We put 
$L(\tau,s)=L(\tau\omega_s)$. 

For a representation $\tau$ not necessarily irreducible, we define $L(\tau)$ 
and $L(\tau,s)$ as above. Put 
$L'(\tau)=\left.\frac{\dd}{\dd s} L(\tau,s)\right|_{s=0}$. On the domain of 
convergence:
\[
	-\frac{L'}{L}(\tau) = \sum_{v\in \Sigma,n>0} \log(\N v) \tr \tau(x_v^n) .
\]
We can generalize this to $\tau$ a virtual representation, i.e.~an element of 
the Grothendieck group of representations of $G$. For $\tau$ unitary and 
$\sigma$ real $>1$, $\omega_\sigma \tau$ in the domain of convergence, we get 
\[
	-\frac{L'}{L}(\omega_\sigma\tau) = \sum_{v\in\Sigma,n>0} (\log(\N v) (\N v)^{-n\sigma}) \tr \tau(x_v^n) .
\]

Let $mu$ be a measure on $G^\natural$. For each virtual unitary representation 
$\tau$ of $G$, we put 
\[
	\widehat\mu(\tau) = \int \tr \tau\, \dd\mu .
\]
The integral converges if the total mass $|\mu|<\infty$. We call the function 
$\tau\mapsto\widehat\mu(\tau)$ the \emph{Fourier transform} of $\mu$. If we 
don't require that $\tau$ be unitary, we call it the \emph{Fourier--Laplace 
transform}. If $\mu$ is positive with finite total mass, we have for each 
unitary virtual representation $\rho$:
\[
	\widehat\mu(\rho\otimes \bar\rho)\geqslant 0 \qquad \textnormal{(for $\mu\geqslant 0$).}
\]

We see that, for each $\sigma>1$, 
$\Lambda_\sigma(\tau) = -\frac{L'}{L}(\omega_\sigma\tau)$ is the Fourier 
transform of the positive measure with finite total mass:
\[
	\mu_\sigma = \sum_{v\in\Sigma,n>0} \log(\N v) \N v^{-n\sigma} \delta_{x_v^n} .
\]
on $G$. For each virtual unitary representation $\rho$, with real character 
$\geqslant 0$, one has $\Lambda_\sigma(\rho)\geqslant 0$ for $\sigma>1$; in 
particular, for each unitary virtual representation $\rho$, one has:
\[
	\Lambda_\sigma(\rho\otimes\bar\rho)\geqslant 0 \qquad \textnormal{(for $\sigma>1$).}
\]

For $\tau\in \widehat G$, let $\nu(\tau)$ be the order of the pole (or opposite 
of the order of zero) of $L$ at $\tau\omega_1$. We extend $\nu$ to an additive 
function on the Grothendieck group of $G$. For $\tau$ in this group, the 
function $-\frac{L'}{L}(\tau\omega_s)$ has at most simple poles, and the residue 
at $\tau\omega_1$ is $\nu(\tau)$. We get that for each unitary virtual 
representation of $G$:
\[
	\nu(\rho\otimes\bar\rho)\geqslant 0 .
\]

It is classic that the non-vanishing of $L(\tau)$ on $\Re(\tau)=1$ implies the 
equidistribution of the $x_v$. The case where $\Gamma\simeq \bR$ is treated in 
detail in Serre's \emph{Abelian $l$-adic representations and elliptic curves}. 
We treat here the case where $\Gamma\simeq \bZ$. Assume the following:
\begin{itemize}
\item[C] The function $L(\tau)$ admits a meromorphic continuation to 
$\Re\tau\geqslant 1$; on this domain at has a simple pole at $\omega_1$, and at 
the other pole, it does not vanish on $\Re(\tau)=1$.
\item[D] $\Gamma\simeq\bZ$. 
\end{itemize}

Let $z$ be a central element of $G$, with degree $d>0$. The space $G^\natural$ 
is the quotient of $G$ by the compact group $G/g^\bZ$ acting by interior 
automorphisms. It is the disjoint union of $G_n^\natural$, where $G_n^\natural$ 
is the set of conjugacy classes of degree $n$, and multiplication by $z^k$ 
induces an isomorphism from $G_n^\natural$ to $G_{n+k d}^\natural$. 

Consider the following measure on $G^\natural$:
\[
	\mu^\natural = \sum_{v\in\Sigma,n>0} \deg(v) q^{-n\deg(v)} \delta_{x_v^n} ,
\]
where 
\begin{itemize}
\item $\dd g=$ the Haar measure, where $G^\circ$ has volume $1$,
\item $\mu_0=$ the product of $\dd g$ with the characteristic function of 
$\{\omega_{-1}>1\}$. 
\item $\mu_0^\natural$ the projection of $\mu_0$ onto $G^\natural$.
\end{itemize}

Prop. If (C) and (D) are true, the Fourier-Laplace transform of 
$\mu^\natural-\mu_0^\natural$ (which converges for $\tau\in \widehat G$ with 
$\Re(\tau)>0$) extends to a holomorphic function on $\{\Re\tau>0\}$. 

The Fourier--Laplace transform of $\mu$ is 
$-\frac{1}{\log q} \frac{L'}{L}(\omega_1\tau)$. That of $\mu_0^\natural$ 
vanishes away from $\omega_s$, where it has value 
$\frac{q^{-s}}{1-q^{-s}}$. On the domain $\{\Re \tau\geqslant 0\}$, 
both $\widehat\mu^\natural$ and $\widehat \mu_0^\natural$ are meromorphic, 
with a simple pole with residue $1/\log q$ at $\omega_0$. The proposition follows. 

Theorem: assuming (C) and (D), for each $i$, the measure 
$z^{-n} \mu^\natural |_{G_{i+n d}^\natural}$ on $G_i^\natural$ converges 
weakly to $z^{-n}\mu_0^\natural|_{G_{i+n d}^\natural}$ as $n\to \infty$. 





\end{document}
