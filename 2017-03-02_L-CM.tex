\documentclass{article}

\usepackage{amsmath,amssymb}
\usepackage[a5paper]{geometry}
\DeclareMathOperator{\End}{End}
\DeclareMathOperator{\N}{N}
\DeclareMathOperator{\res}{R}
\DeclareMathOperator{\ST}{ST}
\newcommand{\bA}{\mathbf{A}}
\newcommand{\bC}{\mathbf{C}}
\newcommand{\bF}{\mathbf{F}}
\newcommand{\Gm}{\mathbf{G}_\mathrm{m}}
\newcommand{\bQ}{\mathbf{Q}}
\newcommand{\bR}{\mathbf{R}}
\newcommand{\bZ}{\mathbf{Z}}
\newcommand{\alg}{\mathrm{alg}}
\newcommand{\an}{\mathrm{an}}
\newcommand{\frob}{\mathrm{fr}}

\title{$L$-functions of elliptic curves with complex multiplication}
\author{Daniel Miller}

\begin{document}
\maketitle





Let $A_{/\bQ}$ be an abelian variety with complex multiplication (over $\bQ$!) 
by $F$. That is, if we write $\End^\circ(A) = \End_\bQ(A)\otimes\bQ$, then 
$F\simeq \End^\circ(A)$. Let $p$ be a prime at which $A$ is unramified. Then 
\[
	\frob_p \in \End^\circ(A_{\bF_p}) = \End^\circ(A) = F .
\]
Moreover, for each prime $l$, the Galois representation $\rho_{A,l}$ takes 
values in $F_l^\times = (F\otimes \bQ_l)^\times$. In fact, if we write $O$ 
for the ring of integers of $F$, then 
$\rho_{A,l}\colon G_\bQ \to O_l^\times$. Moreover, for each $p$, 
$\frob_p\in F^\times$ as a $p$-Weil number of weight $1$, 
i.e.~$|\frob_p| = \sqrt p$ under each embedding $F\hookrightarrow \bC$. 

We will define an algebraic Hecke character associated with $E$. For 
every unramified prime $p$, the Frobenius $\frob_p$ lives in $F^\times$. 
There is, for each $\sigma\colon F \hookrightarrow \bC$, a weight-$1$ Hecke 
character $\chi_\sigma\colon \bA^\times/\bQ^\times \to \bC^\times$, such 
that 
\[
	L^\alg(A,s) = \prod_{\sigma\colon F\hookrightarrow \bC} L(s,\chi_\sigma) .
\]
Thus 
\[
	L^\an(A,s) = L^\alg\left(A,s+\frac 1 2\right) = \prod_\sigma L\left(s+\frac 1 2,\chi_\sigma\right) ,
\]
where $L(s+1/2,\chi_\sigma) = L(s, \chi_\sigma \|\cdot\|^{-1/2})$, for 
$\|\cdot\|\colon \bA^\times \to \bR^+$ the adele norm. 

The Sato--Tate group for $A$ is the compact torus 
$\res_{F/\bQ}\Gm^{\N = 1}(\bR) = F_\infty^{\times, \N=1}$, which is isomorphic 
to $\prod_{\sigma \in \Phi} S^1$. The group of characters of $\ST(A)$ is 
generated by $\{\chi_\sigma\}$, so we know that if the Akiyama--Tanigawa 
conjecture for $A$ is true, then each $L(s,\prod \chi_\sigma^{m_\sigma})$ 
satisfies the Riemann Hypothesis. My counterexample shows: the converse does 
not hold! Even if all the $L(s,\prod \chi_\sigma^{m_\sigma})$ satisfy the 
Riemann Hypothesis, it does not follow that Akiyama--Tanigawa holds. 





\end{document}
