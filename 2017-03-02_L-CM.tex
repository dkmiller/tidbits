\documentclass{article}

\usepackage{amsmath,amssymb}
\usepackage[a5paper]{geometry}
\DeclareMathOperator{\End}{End}
\DeclareMathOperator{\Gal}{Gal}
\DeclareMathOperator{\GL}{GL}
\DeclareMathOperator{\im}{im}
\DeclareMathOperator{\Lie}{Lie}
\DeclareMathOperator{\N}{N}
\DeclareMathOperator{\res}{R}
\DeclareMathOperator{\R}{R}
\DeclareMathOperator{\SL}{SL}
\DeclareMathOperator{\ST}{ST}
\DeclareMathOperator{\X}{X}
\newcommand{\bA}{\mathbf{A}}
\newcommand{\bC}{\mathbf{C}}
\newcommand{\bF}{\mathbf{F}}
\newcommand{\Gm}{\mathbf{G}_\mathrm{m}}
\newcommand{\bQ}{\mathbf{Q}}
\newcommand{\bR}{\mathbf{R}}
\newcommand{\bZ}{\mathbf{Z}}
\newcommand{\fa}{\mathfrak{a}}
\newcommand{\fp}{\mathfrak{p}}
\newcommand{\ab}{\mathrm{ab}}
\newcommand{\alg}{\mathrm{alg}}
\newcommand{\an}{\mathrm{an}}
\newcommand{\frob}{\mathrm{fr}}

\title{$L$-functions of elliptic curves with complex multiplication}
\author{Daniel Miller}

\begin{document}
\maketitle





Let $A_{/\bQ}$ be an abelian variety with complex multiplication (over $\bQ$!) 
by $F$. That is, if we write $\End^\circ(A) = \End_\bQ(A)\otimes\bQ$, then 
$F\simeq \End^\circ(A)$. Let $p$ be a prime at which $A$ is unramified. Then 
\[
	\frob_p \in \End^\circ(A_{\bF_p}) = \End^\circ(A) = F .
\]
Moreover, for each prime $l$, the Galois representation $\rho_{A,l}$ takes 
values in $F_l^\times = (F\otimes \bQ_l)^\times$. In fact, if we write $O$ 
for the ring of integers of $F$, then 
$\rho_{A,l}\colon G_\bQ \to O_l^\times$. Moreover, for each $p$, 
$\frob_p\in F^\times$ as a $p$-Weil number of weight $1$, 
i.e.~$|\frob_p| = \sqrt p$ under each embedding $F\hookrightarrow \bC$. 

We will define an algebraic Hecke character associated with $E$. For 
every unramified prime $p$, the Frobenius $\frob_p$ lives in $F^\times$. 
There is, for each $\sigma\colon F \hookrightarrow \bC$, a weight-$1$ Hecke 
character $\chi_\sigma\colon \bA^\times/\bQ^\times \to \bC^\times$, such 
that 
\[
	L^\alg(A,s) = \prod_{\sigma\colon F\hookrightarrow \bC} L(s,\chi_\sigma) .
\]
Thus 
\[
	L^\an(A,s) = L^\alg\left(A,s+\frac 1 2\right) = \prod_\sigma L\left(s+\frac 1 2,\chi_\sigma\right) ,
\]
where $L(s+1/2,\chi_\sigma) = L(s, \chi_\sigma \|\cdot\|^{-1/2})$, for 
$\|\cdot\|\colon \bA^\times \to \bR^+$ the adele norm. 

The Sato--Tate group for $A$ is the compact torus 
$\res_{F/\bQ}\Gm^{\N = 1}(\bR) = F_\infty^{\times, \N=1}$, which is isomorphic 
to $\prod_{\sigma \in \Phi} S^1$. The group of characters of $\ST(A)$ is 
generated by $\{\chi_\sigma\}$, so we know that if the Akiyama--Tanigawa 
conjecture for $A$ is true, then each $L(s,\prod \chi_\sigma^{m_\sigma})$ 
satisfies the Riemann Hypothesis. My counterexample shows: the converse does 
not hold! Even if all the $L(s,\prod \chi_\sigma^{m_\sigma})$ satisfy the 
Riemann Hypothesis, it does not follow that Akiyama--Tanigawa holds. 





\section{Tate's thesis}

Consider $\bA = \bA_\bQ$. A \emph{Hecke character} is a continuous homomorphism 
$\chi\colon \bA^\times/\bQ^\times \to \bC^\times$. First, note that the 
obvious map $\bR^\times\times \widehat\bZ^\times \to \bA^\times/\bQ^\times$ is 
an isomorphism. The character $\chi$ is \emph{algebraic of weight $w$} if 
$\left. \chi\right|_{\bR^+} = (-)^{-m}$; so $\|\cdot\|$, the adele norm, is 
algebraic of weight $-1$. Since $G_\bQ^\ab = \widehat\bZ^\times$, a ``Hecke 
character'' is just a Dirichlet character + a quasicharacter of $\bR^\times$, 
which is determined by its weight (algebraic or not), and sign. 

Let $\chi$ be an algebraic Hecke character of weight $w$, $l$ a rational prime. 
Then there is a corresponding Galois representation which we'll write 
$\chi_l\colon \widehat\bZ^\times \to S^1$, given by 
\[
	\chi_l(x) = x_l^{-w} \chi_\mathrm{f}(x)
\]





\section{The whole story}

Let $A_{/\bQ}$ be an absolutely simple abelian variety with CM type $(F,\Phi)$ 
defined over $\bQ$, where $F = \End(A)_\bQ$ and 
$\hom(F,\bC) = \Phi\sqcup \overline\Phi$. There is a Galois representation 
$\rho_{A,l}\colon G_\bQ \to F_l^\times \subset \GL_{2g}(\bQ_l)$. We wish first 
to compute the motivic Galois group $G_A$. This will contain the Sato--Tate 
group $\ST(A)$, which is equal to the Mumford--Tate group of $A$. The main 
thing we need to do is compute $\X^\ast(G_A)$, show that it is equal to 
$\widehat{\ST(A)}$, and demonstrate a ``reciprocity law'' relating motivic and 
Hecke $L$-functions between the two. 

Let $L$ be the Galois closure of $F$; there is a norm map 
$\N_{\Phi^{-1}}\colon L^\times \to L^\times$, which sends 
$x\mapsto \prod_{\sigma\in \Phi} \sigma^{-1}(x)$, keeping in mind that 
$\Phi\subset \Gal(L/\bQ)$. Claim: $\N_{\Phi^{-1}}\colon L^\times \to F^\times$; 
call this map $\psi\colon \R_{L/\bQ} \Gm \to \R_{F/\bQ} \Gm$. On the level of 
characters, we have 
\[
	\N_{\Phi^{-1}}^\ast\colon \X^\ast(\R_{L/\bQ} \Gm) \to \X^\ast(\R_{L/\bQ} \Gm) ,
\]
and $\ker(\N_{\Phi^{-1}}^\ast) = \X^\ast(G_A)$. Another definition is: 
$\psi = \N_{\Phi_E} \circ \N_{F/E}$, where $E$ is the reflex field of 
$(F,\Phi)$. 

Let $G_A = \im(\psi)\subset \R_{F/\bQ} \Gm$. 





\section{Correct picture}

The key fact is: \emph{no} abelian variety has CM defined over $\bQ$. Let 
$K$ be a number field (which we may take to contain $F$ and be Galois over 
$\bQ$) and $A_{/K}$ an absolutely simple abelian variety with CM defined over 
$K$, and $F = \End^\circ(A)$. Let $\fa = \Lie(A)$; this is a $K$-vector space 
with $F$-action. The determinant gives us a natural map 
$\det_\fa\colon \R_{K/\bQ} \Gm \to \R_{F/\bQ} \Gm$; Serre--Tate refer to this 
as $\psi$. The motivic Galois group of $A$ is $G_A = \im(\det_\fa)$, and the 
canonical subgroup is $G_A^1 = \im(\det_\fa)^{\N_{F/\bQ} = 1}\subset \SL(2g)$. 
The Sato--Tate group of $A$ is the maximal compact subgroup of $G_A^1(\bC)$; 
this is a compact torus whose representations coincide with the complex 
representations of $G_A^1$. Now, 
$\X^\ast(\R_{F/\bQ} \Gm) \twoheadrightarrow \X^\ast(G_A^1)$, so any 
representation of $G_A^1$ is induced by one of $\R_{F/\bQ}\Gm$. Here, we're on 
familiar ground. For $r\in \X^\ast(\R_{F/\bQ}\Gm)$, the function 
$L(r_\ast \rho_{A,l},s)$ is writable in terms of Hecke characters. That is, 
there is an explicit Hecke character $\omega_r$ such that 
$L(r_\ast \rho_{A,l},s) = L(s,\omega_r)$, possibly up to twist. Let's do the 
details! 

First, for a prime $l$, write 
\[
	\rho_l = \rho_{A,l}\colon G_\bQ \to G_A(\bQ_l)\subset (\R_{F/\bQ} \Gm)(\bQ_l) = F_l^\times 
\]
for the associated $l$-adic Galois representation. For 
$\sigma\colon F\hookrightarrow \bC$, there is a Hecke character $\chi_\sigma$ 
such that 
\[
	\chi_\sigma(\fp) = \sigma(\rho_l(\frob_\fp)) ,
\]
from which it follows that $L^\alg(\sigma \circ \rho_l, s) = L(s,\chi_\sigma)$. 
Now $L(A,s) = L^\alg(A,s+1/2)$, so we set 
$\omega_\sigma = \chi_\sigma \|\cdot\|^{-1/2}$. Then 
\[
	L(A,s) = \prod_{\sigma\colon F\hookrightarrow \bC} L(s,\omega_\sigma) ,
\]
and for any $r = \sum a_\sigma \sigma \in \X^\ast(\R_{F/\bQ} \Gm)$, 
$L(r_\ast \rho_l,s) = L(s,\omega_r)$, where we put 
\[
	\omega_r = \prod_{\sigma\colon F\hookrightarrow \bC} \omega_\sigma^{a_\sigma} .
\]
Since each $\omega_r$ is a Hecke character, it has analytic continuation past 
$\Re = 1$, so the Sato--Tate conjecture holds for $A$. However, the 
``Diophantine Approximation counterexample'' shows that even if each 
$L(r_\ast \rho_l,s)$ satisfies the Riemann Hypothesis, it does not immediately 
follow that the Akiyama--Tanigawa conjecture holds for $A$. 





\end{document}
