\documentclass{article}

\usepackage{amsmath,amssymb}
\DeclareMathOperator{\GL}{GL}
\DeclareMathOperator{\SU}{SU}
\DeclareMathOperator{\tr}{tr}
\newcommand{\bC}{\mathbf{C}}
\newcommand{\bF}{\mathbf{F}}
\newcommand{\bQ}{\mathbf{Q}}
\newcommand{\cF}{\mathcal{F}}

\title{Discrepancy bounds over function fields}
\author{Daniel Miller}

\begin{document}
\maketitle





Let $C_{/\bF_p}$ be smooth proper curve, $\cF$ a smooth $l$-adic sheaf, pure of 
weight zero on $C$. Let $G^\mathrm{geom}$ be the geometric monodromy group of 
$\cF$, and assume what is necessary (see \S9.0 of Katz--Sarnak) for the 
Sato--Tate conjecture for $\cF$ to make sense. Let $K$ be a maximal compact subgroup 
of $G^\mathrm{geom}(\bC)$ and $K^\natural$ the set of conjugacy classes in $K$. 
For each $c\in C$, we have the Frobenius conjugacy class 
$\vartheta(c)\in K^\natural$. Katz--Sarnak prove that, for each irreducible 
$\rho\colon K\to \GL(n,\bC)$, we have the bound
\[
	\left| \frac{1}{\#\{c\in |C|:\deg c\leqslant N\}} \sum_{c\in |C|:\deg c\leqslant N} \tr \rho(\vartheta(c))\right| \ll \#\{c\in |C|:\deg c\leqslant N\}^{-\frac 1 2} .
\]
In simpler terms, if we enumerate the $c\in |C|$ as $c_1,c_2,\dots$ with ascending 
degree, then 
\begin{equation}\label{trace-bound}
	\left| \frac{1}{N} \sum_{n\leqslant N} \tr \rho(\vartheta(c_n))\right| \ll N^{-\frac 1 2} .
\end{equation}

I am wondering whether a similar bound is known on the discrepancy of the $\vartheta(c_n)$. 
For example, assume $K=\SU(2)$, so that $K^\natural=[0,\pi]$. Put 
\[
	D_N(\{\vartheta(c_n)\}) = \sup_{0\leqslant \alpha\leqslant \pi} \left| \frac{\#\{n\leqslant N : \vartheta(c_n)\in [0,\alpha)\}}{N} - \int_0^\alpha \frac{2}{\pi} \sin^2(\theta)\, \mathrm{d}\theta\right| .
\]
It is natural to conjecture (Akiyama and Tanigawa have for elliptic curves over 
$\bQ$) that 
\begin{equation}\label{disc-bound}
	D_N(\{\vartheta(c_n)\}) \ll N^{-\frac 1 2+\epsilon} .
\end{equation}
Is this known in the above case? Via the Koksma--Hlawka inequality, a discrepancy 
bound like \eqref{disc-bound} certainly implies the estimate \eqref{trace-bound}, 
but I am wondering if \eqref{disc-bound} is known even in the case where $\cF$ comes 
from a family of elliptic curves $E_{/C}$. 

I have constructed a sequence of angles $\vartheta_n\in [0,\pi]$ such that 
$|\sum_{n\leqslant N} \tr \rho(\vartheta_n)| \ll_\rho 1$, but for which 
the discrepancy decays like $N^{-1/k}$ for arbitrary $k$. So analytically, there is 
no a priori reason that the truth of \eqref{trace-bound} should imply that of 
\eqref{disc-bound}. 





\end{document}
