\documentclass{article}

\usepackage{amsmath,amssymb,hyperref,microtype}
\usepackage[a5paper,margin=1.5cm]{geometry}
\usepackage[
  hyperref = true,      % links to online documents
  backend  = bibtex,    % use bibtex instead of biber
  sorting  = nyt,       % sorts by (name, year, title)
  style    = alphabetic % citations look like [Har77]
]{biblatex}
\addbibresource{tidbit-sources.bib}
\hypersetup{colorlinks=true,linkcolor=green}

\DeclareMathOperator{\h}{H}
\DeclareMathOperator{\GL}{GL}
\DeclareMathOperator{\norm}{N}
\DeclareMathOperator{\SU}{SU}
\DeclareMathOperator{\sym}{sym}
\DeclareMathOperator{\trace}{tr}
\newcommand{\bC}{\mathbf{C}}
\newcommand{\fa}{\mathfrak{a}}
\newcommand{\fp}{\mathfrak{p}}
\newcommand{\bQ}{\mathbf{Q}}
\newcommand{\bZ}{\mathbf{Z}}
\newcommand{\dd}{\mathrm{d}}
\newcommand{\an}{\mathrm{an}}
\newcommand{\alg}{\mathrm{alg}}
\newcommand{\frob}{\mathrm{fr}}

\title{On Sarnak's letter to Mazur}
\author{Daniel Miller}

\begin{document}
\maketitle





Suppose we have an Euler product of the form 
\[
	L(\rho,s) = \prod_\fp \frac{1}{\det(1-\rho(\frob_\fp) \norm (\fp)^{-s})}
\]
Write the characteristic polyomials 
\[
	\det(1-\rho(\frob_\fp) t) = \prod (1-\lambda_{\fp,i} t) .
\]
What follows is a well-known computation. First, note that 
\[
	\frac{\dd}{\dd s}\log f(s) = \frac{f'}{f}(s) .
\]
Thus, we know that: 
\begin{align*}
	-\frac{L'}{L}(\rho,s) 
		&= -\frac{\dd}{\dd s}\log L(\rho,s) \\
		&= -\frac{\dd}{\dd s} \sum_\fp \log \frac{1}{\prod_i (1-\lambda_{\fp,i} \norm(\fp)^{-s})} \\
		&= \sum_\fp \sum_i \frac{\dd}{\dd s} \log(1-\lambda_{\fp,i} \norm(\fp)^{-s}) \\
		&= -\sum_\fp \sum_i \frac{\dd}{\dd s} \sum_{j\geqslant 1} \frac{(\lambda_{\fp,i} \norm(\fp)^{-s})^j}{j} \\
		&= -\sum_\fp \sum_i \sum_{j\geqslant 1} \frac{\dd}{\dd s} \frac{\lambda_{\fp,i}^j \norm(\fp)^{-j s}}{j} \\
		&= \sum_\fp \sum_i \sum_{j\geqslant 1} (\lambda_{\fp,i}^j \log \norm(\fp)) \norm(\fp)^{-js} \\
		&= \sum_{j\geqslant 1} \sum_\fp \frac{\log \norm(\fp)}{\norm(\fp)^{j s}} \sum_i \lambda_{\fp,i}^j
\end{align*}
This, as a computation, is some general nonsense. What if the 
characteristic polynomials are $(1-e^{i\theta_\fp} t)(1-e^{-i\theta_\fp} t)$, 
and we are taking $\sym^n \rho$? Then the characteristic polynomials of 
$\sym^n\rho$ are 
\[
	\prod_{a+b=n} (1-e^{i \theta_\fp a} e^{-i\theta_\fp b} t)
		= \prod_{a+b=n} (1-e^{i\theta_\fp (a-b)} t) .
\]





\section{Deriving (6)}

Here we prove that 
\[
	\sum_{j=0}^n e^{i(n-2j)\theta} = \frac{\sin((n+1)\theta)}{\sin\theta} .
\]
This is a basic computation:
\begin{align*}
	\sum_{j=0}^n e^{i(n-2j)\theta} 
		&= e^{in\theta} \sum_{j=0}^n (e^{-2i\theta})^j \\
		&= e^{i n \theta} \frac{(e^{-2i\theta})^{n+1}-1}{e^{-2i\theta}-1} \\
		&= \frac{e^{i(n-2(n+1))\theta}-e^{i n\theta}}{e^{-2i\theta}-1} \\
		&= \frac{e^{i(-n-2)\theta}-e^{i n\theta}}{e^{-2i\theta}-1} \\
		&= \frac{e^{-i(n+1)\theta} - e^{i(n+1)\theta}}{e^{-i\theta} - e^{i\theta}} \\
		&= \frac{\sin((n+1)\theta)}{\sin\theta} ,
\end{align*}
the last step following from the well-known identify 
$\sin\theta=\frac{e^{i\theta}-e^{-i\theta}}{2i}$. Define $U_n(\theta)$ to be that 
last function. 

We know that 
\[
	-\frac{L'}{L}(s,\sym^n\pi) = \sum_{r\geqslant 1} \sum_p \frac{\log p}{p^{r s}} U_n(r\theta_p)
\]





\section{General theory}

For the moment, we look at the local theory. Start with an arbitrary invertible 
matrix $A(t)$ depending smoothly on $t$. Then Jacobi's formula tells us that 
\[
	\frac{\dd}{\dd t} \det A(t) = \det A(t) \trace\left(A(t)^{-1} \frac{\dd A}{\dd t}{t}\right) .
\]
In other words, 
$\frac{\dd}{\dd t} \log \det A(t) = \trace\left(A(t)^{-1} \frac{\dd A}{\dd t}(t)\right)$. 

So, for the function $L_\fp(\theta,s) = \det(1-\theta \norm(\fp)^{-s})^{-1}$, 
we can compute 
\begin{align*}
	-\frac{L_\fp'}{L_\fp}(\theta,s) 
		&= \trace \left((1-\theta \norm(\fp)^{-s})^{-1} \frac{\dd}{\dd s}(1-\theta \norm(\fp)^{-s})\right) \\
		&= \trace\left(\sum_{r\geqslant 0} (\theta \norm(\fp)^{-s})^r \theta \norm(\fp)^{-s} \log \norm(\fp)\right) \\
		&= \sum_{r\geqslant 0} \trace(\theta^r \norm(\fp)^{-r s} \theta \norm(\fp)^{-s} \log \norm(\fp)) \\
		&= \log\norm(\fp) \sum_{r\geqslant 1} \frac{\trace(\theta^r)}{\norm(\fp)^{rs}}
\end{align*}

So let's look at a global $L$-function
\[
	L(s) = \prod_\fp \det(1-\theta_\fp \norm(\fp)^{-s})^{-1} .
\]
From the above computation, we have that 
\begin{align*}
	-\frac{L'}{L}(s) = \sum_{r\geqslant 1} \sum_\fp \frac{\log \norm(\fp)}{\norm(\fp)^{r s}} \trace(\theta_\fp^r) .
\end{align*}





\section{A misconception}

Let $f = \sum a_n q^n$ be a modular cusp eigenform of weight $k$. There are two 
cadidates for the local $L$-factors of the $L$-function associated to $f$, 
namely 
\begin{align*}
	L_p^\alg(f,s) &= (1-a_p p^{-s} + p^{k-1} p^{-2s})^{-1} \\
	L_p^\an(f,s) &= (1-a_p p^{-(k-1)/2} p^{-s} + p^{-2s})^{-1} \\
		&= L_p^\alg\left(s+\frac{k-1}{2}\right) .
\end{align*}
Essentially, the analytic $L$-function uses the normalized eigenvalues of 
Frobenius. Since we'll be doing analysis, we will always use the analytic 
$L$-function exclusively, and simply denote it by $L$. In particular, note if 
$f$ is a weight-$2$ modular form correspoding to an elliptic curve $E_{/\bQ}$, 
we have $L^\an(f,s) = L^\alg(E,s+1/2)$, so that $L^\an(f,1/2) = L^\alg(E,1)$ is 
predicted by BSD. 





\section{Fancy approach}

Let $G=\SU(2)$; this is a compact group. Let $f$ be a weight-$k$ modular cusp 
eigenform. We'll start without messing with symmetric powers. For each 
unramified prime $p$, we put 
\[
	x_p = p^{-(k-1)/2}\begin{pmatrix} \alpha_1(p) \\ & \alpha_2(p)\end{pmatrix} \in G^\natural,
\]
where $\alpha_i(p)$ are the eigenvalues of $\rho_{E,l}(\frob_p)$. In other 
words, $x_p$ is the (normalized ?) Satake parameter of $\pi_p$. The Sato-Tate 
conjecture tells us that $\{x_p\}\subset G^\natural$ is equidistributed. 

For any representation $\rho$ of $G$, we put, following Serre:
\[
	L(s,\rho) = \prod_p \det(1-\rho(x_p)p^{-s})^{-1} .
\]
From stuff we already know, we have the computation 
\[
	-\frac{L'}{L}(s,\sym^n) = \sum_{r\geqslant 1} \sum_p \frac{\log p}{p^{r s}} \trace \sym^n(x_p^r) .
\]
By Peter-Weil, the functions $\{\trace\sym^n\}$ form an orthonormal basis for 
$L^2(G^\natural)$. 

More generally, 
\[
	-\frac{L'}{L}(s,\rho) 
		= \sum_{\nu\geqslant 1} \sum_\fp \frac{\log \norm(\fp)}{\norm(\fp)^{\nu s}} \trace \rho(x_\fp^\nu)
\]
In other words, if we put the ``Von-Mangoldt function'' 
\[
	\Lambda_\rho(\fa) = 
	\begin{cases}
		\log \norm(\fp) \trace \rho(x_\fp^\nu) & \text{if }\fa=\fp^\nu \\
		0 & \text{otherwise}
	\end{cases} ,
\]
then 
\[
	-\frac{L'}{L}(s,\rho) = \sum_\fa \frac{\Lambda_\rho(\fa)}{\norm(\fa)^s} .
\]





\section{Functional equation for algebraic L-functions}

Here we follow \cite{fontaine-perrin-riou-1994} in computing the conjectured 
functional equation for symmetric powers of an elliptic curve. 

Let $E_{/\bQ}$ be a non-CM elliptic curve. Consider the motive $\h^1(E)$. As 
Galois representations, we have the isomorphism 
$\h^1(E,\bZ_l)\simeq T_l(E)^\vee$. But since \cite{fontaine-perrin-riou-1994} 
defines local $L$-functions using geometric Frobenius, we have 
\[
	P_p(\h^1(E),u) = \det(1-\rho_{\h^1(E),l}(\text{geom.~frob.}_p)u) = 1-a_p u + p u^2 .
\]
Now consider $\sym^n \h^1(E)$. In general, suppose we have a $2\times 2$ 
diagonal matrix $\theta = \begin{pmatrix} \lambda \\ & \mu \end{pmatrix}$. 
Here are a few of its symmetric powers:
\begin{align*}
	\sym^2 \theta &= \begin{pmatrix} \lambda^2 \\ & \lambda\mu \\ & & \mu^2 \end{pmatrix} \\
	\sym^3 \theta &= \begin{pmatrix} \lambda^3 \\ & \lambda^2 \mu \\ & & \lambda \mu^2 \\ & & & \mu^3 \end{pmatrix} \\
	&\cdots
\end{align*}
This tells us that 
\[
	\det(1-(\sym^n \theta) u) = \prod_{a+b=n} (1-\lambda^a \mu^b u) = \sum_{i=0}^n () u^i
\]
For $\sym^3$, we have 
\begin{align*}
	\det(1-(\sym^3\theta)u) 
		&= (1-\lambda^3 u)(1-\lambda^2 \mu u)(1-\lambda\mu^2)(1-\mu^3 u) \\
		&= 1 - (\lambda^3 + \lambda^2\mu + \lambda\mu^2 + \mu^3)u + \cdots \\
\end{align*}

Notation: $\omega_s(x) = |x|^s$, as $\omega\colon K^\times\to \bC^\times$. Here 
$K$ is a local field. 





\printbibliography
\end{document}
