\documentclass[11pt]{article}

\usepackage{amsmath,amssymb,extarrows,mathrsfs}
\usepackage[colorlinks,linkcolor=blue]{hyperref}
\usepackage[paperwidth=6.75in,paperheight=10in,textwidth=5.25in,textheight=8.5in]{geometry}
\usepackage[all]{xy}
\DeclareMathOperator{\character}{X}
\DeclareMathOperator{\extension}{Ext}
\DeclareMathOperator{\galois}{Gal}
\DeclareMathOperator{\GL}{GL}
\DeclareMathOperator{\h}{H}
\DeclareMathOperator{\lie}{Lie}
\DeclareMathOperator{\norm}{N}
\DeclareMathOperator{\restrict}{R}
\DeclareMathOperator{\shimura}{Sh}
\DeclareMathOperator{\SL}{SL}
\DeclareMathOperator{\SO}{SO}
\DeclareMathOperator{\spectrum}{Spec}
\newcommand{\cA}{\mathcal{A}}
\newcommand{\dA}{\mathbf{A}}
\newcommand{\dC}{\mathbf{C}}
\newcommand{\dF}{\mathbf{F}}
\newcommand{\dG}{\mathbf{G}}
\newcommand{\dR}{\mathbf{R}}
\newcommand{\dQ}{\mathbf{Q}}
\newcommand{\dS}{\mathbf{S}}
\newcommand{\dZ}{\mathbf{Z}}
\newcommand{\fg}{\mathfrak{g}}
\newcommand{\fH}{\mathfrak{H}}
\newcommand{\fk}{\mathfrak{k}}
\newcommand{\sF}{\mathscr{F}}
\newcommand{\dd}{\mathrm{d}}
\newcommand{\analytic}{\mathrm{an}}
\newcommand{\discrete}{\mathrm{disc}}
\newcommand{\etale}{\textnormal{\'et}}
\newcommand{\finite}{\mathrm{f}}
\newcommand{\hecke}{\mathcal{H}}
\newcommand{\iso}{\xlongrightarrow\sim}
\newcommand{\mult}{\mathrm{m}}

\title{Automorphic representations and deformation theory in arithmetic geometry}
\author{Daniel Miller}
\date{July 18, 2014}

\begin{document}
\maketitle
\tableofcontents





\section{The analogy between knots and primes}

We start by recalling the analogy described in \cite[ch.3-4]{m12}. 
\begin{center}
\begin{tabular}{c|c}
Topology & Arithmetic \\ \hline
3-manifold $M$ 
  & $X=\spectrum(O_F)\smallsetminus S$ for a number field $F$ \\
$\extension^\bullet(\sF,\dZ) = \h_c^{3-\bullet}(M,\sF)^\vee$
  & $\extension^\bullet(\sF,\dG_\mult) = \h_c^{3-\bullet}(X,\sF)^\vee$ \\
$S^1=K(\dZ,1)$ 
  & $\spectrum(\dF_q) = K(\widehat\dZ,1)$ \\
knot $S^1\hookrightarrow M$ 
  & prime $\spectrum(\dF_q)\hookrightarrow X$ \\
tubular neighborhood $V_K\hookrightarrow M$ 
  & $\spectrum(O_{F,v})\hookrightarrow X$ \\
boundary $\partial V_K\hookrightarrow M$ 
  & $\spectrum(F_v) \hookrightarrow X$ \\
peripheral group $\pi_1(\partial V_K)$
  & decomposition $D_v=\pi_1(F_v)\to \pi_1(X)$ \\
knot group $\pi_1(M\smallsetminus K)$
  & $\pi_1(X\smallsetminus \{v\})=G_{F,S\cup\{v\}}$ \\
$\dZ^2=\pi_1(\partial V_K) \to \pi_1(M)$ 
  & $D_v \to \pi_1(X\smallsetminus \{v\})$. 
\end{tabular}
\end{center}
Here and in the remainder of this note, if $A$ is a commutative ring we write 
$\pi_1(A)$ for the \'etale fundamental group $\pi_1^\etale(\spectrum A)$ with 
respect to an implicit basepoint. 





\section{Automorphic representations}


\subsection{Adeles}

Let $F$ be a number field. If 
$v$ is a place of $F$, write $F_v$ for the completion of $F$ at $v$, and $O_v$ 
for the ring of integers of $F_v$. Write $\dA=\dA_F$ for the ring of adeles of 
$F$; the most important fact about $\dA$ is that it is a locally compact 
$F$-algebra. It can be defined in many ways: 
\begin{itemize}
  \item The topological direct limit $\varinjlim_S\dA(S)$, where $S$ ranges 
    over all finite sets of valuations of $F$, and 
    \[
      \dA(S) = \prod_{v\in S} F_v \times \prod_{v\notin S} O_v .
    \]
  \item The topological tensor product $(\dR\times\widehat\dZ)\otimes F$.
  \item A restricted direct product $\prod_v' (F_v,O_v)$, consisting of those 
    tuples $(a_v)\in \prod_v F_v$ for which $a_v\in O_v$ for almost all $v$.
  \item Via \cite{gs66}, an initial object in the category of locally compact 
    $F$-algebras with no proper open ideals, and with the intersection of all 
    closed maximal ideals being $0$. 
\end{itemize}
Write $\dA_\finite=\widehat\dZ\otimes F$ for the ring of \emph{finite adeles}. 
It is totally disconnected. 

In \cite{c12}, it is shown that there is a unique product-preserving functor 
$(-)(\dA)$ from affine schemes of finite type over $F$ to topological spaces, 
compatible with closed embeddings, for which $(\spectrum F[t])(\dA)=\dA$ with 
its usual topology. In particular, if $G$ is an algebraic group over $F$, the 
abstract group $G(\dA)$ carries the structure of a locally compact topological 
group. As such, it carries a Haar measure $\dd g$.  


\subsection{Hecke algebras}

A good reference for this section is \cite{f79}. 

For the remainder, $G$ is a connected reductive group over $F$. Let 
$\fg=\lie G$. There exists 
an open subset of $U\subset \spectrum(O_F)$ and a ``spreading out'' of $G$ to 
a reductive group scheme on $U$. Up to finitely many places, this spreading out 
is well-defined. In particular, for almost all finite places $v$, the group 
$G(O_v)$ is well-defined for almost all $v$. It is a maximal (open) compact 
subgroup of $G(F_v)$. We normalize Haar measures on $G(F_v)$ so that 
$G(O_v)$ has volume $1$, and choose the Haar measure on $G(\dA)$ to be the 
product of measure on each $G(F_v)$. 

Let $v$ be a finite place. Write $\hecke_v=\hecke_v(G)$ for the \emph{Hecke 
algebra} consisting of locally constant, compactly supported functions 
$G(F_v) \to \dQ$. Multiplication is convolution: 
\[
  (f\star g)(x) = \int_{G(F_v)} f(g) g(y^{-1} x)\, \dd y .
\]
Even though this is written as an integral, it is a finite sum over double 
cosets of open compact subgroups of $G(F_v)$, so no analysis is involved. For 
almost all $v$, the algebra $\hecke_v$ comes with a canonical idempotent, 
$e_v=\chi_{G(O_v)}$. Write $\hecke_\finite$ for the restricted tensor product 
(in the sense of \cite[\S 2]{f79}) of the $\hecke_v$ with respect to the 
$e_v$. It is the direct limit $\varinjlim_S \hecke(S)$, where 
$\hecke(S)=\bigotimes_{v\in S} \hecke_v$, and for 
$T\supset S$, the injection $\hecke(S)\to \hecke(T)$ is induced by the 
$e_v$ for $v\in T\smallsetminus S$. We will also think of $\hecke_\finite$ as 
the algebra of locally constant, compactly supported functions $f$ on 
$G(\dA_\finite)$ for which $f^{-1}(x)\supset \prod_{v\notin S} G(O_v)$ for each 
$x$, with $S$ depending on $x$. 

The ring $F_\infty=F\otimes\dR$ is a finite product of copies of $\dR$ and 
$\dC$, so  $G(F_\infty)$ is naturally a Lie group. Fix a maximal compact 
subgroup $K_\infty\subset G(F_\infty)$. The Hecke algebra 
$\hecke_\infty=\hecke_\infty(G)$ is the convolution algebra of $K_\infty$-finite 
distributions on $G(F_\infty)$ with support in $K_\infty$. There is an 
isomorphism 
\[
  U(\fg_\dC)\otimes_{U(\fk_\dC)} A_K, \iso \hecke_\infty \qquad D\otimes \mu\mapsto D\star \mu ,
\]
where $\fk=\lie(K_\infty)$ and $A_K$ is the algebra of Borel measures on $K$. 

The \emph{global Hecke algebra} of $G$ is 
$\hecke=\hecke(G)=\hecke_\infty\otimes\hecke_\finite$. We will be interested in 
special classes of representations of $\hecke$. 

For a sufficiently large set $S$ of finite places, it makes sense to define 
$e_S\in \hecke_\finite$ to be the characteristic function of 
$\prod_{v\notin S} G(O_v)$. 

An \emph{admissible} representation of $\hecke_\finite$ is an 
$\hecke_\finite$-module $V$ such that for each $v\in V$, there is a finite set 
$S$ of places for which $e_S\cdot v = v$. 


\subsection{Automorphic representations}

A good reference for this section is \cite{bj79}. 

Let $F$, $G$, \ldots be as above. Let $Z$ be the center of $G$, and choose a 
character $\omega:Z(F)\backslash Z(\dA)\to \dC^\times$. Write 
$L^2(G,\omega)$ for the space of measurable functions 
$f:G(F)\backslash G(\dA)\to \dC$ such that 
\begin{align*}
  f(z x) &= \omega(z) f(x) && z\in Z(\dA) \\
  \|f\|^2 &= \int_{G(F)\backslash G(\dA)} |f(x)|^2\, \dd x < \infty .
\end{align*}
The space $L^2(G,\omega)$ is a representation of $G(\dA)$ in the obvious 
way. Write $L_\discrete^2(G,\omega)$ for the closed subspace generated by all 
irreducible closed subrepresentations. Let 
$\cA(G,\omega)\subset L_\discrete^2(G,\omega)$ be the space of $K$-finite 
vectors. Then $\cA(G,\omega)$ is naturally a $\hecke$-module, and as such, 
decomposes as a countable direct sum of irreducible representations with finite 
multiplicities: 
\begin{equation}\label{eq:aut-decomp}
  \cA(G,\omega) = \bigoplus_\pi m(\pi) \pi .
\end{equation}
We call the irreducible admissible representations of $\hecke$ appearing in 
\eqref{eq:aut-decomp} \emph{automorphic representations} of $G$. By 
\cite[th.4]{f79}, each automorphic representation $\pi$ decomposes as a 
restricted tensor product $\bigotimes \pi_v$ of irreducible admissible 
representations of the $\hecke_v$. 


\subsection{Hecke eigensystems and \texorpdfstring{$L$}{L}-functions}

Let $\pi$ be an automorphic representation of $G$ and choose a nonzero vector 
$u$ in $\pi$. For almost all places $v$, the idempotent $e_v=\chi_{G(O_v)}$ in 
$\hecke_v$ fixes $u$ (in this case, we say that $\pi$ is \emph{unramified} at 
$v$). In particular, the action of $\hecke_v$ on $\pi$ factors 
through that of 
\[
  \hecke_v(O_v) = e_v \hecke_v e_v = C_c^\infty(G(O_v)\backslash G(F_v)/G(O_v)) .
\]
Let $S$ be a set of places outside which $e_v$ fixes $u$. Let 
$\hecke(S)=\bigotimes_{v\notin S} \hecke_v(O_v)$. Then $\pi$ is an irreducible 
admissible module over $\hecke(S)\otimes \bigotimes_{v\in S} \hecke_v$. Since 
$\hecke(S)$ is central in this algebra, it must act via a character 
$\chi:\hecke(S)\to \dC$. The system of homomorphisms 
$\{\chi_v:\hecke_v(O_v) \to \dC:v\notin S\}$ is called a \emph{Hecke 
eigensystem}. 

In the case $G=\GL(n)$, Hecke eigensystems have a particularly easy 
description. A character $\chi:\hecke_v(O_v)\to \dC$ is uniquely determined by 
a semisimple conjugacy class $\sigma_v(\chi)\in \GL(n,\dC)$. If 
$\pi=\bigotimes_v \pi_v$ is an automorphic representation of $\GL(n)$, put 
$\sigma_v(\pi) = \sigma(\chi_{\pi_v})$ and (for finite $v$): 
\[
  L_v(s,\pi) = \det\left(1-\norm(v)\cdot \sigma_v(\pi)^{-s}\right)^{-1} .
\]
For $S$ sufficiently large, we can define the \emph{partial $L$-function} of 
$\pi$ as 
\[
  L_S(s,\pi) = \prod_{v\notin S} L_v(s,\pi) .
\]
This has the expected properties including analytic continuation, a functional 
equation\ldots. In the case $G=\GL(n)$, an automorphic representation $\pi$ is 
determined by $L(s,\pi)$. 





\section{Shimura varieties}

For the rest of this note, we exclusively consider $F=\dQ$ and $G=\GL(n)$. Many 
of the definitions work in greater generality, but technicalities (which we 
wish to avoid) multiply endlessly. Our references for this section are 
\cite{m98,s09}. 


\subsection{Locally symmetric spaces}

Classically, one studies representations of a real semisimple group $G$ by 
fixing a maximal compact $K$, setting $X=G/K$, and studying the regular 
representation of $G$ on $C^\infty(\Gamma\backslash X)$ for $\Gamma\subset G$ 
a discrete group. Big examples are the (affine) modular curves $Y_0(n)$, 
coming from $\Gamma_0(n)\subset \SL(2,\dR)$. We will carry out this 
construction adelically. 

Let $G=\GL(2)_\dQ$. As mentioned above, most constructions here work for 
arbitrary reductive $G$, but that requires introducing the notion of a 
\emph{Shimura datum}, which we omit because of space considerations. Put 
$X=Z_\infty \backslash G(F_\infty) / K_\infty$. Let 
$K\subset G(\dA_\finite)$ be a torsion-free open compact subgroup. We define 
\[
  \shimura_K(G) = G(F)\backslash (X\times G(\dA_\finite)) / K .
\]
A priori, this is only a topological space, but it turns out to canonically 
have the structure of a Riemannian manifold. Let 
$\lambda\in \character^\ast(G)$ be a dominant weight, $V_\lambda$ the induced 
representation of $G$. There is a canonical way of defining a local system
$V_\lambda$ on $\shimura_K(G)$. Put 
\[
  \h(K,\lambda) = \h_{\mathrm{sing},c}^\bullet\left(\shimura_K(G),V_\lambda\right) .
\]
In fact, the vector space $\h(K,\lambda)$ is an $\hecke_\finite$-module! We 
construct an action of the double coset $g C h$ for 
$C\subset G(\dA_\finite)$ sufficiently small. Let $C'=C\cap g^{-1} K h^{-1}$, 
and let $g C h$ act via the correspondence 
\[\xymatrix{
  \shimura_K(G) 
    & \ar@{->>}[l] \shimura_{C'}(G) \ar[r]^-{g(-)h} 
    & \shimura_K(G) .
}\]





\section{Modular representations}

In this section, algebraic groups, adeles, etc.~will be taken over $\dQ$. 
Let $n\geqslant 1$ be an integer. We define the following congruence subgroup 
of $\SL_2(\dZ)$:
\[
  \Gamma_0(n) = \left\{\begin{pmatrix}a & b \\ c & d \end{pmatrix}\in \SL(2,\dZ) : c\equiv 0\pmod n\right\} .
\]
Let $K_0(n)$ be the induced subgroup of $\GL_2(\dA_\finite)$. Write $Y_0(n)$ 
for the induced locally symmetric space: 
\[
  Y_0(n) = \shimura_{K_0(n)}(\GL_2) = \GL_2(\dQ)\backslash \GL_2(\dA) / Z_\infty K_\infty K_0(n) .
\]
Here $Z_\infty=Z(\GL_d(\dR))$ and $K_\infty=\SO(2)\subset \GL_2(\dR)$. Put 
$X=\GL_2(\dR)^+/Z_\infty K_\infty$. Note that $\GL_2(\dR)^+/Z = \SL_2(\dR)$, so 
\[
  \GL_2(\dR)^+ / Z_\infty K_\infty \iso \fH = \{z\in \dC:\Im z>0\} 
\]
via $\gamma\mapsto \gamma\cdot i$. The strong approximation theorem tells us 
that 
\[
  \GL_2(\dA) = \GL_2(\dQ) \GL_2(\dR) K_0(n),
\]
so the quotient $\shimura_{K_0(n)}(\GL_2)$ is just 
\[
  (\GL_2(\dQ)\cap K_0(n))\backslash \fH = \Gamma_0(n) \backslash \fH = Y_0(n).
\]
This will be a singular complex-analytic orbifold. There are two ways of 
realizing $Y_0(n)$ and its compactication $X_0(n)$ as curves over $\dQ$: 
\begin{enumerate}
  \item Interpret $Y_0(n)$ as a moduli space for elliptic curves with level 
    structure. This moduli problem makes sense over $\dQ$, so $Y_0(n)$ descends 
    in a canonical way to $\dQ$. 
  \item Use the general theory of canonical models of Shimura varieties. 
\end{enumerate}
The former approach generalizes to a special class of Shimura varieties 
consisting of those of \emph{PEL type} (standing for 
\textbf{P}olarization, \textbf{E}ndomorphism, and \textbf{L}evel structure). 
The space 
\[
  \fH^\pm = Z_\infty \backslash \GL_2(\dR) / \SO_2(\dR) 
\]
can be interpreted as the set of $\GL_2(\dR)$-conjugacy classes of 
homomorphisms $\dS=\restrict_{\dC/\dR}\dG_\mult\to \GL(2)_\dR$ containing 
\[
  h:(x,y)\mapsto \begin{pmatrix} x & y \\ -y & x \end{pmatrix} .
\]
This is all defined over $\dQ$ (one says $\dQ$ is the \emph{reflex field} of 
the Shimura datum $(\GL_2,\fH^\pm)$, so by \cite[2.18]{m98} if 
$K_\finite\subset\GL_2(\dA_\finite)$ is any open compact subgroup, the quotient 
\[
  \shimura_{K_\finite}(\GL_2) = \GL_2(\dQ)\backslash (\fH^\pm\times \GL_2(\dA_\finite))/K_\finite
\]
descends to a uniquely determined curve over $\dQ$. 


[modular curves via canonical models of Shimura varieties]

[what sheaf on $\shimura_{K_0(n)}(\GL_2)$ do modular forms live in?]

[Hecke operators via Satake isomorphism]





\section{Deformation theory}

Here we follow the exposition in \cite[ch.13-14]{m12}. 





\bibliographystyle{alpha}
\bibliography{tidbit-sources}

\end{document}
