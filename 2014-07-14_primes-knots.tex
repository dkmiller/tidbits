\documentclass[oneside]{amsart}

\usepackage{amsmath,amssymb,extarrows,mathrsfs,newtxtext,newtxmath,stmaryrd}
\usepackage[colorlinks,linkcolor=blue]{hyperref}
\usepackage[paperwidth=6.1in,paperheight=9.25in,textwidth=4.6in,textheight=8in]{geometry}
\usepackage[all]{xy}
\DeclareMathOperator{\annihilator}{ann}
\DeclareMathOperator{\automorphism}{Aut}
\DeclareMathOperator{\automorphic}{\mathcal{A}}
\DeclareMathOperator{\character}{X}
\DeclareMathOperator{\extension}{Ext}
\DeclareMathOperator{\formalspectrum}{Spf}
\DeclareMathOperator{\galois}{Gal}
\DeclareMathOperator{\GL}{GL}
\DeclareMathOperator{\GSp}{GSp}
\DeclareMathOperator{\h}{H}
\DeclareMathOperator{\isometry}{Iso}
\DeclareMathOperator{\lie}{Lie}
\DeclareMathOperator{\norm}{N}
\DeclareMathOperator{\PSL}{PSL}
\DeclareMathOperator{\representation}{Rep}
\DeclareMathOperator{\restrict}{R}
\DeclareMathOperator{\shimura}{Sh}
\DeclareMathOperator{\SL}{SL}
\DeclareMathOperator{\SO}{SO}
\DeclareMathOperator{\spectrum}{Spec}
\DeclareMathOperator{\symmetric}{sym}
\DeclareMathOperator{\trace}{tr}
\DeclareMathOperator{\volume}{vol}
\newcommand{\cA}{\mathcal{A}}
\newcommand{\cX}{\mathcal{X}}
\newcommand{\dA}{\mathbf{A}}
\newcommand{\dC}{\mathbf{C}}
\newcommand{\dF}{\mathbf{F}}
\newcommand{\dG}{\mathbf{G}}
\newcommand{\dH}{\mathbf{H}}
\newcommand{\dR}{\mathbf{R}}
\newcommand{\dQ}{\mathbf{Q}}
\newcommand{\dS}{\mathbf{S}}
\newcommand{\dZ}{\mathbf{Z}}
\newcommand{\eR}{\mathsf{R}}
\newcommand{\fg}{\mathfrak{g}}
\newcommand{\fH}{\mathfrak{H}}
\newcommand{\fk}{\mathfrak{k}}
\newcommand{\fT}{\mathfrak{T}}
\newcommand{\fX}{\mathfrak{X}}
\newcommand{\sF}{\mathscr{F}}
\newcommand{\sV}{\mathscr{V}}
\newcommand{\dd}{\mathrm{d}}
\newcommand{\abelian}{\mathrm{ab}}
\newcommand{\analytic}{\mathrm{an}}
\newcommand{\arithfrob}{\mathrm{fr}}
\newcommand{\cusp}{\mathrm{cusp}}
\newcommand{\discrete}{\mathrm{disc}}
\newcommand{\etale}{\textnormal{\'et}}
\newcommand{\finite}{\mathrm{f}}
\newcommand{\hecke}{\mathcal{H}}
\newcommand{\hida}{\mathbf{h}}
\newcommand{\iso}{\xlongrightarrow\sim}
\newcommand{\mult}{\mathrm{m}}

\setcounter{tocdepth}{1}

\title[Representations and deformation theory in arithmetic]{Automorphic representations and deformation theory in arithmetic geometry}
\author{Daniel Miller}
\address{Department of Mathematics, Cornell University, Malott Hall, Ithaca, NY 14853, USA}
\email{dkmiller@math.cornell.edu}
\keywords{Automorphic representation, Shimura variety, deformation theory, modular form, Hecke algebra, Hida theory}
\date{July 2014}
\urladdr{http://www.math.cornell.edu/~dkmiller}
\subjclass[2000]{11F41,11F70,11F80,11G18,13D10,14G35}

\begin{document}
\maketitle

\begin{abstract}
This is a brief expository note, motivated by \cite{m12}, on the analogy 
between the character variety of the fundamental group of a hyperbolic knot, 
and the $p$-ordinary deformation space of a Galois representation. We have 
two goals: 1) Find an arithmetic analogue of the Jones polynomial. 2) 
Generalize the analogy to arbitrary reductive groups, or at least $\GL(n)$. In 
light of these goals, we begin with a lengthy aside in which we introduce 
modular representations and Hida's $p$-ordinary Hecke algebra from the 
perspective of automorphic Galois representations and eigenvarieties. 
\end{abstract}

\tableofcontents





\section{The analogy between knots and primes}

We start by recalling the analogy described in \cite[ch.3-4]{m12}. 
\begin{center}
\begin{tabular}{c|c}
Topology & Arithmetic \\ \hline
3-manifold $M$ 
  & $X=\spectrum(O_F)\smallsetminus S$ for a number field $F$ \\
$\extension^\bullet(\sF,\dZ) = \h_c^{3-\bullet}(M,\sF)^\vee$
  & $\extension^\bullet(\sF,\dG_\mult) = \h_c^{3-\bullet}(X,\sF)^\vee$ \\
$S^1=K(\dZ,1)$ 
  & $\spectrum(\dF_q) = K(\widehat\dZ,1)$ \\
knot $S^1\hookrightarrow M$ 
  & prime $\spectrum(\dF_q)\hookrightarrow X$ \\
tubular neighborhood $V_K\hookrightarrow M$ 
  & $\spectrum(O_{F,v})\hookrightarrow X$ \\
boundary $\partial V_K\hookrightarrow M$ 
  & $\spectrum(F_v) \hookrightarrow X$ \\
peripheral group $\pi_1(\partial V_K)$
  & decomposition $D_v=\pi_1(F_v)\to \pi_1(X)$ \\
knot group $\pi_1(M\smallsetminus K)$
  & $\pi_1(X\smallsetminus \{v\})=G_{F,S\cup\{v\}}$ \\
$\dZ^2=\pi_1(\partial V_K) \to \pi_1(M)$ 
  & $D_v \to \pi_1(X\smallsetminus \{v\})$. 
\end{tabular}
\end{center}
Here and in the remainder of this note, if $A$ is a commutative ring we write 
$\pi_1(A)$ for the \'etale fundamental group $\pi_1^\etale(\spectrum A)$ with 
respect to an implicit basepoint. 





\section{Automorphic representations}


\subsection{Adeles}

Let $F$ be a number field. If 
$v$ is a place of $F$, write $F_v$ for the completion of $F$ at $v$, and $O_v$ 
for the ring of integers of $F_v$. Write $\dA=\dA_F$ for the ring of adeles of 
$F$; the most important fact about $\dA$ is that it is a locally compact 
$F$-algebra. It can be defined in many ways: 
\begin{itemize}
  \item The topological direct limit $\varinjlim_S\dA(S)$, where $S$ ranges 
    over all finite sets of valuations of $F$, and 
    \[
      \dA(S) = \prod_{v\in S} F_v \times \prod_{v\notin S} O_v .
    \]
  \item The topological tensor product $(\dR\times\widehat\dZ)\otimes F$.
  \item A restricted direct product $\prod_v' (F_v,O_v)$, consisting of those 
    tuples $(a_v)\in \prod_v F_v$ for which $a_v\in O_v$ for almost all $v$.
  \item Via \cite{gs66}, an initial object in the category of locally compact 
    $F$-algebras with no proper open ideals, and with the intersection of all 
    closed maximal ideals being $0$. 
\end{itemize}
Write $\dA_\finite=\widehat\dZ\otimes F$ for the ring of \emph{finite adeles}. 
It is totally disconnected. 

In \cite{c12}, it is shown that there is a unique product-preserving functor 
$(-)(\dA)$ from affine schemes of finite type over $F$ to topological spaces, 
compatible with closed embeddings, for which $(\spectrum F[t])(\dA)=\dA$ with 
its usual topology. In particular, if $G$ is an algebraic group over $F$, the 
abstract group $G(\dA)$ carries the structure of a locally compact topological 
group. As such, it has a unique (up to scalar) Haar measure $\dd g$.  


\subsection{Hecke algebras}

A good reference for this section is \cite{f79}. 

For the remainder, $G$ is a connected reductive group over $F$. Let 
$\fg=\lie G$. There exists 
an open subset of $U\subset \spectrum(O_F)$ and a ``spreading out'' of $G$ to 
a reductive group scheme on $U$. Up to finitely many places, this spreading out 
is well-defined. In particular, for almost all finite places $v$, the group 
$G(O_v)$ is well-defined for almost all $v$. It is a maximal (open) compact 
subgroup of $G(F_v)$. We normalize Haar measures on $G(F_v)$ so that 
$G(O_v)$ has volume $1$, and choose the Haar measure on $G(\dA)$ to be the 
product of measure on each $G(F_v)$. 

Let $v$ be a finite place. Write $\hecke_v$ for the \emph{Hecke 
algebra} consisting of continuous, locally constant, compactly supported 
functions $G(F_v) \to \dQ$. Multiplication is convolution: 
\[
  (f\star g)(x) = \int_{G(F_v)} f(g) g(y^{-1} x)\, \dd y .
\]
Even though this is written as an integral, it is a finite sum over double 
cosets of open compact subgroups of $G(F_v)$, so no analysis is involved. For 
almost all $v$, the algebra $\hecke_v$ comes with a canonical idempotent, 
$e_v=\chi_{G(O_v)}$. Write $\hecke_\finite$ for the restricted tensor product 
(in the sense of \cite[\S 2]{f79}) of the $\hecke_v$ with respect to the 
$e_v$. It is the direct limit $\varinjlim_S \hecke(S)$, where 
$\hecke(S)=\bigotimes_{v\in S} \hecke_v$, and for 
$T\supset S$, the injection $\hecke(S)\to \hecke(T)$ is induced by the 
$e_v$ for $v\in T\smallsetminus S$. We will also think of $\hecke_\finite$ as 
the algebra of locally constant, compactly supported functions $f$ on 
$G(\dA_\finite)$. 

The ring $F_\infty=F\otimes\dR$ is a finite product of copies of $\dR$ and 
$\dC$, so  $G(F_\infty)$ is naturally a Lie group. Fix a maximal compact 
subgroup $K_\infty\subset G(F_\infty)$. The Hecke algebra 
$\hecke_\infty=\hecke_\infty(G)$ is the convolution algebra of $K_\infty$-finite 
distributions on $G(F_\infty)$ with support in $K_\infty$. There is an 
isomorphism 
\[
  U(\fg_\dC)\otimes_{U(\fk_\dC)} M(K_\infty), \iso \hecke_\infty \qquad D\otimes \mu\mapsto D\star \mu ,
\]
where $\fk=\lie(K_\infty)$ and $M(K_\infty)$ is the algebra of measures on 
$K_\infty$. 

The \emph{global Hecke algebra} of $G$ is 
$\hecke=\hecke_\infty\otimes\hecke_\finite$. We will be interested in 
special classes of representations of $\hecke$. 

For a sufficiently large set $S$ of finite places, it makes sense to define 
$e_S\in \hecke_\finite$ to be the characteristic function of 
$\prod_{v\notin S} G(O_v)$. 

An \emph{admissible} representation of $\hecke_\finite$ is an 
$\hecke_\finite$-module $V$ such that for each $v\in V$, there is a finite set 
$S$ of places for which $e_S\cdot v = v$. 


\subsection{Automorphic representations}

A good reference for this section is \cite{bj79}. 

Let $F$, $G$, \ldots be as above. Let $Z$ be the center of $G$, and choose a 
character $\omega:Z(F)\backslash Z(\dA)\to \dC^\times$. Write 
$L^2(G,\omega)$ for the space of measurable functions 
$f:G(F)\backslash G(\dA)\to \dC$ such that 
\begin{align*}
  f(z x) &= \omega(z) f(x) && z\in Z(\dA) \\
  \|f\|^2 &= \int_{G(F)Z(\dA)\backslash G(\dA)} |f(x)|^2\, \dd x < \infty .
\end{align*}
The space $L^2(G,\omega)$ is a representation of $G(\dA)$ in the obvious 
way. Write $L_\discrete^2(G,\omega)$ for the closed subspace generated by all 
irreducible closed subrepresentations. Let 
$\cA(G,\omega)\subset L_\discrete^2(G,\omega)$ be the space of $K$-finite 
vectors. 

[also need $Z(\fg)$-finiteness]

[Don't define $\cA(G,\omega)$ as a subspace of $L^2$ -- the whole thing isn't. 
Instead, just define $\cA(G)$, the space of automorphic forms.]

Then $\cA(G,\omega)$ is naturally a $\hecke$-module, and as such, 
decomposes as a countable direct sum of irreducible representations with finite 
multiplicities: 
\begin{equation}\label{eq:aut-decomp}
  \cA(G,\omega) = \bigoplus_\pi m(\pi) \pi .
\end{equation}
We call the irreducible admissible representations of $\hecke$ appearing in 
\eqref{eq:aut-decomp} \emph{automorphic representations} of $G$. By 
\cite[th.4]{f79}, each automorphic representation $\pi$ decomposes as a 
restricted tensor product $\bigotimes \pi_v$ of irreducible admissible 
representations of the $\hecke_v$. 

In the remainder, we will often pass without comment between admissible 
representations of $G(\dA_\finite)$ and admissible representations of 
$\hecke_\finite$. This is not hard. Suppose $\pi:G(\dA_\finite)\to \GL(V)$ is 
an admissible representation. For $f\in \hecke_\finite$ and $v\in V$, put 
\[
  f \star v = \int_{G(\dA_\finite)} f(x) \pi(x)\cdot v\, \dd x.
\]
This integral is actually a finite sum. Indeed, we can write $f$ as a finite 
sum of scalars multiples of characteristic functions $\chi_{g K}$, where 
$K\subset G(\dA_\finite)$ is open, compact, and fixes $v$. For such a function, 
we see that 
\[
  \chi_{g K}\star v = \int_{G(\dA_\finite)}\chi_{g K}(x) \pi(x) v\, \dd x = \int_K g v\, \dd x = \volume(K) g  v. 
\]
So the action of $\hecke_\finite$ on $V$ makes sense. Going the other way is 
also easy. If $V$ is an admissible $\hecke_\finite$-module and 
$g\in G(\dA_\finite)$, choose open compact normal $K$ such that 
$\chi_K\star v=v$. Inspired by the above, put 
$g v = \volume(K)^{-1} \chi_{g K}\star v$. 


\subsection{Hecke eigensystems and \texorpdfstring{$L$}{L}-functions}

Let $\pi$ be an automorphic representation of $G$ and choose a nonzero vector 
$u$ in $\pi$. For almost all places $v$, the idempotent $e_v=\chi_{G(O_v)}$ in 
$\hecke_v$ fixes $u$ (in this case, we say that $\pi$ is \emph{unramified} at 
$v$). In particular, the action of $\hecke_v$ on $\pi$ factors 
through that of 
\[
  \hecke_v(O_v) = e_v \hecke_v e_v = C_c^\infty(G(O_v)\backslash G(F_v)/G(O_v)) .
\]
Let $S$ be a set of places outside which $e_v$ fixes $u$. Let 
$\hecke(S)=\bigotimes_{v\notin S} \hecke_v(O_v)$. Then $\pi$ is an irreducible 
admissible module over $\hecke(S)\otimes \bigotimes_{v\in S} \hecke_v$. Since 
$\hecke(S)$ is central in this algebra, it must act via a character 
$\chi:\hecke(S)\to \dC$. The system of homomorphisms 
$\{\chi_v:\hecke_v(O_v) \to \dC:v\notin S\}$ is called a \emph{Hecke 
eigensystem}. 

In the case $G=\GL(n)$, Hecke eigensystems have a particularly easy 
description. A character $\chi:\hecke_v(O_v)\to \dC$ is uniquely determined by 
a semisimple conjugacy class $\sigma_v(\chi)\in \GL(n,\dC)$. If 
$\pi=\bigotimes_v \pi_v$ is an automorphic representation of $\GL(n)$, put 
$\sigma_v(\pi) = \sigma(\chi_{\pi_v})$ and (for finite $v$): 
\[
  L_v(s,\pi) = \det\left(1-\norm(v)\cdot \sigma_v(\pi)^{-s}\right)^{-1} .
\]
For $S$ sufficiently large, we can define the \emph{partial $L$-function} of 
$\pi$ as 
\[
  L_S(s,\pi) = \prod_{v\notin S} L_v(s,\pi) .
\]
This has the expected properties including analytic continuation, a functional 
equation\ldots. In the case $G=\GL(n)$, an automorphic representation $\pi$ is 
determined by $L(s,\pi)$. 





\section{Shimura varieties}

For the rest of this note, the reader should keep in mind the example 
$F=\dQ$, $G=\GL(2)$. Many of the definitions work in greater generality, 
but technicalities (which we wish to avoid) multiply endlessly. 


\subsection{Locally symmetric spaces and their cohomology}\label{sec:local-symm}

Classically, one studies representations of a real semisimple group $G$ by 
fixing a maximal compact $K$, setting $X=G/K$, and studying the regular 
representation of $G$ on $C^\infty(\Gamma\backslash X)$ for $\Gamma\subset G$ 
a discrete group. Big examples are the (affine) modular curves $Y_0(n)$, 
coming from $\Gamma_0(n)\subset \SL(2,\dR)$. We will carry out this 
construction adelically. 

Let $G$ be a connected reductive group over $\dQ$. Put 
$X=Z_\infty \backslash G(F_\infty) / K_\infty$. Let 
$K\subset G(\dA_\finite)$ be an open compact subgroup. We define 
\[
  \shimura_K(G) = G(\dQ)\backslash (X\times G(\dA_\finite)) / K .
\]
A priori, this is only a topological space, but the quotient map 
$X\times G(\dA_\finite)/K \to \shimura_K(G)$ gives $\shimura_K(G)$ the 
structure of a Riemannian manifold. Let $(V,\rho)$ be a representation of $G$. 
There is an induced local system $\sV_\rho$ of $F$-vector spaces on 
$\shimura_K(G)$, whose (global) sections are locally constant sections 
$s:X\times G(\dA_\finite)/K \to V$ such that 
$s(\gamma x) = \rho(\gamma) s(x)$ for $\gamma\in G(\dQ)$. 

The cohomology $\h^\bullet(\shimura_K(G),\sV_\rho)$ is naturally an 
admissible $\hecke_\finite$-module. For open compact $C\subset K$, the function 
$\chi_{g C}$ acts via the correspondence 
\[\xymatrix{
  \shimura_K(G) 
    & \ar@{->>}[l] \shimura_{K\cap C}(G) \ar[r]^-{\cdot g} 
    & \shimura_{K\cap g^{-1} C g}(G) \ar@{->>}[r] 
    & \shimura_K(G) .
}\]
There is a standard compactification of $\shimura_K(G)$, namely its 
\emph{Borel-Serre compactification} $\overline\shimura_K(G)$. Define the 
\emph{cuspidal cohomology} to be 
\[
  \h_\cusp^\bullet(\shimura_K(G),\sV_\rho) = \ker\left(\h^\bullet(\overline\shimura_K(G),\sV_\rho) \to \h^\bullet(\partial\shimura_K(G),\sV_\rho)\right) .
\]
The cuspidal cohomology is also an admissible $\hecke_\finite$-module. In 
fact, we have a \emph{generalized Eichler-Shimura isomorphism} \cite[4.1]{s09}. 
\[
  \h_\cusp^\bullet(\shimura_K(G),\sV_\rho) = \bigoplus_{\substack{\pi\in \cA_\cusp(G,\chi_\rho) \\ K\text{-spherical}}}\h^\bullet(\hecke_\infty, \pi_\infty\otimes \rho)\otimes \pi_\finite .
\]
The notation $\h^\bullet(\hecke_\infty,-)$ needs explanation. There is a good 
category of admissible $\hecke_\infty$-modules, and $\hom(\dC,-)$ is 
left-exact. Its derived functor is the $(\fg_\infty,K_\infty)$-cohomology 
$\h^\bullet(\hecke_\infty-)$. 


\subsection{Canonical models}\label{sec:can-model}

A good reference for this is \cite{m98}. In \autoref{sec:local-symm} we 
constructed $\shimura_K(G)$ as a Riemannian manifold. It turns out that there 
is a good definition of ``canonical model'' for Shimura varieties over number 
fields, and in that sense, all Shimura varieties $\shimura_K(G)$ have a 
canonical model over a number field called the \emph{reflex field}. (We have 
been intentionally avoiding use of the Shimura datum necessary to define 
$\shimura_K(G)$ in full generality -- the reflex field depends on this.) 

Let $E$ be the reflex field. Not only does the projective system 
$\shimura(G)=\varprojlim \shimura_K(G)$ descend to the reflex field, but the 
action of $G(\dA_\finite)$ via correspondences descends in a canonical way. 
Moreover, in \cite{harris85} it is shown that our local systems $\sV_\rho$ 
descend in a functorial way to $G(\dA_\finite)$-equivariant local systems 
on $\shimura(G)$. 

The main reason we care about this is that if $\shimura_K(G)$ is smooth and 
$E=\dQ$, then general theorems about \'etale cohomology tell us that 
\[
  \h_{\mathrm{sing},c}^\bullet(\shimura_K(G),\sV_\rho) = \h_{\etale,c}^\bullet(\shimura_K(G)_{\overline\dQ},\sV_\rho(\overline{\dQ_\ell})) 
\]
after choice of an isomorphism $\dC\simeq \overline{\dQ_\ell}$. The choice of 
an \emph{arithmetic compactification} of $\shimura_K(G)$ lets us extend the 
action of $\hecke_\finite$ to the \'etale cohomology of $\shimura_K(G)$. 





\section{Modular representations}


\subsection{Modular curves}

In this section, algebraic groups, adeles, etc.~will be taken over $\dQ$. 
Let $n\geqslant 1$ be an integer. We define the following congruence subgroup 
of $\SL_2(\dZ)$:
\[
  \Gamma_0(n) = \left\{\begin{pmatrix}a & b \\ c & d \end{pmatrix}\in \SL(2,\dZ) : c\equiv 0\pmod n\right\} .
\]
Let $K_0(n)$ be the induced subgroup of $\GL_2(\dA_\finite)$. Write $Y_0(n)$ 
for the induced locally symmetric space: 
\[
  Y_0(n) = \shimura_{K_0(n)}(\GL_2) = \GL_2(\dQ)\backslash \GL_2(\dA) / Z_\infty K_\infty K_0(n) .
\]
Here $Z_\infty=Z(\GL_d(\dR))$ and $K_\infty=\SO(2)\subset \GL_2(\dR)$. Put 
$X=\GL_2(\dR)^+/Z_\infty K_\infty$. Note that $\GL_2(\dR)^+/Z = \SL_2(\dR)$, so 
\[
  \GL_2(\dR)^+ / Z_\infty K_\infty \iso \fH = \{z\in \dC:\Im z>0\} 
\]
via $\gamma\mapsto \gamma\cdot i$. The strong approximation theorem tells us 
that 
\[
  \GL_2(\dA) = \GL_2(\dQ) \GL_2(\dR) K_0(n),
\]
so the quotient $\shimura_{K_0(n)}(\GL_2)$ is just 
\[
  (\GL_2(\dQ)\cap K_0(n))\backslash \fH = \Gamma_0(n) \backslash \fH = Y_0(n).
\]
This will be a singular complex-analytic orbifold. There are two ways of 
realizing $Y_0(n)$ and its compactication $X_0(n)$ as curves over $\dQ$: 
\begin{enumerate}
  \item Interpret $Y_0(n)$ as a moduli space for elliptic curves with level 
    structure. This moduli problem makes sense over $\dQ$, so $Y_0(n)$ descends 
    in a canonical way to $\dQ$. 
  \item Use the general theory of canonical models of Shimura varieties. 
\end{enumerate}
The former approach generalizes to a special class of Shimura varieties 
consisting of those of \emph{PEL type} (standing for 
\textbf{P}olarization, \textbf{E}ndomorphism, and \textbf{L}evel structure). 
The theory of PEL-type Shimura varieties is interesting and useful, but we 
won't go into it here. 

Instead, note that the space 
\[
  \fH^\pm = Z_\infty \backslash \GL_2(\dR) / \SO_2(\dR) 
\]
can be interpreted as the set of $\GL_2(\dR)$-conjugacy classes of 
homomorphisms $\dS=\restrict_{\dC/\dR}\dG_\mult\to \GL(2)_\dR$ containing 
\[
  h:(x,y)\mapsto \begin{pmatrix} x & y \\ -y & x \end{pmatrix} .
\]
This is all defined over $\dQ$, so the theory of canonical models discussed 
in \autoref{sec:can-model} tells us that if 
$K\subset\GL_2(\dA_\finite)$ is any open compact subgroup, the quotient 
\[
  \shimura_K(\GL_2) = \GL_2(\dQ)\backslash (\fH^\pm\times \GL_2(\dA_\finite))/K
\]
descends to a uniquely determined curve over $\dQ$. Moreover, this curve has a 
well-defined smooth compactification also defined over $\dQ$, so we don't need 
to worry about the difference between minimal and toroidal compactifications. 


\subsection{The Eichler-Shimura construction}


%Use the generalized Eichler-Shimura isomorphism given in 
%\cite{harder-1987}. Via the introduction of 
%\url{http://www-users.math.umn.edu/~kwlan/articles/iccm-2010.pdf}, there is an isomorphism 
%\[
%  \h_\cusp^1(\shimura_{K_0(n)}(\GL_2),\symmetric^{k-2}) = S_k(\Gamma_0(n))\oplus \overline{S_k(\Gamma_0(n))} .
%\]
%So the ``weight'' in $S_k$ comes from the weight parameterizing the irreducible 
%representation of $\GL(2)$. To derive this sort of thing from Harder's generalized 
%Eichler-Shimura, I need to check that the $\infty$-type of a weight-$k$ modular 
%is $\symmetric^{k-2}$. 

Let $n\geqslant 3$. 
As above, write $Y_0(n)$ for the Shimura variety $\shimura_{K_0(n))}(\GL_2)$, 
and write $X_0(n)$ for its arithmetic compactification. We are interested in the 
cohomology $\h_\cusp^1(X_0(n),\sV_{\symmetric^{k-2}})$.
At the moment, this is just a $\dC$-vector space with an action of 
$\hecke_\finite$. However, in \autoref{sec:local-symm}, there is an automorphic 
decomposition 
\[
  \h_\cusp^1(X_0(n),\sV_{\symmetric^{k-2}}) = \bigoplus_{\substack{\pi\in \automorphic(\GL_2) \\ \Gamma_0(n)\text{-spherical}}} \h^1(\mathfrak{gl}_2,\SO(2),\pi_\infty\otimes \symmetric^{k-2})\otimes \pi_\finite .
\]
The computation in \cite[\S3.4-3.6]{harder-1987} tells us that 
$\h^1(\mathfrak{gl}_2,\SO(2),\pi_\infty\otimes \symmetric^{k-2})=0$ unless 
$\pi$ is the automorphic representation coming from a weight-$k$ cuspidal 
eigenform of level $n$, in which case 
$\h^1(\mathfrak{gl}_2,\SO(2),\pi_\infty\otimes \symmetric^{k-2})=\dC\oplus \overline\dC$. 
In particular, 
\[
  \h_\cusp^1(X_0(n),\sV_{\symmetric^{k-2}}) = \bigoplus_{f\text{ eigen-cusp}} \dC\oplus \overline{\dC} = S_k(\Gamma_0(n))\oplus \overline{S_k(\Gamma_0(n))}.
\]

Recall that our modular curves are defined over $\dQ$. So we can consider the 
cohomology spaces 
\[
  \h_{\etale,\cusp}^1(X_0(n)_{\overline\dQ},\overline{\dQ_\ell}) \simeq S_2(\Gamma_0(n),\dC) .
\]
These have commuting actions of 
$G_{\dQ,\ell n}=\pi_1(\dZ[\frac{1}{\ell n}])$ and 
$\hecke_\finite$. So $G_{\dQ,\ell n}$ acts on each $\hecke_\finite$-irreducible 
piece. These pieces are 2-dimensional, so we get, for each cuspidal eigenform 
$f$, a Galois representation 
$\rho_{f,\ell}:G_{\dQ,n\ell} \to \GL_2(\overline{\dQ_\ell})$. 
The \emph{Eichler-Shimura relation} basically tells us that the Hecke and 
Frobenius parameters for $\pi_f$ and $\rho_{f,\ell}$ match up. That is, for all 
$p\nmid \ell n$, we have 
$\rho_{f,\ell}(\arithfrob_p) = \sigma_p(\pi_f)$, or equivalently 
$L_p(\rho_{f,\ell},s) = L_p(\pi_f,s)$. 

It is known that $\rho_{f,\ell}:G_{\dQ,\ell n} \to \GL_2(\overline{\dQ_\ell})$ 
factors through $\GL_2(K_{f,\lambda})$, where 
$K_f = \dQ(a_p(f):p\text{ prime})$ is a number field and $\lambda$ is a place 
of $K_f$ dividing $\ell$. An elementary argument shows that we can conjugate 
the image of $\rho_{f,\ell}$ to lie in $\GL_2(O_{f,\lambda})$, where 
$O_f=O_{K_f}$. In particular, we can reduce $\rho_{f,\ell}$ modulo $\lambda$ 
to get a continuous representation 
\[
  \bar\rho_{f,\ell}:G_{\dQ,\ell n} \to \GL_2(O_{f,\lambda}/\lambda) = \GL_2(\dF_\lambda) .
\]
We say that a mod-$\ell$ representation of $G_\dQ$ is \emph{modular} (better, 
\emph{automorphic}) if it is of the form $\bar\rho_{f,\ell}$ for some $f$. 
Similarly, we say that an $\ell$-adic representation of $G_\dQ$ is 
\emph{modular} (better, \emph{automorphic}) if it is of the form 
$\rho_{f,\ell}$ for some $\ell$. 

In what follows, we will ignore the modular form $f$ and just write $\pi$ for 
the corresponding cuspidal automorphic representation of $\GL(2)$, keeping in 
mind that for some automorphic representations (those corresponding to Maass 
forms) we still don't know how to construct the associated Galois 
representations. For an automorphic representation $\pi$, write 
$\rho_{\pi,\ell}$ for the corresponding $\ell$-adic representation (assuming 
it exists). 


\subsection{Interpolating modular representations}

We showed explicitly how to construct the Galois representations associated to 
cuspidal eigenforms of weight $2$. In fact, in \cite{deligne-1973}, Deligne 
showed how to construct $\rho_{f,\ell}$ for \emph{any} cusp-eigenform $f$ of 
weight $k\geqslant 2$. Recall that such forms have a Fourier expansion 
\[
  f(z) = \sum_{n\geqslant 1} a_n(f) e^{2\pi n z} .
\]
Call $f$ \emph{$p$-ordinary} if $a_p(f)$ is a $p$-adic unit. This is equivalent 
to $\bar\rho_{f,p}$ being an extension of an unramified character by a 
character. In \cite{hida-1986a,hida-1986b}, Hida $p$-adically interpolated the 
Galois representations $\rho_{f,p}$ coming from $p$-ordinary $f$ of varying 
weight and level. 

Fix a prime $p\geqslant 5$ and an integer $n$ prime to $p$. Recall that our 
modular curves are defined over $\dQ$. So instead of defining modular forms in 
terms of automorphic representations showing up in the spectral decomposition 
of $\h_\cusp^1(X_0(n),\dC)$, we can (as before) 
consider modular forms as irreducible $\hecke_\finite$-components of 
$\h_{\cusp,\etale}^1(X_0(n)_{\overline\dQ},\dZ_p)$. This lets us define a space 
$S_k(\Gamma_0(n p^r),\dZ_p)$ admitting an action of $\hecke_\finite$. 

Let $K_0\subset \overline{\dQ}$ contain all the Hecke eigenvalues of forms in 
\[
  S_k(n p^\infty,\dC) = \varinjlim_r S_k(n p^r,\dC) .
\]
Let $K$ be the closure of $K_0$ in $\dC_p$, and put 
\[
  S_k(n p^\infty,O_K) = \varinjlim_r S_k(n p^r,O_K) .
\]
For each $r$, let $\hida_{n,r})$ be the image of $\hecke_\finite$ inside 
$\operatorname{End} S_2(n p^r,O_K)$. By \cite[1.1]{hida-1986a}, the projective 
limit $\hida_n = \varprojlim_r \hida_{n,r}$ actually acts on all the 
$S_k(n p^\infty,O_K)$. There is an idempotent $e\in \hida_n$ projecting onto 
the space of ordinary cusp forms; put $\hida_n^\circ=e \hida_n$. 

Suppose $f$ is a $p$-ordinary cusp form of level $n p^r$, and write 
$\bar\rho=\bar\rho_{f,p}$. Write $\hida_n^\circ(\bar\rho)$ for the algebra 
acting on $p$-ordinary forms congruent to $f$ mod $p$. One has to 
``fudge'' $\hida_n^\circ(\bar\rho)$ into a ring 
$\widetilde\hida_n^\circ(\bar\rho)$. Then there is a Galois 
representation $\rho^\mathrm{H}:G_{\dQ,n p} \to \GL_2(\widetilde\hida_n(\bar\rho))$ 
that induces all the $\rho_{f,p}$. 

[check this: details probably not right!]





\section{Deformation theory}


\subsection{Motivation}

First let's consider the 
motivation for studying deformations of Galois representations. If $X$ is a 
nice (that is smooth, projective and geometrically integral) variety over 
$\dQ$, its \'etale cohomology 
$\h_\etale^\bullet(X_{\overline\dQ},\dQ_\ell)$ carries a continuous 
action of $G_\dQ$. The ``right'' way to see this is as follows. Spread out 
$X$ to a smooth proper scheme $\cX$ over an open 
$U= \spectrum(\dZ)\smallsetminus S$. Write $\pi:\cX\to U$ for the structure 
map. Then $\eR^\bullet \pi_\ast \dQ_\ell$ is a local system on 
$U_\etale$. The \'etale version of covering space theory tells us that local 
systems correspond to representations of $\pi_1(U)=G_{\dQ,S}$. 

The motivating example was the representation $\rho_{E,\ell}$ coming from an 
elliptic curve $E\xrightarrow\pi U$ via $\eR^1 \pi_\ast \dQ_\ell$. Langlands' 
conjectural framework tells us that there should exist an automorphic cuspidal 
representation $\pi$ of $\GL(2)$ for which 
$\rho_{E,\ell} \simeq \rho_{\pi,\ell}$. We don't know how to construct $\pi$ 
this directly. However, we \emph{do} know (via Serre's conjecture) that 
$\bar\rho_{E,\ell}$ is automorphic. One of the main applications of 
deformation theory is to prove that the automorphy of $\bar\rho_{E,\ell}$ 
implies that of $\rho_{E,\ell}$. 

More generally, given an $\ell$-adic Galois 
representation $\rho:G_{\dQ,S}\to \GL_n(\dQ_\ell)$ that is suitably nice 
(\emph{geometric}, in the sense of Fontaine-Mazur \cite{fontain-mazur-1995}), 
Langlands' program tells us that we should expect there to be a cuspidal 
automorphic representation $\pi$ of $\GL(n)$ such that 
$\rho\simeq \rho_{\pi,\ell}$ (assuming we knew how to construct $\rho_\pi$ in 
general). Proving that $\rho$ is automorphic is very hard! However, we have a 
much better chance (in theory and in practice) of showing that 
$\bar\rho:G_{\dQ,S} \to \GL_n(\dF_\ell)$ is automorphic. A theorem to the 
effect that ``$\bar\rho$ automorphic $\Rightarrow\rho$ automorphic'' is known 
as a \emph{automorphy lifting theorem}. In practice, one has to impose many 
technical conditions on $\bar\rho$ and the automorphic representation with 
$\bar\rho_\pi\simeq \bar\rho$, and one uses groups like $\GSp(n)$ instead of 
$\GL(n)$. 


\subsection{Representations of knot groups}

Our exposition here follows that of \cite[ch.13-14]{m12}. Let 
$K\subset S^3$ be a hyperbolic knot, $M=S^3\smallsetminus K$ the knot 
complement, $\pi=\pi_1(M)$ the knot group. The uniformization $\dH^3\to M$ 
induces a representation $\pi \to \automorphism(\dH^3) = \PSL_2(\dC)$ which 
lifts to $\rho:\pi \to \SL_2(\dC)$. Introduce the \emph{representation 
variety} $\representation(\pi,\SL_2)$ of homomorphisms $\pi\to \SL_2(\dC)$. 
There are two ways of describing $\representation(\pi,\SL_2)$. One elementary 
but useful approach is to write 
$\pi=\langle g_1,\dots,g_m|r_1,\dots,r_n\rangle$. The variety 
$\representation(\pi,\SL_2)$ is just the subset of $\SL_2(\dC)^m$ cut out by 
the relations $r_1,\dots,r_n$. A more functorial definition is to require that 
for all $\dC$-algebras $A$, a natural isomorphism 
\[
  \hom_\mathsf{grp}(\pi,\SL_2(A)) \simeq \hom_{\mathsf{sch}/\dC}(\spectrum A,\representation(\pi,\SL_2)) .
\]

The \emph{character variety} of $K$ is the geometric quotient 
\[
  X_K = \representation(\pi,\SL_2) /\!\!/ \SL_2 = \spectrum\left(\dC[\representation(\pi,\SL_2)]^{\SL_2(\dC)}\right) ,
\]
via the obvious action of $\SL(2)$ on $\representation(\pi,\SL_2)$ via 
conjugation. The representation $\rho$ is a point in $X_K$, and one is 
interested in the connected component $X_K(\rho)$. 


\subsection{Deformation functors}

The analogous situation in number theory is much more complicated, partly 
because $G_S=\pi_1(\spectrum(\dZ)\smallsetminus S)$ is not a finitely presented 
group -- it's a compact topological group which is (conjecturally) 
\emph{topologically} finitely presented. So instead of looking for 
representations $G_S\to \GL_2(\dC)$, we should look for continuous 
representations $G_S\to \GL_2(A)$, where $A$ is a topological $\dZ_p$-algebra. 

Briefly, a scheme $X$ over $k$ can be thought of in terms of its functor of 
points $X(-):k\textnormal{-}\mathsf{Alg}\to \mathsf{Set}$. In the arithmetic 
context, our deformation spaces will be \emph{formal schemes} over $\dZ_p$. For 
us, this just means that the test category consists of complete local 
pro-artinian $\dZ_p$-algebras with residue field $\dF_p$. If $R$ is such an 
ring, we write $\formalspectrum(R)$ to denote the functor 
$A\mapsto\hom(R,A)$. There is a way of making $\formalspectrum(R)$ into a 
topological space with structure sheaf, but we will not need this. 

The functorial approach to defining representation schemes works well. 
If $\pi$ is an arbitrary profinite group, there is a 
formal scheme $\widehat\representation(\pi,\GL_n)$, satisfying 
\[
  \widehat\representation(\pi,\GL_n)(A) = \hom_\mathrm{cts}(\pi,\GL_n A) ,
\]
for any local pro-artinian $\dZ_p$-algebra $A$ with residue field $\dF_p$. The 
problem is, $\widehat\representation(\pi,\GL_n)$ is really horrible as a space 
-- it generally has infinitely many different connected components. So before 
we do anything else, let's restrict to the connected component of $\bar\rho$, 
where $\bar\rho:\pi\to \GL_n(\dF_p)$ has been fixed beforehand. The component 
$\fX^\square(\bar\rho)$ represents continuous homomorphisms 
$\pi\to \GL_n(A)$ that reduce to $\bar\rho$ modulo $p$. 

As before, we will quotient out by the natural action of $\GL(n)$, but here we 
should be careful because $\GL(n)$ does not preserve the component 
$\fX^\square(\bar\rho)$. The correct thing to do is to first define 
\[
  \widehat\GL_n(A) = \{g\in \GL_n(A):g\equiv 1\pmod p)\} = \ker\left(\GL_n(A)\to \GL_n(\dF_p)\right).
\]
The action of $\widehat\GL(n)$ on $\widehat\representation(\pi,\GL_n)$ 
preserves $\fX^\square(\bar\rho)$. Now a miracle happens. Define 
\[
  \fX(\bar\rho)(A) = \fX^\square(A)/\widehat\GL_n(A).
\]
Then in \cite[pr.1]{mazur-1989}, it is proved that if $\bar\rho$ is absolutely 
irreducible and $\pi$ satisfies a certain technical hypothesis (which will hold 
for all our examples), the functor $\fX(\bar\rho)$ is represented by 
a complete local noetherian $\dZ_p$-algebra $R_{\bar\rho}$ with residue field 
$\dF_p$. That is, there is a representation $\rho:\pi\to \GL_n(R_{\bar\rho})$ 
lifting $\bar\rho$ such that any $\widehat\GL_n(A)$-equivalence class of lifts 
$\pi\to \GL_n(A)$ is induced by a unique continuous homomorphism 
$R_{\bar\rho} \to A$. 

Suppose we have fixed a subgroup $I\subset \pi$. We call a representation 
$\rho:\pi\to \GL_2(A)$ \emph{$I$-ordinary} if $\rho^I$ is a free, rank-one, 
direct summand of $\rho$. Suppose $\bar\rho$ is absolutely irreducible and 
$I$-ordinary. Then we can define a subfunctor $\fX^\circ(\bar\rho)$
of $\fX(\bar\rho)$ by 
\[
  \fX^\circ(\bar\rho)(A) = \{\rho\in \fX(\bar\rho)(A):\rho\text{ is $I$-ordinary}\} .
\]
By \cite[pr.3]{mazur-1989}, $\fX^\circ(\bar\rho)$ is represented by a complete 
local noetherian $\dZ_p$-algebra $R_{\bar\rho}^\circ$ with residue field 
$\dF_p$. 


\subsection{The case \texorpdfstring{$n=1$}{n=1}}

The easiest example is when our representations take values in $\GL(1)$. Let 
$\pi=\pi_1(\dZ[\frac 1 p])$ and 
$\bar\rho=\bar\kappa:\pi\to \GL_1(\dF_p)$ be the \emph{mod-$p$ cyclotomic 
character}. This is defined, for $\sigma\in G_\dQ$, by 
\[
  \sigma(\zeta_p) = \zeta_p^{\bar\kappa(\sigma)} .
\]
Let's start by computing $\widehat\representation(\pi,\GL_1)$. Deformations 
$\rho:\pi\to A^\times$ factor through $\pi^\abelian$. Class field theory tells 
us that $\pi^\abelian \simeq \dZ_p^\times$. So 
\[
  \widehat\representation(\pi,\GL_1) = \formalspectrum\left(\dZ_p\left\llbracket \dZ_p^\times\right\rrbracket\right) \simeq \coprod_{\varepsilon:\pi \to \dF_p^\times} \formalspectrum\left(\dZ_p\llbracket \dZ_p\rrbracket\right) ,
\]
where each $\formalspectrum(\dZ_p\llbracket \dZ_p\rrbracket)$ is the connected 
component of some $\varepsilon$. We see that 
$R_{\bar\kappa}\simeq \dZ_p\llbracket \dZ_p\rrbracket\simeq \dZ_p\llbracket X\rrbracket$, 
via $[p]\leftrightarrow X+1$. 


\subsection{The main example}

Let $U\subset\spectrum(\dZ)$ be open, and put $\pi=\pi_1(U)$. Let 
$E\xrightarrow e U$ be an elliptic curve. Choose a prime $p\notin U$. The 
$p$-torsion subscheme $E[p]$ is an \'etale cover of $U$, so we get an action 
of $\pi$ on the underlying set $E[p]\simeq (\dZ/p)^2$. This action preserves 
the group structure, so we have a representation 
\[
  \bar\rho=\bar\rho_{E,p}:\pi_1(U)\to \GL_2(\dF_p) .
\]
Another way of constructing $\bar\rho$ is to use the equivalence between 
\'etale-local $\dF_p$-systems on $U$ and $\dF_p$-representations of $\pi$ to 
realize $\eR e_\ast \dF_p$ as a mod-$p$ representation of $\pi$. 

We could consider the universal deformation ring $R_{\bar\rho}$ and its 
associated formal spectrum $\fX(\bar\rho)$. Recall that if 
$K\hookrightarrow S^3$ is a hyperbolic knot, the ``knot decomposition group'' 
$D_K=\pi_1(\mathrm{torus})$ is abelian, so 
\[
  \rho_K|_{D_K} \sim \begin{pmatrix} \varepsilon & \ast \\ & \varepsilon^{-1} \end{pmatrix} ,
\]
for some character $\varepsilon:G_K\to \dC^\times$. In particular, 
$\varepsilon(I_K)=1$. So to make the analogy between knots and primes more 
precise, we should require that our residual Galois representation 
$\bar\rho:\pi \to \GL_2(\dF_p)$ be \emph{$p$-ordinary} in the sense that 
\[
  \bar\rho|_{D_p} \sim \begin{pmatrix} \varphi & \ast \\ & \psi \end{pmatrix} ,
\]
where $\psi({I_p})=1$. This is exactly ``$I_p$-ordinary'' as defined above. So 
there is a $p$-ordinary representation 
$\rho^\circ:\pi \to \GL_2(R_{\bar\rho}^\circ)$ that is universal for 
$p$-ordinary deformations, 
i.e.~$\formalspectrum(R_{\bar\rho}^\circ)$ represents the 
functor
\[
  \fX^\circ(\bar\rho)(A) = \{\rho\in \fX(\bar\rho)(A):\rho\text{ is $p$-ordinary}\} .
\]





\section{Deformations of hyperbolic structures and Hida theory}

We put together all the machinery we've developed to see an analogy between the 
space of hyperbolic structures on a 3-manifold and the formal spectrum of 
Hida's big Hecke algebras. 


\subsection{Deformation of hyperbolic structures}

Let $K$ be a hyperbolic knot, $M=S^3\smallsetminus K$ the knot complement. Put 
$\Gamma=\pi_1(M)$. Let $T_K=T(\Gamma)$ be the space of injective 
homomorphisms $\Gamma\to \isometry(\dH^3)$ with discrete image; this is the 
space of hyperbolic structures on $M$. There should be a covering map 
$T_K \to X_K$, which is ``almost'' an isomorphism. The analogue is 
$X_0(n)$ being a moduli space of elliptic curves. It will be harder to justify 
Hida's Hecke algebra $\hida$ (better, $\fT=\formalspectrum(\hida)$) as 
being the ``right'' analogue of $T_K$, but the analogue of 
$T_K \to X_K$ should be $\fT(\bar\rho) \to \fX^\circ(\bar\rho)$, 
and ``almost isomorphism'' should correspond to an ``$R=T$'' theorem. 


\subsection{Hida theory}

Roughly, the idea of Hida theory is to create universal $p$-adic families of 
cuspidal automorphic representations of $\GL(2)$. There is a generalization 
(the eigenvariety of a reductive group satisfying the Harish-Chandra condition) 
but we will not discuss that here. 

Let's start with a Shimura datum $(G,X)$. For simplicity, we'll assume the 
associated Shimura variety $\shimura(G)$ is also defined over $\dQ$. Fix a 
prime $p$, and choose a compact open subgroup (the \emph{tame level}) 
$N\subset G(\dA^p)$. For a representation $\rho$ of $G$, define 
\[
  \h^\bullet_\cusp(\shimura_N(G),\sV_\rho(\dZ_p)) = \varinjlim_L \h_\cusp^\bullet(\shimura_{N L}(G),\sV_\rho(\dZ_p)) ,
\]
where $L$ ranges over all open compact subgroups of $G(\dQ_p)$. Let 
$\hida_{N,\rho}(\dZ_p)$ be the image of the action of 
\[
  \varprojlim_L \hecke(N L,\dZ_p)
\]
on $\h^\bullet_\cusp(\shimura_N(G),\sV_\rho)$. 


\subsection{The analogy}

\noindent is even this Selmer group is not the one considered by Bloch and Kato, but
it is the one that is more pertinent when studying deformation rings. In the
case of $\bar\rho$ even, in some particular cases, deformations of $\bar\rho$ to characteristic $0$ were produced





\bibliographystyle{alpha}
\bibliography{tidbit-sources}

\end{document}
