\documentclass[11pt]{article}

\usepackage{amsmath,amssymb,extarrows,mathrsfs}
\usepackage[nottoc]{tocbibind}
\usepackage[colorlinks,linkcolor=blue]{hyperref}
\usepackage[paperwidth=6.75in,paperheight=10in,textwidth=5.25in,textheight=8.5in]{geometry}
\usepackage[all]{xy}
\DeclareMathOperator{\annihilator}{ann}
\DeclareMathOperator{\automorphism}{Aut}
\DeclareMathOperator{\automorphic}{\mathcal{A}}
\DeclareMathOperator{\character}{X}
\DeclareMathOperator{\extension}{Ext}
\DeclareMathOperator{\formalspectrum}{Spf}
\DeclareMathOperator{\galois}{Gal}
\DeclareMathOperator{\GL}{GL}
\DeclareMathOperator{\GSp}{GSp}
\DeclareMathOperator{\h}{H}
\DeclareMathOperator{\lie}{Lie}
\DeclareMathOperator{\norm}{N}
\DeclareMathOperator{\PSL}{PSL}
\DeclareMathOperator{\representation}{Rep}
\DeclareMathOperator{\restrict}{R}
\DeclareMathOperator{\shimura}{Sh}
\DeclareMathOperator{\SL}{SL}
\DeclareMathOperator{\SO}{SO}
\DeclareMathOperator{\spectrum}{Spec}
\DeclareMathOperator{\symmetric}{sym}
\DeclareMathOperator{\trace}{tr}
\newcommand{\cA}{\mathcal{A}}
\newcommand{\cX}{\mathcal{X}}
\newcommand{\dA}{\mathbf{A}}
\newcommand{\dC}{\mathbf{C}}
\newcommand{\dF}{\mathbf{F}}
\newcommand{\dG}{\mathbf{G}}
\newcommand{\dH}{\mathbf{H}}
\newcommand{\dR}{\mathbf{R}}
\newcommand{\dQ}{\mathbf{Q}}
\newcommand{\dS}{\mathbf{S}}
\newcommand{\dZ}{\mathbf{Z}}
\newcommand{\eR}{\mathsf{R}}
\newcommand{\fg}{\mathfrak{g}}
\newcommand{\fH}{\mathfrak{H}}
\newcommand{\fk}{\mathfrak{k}}
\newcommand{\sF}{\mathscr{F}}
\newcommand{\sV}{\mathscr{V}}
\newcommand{\dd}{\mathrm{d}}
\newcommand{\analytic}{\mathrm{an}}
\newcommand{\arithfrob}{\mathrm{fr}}
\newcommand{\cusp}{\mathrm{cusp}}
\newcommand{\discrete}{\mathrm{disc}}
\newcommand{\etale}{\textnormal{\'et}}
\newcommand{\finite}{\mathrm{f}}
\newcommand{\hecke}{\mathcal{H}}
\newcommand{\iso}{\xlongrightarrow\sim}
\newcommand{\mult}{\mathrm{m}}

\title{Automorphic representations and deformation theory in arithmetic geometry}
\author{Daniel Miller}
\date{July 18, 2014}

\begin{document}
\maketitle
\tableofcontents





\section{The analogy between knots and primes}

We start by recalling the analogy described in \cite[ch.3-4]{m12}. 
\begin{center}
\begin{tabular}{c|c}
Topology & Arithmetic \\ \hline
3-manifold $M$ 
  & $X=\spectrum(O_F)\smallsetminus S$ for a number field $F$ \\
$\extension^\bullet(\sF,\dZ) = \h_c^{3-\bullet}(M,\sF)^\vee$
  & $\extension^\bullet(\sF,\dG_\mult) = \h_c^{3-\bullet}(X,\sF)^\vee$ \\
$S^1=K(\dZ,1)$ 
  & $\spectrum(\dF_q) = K(\widehat\dZ,1)$ \\
knot $S^1\hookrightarrow M$ 
  & prime $\spectrum(\dF_q)\hookrightarrow X$ \\
tubular neighborhood $V_K\hookrightarrow M$ 
  & $\spectrum(O_{F,v})\hookrightarrow X$ \\
boundary $\partial V_K\hookrightarrow M$ 
  & $\spectrum(F_v) \hookrightarrow X$ \\
peripheral group $\pi_1(\partial V_K)$
  & decomposition $D_v=\pi_1(F_v)\to \pi_1(X)$ \\
knot group $\pi_1(M\smallsetminus K)$
  & $\pi_1(X\smallsetminus \{v\})=G_{F,S\cup\{v\}}$ \\
$\dZ^2=\pi_1(\partial V_K) \to \pi_1(M)$ 
  & $D_v \to \pi_1(X\smallsetminus \{v\})$. 
\end{tabular}
\end{center}
Here and in the remainder of this note, if $A$ is a commutative ring we write 
$\pi_1(A)$ for the \'etale fundamental group $\pi_1^\etale(\spectrum A)$ with 
respect to an implicit basepoint. 





\section{Automorphic representations}


\subsection{Adeles}

Let $F$ be a number field. If 
$v$ is a place of $F$, write $F_v$ for the completion of $F$ at $v$, and $O_v$ 
for the ring of integers of $F_v$. Write $\dA=\dA_F$ for the ring of adeles of 
$F$; the most important fact about $\dA$ is that it is a locally compact 
$F$-algebra. It can be defined in many ways: 
\begin{itemize}
  \item The topological direct limit $\varinjlim_S\dA(S)$, where $S$ ranges 
    over all finite sets of valuations of $F$, and 
    \[
      \dA(S) = \prod_{v\in S} F_v \times \prod_{v\notin S} O_v .
    \]
  \item The topological tensor product $(\dR\times\widehat\dZ)\otimes F$.
  \item A restricted direct product $\prod_v' (F_v,O_v)$, consisting of those 
    tuples $(a_v)\in \prod_v F_v$ for which $a_v\in O_v$ for almost all $v$.
  \item Via \cite{gs66}, an initial object in the category of locally compact 
    $F$-algebras with no proper open ideals, and with the intersection of all 
    closed maximal ideals being $0$. 
\end{itemize}
Write $\dA_\finite=\widehat\dZ\otimes F$ for the ring of \emph{finite adeles}. 
It is totally disconnected. 

In \cite{c12}, it is shown that there is a unique product-preserving functor 
$(-)(\dA)$ from affine schemes of finite type over $F$ to topological spaces, 
compatible with closed embeddings, for which $(\spectrum F[t])(\dA)=\dA$ with 
its usual topology. In particular, if $G$ is an algebraic group over $F$, the 
abstract group $G(\dA)$ carries the structure of a locally compact topological 
group. As such, it has a unique (up to scalar) Haar measure $\dd g$.  


\subsection{Hecke algebras}

A good reference for this section is \cite{f79}. 

For the remainder, $G$ is a connected reductive group over $F$. Let 
$\fg=\lie G$. There exists 
an open subset of $U\subset \spectrum(O_F)$ and a ``spreading out'' of $G$ to 
a reductive group scheme on $U$. Up to finitely many places, this spreading out 
is well-defined. In particular, for almost all finite places $v$, the group 
$G(O_v)$ is well-defined for almost all $v$. It is a maximal (open) compact 
subgroup of $G(F_v)$. We normalize Haar measures on $G(F_v)$ so that 
$G(O_v)$ has volume $1$, and choose the Haar measure on $G(\dA)$ to be the 
product of measure on each $G(F_v)$. 

Let $v$ be a finite place. Write $\hecke_v=\hecke_v(G)$ for the \emph{Hecke 
algebra} consisting of continuous, locally constant, compactly supported 
functions $G(F_v) \to \dQ$. Multiplication is convolution: 
\[
  (f\star g)(x) = \int_{G(F_v)} f(g) g(y^{-1} x)\, \dd y .
\]
Even though this is written as an integral, it is a finite sum over double 
cosets of open compact subgroups of $G(F_v)$, so no analysis is involved. For 
almost all $v$, the algebra $\hecke_v$ comes with a canonical idempotent, 
$e_v=\chi_{G(O_v)}$. Write $\hecke_\finite$ for the restricted tensor product 
(in the sense of \cite[\S 2]{f79}) of the $\hecke_v$ with respect to the 
$e_v$. It is the direct limit $\varinjlim_S \hecke(S)$, where 
$\hecke(S)=\bigotimes_{v\in S} \hecke_v$, and for 
$T\supset S$, the injection $\hecke(S)\to \hecke(T)$ is induced by the 
$e_v$ for $v\in T\smallsetminus S$. We will also think of $\hecke_\finite$ as 
the algebra of locally constant, compactly supported functions $f$ on 
$G(\dA_\finite)$. 

The ring $F_\infty=F\otimes\dR$ is a finite product of copies of $\dR$ and 
$\dC$, so  $G(F_\infty)$ is naturally a Lie group. Fix a maximal compact 
subgroup $K_\infty\subset G(F_\infty)$. The Hecke algebra 
$\hecke_\infty=\hecke_\infty(G)$ is the convolution algebra of $K_\infty$-finite 
distributions on $G(F_\infty)$ with support in $K_\infty$. There is an 
isomorphism 
\[
  U(\fg_\dC)\otimes_{U(\fk_\dC)} M(K_\infty), \iso \hecke_\infty \qquad D\otimes \mu\mapsto D\star \mu ,
\]
where $\fk=\lie(K_\infty)$ and $M(K_\infty)$ is the algebra of measures on 
$K_\infty$. 

The \emph{global Hecke algebra} of $G$ is 
$\hecke=\hecke(G)=\hecke_\infty\otimes\hecke_\finite$. We will be interested in 
special classes of representations of $\hecke$. 

For a sufficiently large set $S$ of finite places, it makes sense to define 
$e_S\in \hecke_\finite$ to be the characteristic function of 
$\prod_{v\notin S} G(O_v)$. 

An \emph{admissible} representation of $\hecke_\finite$ is an 
$\hecke_\finite$-module $V$ such that for each $v\in V$, there is a finite set 
$S$ of places for which $e_S\cdot v = v$. 


\subsection{Automorphic representations}

A good reference for this section is \cite{bj79}. 

Let $F$, $G$, \ldots be as above. Let $Z$ be the center of $G$, and choose a 
character $\omega:Z(F)\backslash Z(\dA)\to \dC^\times$. Write 
$L^2(G,\omega)$ for the space of measurable functions 
$f:G(F)\backslash G(\dA)\to \dC$ such that 
\begin{align*}
  f(z x) &= \omega(z) f(x) && z\in Z(\dA) \\
  \|f\|^2 &= \int_{G(F)Z(\dA)\backslash G(\dA)} |f(x)|^2\, \dd x < \infty .
\end{align*}
The space $L^2(G,\omega)$ is a representation of $G(\dA)$ in the obvious 
way. Write $L_\discrete^2(G,\omega)$ for the closed subspace generated by all 
irreducible closed subrepresentations. Let 
$\cA(G,\omega)\subset L_\discrete^2(G,\omega)$ be the space of $K$-finite 
vectors. 

[also need $Z(\fg)$-finiteness]

Then $\cA(G,\omega)$ is naturally a $\hecke$-module, and as such, 
decomposes as a countable direct sum of irreducible representations with finite 
multiplicities: 
\begin{equation}\label{eq:aut-decomp}
  \cA(G,\omega) = \bigoplus_\pi m(\pi) \pi .
\end{equation}
We call the irreducible admissible representations of $\hecke$ appearing in 
\eqref{eq:aut-decomp} \emph{automorphic representations} of $G$. By 
\cite[th.4]{f79}, each automorphic representation $\pi$ decomposes as a 
restricted tensor product $\bigotimes \pi_v$ of irreducible admissible 
representations of the $\hecke_v$. 


\subsection{Hecke eigensystems and \texorpdfstring{$L$}{L}-functions}

Let $\pi$ be an automorphic representation of $G$ and choose a nonzero vector 
$u$ in $\pi$. For almost all places $v$, the idempotent $e_v=\chi_{G(O_v)}$ in 
$\hecke_v$ fixes $u$ (in this case, we say that $\pi$ is \emph{unramified} at 
$v$). In particular, the action of $\hecke_v$ on $\pi$ factors 
through that of 
\[
  \hecke_v(O_v) = e_v \hecke_v e_v = C_c^\infty(G(O_v)\backslash G(F_v)/G(O_v)) .
\]
Let $S$ be a set of places outside which $e_v$ fixes $u$. Let 
$\hecke(S)=\bigotimes_{v\notin S} \hecke_v(O_v)$. Then $\pi$ is an irreducible 
admissible module over $\hecke(S)\otimes \bigotimes_{v\in S} \hecke_v$. Since 
$\hecke(S)$ is central in this algebra, it must act via a character 
$\chi:\hecke(S)\to \dC$. The system of homomorphisms 
$\{\chi_v:\hecke_v(O_v) \to \dC:v\notin S\}$ is called a \emph{Hecke 
eigensystem}. 

In the case $G=\GL(n)$, Hecke eigensystems have a particularly easy 
description. A character $\chi:\hecke_v(O_v)\to \dC$ is uniquely determined by 
a semisimple conjugacy class $\sigma_v(\chi)\in \GL(n,\dC)$. If 
$\pi=\bigotimes_v \pi_v$ is an automorphic representation of $\GL(n)$, put 
$\sigma_v(\pi) = \sigma(\chi_{\pi_v})$ and (for finite $v$): 
\[
  L_v(s,\pi) = \det\left(1-\norm(v)\cdot \sigma_v(\pi)^{-s}\right)^{-1} .
\]
For $S$ sufficiently large, we can define the \emph{partial $L$-function} of 
$\pi$ as 
\[
  L_S(s,\pi) = \prod_{v\notin S} L_v(s,\pi) .
\]
This has the expected properties including analytic continuation, a functional 
equation\ldots. In the case $G=\GL(n)$, an automorphic representation $\pi$ is 
determined by $L(s,\pi)$. 





\section{Shimura varieties}

For the rest of this note, the reader should keep in mind the example 
$F=\dQ$, $G=\GL(2)$. Many of the definitions work in greater generality, 
but technicalities (which we wish to avoid) multiply endlessly. 


\subsection{Locally symmetric spaces}\label{sec:local-symm}

Classically, one studies representations of a real semisimple group $G$ by 
fixing a maximal compact $K$, setting $X=G/K$, and studying the regular 
representation of $G$ on $C^\infty(\Gamma\backslash X)$ for $\Gamma\subset G$ 
a discrete group. Big examples are the (affine) modular curves $Y_0(n)$, 
coming from $\Gamma_0(n)\subset \SL(2,\dR)$. We will carry out this 
construction adelically. 

Let $G$ be a connected reductive group over $\dQ$. Put 
$X=Z_\infty \backslash G(F_\infty) / K_\infty$. Let 
$K\subset G(\dA_\finite)$ be an open compact subgroup. We define 
\[
  \shimura_K(G) = G(\dQ)\backslash (X\times G(\dA_\finite)) / K .
\]
A priori, this is only a topological space, but the quotient map 
$X\times G(\dA_\finite)/K \to \shimura_K(G)$ gives $\shimura_K(G)$ the 
structure of a Riemannian manifold. Let $(V,\rho)$ be a representation of $G$. 
There is an induced local system $\sV_\rho$ on $\shimura_K(G)$, whose (global) 
sections are locally constant sections $s:X\times G(\dA_\finite)/K \to V_\rho$ 
such that $s(\gamma x) = \rho(\gamma) s(x)$ for $\gamma\in G(\dQ)$. Put 
\[
  \h^\bullet(K,\rho) = \h_c^\bullet\left(\shimura_K(G),\sV_\rho\right) .
\]
In fact, the vector space $\h(K,\rho)$ is an $\hecke_\finite$-module! We 
construct an action of a coset $gC$ for $C\subset K$. Let 
$C'=C\cap g^{-1} K$, and let $g C$ act via the correspondence 
\[\xymatrix{
  \shimura_K(G) 
    & \ar@{->>}[l] \shimura_{C'}(G) \ar[r]^-{g\cdot} 
    & \shimura_K(G) .
}\]
As a representation of $\hecke_\finite$, $\h(K,\rho)$ decomposes as a direct 
sum 
\[
  \h^\bullet(K,\rho) = \bigoplus_{\pi\in \automorphic(G)} \h(K,\rho)[\pi]
\]
where $(-)[\pi]$ is the span of all $\hecke_\finite$-submodules of type $\pi$. 
See \cite{s09} for details. 

To compute the cohomology with compact support, it is useful to have a 
compactification of $\shimura_K(G)$ that respects the action of 
$G(\dA_\finite)$. For general $G$, the theory of compactifications of locally 
symmetric spaces is very rich and complicated. Fortunately for us, when 
$G=\GL(2)$ suffice it to say that there is a reasonably canonical 
compactification $\overline{\shimura_K(\GL_2)}$; we define \emph{cuspidal 
cohomology} by 
\[
  \h_\cusp^\bullet(K,\rho) = \ker\left(\h^\bullet(K,\rho) \to \h_c^\bullet(\partial \shimura_K(G),\sV_\rho)\right) .
\]
There is also a spectral decomposition of automorphic cohomology. We call
automorphic representations appearing in the spectral decomposition of a vector 
bundle on some locally symmetric space \emph{cohomological}. 


\subsection{Canonical models}\label{sec:can-model}

A good reference for this is \cite{m98}. In \autoref{sec:local-symm} we 
constructed $\shimura_K(G)$ as a Riemannian manifold. It turns out that there 
is a good definition of ``canonical model'' for Shimura varieties over number 
fields, and in that sense, all Shimura varieties $\shimura_K(G)$ have a 
canonical model over a number field called the \emph{reflex field}. (We have 
been intentionally avoiding use of the Shimura datum necessary to define 
$\shimura_K(G)$ in full generality -- the reflex field depends on this.) 

Let $E$ be the reflex field. Not only does the projective system 
$\shimura(G)=\varprojlim \shimura_K(G)$ descend to the reflex field, but the 
action of $G(\dA_\finite)$ via correspondences descends in a canonical way. 
Moreover, in \cite{harris85} it is shown that our local systems $\sV_\rho$ 
descend in a functorial way to $G(\dA_\finite)$-equivariant local systems 
on $\shimura(G)$. 

The main reason we care about this is that if $\shimura_K(G)$ is smooth and 
$E=\dQ$, then general theorems about \'etale cohomology tell us that 
\[
  \h_{\mathrm{sing},c}^\bullet(\shimura_K(G),\sV_\rho) = \h_{\etale,c}^\bullet(\shimura_K(G)_{\overline\dQ},\sV_\rho(\overline{\dQ_\ell})) 
\]
after choice of an isomorphism $\dC\simeq \overline{\dQ_\ell}$. The choice of 
an \emph{arithmetic compactification} of $\shimura_K(G)$ lets us extend the 
action of $\hecke_\finite$ to the \'etale cohomology of $\shimura_K(G)$. 





\section{Modular representations}

Use the generalized Eichler-Shimura isomorphism given in 
\cite{harder-1987}. Via the introduction of 
\url{http://www-users.math.umn.edu/~kwlan/articles/iccm-2010.pdf}, there is an isomorphism 
\[
  \h_\cusp^1(\shimura_{K_0(n)}(\GL_2),\symmetric^{k-2}) = S_k(\Gamma_0(n))\oplus \overline{S_k(\Gamma_0(n))} .
\]
So the ``weight'' in $S_k$ comes from the weight parameterizing the irreducible 
representation of $\GL(2)$. To derive this sort of thing from Harder's generalized 
Eichler-Shimura, I need to check that the $\infty$-type of a weight-$k$ modular 
is $\symmetric^{k-2}$. 


\subsection{Modular curves}

In this section, algebraic groups, adeles, etc.~will be taken over $\dQ$. 
Let $n\geqslant 1$ be an integer. We define the following congruence subgroup 
of $\SL_2(\dZ)$:
\[
  \Gamma_0(n) = \left\{\begin{pmatrix}a & b \\ c & d \end{pmatrix}\in \SL(2,\dZ) : c\equiv 0\pmod n\right\} .
\]
Let $K_0(n)$ be the induced subgroup of $\GL_2(\dA_\finite)$. Write $Y_0(n)$ 
for the induced locally symmetric space: 
\[
  Y_0(n) = \shimura_{K_0(n)}(\GL_2) = \GL_2(\dQ)\backslash \GL_2(\dA) / Z_\infty K_\infty K_0(n) .
\]
Here $Z_\infty=Z(\GL_d(\dR))$ and $K_\infty=\SO(2)\subset \GL_2(\dR)$. Put 
$X=\GL_2(\dR)^+/Z_\infty K_\infty$. Note that $\GL_2(\dR)^+/Z = \SL_2(\dR)$, so 
\[
  \GL_2(\dR)^+ / Z_\infty K_\infty \iso \fH = \{z\in \dC:\Im z>0\} 
\]
via $\gamma\mapsto \gamma\cdot i$. The strong approximation theorem tells us 
that 
\[
  \GL_2(\dA) = \GL_2(\dQ) \GL_2(\dR) K_0(n),
\]
so the quotient $\shimura_{K_0(n)}(\GL_2)$ is just 
\[
  (\GL_2(\dQ)\cap K_0(n))\backslash \fH = \Gamma_0(n) \backslash \fH = Y_0(n).
\]
This will be a singular complex-analytic orbifold. There are two ways of 
realizing $Y_0(n)$ and its compactication $X_0(n)$ as curves over $\dQ$: 
\begin{enumerate}
  \item Interpret $Y_0(n)$ as a moduli space for elliptic curves with level 
    structure. This moduli problem makes sense over $\dQ$, so $Y_0(n)$ descends 
    in a canonical way to $\dQ$. 
  \item Use the general theory of canonical models of Shimura varieties. 
\end{enumerate}
The former approach generalizes to a special class of Shimura varieties 
consisting of those of \emph{PEL type} (standing for 
\textbf{P}olarization, \textbf{E}ndomorphism, and \textbf{L}evel structure). 
The theory of PEL-type Shimura varieties is interesting and useful, but we 
won't go into it here. 

Instead, note that the space 
\[
  \fH^\pm = Z_\infty \backslash \GL_2(\dR) / \SO_2(\dR) 
\]
can be interpreted as the set of $\GL_2(\dR)$-conjugacy classes of 
homomorphisms $\dS=\restrict_{\dC/\dR}\dG_\mult\to \GL(2)_\dR$ containing 
\[
  h:(x,y)\mapsto \begin{pmatrix} x & y \\ -y & x \end{pmatrix} .
\]
This is all defined over $\dQ$, so the theory of canonical models discussed 
in \autoref{sec:can-model} tells us that if 
$K_\finite\subset\GL_2(\dA_\finite)$ is any open compact subgroup, the quotient 
\[
  \shimura_{K_\finite}(\GL_2) = \GL_2(\dQ)\backslash (\fH^\pm\times \GL_2(\dA_\finite))/K_\finite
\]
descends to a uniquely determined curve over $\dQ$. 


\subsection{The Eichler-Shimura construction}

Let $n\geqslant 3$. 
As above, write $Y_0(n)$ for the Shimura variety $\shimura_{K_0(n))}(\GL_2)$, 
and write $X_0(n)$ for its arithmetic compactification. We define the space of 
\emph{cusp forms} of weight 2 and level $\Gamma_0(n)$ to be 
\[
  S_2(\Gamma_0(n)) = \h_\cusp^1(X_0(n),\dC) .
\]
As a $\hecke_\finite$-module, there is a decomposition 
\[
  S_2(\Gamma_0(n),\dC) = \bigoplus_{\pi\in \automorphic(\GL_2)} m(\Gamma_0(n),\pi) \pi ,
\]
where in our case $m(\Gamma_0(n),\pi) \leqslant 1$. Representations $\pi$ with 
$m(\Gamma_0(n),\pi)=1$ are of the form $\hecke_\finite\star f$ for $f$ a 
normalized cupsidal eigenform. Write $\pi_f$ for the corresponding cuspidal 
automorphic representation of $\GL(2)$. 

Recall that our modular curves are defined over $\dQ$. In particular, we could 
consider the cohomology spaces 
\[
  \h_{\etale,\cusp}^1(X_0(n)_{\overline\dQ},\overline{\dQ_\ell}) \simeq S_2(\Gamma_0(n),\dC) .
\]
These have commuting actions of 
$G_{\dQ,\ell n}=\pi_1(\dZ[\frac{1}{\ell n}])$ and 
$\hecke_\finite$. There is a decomposition of 
$\hecke_\finite\times G_{\dQ,\ell n}$-modules 
\[
  \h_{\etale,\cusp}^1(X_0(n)_{\overline\dQ},\overline{\dQ_\ell}) = \bigoplus_f \pi_f\otimes \rho_{f,\ell} ,
\]
and the \emph{Eichler-Shimura relation} basically tells us that the Hecke and 
Frobenius parameters for $\pi_f$ and $\rho_{f,\ell}$ match up. That is, for all 
$p\nmid \ell n$, we have 
$\rho_{f,\ell}(\arithfrob_p) = \sigma_p(\pi_f)$, or equivalently 
$L_p(\rho_{f,\ell},s) = L_p(\pi_f,s)$. 

It is known that $\rho_{f,\ell}:G_{\dQ,\ell n} \to \GL_2(\overline{\dQ_\ell})$ 
factors through $\GL_2(K_{f,\lambda})$, where 
$K_f = \dQ(a_p(f):p\text{ prime})$ is a number field and $\lambda$ is a place 
of $K_f$ dividing $\ell$. An elementary argument shows that we can conjugate 
the image of $\rho_{f,\ell}$ to lie in $\GL_2(O_{f,\lambda})$, where 
$O_f=O_{K_f}$. In particular, we can reduce $\rho_{f,\ell}$ modulo $\lambda$ 
to get a continuous representation 
\[
  \bar\rho_{f,\ell}:G_{\dQ,\ell n} \to \GL_2(O_{f,\lambda}/\lambda) = \GL_2(\dF_\lambda) .
\]
We say that a mod-$\ell$ representation of $G_\dQ$ is \emph{modular} (better, 
\emph{automorphic}) if it is of the form $\bar\rho_{f,\ell}$ for some $f$. 
Similarly, we say that an $\ell$-adic representation of $G_\dQ$ is 
\emph{modular} (better, \emph{automorphic}) if it is of the form 
$\rho_{f,\ell}$ for some $\ell$. 

In what follows, we will ignore the modular form $f$ and just write $\pi$ for 
the corresponding cuspidal automorphic representation of $\GL(2)$, keeping in 
mind that for some automorphic representations (those corresponding to Maass 
forms) we still don't know how to construct the associated Galois 
representations. For an automorphic representation $\pi$, write 
$\rho_{\pi,\ell}$ for the corresponding $\ell$-adic representation (assuming 
it exists). 





\section{Deformation theory}


\subsection{Motivation}

First let's consider the 
motivation for studying deformations of Galois representations. If $X$ is a 
nice (that is smooth, projective and geometrically integral) variety over 
$\dQ$, its \'etale cohomology 
$\h_\etale^\bullet(X_{\overline\dQ},\dQ_\ell)$ carries a continuous 
action of $G_\dQ$. The ``right'' way to see this is as follows. Spread out 
$X$ to a smooth proper scheme $\cX$ over an open 
$U= \spectrum(\dZ)\smallsetminus S$. Write $\pi:\cX\to U$ for the structure 
map. Then $\eR^\bullet \pi_\ast \dQ_\ell$ is a local system on 
$U_\etale$. The \'etale version of covering space theory tells us that local 
systems correspond to representations of $\pi_1(U)=G_{\dQ,S}$. 

The motivating example was the representation $\rho_{E,\ell}$ coming from an 
elliptic curve $E\xrightarrow\pi U$ via $\eR^1 \pi_\ast \dQ_\ell$. Langlands' 
conjectural framework tells us that there should exist an automorphic cuspidal 
representation $\pi$ of $\GL(2)$ for which 
$\rho_{E,\ell} \simeq \rho_{\pi,\ell}$. We don't know how to construct $\pi$ 
this directly. However, we \emph{do} know (via Serre's conjecture) that 
$\bar\rho_{E,\ell}$ is automorphic. One of the main applications of 
deformation theory is to prove that the automorphy of $\bar\rho_{E,\ell}$ 
implies that of $\rho_{E,\ell}$. 

More generally, given an $\ell$-adic Galois 
representation $\rho:G_{\dQ,S}\to \GL_n(\dQ_\ell)$ that is suitably nice 
(\emph{geometric}, in the sense of Fontaine-Mazur \cite{fontain-mazur-1995}), 
Langlands' program tells us that we should expect there to be a cuspidal 
automorphic representation $\pi$ of $\GL(n)$ such that 
$\rho\simeq \rho_{\pi,\ell}$ (assuming we knew how to construct $\rho_\pi$ in 
general). Proving that $\rho$ is automorphic is very hard! However, we have a 
much better chance (in theory and in practice) of showing that 
$\bar\rho:G_{\dQ,S} \to \GL_n(\dF_\ell)$ is automorphic. A theorem to the 
effect that ``$\bar\rho$ automorphic $\Rightarrow\rho$ automorphic'' is known 
as a \emph{automorphy lifting theorem}. In practice, one has to impose many 
technical conditions on $\bar\rho$ and the automorphic representation with 
$\bar\rho_\pi\simeq \bar\rho$, and one uses groups like $\GSp(n)$ instead of 
$\GL(n)$. 


\subsection{Representations of knot groups}

Our exposition here follows that of \cite[ch.13-14]{m12}. Let 
$K\subset S^3$ be a hyperbolic knot, $M=S^3\smallsetminus K$ the knot 
complement, $\pi=\pi_1(M)$ the knot group. The uniformization $\dH^3\to M$ 
induces a representation $\pi \to \automorphism(\dH^3) = \PSL_2(\dC)$ which 
lifts to $\rho:\pi \to \SL_2(\dC)$. Introduce the \emph{representation 
variety} $\representation(\pi,\SL_2)$ of homomorphisms $\pi\to \SL_2(\dC)$. 
There are two ways of describing $\representation(\pi,\SL_2)$. One elementary 
but useful approach is to write 
$\pi=\langle g_1,\dots,g_m|r_1,\dots,r_n\rangle$. The variety 
$\representation(\pi,\SL_2)$ is just the subset of $\SL_2(\dC)^m$ cut out by 
the relations $r_1,\dots,r_n$. A more functorial definition is to require that 
for all $\dC$-algebras $A$, a natural isomorphism 
\[
  \hom_\mathsf{grp}(\pi,\SL_2(A)) \simeq \hom_{\mathsf{sch}/\dC}(\spectrum A,\representation(\pi,\SL_2)) .
\]

The \emph{character variety} of $K$ is the geometric quotient 
\[
  X_K = \representation(\pi,\SL_2) /\!\!\!/ \SL_2 = \spectrum\left(\dC[\representation(\pi,\SL_2)]^{\SL_2(\dC)}\right) ,
\]
via the obvious action of $\SL(2)$ on $\representation(\pi,\SL_2)$ via 
conjugation. The representation $\rho$ is a point in $X_K$, and one is 
interested in the connected component $X_K(\rho)$. 


\subsection{Deformation functors}

The analogous situation in number theory is much more complicated, partly 
because $G_S=\pi_1(\spectrum(\dZ)\smallsetminus S)$ is not a finitely presented 
group -- it's a compact topological group which is (conjecturally) 
\emph{topologically} finitely presented. So instead of looking for 
representations $G_S\to \GL_2(\dC)$, we should look for continuous 
representations $G_S\to \GL_2(A)$, where $A$ is a topological $\dZ_p$-algebra. 

Here, the functorial approach to defining representation schemes is the 
correct one. If $\pi$ is an arbitrary profinite group, there is a 
formal scheme $\widehat\representation(\pi,\GL_2)$, satisfying 
\[
  \hom\left(\formalspectrum(A),\widehat\representation(\pi,\GL_2)\right) = \hom_\mathrm{cts}(\pi,\GL_2 A) ,
\]
for any local pro-artinian $\dZ_p$-algebra $A$ with residue field $\dF_p$. The 
problem is, $\widehat\representation(\pi,\GL_2)$ is really horrible as a space 
-- it generally has infinitely many different connected components. So before 
we do anything else, let's restrict to the connected component of $\bar\rho$, 
where $\bar\rho:G_S\to \GL_2(\dF_p)$ has been fixed beforehand. The component 
$X^\square(\bar\rho)$ represents continuous homomorphisms 
$G_S\to \GL_2(A)$ that reduce to $\bar\rho$ modulo $p$. 





\bibliographystyle{alpha}
\bibliography{tidbit-sources}

\end{document}
