\documentclass{article}

\usepackage{amsmath,amssymb,fullpage,mathpazo}
\usepackage[all]{xy}
\DeclareMathOperator{\gal}{Gal}
\DeclareMathOperator{\h}{H}
\DeclareMathOperator{\genlin}{GL}
\DeclareMathOperator{\lie}{Lie}
\newcommand{\ab}{\textnormal{ab}}
\newcommand{\dA}{\mathbf{A}}
\newcommand{\dC}{\mathbf{C}}
\newcommand{\dG}{\mathbf{G}}
\newcommand{\dQ}{\mathbf{Q}}
\newcommand{\dZ}{\mathbf{Z}}

\title{On computable elements of $\gal(\overline\dQ/\dQ)$}
\author{Daniel Miller}

\begin{document}
\maketitle





Let $G_\dQ=\gal(\overline\dQ/\dQ)$. Our goal is to prove that ``computable 
elements'' of $G_\dQ$ are dense. 

First note that for each place $v$ of $\dQ$, there is a canonical algebraically 
closed field $\Omega_v$, which has valuation extending $v$, and which is 
spherically complete. For example, $\Omega_\infty=\dC$, and 
$\Omega_p$ is the field of ``generalized Hahn series'' considered by Poonen. 
There are canonical embeddings $\dQ_v\hookrightarrow \Omega_v$, so we can 
consider $\overline\dQ\subset \overline\dQ_v\subset \Omega_v$ for any $v$, 
giving a canonical system of embeddings embedding 
$G_{\dQ_v}\hookrightarrow G_{\dQ,v}$, where 
\[
  G_{\dQ,v} = \gal((\overline\dQ\cap \Omega_v)/\dQ) .
\]
As topological groups, the $G_{\dQ,v}$ are all isomorphic, but the isomorphisms 
do not seem to be computable in any reasonable sense. 

Each $G_{\dQ,v}$ should have a distinguished ``Frobenius element.'' 

By writing $\overline\dQ$ as a ``computable inductive limit of finite 
extensions of $\dQ$,'' the hope would be to write elements of 
$G_\dQ$ as ``computable compatible sequences of automorphisms.'' 









\section{Conjectures in explicit class field theory}

Let $k$ be a local or global field. Class field theory (CFT) gives an explicit 
description of the abelianized absolute Galois group $G_k^\ab$. This is 
\emph{not} explicit class field theory. Explicit class field theory (ECFT) 
attempts to construct the field $k^\ab=(k^s)^{G_k^\ab}$. For example, if 
$k=\dQ$, let $\chi:G_\dQ \to \widehat\dZ^\times$ be the cyclotomic character 
coming from torsion points on $\dG_m$. The map $\chi$ induces an isomorphism 
$G_\dQ^\ab \xrightarrow\sim \widehat\dZ^\times$. In other words, 
$\dQ^\ab = \dQ(\operatorname{Tors}\dG_m)$. 

Similarly, let $k$ be a CM field of degree $2 g$ over $\dQ$, and let $A$ be a 
$2 g$-dimensional abelian variety over $k$ with 
$\operatorname{End}_k^\circ(A)=k$. Let $V(A)$ be the ``total rational Tate 
module of $k$.'' This is a free $\dA_k^f$-module of rank one, the action of 
$G_k$ commutes with that of $\dA_k^f$, so we get a homomorphism 
$G_k \to \genlin_1(\dA_k^f)$. 

The problem with the functor 
$\lie:\{\text{algebraic groups}\} \to \{\text{Lie algebras}\}$ is that it only 
retains information about the connected component. We would also like to retain 
some information about the (nilpotent) completion of the connected components 
of an algebraic group. $G^0=G^\circ$

There should be a functor 
\[
  \{\text{pro-algebraic groups over $k$}\} \to 
\]





\end{document}
