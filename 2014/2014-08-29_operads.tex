\documentclass{article}

\usepackage{amsmath,amssymb,amsthm,fullpage}
\DeclareMathOperator{\complexes}{\mathsf{Com}}
\DeclareMathOperator{\cone}{cone}
\DeclareMathOperator{\holim}{holim}
\newcommand{\cC}{\mathcal{C}}
\newcommand{\dN}{\mathbf{N}}
\newcommand{\dZ}{\mathbf{Z}}
\newcommand{\tuples}{\mathsf{tup}}
\newtheorem{definition}{Definition}

\title{Operads for number theory}
\author{Daniel Miller}

\begin{document}
\maketitle






Our main source is \cite{hinich-schechtman-1987}. 
Let $\cC$ be a strict monoidal category. Our main example is the category 
$\complexes(k)$ of chain complexes over a (commutative, unital) ring $k$. The 
main objects of interest will be collections $O=\{O_n:n\geqslant 0\}$ in $\cC$. 
To handle these, we introduce some formalism. Note that $\dN=\{1,2,\ldots\}$. 

Let $\tuples=\coprod_{d\geqslant 0} \tuples_d$ be the set of \emph{tuples}. 
For $d\geqslant 0$, $\tuples_d$ is the set of pairs $(m,S)$, where 
$S$ is a finite subset of $\dN^d$ and $m:S\to \dZ_{\geqslant 0}$. We write 
$\dim(m)=d$ if $m\in \tuples_d$. Define 
\begin{align*}
  |\cdot|:\tuples &\to \dZ_{\geqslant 0} && |(m,S)| = \sum_{s\in S} m_s \\
  l:\tuples_{d+1} &\to \tuples_d && l(m,S) = \left((s_1,\dots,s_d)\mapsto \sup_{(s_1,\dots,s_{d+1})\in S}s_{d+1}, \{(s_d,\dots,s\right)
\end{align*}

We 
will write $m = (m_1,\dots,m_n)$ for an $n$-tuple of integers; we put 
$l(m) = n$ and $|m| = \sum m_i$. 
In general, we say a tuple $m=(m_{i_1,\dots,i_d})$ has \emph{dimension $d$}, 
write $\dim(m)=d$, $|m|=\sum m_{i_1,\dots,i_d}$ and 

\begin{definition}
An \emph{operad} $O$ in $\cC$ consists of the following data: 
\begin{itemize}
  \item A family of objects $\{O_n:n\geqslant 0\}$ in $\cC$. For 
    $m = (m_1,\dots,m_n)$, we put 
    $O_m = O_{m_1}\otimes \cdots \otimes O_{m_n}$. 
  \item Multiplication maps 
    $\gamma_m:O_{\ell(m)}\otimes O_m \to O_{|m|}$. 
  \item A unit $e:1\to O_0$. 
\end{itemize}
If $m$ is a two-dimensional tuple, put 
$O_m = \bigotimes_{i_1,i_2} O_{m_{i_1},m_{i_2}}$. 

This data is required to satisfy the following properties: 
\end{definition}





\section{Application}

Here our inspiration is from \cite{beilinson-2012}. The fundamental idea is 
that a ring like $\dZ_p$, a limit of non-free $\dZ$-modules, should be 
``resolved.'' More precisely, consider 
\[
  \dZ_p^\flat = \holim_n \cone(\dZ \xrightarrow{p^n} \dZ) .
\]
The functor $F\widehat\otimes \dZ_p := \holim \cone$





\bibliographystyle{alpha}
\bibliography{tidbit-sources}

\end{document}
