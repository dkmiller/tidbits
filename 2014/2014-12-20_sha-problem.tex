\documentclass{article}

\usepackage{amsmath,amssymb,amsthm,hyperref,thmtools,tikz-cd}
\usepackage[a5paper]{geometry}
\usepackage[cm]{fullpage}
\DeclareMathOperator{\Adjoint}{Ad}
\DeclareMathOperator{\galois}{Gal}
\DeclareMathOperator{\GL}{GL}
\DeclareMathOperator{\h}{H}
\DeclareMathOperator{\image}{im}
\DeclareMathOperator{\inflate}{inf}
\DeclareMathOperator{\restrict}{res}
\newcommand{\dQ}{\mathbf{Q}}
\newcommand{\dZ}{\mathbf{Z}}
\newcommand{\abelian}{\mathrm{ab}}
\newcommand{\iso}{\xrightarrow\sim}
\newcommand{\monic}{\hookrightarrow}
\newcommand{\unramified}{\mathrm{ur}}
\newtheorem{lemma}{Lemma}
\newtheorem{theorem}{Theorem}
\usepackage[
  hyperref = true,      % links to online documents
  backend  = bibtex,    % use bibtex instead of biber
  sorting  = nyt,       % sorts by (name, year, title)
  style    = alphabetic % citations look like [Har77]
]{biblatex}
\addbibresource{tidbit-sources.bib}
\hypersetup{
  colorlinks = true,
  linkcolor  = blue,
  urlcolor   = cyan
}
\hypersetup{colorlinks=true,linkcolor=green}

% Tate-Shafarevich groups
  \DeclareFontFamily{U}{wncy}{}
  \DeclareFontShape{U}{wncy}{m}{n}{<->wncyr10}{}
  \DeclareSymbolFont{mcy}{U}{wncy}{m}{n}
  \DeclareMathSymbol{\sha}{\mathord}{mcy}{"58} 

\title{A problem of Tate-Shafarevich groups}
\author{Daniel Miller}

\begin{document}
\maketitle





\section{Generalities on group cohomology}

Let $\Gamma$ be a profinite group. A \emph{continuous $\Gamma$-module} (later: 
just a $\Gamma$-module) is a $\Gamma$-module $M$ such that the action 
$\Gamma\times M\to M$ is continuous when $M$ is given the discrete topology. 
One puts $\h^\bullet(\Gamma,-)$ for the derived functors of $\h^0(\Gamma,-)$, 
taken in the category of all continuous $\Gamma$-modules. We will frequently 
use the inflation-restriction exact sequence \cite[1.6.7]{nsw08}: let 
$1\to \Gamma'\to \Gamma\to \Gamma''\to 1$ be a short exact sequence of profinite 
groups, $M$ a $\Gamma$-module. Then the following sequence is exact:
\[
\begin{tikzcd}
  0 \ar[r] 
    & \h^1(\Gamma'',M^{\Gamma'}) \ar[r, "\inflate"]
    & \h^1(\Gamma,M) \ar[r, "\restrict"]
    & \h^1(\Gamma',M) .
\end{tikzcd}
\]

\begin{lemma}
Let $\Gamma'\subset \Gamma$ be a closed subgroup of a profinite group. Then the 
kernel of $\h^1(\Gamma,M)\xrightarrow{\restrict}\h^1(\Gamma',M)$ does not 
depend on the conjugacy class of $\Gamma'$. 
\end{lemma}
\begin{proof}
Let $c:\Gamma\to M$ represent an element of $\h^1(\Gamma,M)$. Elementary 
manipulations show that $c_{\gamma^{-1}} = -\gamma^{-1} c_\gamma$ for all 
$\gamma\in \Gamma$. For $\sigma\in \Gamma'$, we compute 
\begin{align*}
  c_{\gamma\sigma\gamma^{-1}} 
    &= \gamma\sigma c_{\gamma^{-1}} + \gamma c_\sigma + c_\gamma \\
    &= (1-\gamma\sigma\gamma^{-1})c_\gamma + \gamma c_\sigma .
\end{align*}
Thus $c|_{\gamma \Gamma'\gamma^{-1}}$ is equivalent to the cocycle 
\begin{align*}
  \gamma\sigma\gamma^{-1} &\mapsto \gamma(\sigma-1) m \\
    &= (\gamma\sigma\gamma^{-1}-1)\gamma m ,
\end{align*}
which is a coboundary. We have shown that 
\[
  \ker\left(\h^1(\Gamma,M)\to \h^1(\Gamma',M)\right) \subset \ker\left(\h^1(\Gamma,M)\to \h^1(\gamma\Gamma'\gamma^{-1},M)\right) .
\]
To obtain the other inclusion, replace $\Gamma'$ by 
$\gamma \Gamma'\gamma^{-1}$ and $\gamma$ by $\gamma^{-1}$. 
\end{proof}





\section{Galois cohomology of number fields}

Let $k$ be a number field, $v$ a place of $k$. We write 
$\Gamma_v=\galois(\overline{k_v}/k)$ for the decomposition group, and assume 
given a conjugacy class of embeddings $\Gamma_v\monic \Gamma$. Let 
$I_v\subset \Gamma_v$ be the inertia group. If $S$ is a finite set of places, 
we write $\Gamma^S\subset \Gamma$ for the normal subgroup generated by the 
images of $I_v\to \Gamma$ ($v\notin S$), and put $\Gamma_S=\Gamma/\Gamma^S$. 
If $M$ is a $\Gamma_v$-module, put 
\begin{align*}
  \h^1_\unramified(\Gamma_v,M) 
    &= \ker\left(\h^1(\Gamma_v,M)\to \h^1(I_v,M)\right) \\
    &= \image\left(\h^1(\widehat\dZ,M^{I_v})\to \h^1(\Gamma_v,M)\right) .
\end{align*}

\begin{lemma}
Let $M$ be a $\Gamma$-module unramified outside $S$. Then 
\begin{align*}
  \h^1(\Gamma_S,M) 
  &\iso \ker\left(\h^1(\Gamma,M)\to \bigoplus_{v\notin S} \frac{\h^1(\Gamma_v,M)}{\h^1_\unramified(\Gamma_v,M)}\right) \\
  &= \ker\left(\h^1(\Gamma,M)\to \bigoplus_{v\notin S} \h^1(I_v,M)\right) .
\end{align*}
\end{lemma}
\begin{proof}
By the inflation-restriction exact sequence, we know that 
\[
  \h^1(\Gamma_S,M) = \{c\in \h^1(\Gamma,M):c|_{\Gamma^S}=0\} .
\]
Moreover, we know that the map 
$\prod_{v\notin S} I_v^\abelian\to \Gamma^{S,\abelian}$ is surjective. Since 
\begin{align*}
  \h^1(\Gamma^S,M) 
    &= \hom(\Gamma^{S,\abelian},M) \\
    &\monic \prod_{v\notin S} \hom(I_v,M) \\
    &= \prod_{v\notin S} \h^1(I_v,M) ,
\end{align*}
it is clear that $c|_{\Gamma^S}=0$ if and only if $c|_{I_v}=0$ for all 
$v\notin S$. 
\end{proof}

As before, let $M$ be a $\Gamma$-module unramified outside $S$. Define 
\[
  \sha_S^1(M) = \ker\left(\h^1(\Gamma_S,M)\to \bigoplus_{v\in S} \h^1(\Gamma_v,M)\right) .
\]

\begin{theorem}\label{main-result}
If $M$ is unramified outside $S$ and $T\supset S$ is a finite set of places, 
the image of $\sha_S^1(M)$ under the inflation map 
$\h^1(\Gamma_S,M)\monic \h^1(\gamma_T,M)$ contains $\sha_T^1(M)$. 
\end{theorem}
\begin{proof}
We need to show that if $c\in \h^1(\Gamma,M)$, then 
\[
    \left(
        \begin{array}{cc} 
            c|_{\Gamma_v} \in \h^1_\unramified(\Gamma_v,M) & v\notin T \\
            c|_{\Gamma_v}=0 & v\in T
        \end{array}
    \right) \Rightarrow
    \left(
        \begin{array}{cc} 
            c|_{\Gamma_v} \in \h^1_\unramified(\Gamma_v,M) & v\notin S \\
            c|_{\Gamma_v}=0 & v\in S
        \end{array}
    \right) ,
\]
but this is obvious.
\end{proof}

\emph{Note}: elliptic curves $E_{/\dQ}$ with $\sha(E)\ne 0$ (or failures of the 
Grunwald-Wang theorem over number fields having no unramified extensions) seemed to 
be counterexamples to \autoref{main-result}, as in those cases 
$\Gamma_\varnothing=1$ but $\sha^1_S\ne 0$. The problem is, in all such cases the 
module in question is \emph{not} everywhere unramified.





\printbibliography
\end{document}
