\documentclass{article}

\usepackage{
  amsmath,
  amssymb,
  amsthm,
  bm,
  bookmark,
  fullpage,
  mathpazo,
  mathtools,
  mathrsfs,
  microtype,
  thmtools
}
\DeclareMathOperator{\adjoint}{Ad}
\DeclareMathOperator{\End}{End}
\DeclareMathOperator{\galois}{Gal}
\DeclareMathOperator{\GL}{GL}
\DeclareMathOperator{\h}{H}
\DeclareMathOperator{\height}{ht}
\DeclareMathOperator{\induce}{ind}
\DeclareMathOperator{\lie}{Lie}
\DeclareMathOperator{\norm}{N}
\DeclareMathOperator{\orbital}{O}
\DeclareMathOperator{\reciprocity}{rec}
\DeclareMathOperator{\sheaves}{Sh}
\DeclareMathOperator{\shimura}{Sh}
\DeclareMathOperator{\trace}{tr}
\DeclareMathOperator{\twistedorbital}{TO}
\DeclareMathOperator{\witt}{W}
\newcommand{\etale}{\textnormal{\'et}}
\newcommand{\finite}{\mathrm{f}}
\newcommand{\frob}{\mathrm{fr}} % Frobenius
\newcommand{\Ga}{\mathbf{G}_\mathrm{a}}
\newcommand{\Gm}{\mathbf{G}_\mathrm{m}}
\newcommand{\iso}{\xrightarrow\sim}
\newcommand{\transpose}[1]{\prescript{\mathrm{t}}{}{#1}}
\newcommand{\WD}{\mathrm{WD}} % Weil-Deligne group
\newcommand{\cA}{\mathcal{A}}
\newcommand{\cG}{\mathcal{G}}
\newcommand{\cO}{\mathcal{O}}
\newcommand{\dA}{\mathbf{A}}
\newcommand{\dC}{\mathbf{C}}
\newcommand{\dD}{\mathbf{D}}
\newcommand{\dF}{\mathbf{F}}
\newcommand{\dmu}{\bm\mu}
\newcommand{\dN}{\mathbf{N}}
\newcommand{\dQ}{\mathbf{Q}}
\newcommand{\dR}{\mathbf{R}}
\newcommand{\dZ}{\mathbf{Z}}
\newcommand{\eR}{\mathrm{R}}
\newcommand{\fm}{\mathfrak{m}}
\newcommand{\fV}{\mathfrak{V}}
\newcommand{\fX}{\mathfrak{X}}
\newcommand{\sF}{\mathscr{F}}
\newcommand{\sO}{\mathscr{O}}
\newcommand{\sV}{\mathscr{V}}
\newtheorem{conjecture}[subsection]{Conjecture}
\newtheorem{definition}[subsection]{Definition}
\newtheorem{theorem}[subsection]{Theorem}

\usepackage[
  hyperref = true,      % links to online documents
  backend  = bibtex,    % use bibtex instead of biber
  sorting  = nyt,       % sorts by (name, year, title)
  style    = alphabetic % citations look like [Har77]
]{biblatex}
\addbibresource{tidbit-sources.bib}
\hypersetup{
  colorlinks = true,
  linkcolor  = blue,
  urlcolor   = cyan
}

\title{Local Langlands for \texorpdfstring{$\GL(n)$}{GL(n)} over \texorpdfstring{$p$}{p}-adic fields [after Peter Scholze]}
\author{Daniel Miller}
\date{November 19, 2014}

\begin{document}
\maketitle





This is an exposition of Scholze's paper \cite{scholze-2013}. For the 
remainder of this note, let $F$ be a finite extension of $\dQ_p$. Let 
$\cO$ be the ring of integers of $F$, and let $k=\cO/\fm$ be the residue field 
of $\cO$. Write $\varpi$ for a uniformizer of $F$. 





\section{Statement of the conjectures}

Here we follow \cite{wedhorn-2008}. Let $\Gamma$ be an arbitrary locally 
profinite group. 

\begin{definition}
An \emph{admissible representation of $\Gamma$} is a (possibly 
infinite-dimensional) vector space $\pi$ with $\Gamma$-action such that: 
\begin{enumerate}
  \item For all open subgroups $U\subset \Gamma$, the space 
    $\h^0(U,\pi)$ is finite-dimensional. 
  \item $\pi=\varinjlim_{U\subset \Gamma} \h^0(U,\pi)$. 
\end{enumerate}
\end{definition}

Note that we did not mention what field $\pi$ is a vector space over. This 
is intentional. Since no mention is made of the topology on the field, the 
notion of ``admissible representation'' only depends on its isomorphism type 
as an abstract field. We will only consider vector spaces over fields 
abstractly isomorphic to $\dC$, e.g.~$\overline{\dQ_l}$. 

For each $n\geqslant 1$, let $\cA_n(F)$ be the set of isomorphism classes of 
irreducible admissible representations of $\GL_n(F)$. This has 
a subset $\cA_n^\mathrm{cusp}(F)$ consisting of those $\pi$ for which 
$\hom(\pi,\induce_P^{\GL_n(F)} \rho)=0$ for all parabolic subgroups 
$P\subset \GL_n(F)$. Such $\pi$ are called \emph{supercuspidal}; equivalently 
the matrix coefficients of $\pi$ are compact support modulo the center of 
$\GL_n(F)$. 

Let $W_F\subset\galois(\overline F/F)$ be the set of $\sigma$ such that 
for some $r\in \dZ$, we have $\sigma(x)\equiv \frob^r(x)\mod \fm_{\overline F}$ 
for all $x\in \cO_{\overline F}$. This is the \emph{Weil group} of $F$. 

\begin{definition}
A \emph{Weil-Deligne representation} of $W_F$ is a pair $\rho=(r,N)$, where 
$r:W_F\to \GL(V)$ is a continuous representation ($V$ is a finite-dimensional 
vector space with the discrete topology) and $N:V\to V$ is a nilpotent linear 
map such that 
\[
  \adjoint r(\gamma)(N) = |\gamma|\cdot N 
\]
for all $\gamma\in W_F$. 
\end{definition}

Here $|\cdot|:W_F\to \dR^\times$ is defined by $|\sigma|=|\varpi^r|$, where 
$r\in \dZ$ is such that $\sigma\equiv \frob^r\pmod p$. One says 
$\rho=(r,N)$ is \emph{Frobenius semisimple} if $r:W_F\to \GL(V)$ is semisimple. 
Note that as with admissible representations, no mention is made of the field 
over which $V$ is a vector space. As above, we will work with either $\dC$ or 
$\overline{\dQ_l}$. It turns out that a Weil-Deligne representation is the same 
thing as an ``honest'' representation of the pro-algebraic \emph{Weil-Deligne 
group} $\WD_F=W_F\ltimes \Ga$, via the action 
$\gamma\cdot x = |\gamma|^{-1} x$. Essentially, this is because representations 
of $\Ga$ are exactly ``choices of nilpotent endomorphisms.'' 

For each $n\geqslant 1$, let $\cG_n(F)$ be the set of equivalence classes of 
Frobenius-semisimple $n$-dimensional Weil-Deligne representations of $W_F$. It 
has a distinguished subset $\cG_n^\mathrm{irr}(F)$ consisting of irreducible 
representations. 

\begin{conjecture}[local Langlands]\label{conj:local-langlands}
There is a unique set of bijections $\{\reciprocity_n:\cA_n(F)\to \cG_n(F)\}$ 
such that: 
\begin{enumerate}
\item $\reciprocity_1$ is induced by local Class Field Theory. 

\item The maps $\reciprocity_n$ preserve $L$- and $\varepsilon$- factors: 
\begin{align*}
  L(\pi_1\oplus \pi_2) 
    &= L(\reciprocity(\pi_1)\otimes \reciprocity(\pi_2)) \\
  \varepsilon(\pi_1\oplus \pi_2) 
    &= \varepsilon(\reciprocity(\pi_1)\otimes \reciprocity(\pi_2)) .
\end{align*}

\item If $\chi\in \cA_1$, then 
$\reciprocity(\pi\otimes \chi)=\reciprocity(\pi)\otimes \reciprocity_1(\chi)$. 

\item If $\pi$ has central character $\chi$, then 
$\det\circ \reciprocity(\pi) = \reciprocity_1(\chi)$. 
\end{enumerate}
\end{conjecture}

We will not define the $L$- and $\varepsilon$- factors for general 
representations. The one example is: 
\[
  L(\rho,s) = \det\left(1-q^{-s} r(\frob^{-1}), (\ker N)^I\right)^{-1} ,
\]
where $q=\# k$ and $I\subset W_F$ is the \emph{inertia group} of $F$.  
Henniart proved in \cite{henniart-1993} that these requirements characterize 
the reciprocity map uniquely. Moreover, for such a correspondence, $\pi$ is 
supercuspidal if and only if $\reciprocity(\pi)$ is irreducible. Moreover, 
$\pi$ is a subquotient of the parabolic induction of some 
$\pi_1\boxtimes\cdots\boxtimes \pi_r$ if and only if 
$\reciprocity(\pi) = \reciprocity(\pi_1)\oplus \cdots \oplus \reciprocity(\pi_r)$. 
The local Langlands conjecture for $\GL(n)$ was originally proved by 
Harris and Taylor \cite{harris-taylor-2001}. 





\section{Moduli spaces of \texorpdfstring{$p$}{p}-divisible groups}

The presentation of $p$-divisible group here follows that of 
Messing in \cite{messing-1972}. Recall that $p$ is the residue characteristic 
of $F$. Let $\dZ(p^\infty)=\dQ_p/\dZ_p$; this is an ind-cyclic $p$-torsion 
group sometimes called the Pr\"ufer group. 

\begin{definition}
Let $S$ be a scheme. A \emph{$p$-divisible group} over $S$ is an fppf 
sheaf $G_{/S}$ such that: 
\begin{enumerate}
  \item $\hom(\dZ(p^\infty),G)\iso G$ ``$p$-torsion''
  \item $G\xrightarrow p G$ is an epimorphism ``$p$-divisible''
  \item Each $G[p^n]=\hom(\dZ/p^n,G)$ is a finite flat group scheme on $S$. 
\end{enumerate}
\end{definition}

It follows that $G=\varinjlim G[p^n]$, where each $G[p^n]$ is a finite flat 
group scheme for which multiplication by $p^i$ induces an isomorphism 
$G[p^{n+i}]\xrightarrow{p^i}G[p^n]$. Thus this definition agrees with the more 
traditional one. The main examples are: 
\begin{enumerate}
  \item The constant $p$-divisible group $\dZ(p^\infty)$. 
  \item $\Gm[p^\infty] = \dmu_{p^\infty} = \varinjlim \dmu_{p^n}$. 
  \item If $A_{/S}$ is an abelian scheme, $A[p^\infty] = \varinjlim A[p^n]$. 
\end{enumerate}

If $G_{/S}$ is a $p$-divisible group, we put $\lie(G)=\lie(G[p^n])$ for any 
$n\gg 0$. This is a locally free $\sO_S$-module. We put 
$\dim(G)=\dim(\lie G)$. Each $G[p^n]$ will be locally free of rank 
$p^{n h}$ for some fixed $h=\height(G)$, called the \emph{height} of $G$. 
Recall that $\varpi$ is a uniformizer in $\cO=\cO_F$. 

\begin{definition}
Let $S_{/\cO}$ be a scheme. A \emph{$\varpi$-divisible group} on $S$ is a 
$p$-divisible group $G_{/S}$ together with a homomorphism 
$\cO\to \End(G)$ such that the induced action of $\cO$ on $\lie(G)$ agrees 
with the usual one. 
\end{definition}

Let ${G_0}_{/k}$ be a $\varpi$-divisible group. We define a functor 
$\fX_{G_0}$ on connected artinian schemes over $\cO$ by letting 
$\fX_{G_0}(S)$ be the set of isomorphism classes of $\varpi$-divisible 
groups $G_{/S}$ with an isomorphism $G\otimes k\iso G_0$. Suppose now that 
$\dim(G_0)=1$ and $\height(G_0)=n$. Then for each 
$m\geqslant 1$, let $\fX_{G_0,m}(S)$ be the set of isomorphism classes of 
$G\in \fX_{G_0}(S)$ together with $x_1,\dots,x_n\in G[p^m]$ such that 
$\langle x_1,\dots,x_n\rangle = G[p^m]$ as relative Cartier divisors. 
Clearly $\GL_n(\cO/\fm^{m+1})$ acts on $\fX_{G_0,m}$ for each $m$. So if 
we put 
\[
  H_{G_0} = \varinjlim_m \h^0\left(\eR \psi_{\fX_{G_0}}\overline{\dQ_l}\right) ,
\]
then $H_{G_0}$ is a complex of $\overline{\dQ_l}$-vector spaces with 
$\GL_n(\cO)\times W_F$-action. 

Here we use formal nearby cycle sheaves in the sense of Berkovich 
\cite{berkovich-1996}, though he calls them vanishing cycle sheaves. If 
$\fX_{/\cO}$ is a formal scheme, there is a functor 
$\psi:\sheaves_\etale(\fX_{\overline F}) \to \sheaves_\etale(\fX_{\overline k})$, 
where $(\psi \sF)(U)=\sF(V)$, where $V=\fV_{\overline F}$ and 
$\fV_{/\cO_{\overline F}}$ is the unique \'etale cover whose special fiber is 
$U$. 

For each $r\geqslant 1$, let $F_r/F$ be the unique degree-$r$ unramified 
extension of $F$. Using a relative version of Dieudonn\'e theory, Scholze 
notices that to each 
$\beta\in \Gamma_0(\varpi^m)\backslash \GL_n(\cO_r)/\Gamma_0(\varpi^m)$, 
there is associated a unique one-dimensional $\varpi$-divisible group 
$\overline G_\beta$ of height $n$ over $\cO_r$, 
hence a formal scheme $\fX_{\beta,m}$ parameterizing $\varpi$-divisible 
lifts of $\overline G_\beta$. The scheme $\fX_{\beta,m}$ admits a 
$\GL_n(\cO_r/\fm^m)$-action. Put 
\[
  H_\beta = \varinjlim_m \h^0\left(\eR \psi_{\fX_{\beta,m}} \overline{\dQ_l}\right) ;
\]
this is a virtual $ W_{F_r}\times\GL_n(\cO_r)$-representation. 





\section{Orbital integrals and transfer}

The basic idea is follows. Let $\tau\in I\cdot \frob^r\subset W_F$. Recall 
that $F_r$ is the degree-$r$ unramified extension of $F$. For 
$h\in C_c^\infty(\GL_n(F_r))$, define $h^\vee(x) = h(\transpose x^{-1})$. 
Define an element $\phi_{\tau,h}\in C_c^\infty(\GL_n(F_r))$ by 
\begin{equation*}\tag{$\ast$}\label{eq:fun-def}
  \phi_{\tau,h}(\beta) = \begin{cases} \trace(\tau\times h^\vee,H_\beta) & \beta\in \GL_n(\cO_r)\operatorname{diag}(\varpi,1,\dots,1)\GL_n(\cO_r) \\ 0 & \text{otherwise} \end{cases}
\end{equation*}
Then $\phi_{\tau,h}\in C_c^\infty(\GL_n(F_r))$ has values in $\dQ$, independent 
of $l$. Let $\pi$ be an irreducible admissible representation of $\GL_n(F)$. 
We would like to characterize $\reciprocity(\pi)$ by: 
\[
  \trace(\phi_{\tau,h},\pi) = \trace(\tau,\reciprocity(\pi))\trace(h,\pi) ,
\]
but this does not work because $\phi_{\tau,h}$ does not act on $\pi$. We must 
``push down'' $\phi_{\tau,h}$ to a function $f_{\tau,h}\in \GL_n(F)$. This is 
done via requiring that orbital integrals match as follows. 

Here, we loosely follow \cite[1.3]{arthur-clozel-1989}. For 
$\gamma\in \GL_n(F)$ and $\delta\in \GL_n(F_r)$, put 
\begin{align*}
  G_\gamma &= \{x\in \GL_n(F):x^{-1} \gamma x = \gamma\} \\
  G_{\delta,\frob} &= \{x\in \GL_n(F_r):x^{-1} \gamma \frob(x) = \gamma\} ,
\end{align*}
where $\frob:F_r\to F_r$ is the Frobenius map. Define 
\begin{align*}
  \orbital_\gamma(f) 
    &= \int_{G_\gamma\backslash \GL_n(F)} f(x^{-1} \gamma x)\, \mathrm{d} \dot x && f\in C_c^\infty(\GL_n(F)) \\ 
  \twistedorbital_\delta(\phi) 
    &= \int_{G_{\delta,\frob}\backslash \GL_n(E)} \phi(x^{-1} \delta \frob(x))\, \mathrm{d}\dot x && \phi\in C_c^\infty(\GL_n(F_r)) .
\end{align*}

Define the \emph{norm map} $\norm:\GL_n(F_r)\to \GL_n(F)$ by 
\[
  \norm(g) = g\cdot \frob(g)\dotsm \frob^{r-1}(g) .
\]
Given $\phi\in C_c^\infty(\GL_n(F_r))$, there exists 
$f\in C_c^\infty(\GL_n(F))$, called the function \emph{associated to} $\phi$, 
such that for all regular $\delta\in \GL_n(F_r)$, we have 
\[
  \orbital_\gamma(f) = 
  \begin{cases} 
    0 & \gamma\text{ is not a norm} \\ 
    \twistedorbital_\delta(\phi) & \gamma=\norm(\delta)
  \end{cases} .
\]
Even though the function $f$ associated to $\phi$ is not well-defined, 
its traces are, so we have the following result. 

\begin{theorem}\label{thm:main-thm}
Let $\pi$ be an irreducible admissible representation of $\GL_n(F)$. Then there 
exists a unique $\rho\in \cG_n(F)$ such that 
\[
  \trace(f_{\tau,h},\pi) = \trace(\tau,\rho)\trace(h,\pi) 
\]
for all $\tau\in \frob^r\cdot I$ and $h\in C_c^\infty(\GL_n(F))$, where 
$f_{\tau,h}$ is associated to $\phi_{\tau,h}$ defined as in 
\eqref{eq:fun-def}. If we put 
\[
  \reciprocity(\pi) = \rho\left(\frac{n-1}{2}\right) \qquad \text{(Tate twist)},
\]
then $\pi\mapsto \reciprocity(\pi)$ realizes the local Langlands correspondence 
of \autoref{conj:local-langlands}. 
\end{theorem}





\section{Global theory}

Just as local class field theory was first proved via global class field 
theory, we need to embed the ``local problem'' of defining 
$\reciprocity(\pi)$ into the ``global problem'' of associating Galois 
representations to automorphic representations. 

First, Scholze uses induction on $n$ to show that \autoref{thm:main-thm} 
follows from the result restricted to the class of either essentially 
square-integrable (square-integrable up to a twist by a power of the 
determinant) or ``generalized Speh representations.'' 

One can realize $F$ as $\dF_v$ for $v\mid p$ a place of a CM field $\dF$ 
satisfying certain technical hypotheses \cite[\S 8]{scholze-2013}. One then 
constructs a central division algebra $\dD$ over $\dF$ together with an 
involution $\dagger$ of the second kind such that the $\dF^\dagger$-group 
$G_0$ defined by 
\[
  G_0(A) = \{g\in (A\otimes_{\dF^\dagger} \dD)^\times:g g^\dagger = 1\}
\]
is unitary of signature $(1,n-1)$ at one infinite place, and $(0,n)$ at all 
other infinite places. Define an $\dF$-group by 
\[
  G(A) = \{g\in (A\otimes_\dF \dD)^\times:g g^\dagger = 1\} .
\]
For each irreducible representation $\xi$ of $G$, one has a 
$\Gamma_\dF\times G(\dA_\finite)$ representation 
\[
  H_\xi = \varinjlim \h^\bullet_\etale\left(\shimura_K(G), \sV_\xi(\overline{\dQ_l})\right) ,
\]
in which $\sV_\xi$ is the standard automorphic vector bundle on the Shimura 
variety $\shimura_K(G)$ associated to $\xi$. 

If $\pi$ is an automorphic representation of $G$, put 
$W_\xi(\pi) = \hom_{G(\dA_\finite)}(\pi_\finite, H_\xi)$. This is a 
$\Gamma_\dF$-module. For a representation $\pi_p$ of $\GL_n(F)$ of the special 
type specified above, the correspondence 
$\pi_p\leftrightarrow \reciprocity(\pi_p)$ roughly occurs in 
$W_\xi(\pi)$ for some $\xi$. Since 
$G(\dQ_p) = \GL_n(F)\times D^\times \times \dQ_p^\times$ for the division 
algebra $D_{/\dQ_p}$ of invariant $\frac 1 n$, this is at least plausible. In 
fact, $\pi_p\leftrightarrow \reciprocity(\pi_p)$ only occurs at the level of 
``Grothendieck groups tensored with $\dQ$.'' One needs Scholze's Lemma 3.2 to 
deduce \autoref{thm:main-thm}. 





\printbibliography

\end{document}
