\documentclass{article}

\title{Equidistribution for function fields}
\author{Daniel Miller}

\usepackage{amsmath,amssymb,hyperref}
\DeclareMathOperator{\gl}{GL}
\newcommand{\dC}{\mathbf{C}}
\newcommand{\dF}{\mathbf{F}}
\newcommand{\dQ}{\mathbf{Q}}

\begin{document}
\maketitle




We mostly follow chapter 3 of \cite{ka88}.

Let $k$ be a function field of characteristic $p$. Let $\dF_q$ be the field of 
constants of $k$. We interpret $k$ as the function field of a unique (up to 
isomorphism) smooth proper geometrically connected curve $C$ over $\dF_q$. Let 
$S\subset C$ be a finite set of closed points, and let $U=C\smallsetminus S$. 
We pick a generic point $\eta$ of $U$, and let $\bar\eta$ be the corresponding 
geometric point. The group $\pi_1(U,\bar\eta)$ is the Galois group of the 
maximal extension of $k$ unramified outside $S$, i.e. 
$\pi_1(U,\bar\eta)=G_{k,S}$. 

Following \cite[2.2.4]{de77}, a \emph{lisse $\overline{\dQ_\ell}$-sheaf} on $U$ 
corresponds (uniquely) to a continuous representation 
$\rho:\pi_1(U,\bar\eta) \to \gl(n,E_\lambda)$ for a finite extension 
$E_\lambda/\dQ_\ell$. We will start with such a sheaf. In other words, we have 
a continuous representation $\rho:G_k \to \gl(n,E_\lambda)$ that is unramified 
outside $S$. Let $k^g$ be the function field of $U_{\overline{\dF_q}}$; note 
that there is a natural embedding $G_{k^g}\hookrightarrow G_k$, corresponding 
to the sequence:
\[
  1 \to \pi_1^g(U) \to \pi_1^a(U) \to G_{\dF_q} \to 1\text{,}
\]
where $\pi_1^g(U)=\pi_1(U_{\overline{\dF_q}},\bar\eta)$ is the \emph{geometric 
fundamental group} of $U$, and $\pi_1^a(U)=\pi_1(U,\bar\eta)$ is the 
\emph{arithmetic fundamental group} of $U$. 

Let $G\subset \gl(n)_{\overline{\dQ_\ell}}$ be the Zariski closure 
of $\rho(\pi_1^g)$. After fixing an embedding 
$\overline{\dQ_\ell}\hookrightarrow\dC$, we regard $G$ as a algebraic group 
over $\dC$. As such, $G(\dC)$ is a complex Lie group. 

It is easy to construct an example. Let $A$ be a $d$-dimensional abelian variety 
over $k$, and let $\rho=\rho_{A,\ell}:G_k \to \gl(2 d,\dQ_\ell)$ be the 
associated $\ell$-adic representation. If, for example $A=E$ is an elliptic 
curve without complex multiplication, then for all $\ell$, we have 
$G=\overline{\rho_{E,\ell}(\pi_1^g)}=\gl(2)$. 

Let $K$ be a maximal compact subgroup of $G(\dC)$. Katz claims that restriction 
induces an equivalence of categories 
\[
  \mathsf{Rep}_\dC^\text{alg}(G) \xrightarrow\sim \mathsf{Rep}_\dC^\text{top}(K) \text{.}
\]
\[
  \left|\sum_{\deg(v)\mid n} \deg(v) \operatorname{tr} \psi\left(\theta(v)^{n/\deg v}\right)\right| \leqslant \left(2 g-2+N-\sum r_i\right) \dim(\psi) q^{n/2}
\]





\bibliographystyle{alpha}
\bibliography{tidbit-sources}

\end{document}
