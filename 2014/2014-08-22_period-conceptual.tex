\documentclass{article}

\usepackage{amsmath,amssymb,amsthm,bookmark,fullpage,tikz-cd}
\DeclareMathOperator{\filtration}{fil}
\DeclareMathOperator{\graded}{gr}
\DeclareMathOperator{\symmetric}{Sym}
\newcommand{\cT}{\mathcal{T}}
\newcommand{\dF}{\mathbf{F}}
\newcommand{\dQ}{\mathbf{Q}}
\newcommand{\dZ}{\mathbf{Z}}
\newcommand{\fp}{\mathfrak{p}}
\newcommand{\algebra}{\mathsf{Alg}}
\newcommand{\frobenius}{\mathrm{fr}}
\newcommand{\Perf}{\mathsf{Pf}}
\newcommand{\perf}{\mathsf{pf}}
% period rings and related objects
  \newcommand{\Ainf}{\mathbf{A}_\mathrm{inf}}
  \newcommand{\Binf}{\mathbf{B}_\mathrm{inf}}
  \newcommand{\Acris}{\mathbf{A}_\mathrm{cris}}
  \newcommand{\Bcris}{\mathbf{B}_\mathrm{cris}}
  \newcommand{\Bdr}{\mathbf{B}_\mathrm{dR}}
  \newcommand{\Bht}{\mathbf{B}_\mathrm{HT}}
  \newcommand{\Bst}{\mathbf{B}_\mathrm{st}}
  \DeclareMathOperator{\units}{U}
\newtheorem{theorem}{Theorem}

\title{Conceptual approach to Fontaine's period rings}
\author{Daniel Miller}

\begin{document}
\maketitle





\section{Brief overview}

The ideas here are inspired by \cite{fontaine-1994} and \cite{scholze-2012}. 
Fix a ring $\Lambda$, an ideal $\fp\subset \Lambda$, and a $\Lambda$-algebra 
$V$. The category $\cT^{\leqslant m}(\fp) = \cT_\Lambda^{\leqslant m}(V,\fp)$ 
of \emph{$\fp$-adic pro-infinitesimal $\Lambda$-thickenings of $V$ of order 
$\leqslant m$} consists of pairs $(D,\theta)$, where $D$ is a $\fp$-adically 
complete $\Lambda$-algebra, $\theta:D\to V$ is a surjection, and moreover 
$(\ker\theta)^{m+1}=0$. The category $\cT^\infty(\fp)$ of 
\emph{$\fp$-adic pro-infinitesimal $\Lambda$-thickenings} of $V$ consists of 
similar pairs, except that now we require $D$ to be separated and complete with 
respect to $I_D=\ker(\theta)$. 

\begin{theorem}
Let $\fp=(p)$, and suppose $V$ is $p$-adically complete and 
$\frobenius:V/p\to V/p$ is surjective. Then $\cT^\infty(p)$ has an initial 
object $\Ainf(V/\Lambda)$. 
\end{theorem}
\begin{proof}
One constructs $\Ainf$ directly. Start by putting 
$V^\flat = \varprojlim_\frobenius V/p$. Our assumptions on $V$ make the map 
$(-)^\sharp:V^\flat \to V$ given by 
\[
  x^\sharp = \lim_{n\to \infty} \widetilde{x_n}^{p^n} 
\]
a well-defined multiplicative map. It induces a ring map 
$\theta:W(V^\flat) \to V$ by $[x]\mapsto x^\sharp$. The ring 
$\Ainf(V/\Lambda)$ is the completion of 
$W(V^\flat)_\Lambda$ with respect to $\ker(\theta:W(V^\flat)_\Lambda\to V)$. 
The basic idea is: suppose we have $\theta:D\twoheadrightarrow V$. Define 
$(-)^\sharp:V^\flat\to D$ just as above; this is multiplicative, and induces 
a unique $\theta:W(V^\flat)_\Lambda\to D$, which in turn factors uniquely 
through its completion $\Ainf(V/\Lambda)$. 
\end{proof}

The ring $\Bdr^+(V/\Lambda)$ is the completion of $\Ainf(V/\Lambda)[\frac 1 p]$ 
with respect to $\ker(\theta)$. 
\[
  \log[\cdot]:\mathrm{U}^\times \to \Bdr
\]
The category $\perf$ consists of $\dF_p$-algebras $A$ such that  
$\frobenius:R\to R$ is surjective. The category $\Perf$ consists of 
$p$-adically separated and complete rings $A$ such that $A/p\in \perf$. 
There are functors $-\otimes \dF_p:\Perf\to \perf$ and 
$W(-):\perf\to \Perf$. Note that $(W,\otimes\dF_p)$ is an adjoint pair.
\[\begin{tikzcd}
  \mathsf{Pf} \ar[r, "W"] 
    & \mathsf{pf}
\end{tikzcd}\]
If $\Bcris(V/\Lambda)$ is a divided power envelope, then its universal 
property should give a map $\Bcris\to \Bdr$. The trickier ring is 
$\Bst$. Also note that 
\[
  \operatorname{gr}^\bullet\Bdr = \mathbf{B}_\mathrm{HT} .
\]





\section{Some categories and functors}

Let $A$ be a reduced $\dF_p$-algebra. Then $\frobenius:A\to A$ is injective, so 
$\varprojlim_\frobenius A = \bigcap \frobenius^n(A)$. This motivates our 
general definition of $\frobenius^\infty(A)=\varprojlim_\frobenius A$. The ring 
$\frobenius^\infty(A)$ is an ``$\dF_p$-algebra with splitting.'' That is, it 
comes with a canonical section $(-)^{1/p}:(a_0,a_1,\dots)\mapsto(a_1,\dots)$ of 
Frobenius. Let $\perf$ denote the category of such algebras. That is, an object 
of $\perf$ is a pair $(A,(-)^{1/p})$, where $A$ is an $\dF_p$-algebra and 
$(-)^{1/p}:A\to A$ is a section of $\frobenius:A\to A$. So $\frobenius^\infty$ 
is a functor $\algebra(\dF_p)\to \perf$. In fact, one can easily check that 
\[
  \hom_{\algebra(\dF_p)}(A,B) = \hom_\perf((A,(-)^{1/p}),\frobenius^\infty B) ,
\]
i.e.~$\frobenius^\infty$ is right-adjoint to the forgetful functor 
$\perf\to \algebra(\dF_p)$. 

Let $\algebra(p)$ be the category of $p$-adically complete algebras, and let 
$\Perf$ be the full subcategory mapping onto $\perf\subset \algebra(\dF_p)$. 
There is the obvious functor $\otimes\dF_p:\Perf\to \perf$. Moreover, there 
is a functor ``take Witt vectors'' $W:\perf\to \Perf$ that satisfies 
\[
  \hom_\Perf(W(A),B) = \hom_\perf(A,B/p) .
\]





\section{A bestiary of period rings}

Call a \emph{quasi-perfectoid ring} a commutative $p$-adic Banach algebra 
$A$ such that $\frobenius:A^\circ/p\to A^\circ/p$ is surjective. For the 
remainder, let $A$ be a quasi-perfectoid ring. We agree that 
$\mathbf{A}_\ast$ will take values in $\dZ_p$-algebras, while 
$\mathbf{B}_\ast$ will take values in $\dQ_p$-algebras. In fact, when both are 
defined, $\mathbf{B}_\ast=\mathbf{A}_\ast[\frac 1 p]$. For simplicity, we 
assume $p$ is odd. 


\subsection{\texorpdfstring{$\Ainf$}{Ainf}}

The ring $\Ainf(A)$ is the ``universal $p$-adic pro-infinitesimal formal 
thickening of $A^\circ$.'' That is, it has a surjective ring map 
$\theta:\Ainf(A)\to A^\circ$ for which $\Ainf(A)$ is complete with respect to 
$\ker(\theta)$. Explicitly, 
\[
  \Ainf(A) = W(\frobenius^\infty A^\circ/p) ,
\]
and $\theta([a]) = a^\sharp$. Note that $\Ainf(A)$ has a natural filtration 
in which $\filtration^r \Ainf = (\ker\theta)^r$. 


\subsection{\texorpdfstring{$\Binf$}{Binf}}

This is just $\Binf(A) = \Ainf(A)[\frac 1 p]$. Note that 
$\theta:\Ainf(A)\to A^\circ$ extends uniquely to 
$\theta:\Binf(A)\to A$, so $\Binf$ inherits the filtration from 
$\Ainf$. 


\subsection{\texorpdfstring{$\units^\times$}{Ux}}

We define two subgroups of $\frobenius^\infty(A^\circ/p)$:
\begin{align*}
  \units^1(A) 
    &= \{x\in \frobenius^\infty(A^\circ/p):x^\sharp\equiv 1\pmod p\} \\
  \units^\times(A) 
    &= \{x\in \frobenius^\infty(A^\circ/p):|x^\sharp-1|<1\} .
\end{align*}
Clearly $\units^1\subset \units^\times$. Note that a better-motivated definition 
would be $\units^1(A)=\{a\in A:|a-1|<\frac 1 p\}$. The logarithm function 
\[
  \log(a) = \sum_{n\geqslant 1} (-1)^{n+1} \frac{(a-1)^n}{n}
\]
is easily checked to converge, so it gives a continuous homomorphism (with this 
definition of $\units^1$) $\units^1(A)\to A$. 


\subsection{\texorpdfstring{$\Bdr$}{Bdr}}

First we define $\Bdr^+(A)$ to be the completion of 
$\Binf(A)$ with respect to $\ker(\theta)$. There is a continuous homomorphism 
$\log[\cdot]:\units^\times \to \Bdr^+$. The ring $\Bdr$ should be an 
appropriate localization of $\Bdr^+$. 


\subsection{\texorpdfstring{$\Bht$}{Bht}}

We set $\Bht^+=\graded^{\geqslant 0} \Bdr$, and 
$\Bht=\graded^\bullet \Bdr$. Hopefully, 
$\Bht(A)=\bigoplus_{n\in \dZ} \units^\times(A)^{\otimes n}$. But this only 
works if $\dim_{\dQ_p} \units^\times(A)=1$. Indeed, we needed that to define 
$\langle \xi\rangle = \graded^1\Bdr$ anyways. 


\subsection{\texorpdfstring{$\Bcris$}{Bcris}}

Let $\Acris(A)$ be the universal $p$-adically complete formal 
divided-power thickening of $A^\circ$, and $\Bcris(A)=\Acris(A)[\frac 1 p]$. By 
definition, there is a map $\theta:\Acris(A)\to A^\circ$ inducing 
$\theta:\Bcris(A)\to A$. Moreover, we get a natural map (injective if $A$ is a 
field) $\Bcris(A)\to \Bdr(A)$. Moreover, the map 
$\log[\cdot]:\units^\times \to \Bdr$ factors through 
$\log[\cdot]:\units^\times \to \Bcris$. 


\subsection{\texorpdfstring{$\Bst$}{Bst}}

Let $\Bst^+(A)$ be the initial object among $\Bcris^+(A)$-algebras $S$ together 
with $\lambda:\operatorname{frac}(\frobenius^\infty A^\circ/p)^\times\to S$ 
extending $\log[\cdot]:\units^\times(A)\to S$. One has 
\[
  \Bst^+=\symmetric^\bullet(\operatorname{frac}(\frobenius^\infty A^\circ/p)^\times)\otimes_{\symmetric^\bullet((\frobenius^\infty A^\circ/p)^\times)} \Bcris^+(A) .
\]
Again, if $A$ is a field, then 
$\operatorname{frac}(\frobenius^\infty A^\circ/p)^\times/\frobenius^\infty(A^\circ/p)^\times = A^{\flat\times} / A^{\flat\circ\times}$ is a one-dimensional $\dQ_p$-vector space, so we have a 
(non-canonical) isomorphism $\Bst^+=\Bcris[X]$. Finally, 
$\Bst=\Bst^+\otimes_{\Bcris^+} \Bcris$. 





\bibliographystyle{alpha}
\bibliography{tidbit-sources}

\end{document}
