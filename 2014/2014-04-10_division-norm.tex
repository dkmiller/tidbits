\documentclass{article}

\usepackage{amsmath,amssymb,fullpage,mathpazo}
\usepackage[all]{xy}
\DeclareMathOperator{\h}{H}
\DeclareMathOperator{\genlin}{GL}
\DeclareMathOperator{\lie}{Lie}
\DeclareMathOperator{\ortho}{O}
\DeclareMathOperator{\pin}{Pin}
\newcommand{\dA}{\mathbf{A}}
\newcommand{\dC}{\mathbf{C}}
\newcommand{\dG}{\mathbf{G}}
\newcommand{\dQ}{\mathbf{Q}}
\newcommand{\dZ}{\mathbf{Z}}

\title{Division algebras and spin groups}
\author{Daniel Miller}

\begin{document}
\maketitle


\section{Totally unrelated but possibly interesting question}

Let $k$ be a field, $D$ a division algebra with center $k$. One can define a 
norm map $N:D\to k$ that is multiplicative. Clearly 
$N(k^\times) = (k^\times)^n$, where $n=\dim_k D$. So 
$N(D^\times)\supset (k^\times)^n$, and it is natural to ask 
whether they are equal. Put 
$\operatorname{H}(D) = N(D^\times) / N(k^\times)$. 
The question is: when is $\operatorname{H}(D)=0$?

This is essentially the same question as the following. Put 
$D^{(1)}=\ker(N:D^\times \to k^\times)$. There is a natural isomorphism 
$D^\times / k^\times \xrightarrow\sim \operatorname{Aut}(D)$, sending 
$a\in D^\times$ to $b\mapsto a b a^{-1}$. Hubbard asked whether 
$D^{(1)} \to \operatorname{Aut}(D)$ is surjective, i.e. whether 
$D^{(1)} \to D^\times / k^\times$ is surjective. Choosing $a\in D^\times$, we 
see that the only way that $a$ can be in the image of 
$D^{(1)}$ is for $N(a)\in (k^\times)^n$. So we could say that the failure of 
$D^{(1)} \to D^\times / k^\times$ to be surjective is measured by 
$N(D^\times) / N(k^\times)$. Alternatively, we could think of 
$D^{(1)}$ and $D^\times$ as affine group schemes over $k$. The image of 
$D^{(1)} \to D^\times / \dG_m$ is normal, so the quotient 
$D^{(1)}\backslash D^\times / \dG_m$ exists as a variety over $k$, and we could 
ask about its structure. 

Even better, the quotient $D^\times / \dG_m D^{(1)}$ exists as a commutative 
group scheme over $k$. It should have dimension zero, so if it is \'etale, 
it will correspond to a $G_k$-module. 

I think there is an easy solution. Write $D^1$ instead of $D^{(1)}$ and consider 
both $D^1$ and $D^\times$ as algebraic groups over $k$. The sequence 
\[\xymatrix{
  1 \ar[r] 
    & D^1 \ar[r] 
    & D^\times \ar[r]^-N 
    & \dG_m \ar[r] 
    & 1
}\]
of algebraic groups over $k$ is exact in the \'etale topology, so we get a 
long exact sequence in cohomology. 
\[\xymatrix{
  1 \ar[r] 
    & D^1 \ar[r] 
    & D^\times \ar[r]^-N 
    & k^\times \ar[r] 
    & \h^1(k,D^1) \ar[r] 
    & \h^1(k,D^\times) \ar[r] 
    & 0
}\]
It follows that $N(D^\times) = \ker(k^\times \to \h^1(k,D^1))$. That kernel is 
actually computable. Recall that $\h^1(k, D^1) = \h^1(G_k, D^1(k^s))$. An 
element $\lambda\in k^\times$ is send to the cocycle 
$\varphi_\lambda:G_k \to D^1(k^s)$ defined as follows. Choose a lift 
$\widetilde\lambda\in D^1(k^s)$ of $\lambda$, and put 
$\varphi_\lambda(\sigma) = \sigma(\widetilde\lambda) / \widetilde\lambda$. 
It is easy to check that $\varphi_\lambda=0$ in $\h^1(k, D^1)$ if and only if 
there exists $\widetilde\lambda$ with $\sigma(\widetilde\lambda)=\widetilde\lambda$, 
i.e. if and only if $\lambda$ is in the image of $N(D^1)$, i.e. I'm not sure 
where this proof is going\ldots it turns out to be a rather sophisticated proof 
that $N(D^\times) = N(D^\times)$. 





\section{A problem of Hubbard}

John Hubbard suggested the following problem to me. Let $k$ be a field (the 
main example I have in mind is a number fields, but the problem can be stated 
in much greater generality). Let $(V,q)$ be a ``quadratic space over $k$,'' 
i.e. $V$ is a finite-dimensional $k$-vector space and $q$ is a quadratic form 
on $V$. For example, we could have $k=\dQ$, $V=\dQ^{\oplus 4}$, and 
$q(x_1,\dots,x_4) = x_1^2 + x_2^2 + x_3^2 - 7 x_4^2$. We can form the 
\emph{orthogonal group} of $q$, $\ortho(q)\subset \genlin(V)$; this is an 
algebraic group over $k$. The subgroup $\ortho^+(q)\subset \ortho(q)$ of 
``isometries'' with determinant one is called the special orthogonal group of 
$q$. It turns out that there is a natural embedding of $\ortho^+(q)$ into the 
group of units of a particular associative algebra. 

Let $C(V,q)$ be the \emph{Clifford algebra} of $(V,q)$, i.e. the quotient 
\[
  C(V,q) = T(V) / \langle v\otimes v-q(v):x\in V\rangle 
\]
where $T(V)=\bigoplus_{n\geqslant 0} V^{\otimes n}$ is the Tensor algebra of 
$V$. The involution $-1:V\to V$ induces an involution $\gamma$ of $C(V,q)$, and 
we put 
\[
  C^\circ(V,q) = C(V,q)^\gamma .
\]

Key fact: $k^\times / (k^\times)^2 = \h^2(k, \mu_2) = \h^2(k, \dG_m)[2]$. 
Moreover, all division algebras associated to elements of 
$\h^2(k, \mu_2)$ are quaternion algebras. We have the following exact sequences: 
\[\xymatrix{
  1 \ar[r] 
    & \ortho^+(q) \ar[r] 
    & \ortho(q) \ar[r]^-{\det} 
    & \mu_2 \ar[r] 
    & 1 \\
  1 \ar[r] 
    & \dG_m \ar[r] 
    & \Gamma(q) \ar[r]^-\alpha 
    & \ortho(q) \ar[r] 
    & 1 \\
  1 \ar[r] 
    & \pin(q) \ar[r] 
    & \Gamma(q) \ar[r]^-N 
    & \dG_m \ar[r] 
    & 1 \\
  1 \ar[r] 
    & \mu_2 \ar[r] 
    & \pin(q) \ar[r]^-\alpha 
    & \ortho(q) \ar[r] 
    & 1
}\]
From the last sequence, we get $\h^1(\ortho_n) \to \h^2(\mu_2)$, which will 
associate a division algebra to each $n$-dimensional quadratic form over $k$. 

In general, if $D$ is an $n^2$-dimensional division algebra over $k$, then 
the map $N:D^\times \to k^\times$ will factor through the $n$-th power map, 
i.e. $N=(-)^n\circ N_{rd}$, where $N_{rd}$ is the so-called \emph{reduced norm}. 




\end{document}