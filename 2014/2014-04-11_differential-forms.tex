\documentclass{article}

\usepackage{amsmath,amssymb,calligra,fullpage,mathpazo,mathtools}
\usepackage[mathscr]{euscript}
\usepackage[all]{xy}
\DeclareMathOperator{\curl}{curl}
\DeclareMathOperator{\derivation}{\mathscr{D}\text{\kern -2pt {\calligra\large er}}}
\DeclareMathOperator{\generallinear}{GL}
\DeclareMathOperator{\grad}{grad}
\DeclareMathOperator{\locallyfree}{\mathsf{LF}}
\DeclareMathOperator{\sym}{Sym}
\renewcommand{\div}{\operatorname{div}}
\newcommand{\ii}{\mathbf{i}}
\newcommand{\jj}{\mathbf{j}}
\newcommand{\kk}{\mathbf{k}}
\newcommand{\df}{\mathbf{f}}
\newcommand{\dR}{\mathbf{R}}
\newcommand{\sO}{\mathscr{O}}
\newcommand{\sE}{\mathscr{E}}
\newcommand{\sI}{\mathscr{I}}
\newcommand{\sL}{\mathscr{L}}
\newcommand{\sM}{\mathscr{M}}
\newcommand{\sX}{\mathscr{X}}
\newcommand{\hodge}[1]{\prescript{\star}{}{#1}}





\title{Multivariable calculus and differential forms}
\author{Daniel Miller}

\begin{document}
\maketitle





Let $U$ be a connected, simply-connected subset of the smooth manifold $\dR^3$. 
We have the de Rham sheaf $\Omega^\bullet$ on $U$. Let $\sO$ be the structure 
sheaf of $U$, and let $\sX=(\Omega^1)^\vee$ be the sheaf of vector fields on 
$U$. We identify $\sX$ with $\sO^3$ in the usual way, i.e. 
$X=f \partial_x + g \partial_y + h\partial_z$ corresponds to 
$f\ii + g \jj + h \kk$. One defines maps 
\begin{align*}
  \grad: \sO&\to \sX \qquad f\mapsto \nabla f = f_x \ii + f_y \jj + f_z \kk \\
  \curl:\sX&\to \sX \qquad \df \mapsto \nabla\times \df = \det\begin{pmatrix} \ii & \jj & \kk \\ \partial_x & \partial_y & \partial_z \\ f_1 & f_2 & f_3\end{pmatrix} = (\partial_x f_2 - \partial_y f_1) \ii\\
  \div:\sX &\to \sO \qquad \df\mapsto \nabla \cdot \df = \partial_x f_1  + \partial_y f_2  + \partial_z f_3  
\end{align*}
The key fact is that we have a commutative diagram:
\[\xymatrix{
   \sO \ar[r]^-\grad \ar@{=}[d] 
    & \sX \ar[r]^-\curl \ar[d]^-{(-)^\flat} 
    & \sX \ar[r]^-\div \ar[d]^-\alpha 
    & \sO \ar[d]^-\beta \\
  \sO \ar[r]^-d 
    & \Omega^1 \ar[r]^-d 
    & \Omega^2 \ar[r]^-d 
    & \Omega^3 
}\]

Since $U$ is a Riemannian manifold, it comes with a metric, i.e. an isomorphism 
$\sO\xrightarrow g \sym^2 \Omega^1$. It yields an isomorphism (the 
musical isomorphism) $(-)^\flat:\sX\to \sX^{\vee\vee} = \Omega^1$, given by 
$X^\flat(Y) = \langle X,Y\rangle = (X\otimes Y)(g)$. 

The maps $\alpha$ and $\beta$ are 
\[
  \alpha:X\mapsto (Y\otimes Z\mapsto \langle X, Y\times Z\rangle)
\]
d
\[
  \nabla\times X = \left(\hodge(d X^\flat)\right)^\sharp
\]










\section{General differential geometry}

Let $(X,\sO)$ be a ringed topos, and $\sE$ a locally free $\sO$-module of 
finite type. General nonsense tells us that the natural map 
$(\sE\otimes\sE)^\vee \to \hom(\sE,\sE^\vee)$ is an isomorphism. In particular, 
to any $g\in (\sE\otimes \sE)^\vee$ we can associate the ``musical map'' 
$(-)^\flat:\sE\to \sE^\vee$. We call $g$ non-degenerate if $(-)^\flat$ is an 
isomorphism. 


Suppose we have a derivation $d:\sO\to \Omega^1$, $\Omega^1$ is locally free, 
and put $\sX=(\Omega^1)^\vee$. Let $g\in (\sX\otimes \sX)^\vee$ be a non-degenerate 
symmetric inner product. Put $\langle X,Y\rangle = g(X,Y)$. Since $g$ is 
non-denerate, 

Again, let $(X,\sO)$ be a ringed topos. The motivating problem is that if we 
define $\Omega^1_X$ in the obvious way, (i.e.\ as $\sI/\sI^2$, where $\sI$ is 
the kernel of $\sO\otimes \sO \to \sO$, then $\Omega_X^1$ for $X$ a smooth 
manifold doesn't agree with the ``classical definition'' in differential 
geometry. [This may not be the case.] One can define a sheaf (for any 
$\sO$-module $\sM$) $\derivation(\sO,\sM)$ in the usual manner. This agrees 
with the classical definition. However, $\derivation(\sO,\sO)^\vee$ is 
\emph{not} necessarily isomorphic to $\Omega^1$, if $(X,\sO)$ is a smooth 
manifold. A kludge is to put $\sX=\derivation(\sO,\sO)$ for vector fields, and 
not worry directly about $\Omega^1$. 





\subsection{Vector bundles and torsors}

Let $(X,\sO)$ be a smooth manifold. Recall that a \emph{vector bundle} on 
$X$ is a locally free $\sO$-module. Let $\locallyfree(\sO)$ be the category of 
vector bundles on $X$. 





\end{document}
