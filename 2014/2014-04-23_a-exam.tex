\documentclass{article}

\usepackage{amsmath,amssymb,amsthm,extarrows,fullpage,mathpazo,mathrsfs,mathtools}
\usepackage[hidelinks]{hyperref}
\usepackage[all]{xy}
\DeclareMathOperator{\adjoint}{ad}
\DeclareMathOperator{\automorphism}{Aut}
\DeclareMathOperator{\characters}{X}
\DeclareMathOperator{\divisor}{div}
\DeclareMathOperator{\divisors}{Div}
\DeclareMathOperator{\galois}{Gal}
\DeclareMathOperator{\generallinear}{GL}
\DeclareMathOperator{\graded}{gr}
\DeclareMathOperator{\h}{H}
\DeclareMathOperator{\height}{ht}
\DeclareMathOperator{\image}{im}
\DeclareMathOperator{\induce}{ind}
\DeclareMathOperator{\lie}{Lie}
\DeclareMathOperator{\picard}{Pic}
\DeclareMathOperator{\restrict}{res}
\DeclareMathOperator{\sign}{sgn}
\DeclareMathOperator{\speciallinear}{SL}
\DeclareMathOperator{\specialunitary}{SU}
\DeclareMathOperator{\spectrum}{Spec}
\DeclareMathOperator{\symmetricpower}{Sym}
\DeclareMathOperator{\trace}{tr}
\DeclareMathOperator{\volume}{vol}
\newcommand{\dA}{\mathbf{A}}
\newcommand{\dC}{\mathbf{C}}
\newcommand{\dF}{\mathbf{F}}
\newcommand{\dG}{\mathbf{G}}
\newcommand{\dI}{\mathbf{I}}
\newcommand{\dN}{\mathbf{N}}
\newcommand{\dQ}{\mathbf{Q}}
\newcommand{\dR}{\mathbf{R}}
\newcommand{\dZ}{\mathbf{Z}}
\newcommand{\fa}{\mathfrak{a}}
\newcommand{\fe}{\mathfrak{e}}
\newcommand{\fg}{\mathfrak{g}}
\newcommand{\fh}{\mathfrak{h}}
\newcommand{\fk}{\mathfrak{k}}
\newcommand{\fm}{\mathfrak{m}}
\newcommand{\fn}{\mathfrak{n}}
\newcommand{\fp}{\mathfrak{p}}
\newcommand{\fo}{\mathfrak{o}}
\newcommand{\sD}{\mathscr{D}}
\newcommand{\sI}{\mathscr{I}}
\newcommand{\sM}{\mathscr{M}}
\newcommand{\sO}{\mathscr{O}}
\newcommand{\abelianize}{\textnormal{ab}}
\newcommand{\complementaryseries}{\mathcal{C}}
\newcommand{\discreteseries}{\mathcal{D}}
\newcommand{\finitedimensional}{\mathcal{F}}
\newcommand{\isomorphism}{\xlongrightarrow{\sim}}
\newcommand{\laurent}[1]{(\! ({#1} )\! )}
\newcommand{\principalseries}{\mathcal{P}}
\newcommand{\smat}[4]{\bigl(\begin{smallmatrix} {#1} & {#2} \\ {#3} & {#4} \end{smallmatrix} \bigr)}
\newcommand{\transpose}[1]{\prescript{\textnormal{t}}{}{#1}}
\newcommand{\unramified}{\textnormal{ur}}
\newtheorem{theorem}[subsubsection]{Theorem}
\theoremstyle{definition}
\newtheorem{definition}[subsubsection]{Definition}

\title{Topics for A-exam}
\author{Daniel Miller}

\begin{document}
\maketitle
\tableofcontents










\section{Algebraic number theory}





\subsection{Extensions of dedekind schemes}

Recall that a \emph{dedekind scheme} is an integral noetherian normal scheme 
of dimension $1$. If $X$ is a Dedekind scheme, then all the stalks 
$\sO_{X,x}$ are one-dimensional regular local rings, i.e.\ discrete valuation 
rings. In particular, each closed point $x\in X$ induces a discrete valuation 
$v_x$ on the field of fractions $k$ of $X$. 

If $K/k$ is a finite separable extension, let $Y$ be the normalization of $X$ 
in $K$, i.e.\ the relative spectrum of the sheaf 
\[
  U\mapsto \text{integral closure of $\sO_X(U)$ in $K$} .
\]
Then $p:Y\to X$ is a finite surjective morphism. Each fiber $p^{-1}(x)$ is a 
finite set, possibly with nilpotents. 





\subsection{Discrete valuation fields}

Let $k$ be a field with a discrete valuation 
$v:k^\times \twoheadrightarrow \dZ$. Let $\fo=\{x\in k:v(x)\geqslant 0\}$ be 
the valuation ring of $k$, and let $\fp\subset \fo$ be the unique maximal 
ideal. Put $U=1+\fp$, and give $\fp$ and $U$ filtrations by 
\begin{align*}
  \fp^r &= \text{the $r$-th power of $\fp$} \\
  U^r &= 1+\fp^r .
\end{align*}
There is a canonical isomorphism 
$\graded(\fp^\bullet) \isomorphism \graded(U^\bullet)$, given by 
$x\mapsto 1+x$. 






\subsection{Henselian Fields}

Let $k$ be a field with a discrete valuation. One calls $k$ \emph{Henselian} 
if Hensel's lemma holds for $\fo_k$. Alternatively, one requires that 
valuations extend uniquely to algebraic extensions of $k$. Complete fields 
are Henselian. 

One can give a reasonable description of the absolute Galois group $G_k$ of a 
Henselian field $k$. Let $\kappa$ be the residue field of $k$, and let 
$p\geqslant 0$ be the characteristic of $\kappa$. If $K/k$ is a \emph{finite} 
extension, then define for $r\geqslant 0$, 
\[
  \galois(K/k)_r = \ker\left(\galois(K/k) \to \automorphism_{\fo_k}(\fo_L/\fp^{r+1})\right) .
\]
There is a canonical embedding 
$\graded(\galois(K/k)_\bullet)\hookrightarrow \graded(U_K^\bullet)$, 
given by $\sigma\mapsto \sigma \pi/\pi$, for $\pi\in \fo_K$ an arbitrary 
uniformizer. In particular, $\galois(K/k)$ is solvable. 

upper ramification numbering





\subsection{Local fields}

A \emph{local field} is a locally compact topological field. Local fields are 
known to be finite extensions of either $\dF_p\laurent t$ or $\dQ_p$. 





\subsection{Global fields}


\subsection{Classical geometry of numbers}

Our main reference is Chapter I, \S5-7 of \cite{ne99}. Let $k$ be a number 
field, and write $k_\infty = k_\dR=k\otimes_\dQ \dR$. This is a finite \'etale 
$\dR$-algebra isomorphic to $\dR^r\times \dC^s$, where $r$ is the 
number of real places and $s$ is the number of complex places of $k$. One gives 
$k_\infty$ a standard measure (twice Lebesgue on copies of $\dC$ and Lebesgue 
on copies of $\dR$) under which the lattice $\fo$ has volume 
$\volume(\fo) = \volume(k_\infty/\fo) = |d_k|^{1/2}$. As a corollary, if 
$S\subset k_\infty$ is open, convex, and centrally symmetric, then 
$\volume(S) \geqslant \volume(\fo)$ implies $S\cap \fo\ne 0$. The same type of 
theorem holds for any $\fa\subset \fo$, where 
$\volume(\fa) = [\fo:\fa] |d_k|^{1/2}$. Considering $S=\{a:|N(a)| \leqslant ?\}$ 
gives that for every nonzero ideal 
$\fa\subset \fo$, there is $a\in \fa\smallsetminus 0$ such that (there exist 
$a$ for both of these) 
\begin{align*}
  |N_{k/\dQ}(a)| &\leqslant \frac{n!}{n^n}\left(\frac 4 \pi\right)^s |d_k|^{1/2} [\fo:\fa] \\
  |N_{k/\dQ}(a)| &\leqslant \left(\frac 2 \pi\right)^s |d_k|^{1/2} [\fo:\fa] .
\end{align*}

There is a natural map $\log|\cdot|:k_\infty^\times$. 


\subsection{Adelic geometry of numbers}

A key fact is that $\dI^1_k/k^\times$ and $\dA_k/k$ are compact. 

Our main reference here is \cite{pr94}. Let $k$ be a number field, and $G$ an 
algebraic group over $k$. Let $S$ be a finite set of places of $k$, containing 
all the infinite places, and let $k_S=\prod_{v\in S} k_v$. It is a theorem of 
Borel (essentially Theorem 5.1 of \cite{pr94}) that the double quotient 
$G(k)\backslash G(\dA_k) / G(k_S) K$ is finite, whenever 
$K\subset G(\dA)$ is open compact. 

Let $\characters(G)=\hom_k(G,\dG_m)$; this is a finite free $\dZ$-module. 
There is a canonical homomorphism 
\[
  c:G(\dA) \to \characters(G)_\dR^\vee \qquad (g_v) \mapsto \chi\mapsto \prod_v |\chi(g_v)|_v .
\]
Let $G(\dA)^1 = \ker(c)$. Then Theorem 5.6 of \cite{pr94} tells us that 
$G(\dA)^1 / G(k)$ is compact if and only if the semisimple part of $G$ is 
anisotropic over $k$. 










\section{Class field theory}

The main references are Chapter V of \cite{ne99} and Chapter VI of 
\cite{cf86}. 


\subsection{Local class field theory}

These theorems are from chapter V of \cite{ne99}. Recall that if $\kappa$ is a 
finite field, we regard $\dZ$ as a subgroup of $G_\kappa$ by letting $1\in \dZ$ 
correspond to the (arithmetic) Frobenius of $\kappa$. We normalize all 
valuations so that uniformizers have valuation $1$. 

\begin{theorem}
Let $k$ be a local field with residue field $\kappa$. Then there is a unique 
continuous homomorphism $r:k^\times \to G_k^\abelianize$ such that 
\begin{enumerate}
  \item the following diagram commutes:
    \[\xymatrix{
      k^\times \ar[r]^r \ar[d]^-v 
        & G_k^\abelianize \ar@{->>}[d] \\
      \dZ \ar@{^{(}->}[r] 
        & G_\kappa
    }\]
  \item for all finite abelian $K/k$, the map $r$ induces an isomorphism 
    $k^\times/N(K^\times) \isomorphism \galois(K/k)$
\end{enumerate}
This homomorphism (called the \emph{reciprocity map}) induces an isomorphism 
$\widehat{k^\times} \isomorphism G_k^\abelianize$. Finally, a subgroup 
$G\subset k^\times$ is of the form $N(K^\times)$ for some finite $K/k$ if and 
only if $G$ is open and has finite index. 
\end{theorem}
\begin{proof}
This proof is inspired by that of Theorem 1.13 in \cite{mi-cft}. 
We only prove the uniqueness of $r$. Since $k^\times$ is topologically 
generated by uniformizers, it suffices to show that conditions determine 
$r(\pi)$ for any $\pi\in k^\times$ with $v(\pi)=1$.  For any such $\pi$, we 
get a decomposition $k^\abelianize = k^\unramified \cdot k_\pi$, where $k_\pi$ 
is the fixed field of $r(\pi)$. Since $r(\pi)$ has to act on $k^\unramified$ as 
Frobenius, and (by definition), $r(\pi)$ acts trivially on $k_\pi$, the action 
of $r(\pi)$ on $k^\abelianize$ is determined. 
\end{proof}

This theorem is true even if $k$ is archimedean. Just ignore part 1, and note 
that $G_\dR=\dZ/2$, which has no nontrivial automorphisms. In other words, 
there is a unique homomorphism $\dR^\times \to G_\dR$ inducing an isomorphism 
$\widehat{\dR^\times}\isomorphism G_\dR$. 


\subsection{Global class field theory}

A good references is chapter VI of \cite{ne99}. For a global field $k$, write 
$\dA_k$ for its ring of adeles, and define 
$C_k=\generallinear(1,\dA_k) / \generallinear(1,k)$. For a place $v$ of $k$, 
write $r_v$ for the reciprocity map of $k_v$. Since the decomposition groups 
$D_v=G_{k_v}$ are well-defined up to conjugacy in $G_k$, the abelianized 
decomposition groups $D_v^\abelianize$ are well-defined as subgroups of 
$G_k^\abelianize$. 

\begin{theorem}
There is a unique continuous homomorphism $r:C_k \to G_k^\abelianize$ such that 
\begin{enumerate}
  \item the following diagram commutes for all $v$:
    \[\xymatrix{
      k_v^\times \ar@{^{(}->}[r] \ar[d]^-{r_v} 
        & C_k \ar[d]^-r \\
      D_v^\abelianize \ar@{^{(}->}[r] 
        & G_k^\abelianize
    }\]
  \item if $K/k$ is finite abelian, $r$ induces an isomorphism 
    $C_k/N(C_K) \isomorphism \galois(K/k)$. 
\end{enumerate}
This homomorphism (also called the \emph{reciprocity map}) induces an 
isomorphism $\widehat{C_k}\isomorphism G_k^\abelianize$. Finally, a subgroup 
$G\subset C_k$ is of the form $N(K^\times)$ for some finite $K/k$ if and only 
if $G$ is open and has finite index. 
\end{theorem}





\section{Algebraic geometry}

\cite{ha77}, \cite{si09}, and \cite{eh00}

Zariski's main theorem

tangent space on affine / projective variety as special case of tangent sheaf 
for arbitrary morphism of schemes


\subsection{Cartier divisors}

For a ringed space $X$, let $\sM$ be the sheaf of meromorphic functions, and 
put $\sD=\sM^\times/\sO^\times$. This is a sheaf of abelian groups on $X$, 
called the \emph{sheaf of Cartier divisors}. A global section of $\sD$ is 
called a \emph{Cartier divisor} on $X$. The (tautological) short exact 
sequence 
\[
  1 \to \sO^\times \to \sM^\times \to \sD \to 0 
\]
yields a long exact sequence in sheaf cohomology:
\[
  1 \to \Gamma(\sO^\times) \to \Gamma(\sM^\times) \to \divisors(X) \to \picard(X) \to \h^1(\sM^\times) \to \cdots
\]
If $X$ is integral, $\sM$ is flasque, so $\h^1(\sM^\times)=0$, whence 
$\picard(X) = \divisors(X)/\divisor\Gamma(\sM^\times)$. If $X$ is a 
one-dimensional scheme over $S$, then points $x\in X(S)$ yield Cartier 
divisors, denoted $\sI(x)$, on $X$. If we think of $\sI(x)$ as a sheaf, 
then it makes sense to talk about $\sI^{-1}(x)$. 


\subsection{Algebraic groups and their Lie algebras}

We would like to interpret the notion of a representation of an algebraic 
group over a field in terms of arbitrary ringed topoi. The main source here is 
\cite{sga3}. We start with the following list of analogies. 
\begin{center}
\begin{tabular}{c|c}
  classical & topos-theoretic \\ \hline
  field $k$ & base scheme $S$ \\
  finite-dimensional $k$-space $V$ & coherent $\sO_X$-module $\sM$ \\
  $R\mapsto V_R=V\otimes_k R$ & $(X\xrightarrow f S)\mapsto \Gamma(f^\ast \sM)$ \\
  $\spectrum(k[V^\vee])$ & $\mathbf V(\sM) = \spectrum(\sO_S[\sM^\vee])$ \\
  $\generallinear(V):R\mapsto \automorphism_R(V_R)$ & $\generallinear(\sM):X\mapsto \automorphism_{\Gamma(X)}(\Gamma(f^\ast \sM))$ \\
  $\rho:G\to \generallinear(V)$ & $\rho:G\to \generallinear(\sM)$ \\
  $G(R) \to \automorphism_R(V_R)$ & $G(X) \to \automorphism_{\Gamma(X)}(\Gamma(f^\ast \sM))$
\end{tabular}
\end{center}

An algebraic group $G$ over a field $k$ is \emph{anisotropic} if it contains 
no $k$-split tori. If $k$ is a non-algebraically closed local field, then $G$ 
is anisotropic if and only if $G(k)$ is compact. 





\section{Geometry of curves}

The main references for this section are Chapter IV of \cite{ha77} and Chapter 
7 of \cite{li02}:


\subsection{Divisors and invertible sheaves on curves}


\subsection{The Riemann-Roch theorem}





\section{Elliptic curves}

The main reference on elliptic curves is \cite{si09}, especially chapters 
II-IV and VI-VIII. Chapter II mostly covers the geometry of curves. 

elliptic curves over local fields

elliptic curves over global fields


\subsection{The group law on an elliptic curve}

If $X$ is a curve over $S$, define 
\[
  \picard_{X/S}^1(T) = \picard^1(X_T) / \picard(T) .
\]

\begin{theorem}
Let $E$ be an elliptic curve over $S$. Then there is a natural isomorphism 
of functors $E\isomorphism \picard_{E/S}^1$ given by $x\mapsto \sI^{-1}(x)$. 
This induces the structure of an $S$-group on $E$, where $x+y+z=0$ if and only 
if 
\[
  (\sI(x) - \sI(0)) + (\sI(y) - \sI(0)) + (\sI(z) - \sI(0)) = 0\qquad \text{in } \picard_{X/S}(T) . 
\]
\end{theorem}
\begin{proof}
This is Theorem 2.1.2 of \cite{km85}
\end{proof}


\subsection{Formal groups of elliptic curves}

Let $F$ be a formal group. By Proposition IV.4.2 of \cite{si09}, the 
invariant differential of $F$ is 
$\omega = \frac{\partial F}{\partial X}(0,T)^{-1} dT$. Put 
$\log_F=\int \omega$. Then $\log_F:F\to \hat\dG_a$ is an isomorphism over 
$\dQ$. If $F$ is defined over a mixed-characteristic complete discrete 
valuation ring and $v(p)>0$, then for $r>v(p)/(p-1)$, $\log_F$ induces 
an isomorphism $F(\fm^r) \isomorphism \hat\dG_a(\fm^r)$. 

If $E$ is an elliptic curve, let $\hat E$ denote the corresponding formal 
group. A nice theorem (IV.7.4 in \cite{si09}) is that if $f:E_1\to E_2$ is an 
isogeny of characteristic $p$ elliptic curves, then 
$\deg_i(f) = p^{\height{\hat f}}$, where $\deg_i$ denotes inseparable degree. 


\subsection{Elliptic curves over \texorpdfstring{$\dC$}{C}}

Let $E$ be an elliptic curve over $\dC$, which we will take to mean that $E$ is 
a compact connected one-dimensional complex Lie group of genus one. Let 
$\fe=\lie E$. The exponential map $\exp:\fe\to E$ is a surjective homomorphism 
of Lie groups, so we get a short exact sequence 
\[
  0 \to \Lambda \to \fe \xrightarrow{\exp} E \to 0 .
\]

It is possible to go the other direction. Start with a lattice 
$\Lambda\subset \dC$, and define 
\begin{align*}
  \wp_\Lambda(z) &= \sum_{\lambda\in \Lambda\smallsetminus 0} \left((z-\lambda)^{-2} - \lambda^{-2} \right) \\
  G_k(\Lambda) &= \sum_{\lambda\in \Lambda\smallsetminus 0} \lambda^{-k} .
\end{align*}

\begin{theorem}
Let $\Lambda\subset \dC$ be a lattice. Let $g_2=60 G_2(\Lambda)$ and 
$g_3 = 140 G_6(\Lambda)$. Let $E_\Lambda$ be the elliptic curve 
$y^2=4 x^3+ g_2 x + g_3$. Then the map 
$(\wp_\Lambda:\wp_\Lambda':1):\dC/\Lambda \to E_\Lambda$ is an analytic isomorphism. 
\end{theorem}
\begin{proof}
This is Proposition VI.3.6 of \cite{si09}. 
\end{proof}

In fact, $\Lambda \mapsto E_\Lambda$ induces an equivalence of categories 
between lattices over $\dC$ and elliptic curves over $\dC$. 





\section{Representation theory}

The main references are \cite{kn79} and \cite{do97}. 


\subsection{Representations of reductive Lie groups}

Following \cite{wa88}, we say that a real Lie group $G$ is \emph{real 
reductive group} if it is a finite cover of the real points of a Zariski-closed 
subgroup of $\generallinear_n(\dC)$, defined over $\dR$, which is closed under 
conjugate-transpose. 

First, note that if $G$ is an arbitrary locally compact group, $f\in C_c(G)$, 
and $\pi$ is a continuous Banach representation of $G$, then we can put 
\[
  \pi(f) = \int_G f(g) \pi(f)\, dg .
\]
This is a representation of the algebra $C_c(G)$. We apply this construction 
to the special case where $G$ be a real reductive group with maximal compact 
$K$. Let $\widehat K$ be the unitary dual of $K$. If $\tau\in \widehat K$, 
put $\alpha_\tau(k) = \dim(\tau) \trace \tau(k^{-1})$. For any representation 
$\pi$ of $K$, set $\Pi_\tau = \pi(\alpha_\tau)$ for any $\tau\in \widehat K$. 
It turns out that $\Pi_\tau$ is a projection operator, and that 
\[
  \pi = \widehat{\displaystyle\bigoplus_{\tau\in \widehat K}} \image(\Pi_\tau) .
\]
One says that $\pi$ is \emph{admissible} if each $\image(\Pi_\tau)$ is 
finite-dimensional (equivalently, if each $\tau$ has finite multiplicity in 
$\pi$). Call a representation $\pi$ of $G$ \emph{admissible} if the restriction 
$\restrict_K \pi$ is admissible. 


\subsection{Decompositions of groups}

Let $G$ be a real reductive group with lie algebra $\fg$ and Cartan involution 
$\theta$. Put 
\begin{align*}
  \fk &= \{X\in \fg:\theta X=X\} \\
  \fp &= \{X\in \fg:\theta X=-X\} .
\end{align*}
Then $\fk$ is the Lie algebra of a ``canonical'' maximal compact $K$ of $G$. 
The \emph{Cartan decomposition} of $G$ is the fact that the map 
$\fp\times K\to G$, $(X,k)\mapsto \exp(X) k$, is a diffeomorphism. 
\textbf{Warning}: $\fp$ is \emph{not} a subalgebra of $\fg$. The next 
decomposition is trickier. Let $\fa\subset \fp$ be a maximal abelian 
subalgebra. For $\alpha\in \fa^\vee$, put 
\[
  \fg_\alpha=\{X\in \fg:\adjoint Y(X)=\lambda(Y) X\text{ for all }Y\in \fa\} .
\]
Let $R=\{\alpha\in \fa^\vee\smallsetminus 0:\fg_\alpha\ne 0\}$. It is known 
that $\fg=\fg_0\oplus \sum_{\alpha\in R} \fg_\alpha$. Let $\Delta^+\subset R$ 
be some choice of positive roots, and put 
\[
  \fn = \sum_{\alpha\in \Delta^+} \fg_\alpha .
\]
Then $\fg=\fk\oplus \fa\oplus \fn$, inducing an isomorphism 
$G=K A N$ (at least, if everything is appropriately connected). 

One interprets $A$ and $N$ as depending on $\fp$, so write $A_\fp$, $N_\fp$. 
Define 
\[
  M = M_\fp = Z_K(A_\fp) = \{k\in K:k a = a k\text{ for all }a\in A_\fp\} .
\]
Then $S=M A N$ is the ``standard minimal parabolic.'' In the main example 
$G=\speciallinear_2(\dR)$, we have 
\begin{align*}
  \fp &= \left\{\smat{a}{b}{b}{-a}:a,b\in \dR\right\} = \left\langle \smat{1}{}{}{-1},\smat{}{1}{1}{}\right\rangle \\
  \fk &= \left\langle \smat{}{1}{-1}{}\right\rangle  = \lie K = \lie\left\{k_\theta = \smat{\cos\theta}{\sin\theta}{-\sin\theta}{\cos\theta} :0\leqslant \theta < 2\pi\right\} \\
  \fa &= \left\langle \smat{1}{}{}{-1} \right\rangle  = \lie A = \lie\left\{\smat{a}{}{}{a^{-1}} : a>0\right\} \\
  \fn &= \left\langle \smat{}{1}{}{}\right\rangle = \lie N = \lie\smat{1}{\ast}{}{1} \\
  \fm &= 0 = \lie M = \lie(\pm 1)
\end{align*}
One puts $\fn_+=\fn$, $\fn_-=\sum_{\alpha\in -\Delta^+} \fg_\alpha$. 


\subsection{Induced representations}

Let $G$ be a real reductive group, and consider the canonical decomposition 
$G = K S = K M A N$. Suppose $\rho$ is a unitary representation of $S$ with 
space $V_\rho$. The induced space $\induce_S^G\rho$ is the completion of 
\[
  \{f:G\to V_\rho:f(k s) = \rho(s^{-1}) f(k)\} 
\]
with respect to the norm 
\[
  \|f\|^2 = \int_K |f(k)|^2\, dk .
\]
The action is $(\gamma \cdot f)(g) = f(\gamma^{-1} g)$. A function 
$f\in \induce_S^G\rho$ is determined by both its restrictions 
$f|_K$ and $f|_{N_-}$, so $\induce_S^G\rho$ can be realized as function spaces 
on both those subgroups of $G$. 


\subsection{Representations of \texorpdfstring{$\speciallinear_2(\dR)$}{SL2(R)}}

Let's start out with the standard families of unitary representations of 
$\speciallinear_2(\dR)$. Each of these is defined as the completion of some 
space of smooth (or analytic) functions with respect to a specified norm. 

\begin{definition}[finite-dimensional: $\finitedimensional_n$ for $n\geqslant 0$]
Let $\rho_0:\speciallinear_2(\dR)\hookrightarrow\generallinear_2(\dC)$ be the 
inclusion morphism. Put $\rho_n = \symmetricpower^n \rho_0$. The representation 
$\rho_n$ can be realized in the space $\finitedimensional_n$ of homogeneous 
polynomials $f\in \dC[x,y]$ of degree $n$. A matrix 
$\gamma\in \speciallinear_2(\dR)$ acts by $(\gamma f)(v) = f(\gamma^{-1} v)$. 
If we reinterpret $f\in \dC[x,y]_n$ as an element of $\dC[z]_{\leqslant n}$, 
then $\gamma=\smat a b c d$ acts by 
$(\gamma f)(z) = (b z+d)^n f\bigl(\frac{a z+c}{b z+d}\bigr)$. 
\end{definition}

\begin{theorem}
Every finite-dimensional irreducible representation of $\speciallinear_2(\dR)$ 
is one of $\{\finitedimensional_n:n\geqslant 0\}$. 
\end{theorem}

The finite-dimensional representations of $\speciallinear_2(\dR)$ are 
\emph{not} unitary, except for the trivial representation. 

\begin{definition}[discrete series: $\discreteseries_n^\pm$ for $n\geqslant 2$]
Let $G$ act on the upper half-plane $\fh$ in the usual way: 
$\gamma = \smat a b c d:z\mapsto\frac{a z+b}{c z+d}$. The group $G$ has a 
right action on functions on $\fh$: 
\[
  (f\cdot \gamma)(z) = \frac{1}{(c z+d)^n} f\left(\frac{a z+b}{c z+d}\right) = j(z,\gamma)^{-n} f(\gamma z) .
\]
Composing with the inverse gives us a left action: 
$(\gamma f)(z) = j(z,\gamma^{-1})^{-n} f(\gamma^{-1} z)$. The norm on 
$\discreteseries_n^+$ is 
\[
  \|f\|^2 = \int_\fh |f(z)|^2 y^n \, \frac{dx dy}{y^2}  .
\]
To be precise: 
$\discreteseries_n^+ = \{f:\fh\to \dC\text{ analytic}: \|f\|<\infty\}$. 
\end{definition}

\begin{theorem}
The representations $\discreteseries_n^\pm$ are irreducible for $n\geqslant 1$. 
\end{theorem}
\begin{proof}
We only consider $\discreteseries_n^+$. 
Let $K=\specialunitary(2)\subset \speciallinear_2(\dR)$ be the standard 
maximal compact subgroup. Its unitary representations are of the form 
$\chi_n:\smat{\cos\theta}{\sin\theta}{-\sin\theta}{\cos\theta}\mapsto e^{i n\theta}$ 
for $n\in \dZ$. Let $\Pi_n=\Pi_{\chi_n}$. This is the projection operator on 
$\discreteseries_n^+$ defined by 
\[
  (\Pi_n f)(z) = \frac{1}{2\pi} \int_0^{2\pi} e^{-i n \theta} (-z\sin\theta + \cos\theta)^{-n} f\left(\frac{z\cos\theta + \sin\theta}{-z\sin\theta + \cos\theta}\right)\, d\theta
\]
The function $f_n(z) = (z+i)^{-n}$ is an eigenvector for $\Pi_n$, and an element 
of $\discreteseries_n^+$. Moreover, a computation in complex analysis shows 
that if $f\in \discreteseries_n^+$ has $f(i)\ne 0$, then 
$\Pi_n f\in \dC\cdot f_n$. Using the transitivity of the action of 
$\speciallinear_2(\dR)$ on $\fh$, the result easily follows. 
\end{proof}


\begin{definition}[principal series: $\principalseries^{\pm, i v}$ for $v\in \dR$]
The space underlying $\principalseries^{\pm, i v}$ is $L^2(\dR)$. The action is 
\[
  (\gamma\cdot f)(x) = 
  \begin{cases}
    \frac{1}{|-b x+d|^{1+i v}} f\left(\frac{a x-c}{-b x+d}\right) & \text{if $+$} \\
    \frac{\sign(-b x+d)}{|-b x+d|^{1+i v}} f\left(\frac{a x-c}{-b x+d}\right) & \text{if $-$}
  \end{cases}
\]
These are induced from unitary representations of $S=\smat{\ast}{\ast}{}{\ast}$. 
They can be realized as functions on $K$ or $N_-$. 
\end{definition}

The only reducible principal series is 
$\principalseries^{-,0}\simeq \discreteseries_1^+ \oplus \discreteseries_1^-$. 
There is a ``nonunitary principal series'' $\principalseries^{k, z}$ for 
$k\in \dZ$, $z\in \dC$, where the space is $C_c^\infty(\dR)$, and the action 
is 
\[
  (\gamma f)(x) = \frac{\pm(-b x+d)}{|-b x+d|^{1+z}} f\left(\frac{a x-c}{-b x+d}\right) .
\]

\begin{definition}[complementary series $\complementaryseries^s$ for $0<s<1$]
Let $0<s<1$. The space underlying $\complementaryseries^s$ is the set of 
$f\in L_\text{loc}^1(\dR)$ such that 
\[
  \|f\|^2 = \int_{\dR^2} \frac{f(x)\overline{f(y)}}{|x-y|^{1-u}}\, dxdy <\infty .
\]
The action is what we would have with $\principalseries^{+, u}$, i.e. 
\[
  (\gamma f)(x) = \frac{1}{|-b x+d|^{1-u}} f\left(\frac{a x-c}{-b x+d}\right) .
\]
\end{definition}

\begin{definition}[limits of discrete series: $\discreteseries_1^{\pm}$]
The action here is the same as in the $\discreteseries_n^\pm$, but the norm is 
\[
  \|f\|^2 = \sup_{y>0} \int_\dR |f(x+i y)|^2\, dx .
\]
\end{definition}


\begin{theorem}
Every irreducible unitary representation of $\speciallinear_2(\dR)$ is one 
of the following:
\begin{itemize}
  \item $\discreteseries_n^\pm$ for $n\geqslant 2$
  \item $\discreteseries_1^\pm$
  \item $\principalseries^{\pm, i v}$ for $v\in \dR$, and $v\ne 0$ if $-$. 
  \item $\complementaryseries^s$ for $0<s<1$. 
\end{itemize}
The only isomorphisms between items in this list are: 
\begin{align*}
  \principalseries^{+, i v} &\simeq \principalseries^{+, -i v} \\
  \principalseries^{-, i v} &\simeq \principalseries^{-,-i v} .
\end{align*}
\end{theorem}


\subsection{Representations of \texorpdfstring{$\generallinear_2(\dR)$}{GL2R}}

This is discussed in section 2 of \cite{kn79}. 
Let $\speciallinear_2^\pm(\dR)=\{g\in \generallinear_2(\dR):\det g=\pm 1\}$, 
and consider $\induce_{\speciallinear_2(\dR)}^{\speciallinear_2^\pm(\dR)}\pi$ for 
irreducible unitary representations $\pi$ of $\speciallinear_2(\dR)$. We have 
\[
  \induce_{\speciallinear_2}^{\speciallinear_2^\pm}(\principalseries^{\pm, i v}) \simeq P^{\pm, i v}\oplus P^{\pm , i v}
\]
for a canonical irreducible unitary representation $P^{\pm, i v}$ of 
$\speciallinear_2^\pm(\dR)$. For $n\geqslant 2$, the representation 
$\induce_{\speciallinear}^{\speciallinear^\pm}(\discreteseries_n^\pm)$ is 
irreducible unitary. These (also with the complementary series) exhaust the 
irreducible unitary representations of $\speciallinear_2^\pm(\dR)$. The 
irreducible unitary representations of $\generallinear_2(\dR)$ are of the 
form 
\[
  \smat{x}{}{}{x} \smat a b c d \mapsto \chi(x)\pi\smat a b c d
\]
for $\chi$ a unitary character of $\dR^+$ and $\pi$ a representation of 
$\speciallinear_2^\pm(\dR)$. 


\subsection{Representations of \texorpdfstring{$\speciallinear_2(\dC)$}{SL2C}}

A good reference is II.4 of \cite{kn86}. 

\begin{definition}[principal series $\principalseries^{n, i v}$ for $n\in \dZ$ and $v\in \dR$]
The underlying space is $L^2(\dC)$. The action is, for $\gamma=\smat a b c d$: 
\[
  (\gamma f)(z) = |-b z+d|^{-2-i v}\left(\frac{-b z+d}{|-b z+d|}\right)^{-n} f\left(\frac{a z-c}{-b z+d}\right) . 
\]
\end{definition}

\begin{definition}[complementary series $\complementaryseries^s$ for $0<s<2$]
As for $\speciallinear_2(\dR)$, the space lives inside $L^1_\text{loc}(\dC)$, 
has action just like $\principalseries^{0, s}$: 
\[
  (\gamma f)(z) = |-b z+d|^{-2-s} f\left(\frac{a z-c}{-b z+d}\right) .
\]
The norm is 
\[
  \|f\|^2 = \int_{\dC^2} \frac{f(x)\overline{f(y)}}{|x-y|^{2-s}}\, dxdy .
\]
\end{definition}

\begin{theorem}
Every irreducible unitary representation of $\speciallinear_2(\dC)$ is one of 
the following:
\begin{itemize}
  \item $\principalseries^{n, i v}$ for $n\in \dZ$, $v\in \dR$ 
  \item $\complementaryseries^s$ for $0<s<2$
\end{itemize}
The only isomorphisms between items in this list are: 
\[
  \principalseries^{n, i v} \simeq \principalseries^{-n, -i v}. 
\]
\end{theorem}
\begin{proof}
This is Theorem 16.2 of \cite{kn86}. 
\end{proof}


\subsection{Representations of \texorpdfstring{$\generallinear_2(\dC)$}{GL2C}}

It is claimed in the last paragraph of \cite{kn79} that every irreducible 
unitary representation of $\generallinear_2(\dC)$ is of the form 
\[
  \smat{z}{}{}{z} \smat a b c d \mapsto \chi(z) \pi\smat a b c d ,
\]
for $z\in \dC^\times$, $\smat a b c d\in \speciallinear_2(\dC)$, and 
$\chi$ a unitary character of $\dC^\times$ that agrees with $\pi$ on 
$\smat{-1}{}{}{-1}$. 





\bibliographystyle{alpha}
\bibliography{tidbit-sources}

\end{document}
