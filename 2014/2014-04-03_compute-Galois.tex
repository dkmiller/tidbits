\documentclass{article}

\usepackage{amsmath,amssymb,amsthm,bookmark}
\usepackage[all]{xy}
\DeclareMathOperator{\gal}{Gal}
\newcommand{\dA}{\mathbf{A}}
\newcommand{\dC}{\mathbf{C}}
\newcommand{\dP}{\mathbf{P}}
\newcommand{\dQ}{\mathbf{Q}}
\newcommand{\dR}{\mathbf{R}}
\newcommand{\dZ}{\mathbf{Z}}
\newcommand{\iso}{\xrightarrow{\sim}}
\newtheorem{definition}{Definition}

\title{On computable elements of $\gal(\overline\dQ/\dQ)$}
\author{Daniel Miller}

\begin{document}
\maketitle




M. Mignotte: An inequality about factors of polynomials. [Do this for general 
number fields with height functions.] Are heights effectively computable? 

Also see \url{http://math.stackexchange.com/questions/412010}. 

And \url{http://mathoverflow.net/questions/24047/}





\section{Factoring polynomials in number fields}

\begin{definition}
A \emph{pinned number field} is a monic irreducible polynomial 
$f\in \dZ[x]$. 
\end{definition}

When we say ``let $F$ be a pinned number field,'' we mean ``let $f$ be a monic 
irreducible polynomial and $F=\dQ[x]/(f)$. 

For us, a \emph{pinned number field} consists of a number field $F$ together 
with a chosen isomorphism $\dQ[T]/f\iso F$, where $f\in \dZ[T]$ is a specified 
monic irreducible polynomial. 

Claim: the height function $H:O_F\to \dR_{>0}$ is computable. Indeed, 
\[
  H(x) = \#(O_F/x)\cdot \prod_{v\mid\infty} \max\{\|x\|_v,1\} .
\]
and both the set $F(\infty)$ of infinite places of $F$ and the 
normalized ``absolute value'' $\|x\|_v$ are computable. For any 
$c\in \dR$, we want the set $\{H<c\}$ to be effectively computable. 

Better, use Theorem 2 from \cite{mignotte-1974}: if $g=\sum g_i t^i$ is a 
factor of $f=\sum f_i t^i$ in $\dC[t]$, then
\[
  \max\{|g_i|\} \leqslant \deg(f)! \left(\sum |f_i|^2\right)^{1/2}
\]
In other words, there is an effectively computable constant 
$C(f)$ such that all coefficients in $g$ have absolute value 
$\leqslant C(f)$. 

\section{Conclusion}

The idea was to show that there is a dense computable subset of 
$\gal(\overline\dQ/\dQ)$. It turns out the idea was already figured out by 
Greg Kuperberg. See his answer to \url{http://mathoverflow.net/questions/6802}. 





\bibliographystyle{alpha}
\bibliography{tidbit-sources}

\end{document}
