\documentclass{article}

\title{Abstract class field theory}
\author{Daniel Miller}

\usepackage{amsmath,amssymb,amsthm,fullpage}
\usepackage[all]{xy}
\newcommand{\bb}{\mathbb}
\newcommand{\f}{\mathfrak}
\newcommand{\frob}{F}
\newcommand{\hZ}{\hat{\mathbb{Z}}}
\newcommand{\tH}{\hat H}
\newtheorem{theorem}{Theorem}

\begin{document}
\maketitle





This is a reworking of the abstract class field theory in Neukirch's book 
\emph{Algebraic Number Theory}. 





\section{The setup}

Let $G$ be a profinite group. We will denote closed subgroups of $G$ by 
lowercase letters $g$, $h$, \ldots. If $g,h$ are two subgroups of $G$, then 
$(g,h)$ will denote the closed subgroup generated by $g$ and $h$. Let $A$ be a 
continuous $G$-module, written additively. If $g\subset G$, write $A^g$ for 
the submodule of $A$ consisting of elements fixed by $g$. If $g\subset h$ and 
the index $[h:g]$ is finite, we have the \emph{norm map} 
\[
  N_{h/g} : A^g\to A^h\qquad\qquad
  N_{h/g} x = \sum_{\sigma\in h/g} \sigma x
\]

Suppose we have a continuous surjective homomorphism $d:G\to \hZ$. We will 
write $I = \ker d$ and $\tilde g = g\cap I$. If $g\subset G$ is of finite 
index, then $d g\subset \hZ$ is also of finite index. We define 
\begin{align*}
  e_g &= [I : \tilde g] \\
  f_g &= [\hZ : dg]
\end{align*}
We define $d_g = \frac{1}{f_g}$; the map $d_g:g\to \hZ$ is also a continuous 
surjection. There are relative ``ramification'' and ``inertia'' degrees:
\begin{align*}
  e_{h/g} &= [\tilde h : \tilde g] \\ 
  f_{h/g} &= [d h : d g]
\end{align*}

\begin{theorem}
If $g\subset h$ are subgroups of $G$, we have $[h:g] = e_{h/g} f_{h/g}$. 
\end{theorem}
\begin{proof}
If $g$ is normal in $h$, the following exact sequence 
\[
  1 \to \tilde h/\tilde g \to h/g \to d h/d g \to 1
\]
yields the desired identity. For the general case, use the transitivity of $e$ 
and $f$.
\end{proof}

We also define the \emph{Weil group} $W=d^{-1}(\bb Z)$. As usual, there is a 
relative version:
\[
  W_{h/\tilde g} = d_h^{-1}(\bb Z) \subset h/\tilde g
\]
I should prove that the obvious map $W_{h/\tilde g} \to h/g$ is surjective. In 
fact, we have the following general theorem:

\begin{theorem}
Let $d:G\to H$ be a continuous surjective homomorphism between profinite 
groups. If $X\subset H$ is dense and $W=d^{-1}(X)$, then $X$ is dense in $G$. 
\end{theorem}
\begin{proof}
Let $g$ be an open normal subgroup of $G$; we have to show that $W\to G/g$ is 
surjective. Note that for $I=\ker d$ we have $(g,I) = g I$ because $g$ is open 
and the two subgroups are normal. Since $G/g I$ is a finite quotient of 
$G/I\simeq H$, the map $W\to G/g I$ is surjective. Thus for $\sigma\in G/g$, 
we have $\sigma = w$ in $G/g I$ for some $w\in W$. In other words, 
$\sigma w^{-1} = \tau i$, where $\tau\in g$ and $i\in I$. But then in $G/g$, 
we have $\sigma = iw$ which is in the image of $W$. 
\end{proof}





\section{Abstract valuation theory}

If $g\subset G$, then $d_g$ induces an isomorphism $g/\tilde g \to \hZ$. The 
element of $g$ corresponding to $1\in \hZ$ will be written $\phi_g$ and called 
the \emph{Frobenius} of $g$. We let $\frob_{h/\tilde g} = d_h^{-1}(\bb N)$. 
Since $\bb N$ surjects onto $\bb Z/n$ for all $n$, $\bb N$ is dense in $\hZ$, 
whence $\frob_{h/\tilde g}$ is dense. 

\begin{theorem}
Let $\sigma\in \frob_{h/\tilde g}$, and let $l=(\sigma,\tilde g)$. Then 
\begin{enumerate}
  \item $f_{h/l} = d_h(\sigma)$ 
  \item $[h:l]<\infty$
  \item $\tilde l = \tilde g$
  \item $\sigma = \phi_l$. 
\end{enumerate}
\end{theorem}
\begin{proof}
1. We have $f_{h/l} = [d h : d l] = [\hZ : d_h\sigma] = d_h\sigma$.

2. From the fundamental identity $[h:l] = e_{h/l} f_{h/l}$, it is sufficient to 
prove $e_{h/l}<\infty$. But 
$e_{h/l} = [\tilde h:\tilde l]\leqslant [\tilde h:\tilde g] = e_{h/g}<\infty$. 

3. There is an obvious surjection 
$l/\tilde g\twoheadrightarrow l/\tilde l\simeq \hZ$. But $l/\tilde g$ is 
procyclic, being generated by $\sigma$, and it is a general theorem that if a 
surjection between procyclic groups must be an isomorphism. 

4. The group $d l$ is generated by $d_l(\sigma)$, so when we normalize, 
$d_l(\sigma) = 1$, i.e. $\sigma = \phi_l$. 
\end{proof}

We now consider valuations on the $G$-module $A$. A \emph{Henselian valuation} 
on $A$ is a homomorphism $v:A^G\to \hZ$ with image containing $\bb Z$, such 
that for all finite index $g\subset G$, we have 
\[
  v(N_{G/g} A^g) = f_g v(A^G)
\]
This lets us define the valuations $v_g = \frac 1{f_g} v\circ N_{G/g}$; these 
satisfy the obvious compatibility properties. 

We call $\pi\in A^g$ a \emph{prime} if $v_g(\pi) = 1$, and write 
$U_g=\ker v_g$. We will wrote $\pi_g$ for the prime of $g$.





\section{The reciprocity map}

Recall that if $G$ is a finite group, the \emph{Tate cohomology} of a 
$G$-module $A$ is a $\bb Z$-graded abelian group $\tH^\bullet(G,A)$ defined as 
follows:
\[
  \tH^n(G,A) = \begin{cases}
                 H^n(G,A)         & \mbox{if $n>0$} \\
                 A^G / N A        & \mbox{if $n=0$} \\
                 _N A/\f j_G A    & \mbox{if $n=-1$} \\
                 H_{-n-1}(G,A)    & \mbox{if $n<-1$}
               \end{cases}
\]
Here $N = N_G = \sum_{\sigma \in G} \sigma$, $\f j_G$ is the \emph{augmentation 
ideal} generated by $\{\sigma-1 : \sigma\in G\}$, and 
$_N A = \{x\in A : N x = 0\}$. It turns out (though we will not use this fact) 
that $\tH^\bullet$ forms a cohomology theory. We will only use $\tH^0$ and 
$\tH^{-1}$. 

Returning to the case where $G$ is profinite: from here on out, we will assume 
the following axiom:
\[
  \tH^i(h/g,U_g) = 0 \qquad\mbox{for $i\in\{-1,0\}$}
\]

If $h\supset g\supset k$ are subgroups of $G$ with $[h:k]<\infty$, then 
$N_{h/k} = N_{h/g}N_{g/k}$ implies $N_{h/k} A^k\subset N_{h/g} A^g$. Set, for 
arbitrary inclusions $h\supset g$:
\[
  N_{h/g} A^g = \bigcap_k N_{h/k} A^k
\]
where $k$ ranges over $h\supset k\supset g$ with $[h:k]<\infty$. The previous 
remark shows that this agrees with the usual definition if we already have 
$[h:g]<\infty$. 

We define the \emph{reciprocity map} first in a special case. It will be, for 
$G\supset h\supset g$ with $[G:h]$ finite, a map 
\[
  r_{h/\tilde g} : \frob_{h/\tilde g} \to A^h/N_{h/\tilde g} A^{\tilde g}
\]
For $\sigma\in \frob_{h/\tilde g}$, let $l=(\sigma,\tilde g)$. A previous 
theorem shows that $[h:l]<\infty$, so it makes sense to define $v_l$ and 
say that 
\[
  r_{h/\tilde g}(\sigma) = N_{h/l}(\pi_l) \mod N_{h/\tilde g} A^{\tilde g} 
\]
First, we need to show that this is independent of the choice of $\pi_l$. Two 
different choices will differ by an element $u\in U_l$; it suffices to prove 
that $N_{h/l} u\in N_{h/\tilde g} A^g$. So we need $N_{h/l} u\in N_{h/k} A^k$ 
for all $h\supset k\supset \tilde g$ with $[h:k]<\infty$. Replacing $k$ by 
$k\cap l$ if necessary, we may assume $k\subset l$. We then apply the axiom 
$\tH^0(l/k,U_k) = 0$ to find $x\in U_k$ with $N_{l/k} x = u$. It follows that 
$N_{h/l} u = N_{h/l} N_{l/k} x = N_{h/k} x\in N_{h/k} A^k$, so $r_{h/\tilde g}$ 
is well-defined. 

For the remainder of this section, fix $g\lhd h\subset G$ with 
$[G:g]<\infty$. We set $N = N_{\tilde h/\tilde g}$, and for general $\sigma$ we 
set $\sigma_n = 1+\cdots + \sigma^{n-1}$. This yields the formal identity 
\[
  (\sigma-1)\sigma_n = \sigma_n(\sigma-1) = \sigma^n-1
\]

\begin{theorem}
Fix $\phi,\sigma\in \frob_{h/\tilde g}$ with $d_h \phi = 1$ and 
$d_h \sigma = n$. If $l=(\sigma,\tilde g)$, then 
\[
  N_{h/l} = \phi_n N = N \phi_n
\]
\end{theorem}
\begin{proof}
Let $l=(\sigma,\tilde h)$; we clearly have $N_{h/l} = N_{h/l_0} N_{l_0/l}$. 
Moreover, it is easy to see that $N_{h/l_0} = \phi_n$. To see that 
$N_{\tilde h/\tilde g} = N_{l_0/l}$, one simply needs to check that 
$l_0 = (l,\tilde h)$ and $\tilde g = l\cap \tilde h$, which is not hard. 
Finally, $\phi_n$ and $N$ commute because $\tilde h$ is normal in $h$. 
\end{proof}

For arbitrary groups $G$ and $G$-modules $A$, let $H_0(G,A) = A/\f j_G A$. In 
our case, we will consider $H_0(\tilde h/\tilde g,U_{\tilde g})$ and set 
$\f j = \f j_{\tilde h/\tilde g}$. It is easy to check that $N$ descends to a 
map $H_0(\tilde h/\tilde g,U_{\tilde g}) \to U_{\tilde h}$. In fact, we have 
the following 

\begin{theorem}
Suppose $\phi\in h$ has $d_h \phi = 1$. Then $N$ restricts to a map 
\[
  N : H_0(\tilde h/\tilde g,U_{\tilde g})^\phi \to N_{h/\tilde g} U_{\tilde g}
\]
\end{theorem}
\begin{proof}
First, note that the action of $h$ on $U_{\tilde g}$ is well-defined because 
$\tilde g$ is normal in $h$. Suppose $x = \bar u\in U_{\tilde g}$ with 
$\phi x = x$. Then we have 
\[
  (\phi-1) u = \sum (\tau_i - 1) u_i
\]
for some $\tau_i\in \tilde h$ and $u_i\in U_{\tilde g}$. We wish to show that 
$N u\in N_{h/m} U_m$ for all $h\supset m\supset \tilde g$ with $[h:m]<\infty$. 
It is clearly sufficient to assume $m\subset g$. 
\end{proof}






\end{document}
