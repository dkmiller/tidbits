\documentclass{article}

\usepackage{amsmath,amssymb,amsthm,bookmark,fullpage,microtype,thmtools}
\DeclareMathOperator{\gal}{Gal}
\DeclareMathOperator{\h}{H}
\DeclareMathOperator{\trace}{tr}
\newcommand{\fa}{\mathfrak{a}}
\newcommand{\fA}{\mathfrak{A}}
\newcommand{\fb}{\mathfrak{b}}
\newcommand{\fC}{\mathfrak{C}}
\newcommand{\fD}{\mathfrak{D}}
\newcommand{\fo}{\mathfrak{o}}
\newcommand{\fO}{\mathfrak{O}}
\newcommand{\fp}{\mathfrak{p}}
\newcommand{\fP}{\mathfrak{P}}
\newcommand{\iso}{\xrightarrow\sim}
\newtheorem{definition}[subsection]{Definition}
\newtheorem{theorem}[subsection]{Theorem}

\usepackage[
  hyperref = true,      % links to online documents
  backend  = bibtex,    % use bibtex instead of biber
  sorting  = nyt,       % sorts by (name, year, title)
  style    = alphabetic % citations look like [Har77]
]{biblatex}
\addbibresource{tidbit-sources.bib}
\hypersetup{
  colorlinks = true,
  linkcolor  = blue,
  urlcolor   = cyan
}

\title{A pseudo-associated order}
\author{Daniel Miller}
\date{November 12, 2012}

\begin{document}
\maketitle





This note is essentially an explication of the arguments in the first section 
of Bondarko's paper \cite{bondarko-2000}. 





\section{Duality for group algebras}

Let $k$ be a commutative ring, $\Gamma$ a finite group. We can consider the 
group algebra $k[\Gamma]$ as a Hopf algebra in the usual way. The dual of 
$k[\Gamma]$ is spanned as a $k$-module by the functionals 
$\hat\sigma(\tau)=\delta_{\sigma\tau}$. Instead of using different notation for 
$k[\Gamma]$ and its dual, we will use the $\ast$ to denote the product in the 
dual of $k[\Gamma]$. In other words, 
\[
  \left(\sum a_\sigma \sigma\right) \ast \left(\sum b_\sigma \sigma\right) = \sum a_\sigma b_\sigma \sigma .
\]

For the rest of this note, the \emph{standard situation} consists of a 
Dedekind domain $\fo$ with field of fractions $k$, along with a finite Galois 
extension $K/k$. Put $\Gamma=\gal(K/k)$, and write $\fO$ for the integral 
closure of $\fo$ in $K$. For fractional ideals $\fa,\fb\subset\fO$, put 
\[
  \fA(\fa,\fb)=\{f\in k[\Gamma]:f\fa\subset \fb\} .
\]
Write $\fA=\fA(\fO,\fO)$. Note that $\fA$ is not, in general, closed under the 
multiplication $\ast$. We will define an $\fo$-subalgebra 
$Q\subset k[\Gamma]$ that is closed under $\ast$, and show that in certain 
instances, $Q$ is monogenic under $\ast$, and gives an easily computable 
$\fo$-basis for $\fA$. 





\section{General nonsense with projectives}

As before, let $k$ be an arbitrary commutative ring. If $E$ is a $k$-module, 
write $E^\vee=\hom_k(E,k)$. If $E,F$ are finitely generated projective 
$k$-modules, there is a natural isomorphism 
\[
  E^\vee\otimes F\iso \hom(E,F) \qquad f\otimes x\mapsto f(-) x .
\]

Return to the standard situation. There is a natural isomorphism  
$K[\Gamma]\to \hom_k(K,K)$. For fractional ideals $\fa,\fb\subset K$ put 
\[
  \fC(\fa,\fb) = \hom_\fo(\fa,\fb) = \{f\in K[\Gamma]:f\fa\subset \fb\} .
\]
If $\fa\subset K$ is a fractional ideal, we may identify $\fa^\vee$ with 
$\fa^\ast:=\{x\in K:\trace(x\fa)\subset \fo\}$ via 
\[
  \fa^\ast \to \fa^\vee \qquad x\mapsto \trace(x-) .
\]
In light of the above identifications, we have the following isomorphism of 
$\fo$-modules: 
\[
  \fa^\ast\otimes \fb \iso \fa^\vee\otimes \fb \iso \hom_\fo(\fa,\fb)=\fC(\fa,\fb) \qquad x\otimes y\mapsto \sum \sigma(x) y\sigma .
\]
This suggests we define a map 
\[
  \phi:K\otimes K\to K[\Gamma] \qquad x\otimes y\mapsto \sum \sigma(x) y \sigma .
\]
It is well-known that if we give $K[\Gamma]$ the multiplication $\ast$, then 
$\phi$ is an isomorphism of $k$-algebras. 

Write $\fD^{-1}=\fO^\ast$ for the inverse different of the extension $K/k$. 

\begin{definition}
$Q=\fC(\fD^{-1},\fO)\cap k[\Gamma]$. 
\end{definition}

Since $Q=\fO\otimes \fO$ as a subset of $K\otimes K$, it is closed under 
$\ast$. 





\section{Local fields}

We remain in the standard situation, and moreover assume that $K/k$ is a 
totally ramified extension of local fields. Write $\fp$ (resp.~$\fP$) for the 
maximal ideal of $\fo$ (resp.~$\fO$). We also assume that 
$\fD=\delta\fO$ for some $\delta\in k$. We call such extensions \emph{perfectly 
ramified}. In this case, one has $Q=\delta\fA$. As a result, $\fO$ is free over 
$\fA$ if and only if $\fO=Qx$ for some $x\in \fD^{-1}$. 

For the moment, we drop the assumption that $K/k$ is perfectly ramified (but 
still assume it is totally ramified). Let $n=[K:k]$ and let $v$ be the 
normalized valuation on $K$. Put $d=v(\fD)$ and $r=n-d-1$. 

\begin{theorem}
If $v(a)=r$, then $\trace a\in \fo^\times$. 
\end{theorem}
\begin{proof}
Let $\pi\in K$ be a uniformizer, $f\in \fo[X]$ its minimal polynomial. It is 
well-known that $d=v(f'(\pi))$. So, if we let $x=\pi^{n-1}/f'(\pi)$, then 
$v(x)=n-d-1$, and by a theorem of Euler, $\trace x=1$. If $v(y)=v(x)$, then 
$y=\varepsilon x+z$, where $\varepsilon\in \fo^\times$ and $v(z)>0$. Then 
$v(\trace z)>0$, so $\trace y=\varepsilon + \trace z\in \fo\times$. 
\end{proof}

\begin{theorem}\label{thm:compute-val}
Let $\alpha=1\otimes x+\sum x_i\otimes y_i\in \fO\otimes\fO$ with $v(x)<n$ and 
$v(x_i)>0$. If $v(a)=r$, then $v(\phi(\alpha)(a))=v(x)$. 
\end{theorem}
\begin{proof}
Consider the following computation, where $\pi_0$ is a uniformizer in $\fo$:
\begin{align*}
  \phi(\alpha)(a) 
    &= x\trace(a) + \sum \trace(x_i a) y_i \\
    &= x\trace(a) + \sum \pi_0 y_i \trace\left(\frac{x_i a}{\pi_0}\right) .
\end{align*}
Since $v(a)=r$, the arguments of the trace on the right have valuation at 
least $-d$, so the sum on the right has valuation $>n$, whence the result. 
\end{proof}

In particular, if $\trace a=1$, then $\phi(\alpha)(a)\equiv x\pmod{\fP^e}$, where 
$e=v(\fp)$ is the index of ramification. 

\begin{theorem}\label{thm:val-product}
Let $f,g\in Q$ and $v(a)=r$. If $v(f(a))=i$, $v(g(a))=j$ and 
$i,j,i+j<n$, then $v((f\ast g)(a))=i+j$. 
\end{theorem}
\begin{proof}
Write $f=\phi(\alpha)$, $g=\phi(\beta)$ where 
$\alpha=1\otimes x+\sum x_i\otimes y_i$ and 
$\beta=1\times u+\sum u_j\otimes v_j$ as in 
\autoref{thm:compute-val}. Then $v(x)=i$ and $v(y)=j$. We have 
\[
  f\ast g=\phi(\alpha\beta) = 1\otimes u x+\sum x_i u_j\otimes y_i v_j .
\]
By \autoref{thm:compute-val}, it follows that $(f\ast g)(a)$ has valuation 
$v(u x)=i+j$. 
\end{proof}

As a matter of fact, note that if $\trace a=1$, then 
$(f\ast g)(a)\equiv f(a) g(a)\pmod{\fP^e}$. 

For $f\in K[\Gamma]$, define $f^{\ast 0}=\trace$, and 
$f^{\ast(i+1)}=f\ast f^{\ast i}$. If $f\in Q$, then $f^{\ast i}\in Q$ for all 
$i$. Suppose $v(f(a))=1$ for some $a$ with $v(a)=r$. Then by 
\autoref{thm:val-product}, $f^{\ast i}(a)$ has valuation $i$ for all 
$0\leqslant i < n$. Thus $\{f^{\ast 0}(a),\dots,f^{\ast(n-1)}(a)\}$ is an 
$\fo$-basis for $\fO$. 

\begin{theorem}
If $v(a)=r$ and $f(a)$ is a uniformizer, then 
$\{f^{\ast 0},\dots,f^{\ast(n-1)}\}$ is an $\fo$-basis for $Q$. 
\end{theorem}
\begin{proof}
First note that the $f^{\ast i}$ form a $k$-basis for $K[\Gamma]$. Moreover, 
for $g\in k[\Gamma]$, $g=0$ if and only if $g(a)=0$. Indeed, writing  
$g=\sum a_i f^{\ast i}$, we get $g(a)=\sum a_i f^{\ast i}(a)$. The $i$th term 
on the right has valuation $i\pmod n$, so the only way $g(a)$ can be zero is 
for each $a_i=0$. Thus if $g\in Q$, then we have $g(a)=a_i f^{\ast i}(a)$ for 
some $a_i\in \fo$ for each $0\leqslant i<n$. It follows that 
$g-\sum a_i f^{\ast i}$ kills $a$, whence $g=\sum a_i f^{\ast i}$. 
\end{proof}





\section{Back to the associated order}

Suppose $K/k$ is perfectly ramified, that is $\fD=\fO\delta$ for $\delta\in k$. 
Then $\fA=\delta^{-1} Q$. So, if there exists $f\in Q$ such that $f(a)$ is 
prime for some $a$ with $v(a)=r$, we have $\fA\cdot a=\fO$, and 
$\{\delta^{-1}f^{\ast i}\}$ forms an $\fo$-basis for $\fA$. 

\begin{theorem}
$\fA$ is Hopf if and only if $\Delta(\delta^{-1}\trace)\in \fA\otimes\fA$. 
\end{theorem}
\begin{proof}
Since $\Delta$ is $k$-linear, it suffices to prove that 
$\Delta(Q)\subset \fA\otimes Q$ if and only if 
$\Delta(\trace)\in \fA\otimes Q$. In fact, this always holds (even if $K/k$ is 
not perfectly ramified). One implication is obvious. To prove the other, we 
show that for $f\in k[\Gamma]$, we have 
\[
  \Delta(f) = (f\otimes \trace)\ast \Delta(\trace) .
\]
Indeed, writing $f=\sum a_\sigma \sigma$, we compute 
\begin{align*}
  \Delta(f) 
    &= \sum a_\sigma \sigma\otimes \sigma \\
    &= \left( \sum_{a,\tau}a_\sigma \sigma\otimes \tau\right) \ast \left(\sum_\gamma \gamma\otimes \gamma\right) \\
    &= (f\otimes \trace)\ast \Delta(\trace). 
\end{align*}
Since $\Delta(\trace)\in \fA\otimes Q$ and 
$(Q\otimes Q)(\fA\otimes Q)\subset \fA\otimes Q$, the result follows. 
\end{proof}





\section{Functoriality}

Finally, we show that $Q$ is functorial in $K$. Let $K'/K$ be a finite Galois 
extension, $\Gamma'=\gal(K'/K)$, and $G=\gal(K'/k)$. There is an obvious 
functorial surjection $\Gamma'\twoheadrightarrow\Gamma$. This extends to a 
(functorial) map of $k$-Hopf algebras $k[\Gamma']\to k[\Gamma]$. It induces 
an injection $\rho:k[\Gamma]^\vee\to k[\Gamma']^\vee$, via 
\[
  \rho(\sigma) = \sum_{\tau\mapsto\sigma} \tau .
\]
One can easily check that if $f\in Q_K$, then $\rho(f)\in Q_{K'}$. Since the 
maps $\Gamma'\to \Gamma$ are functorial, so are the $\rho$. To conclude, $Q$ 
defines a functor from the category of finite Galois extensions of $k$ to the 
category of $\fo$-algebras. 





\printbibliography

\end{document}
