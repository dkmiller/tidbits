\documentclass{article}

\title{Compatible families of Galois representations}
\author{Daniel Miller}

\usepackage{amsmath,amssymb,fullpage}
\DeclareMathOperator{\gl}{GL}
\newcommand{\dA}{\mathbb{A}}
\newcommand{\dB}{\mathbb{B}}
\newcommand{\fo}{\mathfrak{o}}
\newcommand{\frob}{\varphi}
\newcommand{\scomp}{\mathsf{C}}

\begin{document}
\maketitle





Let $K/k$ be an extension of global fields. We say that a continuous 
homomorphism $\rho:G_K\to \gl(k_v)$ is \emph{rational at $w$} if $\rho(I_w)=1$ 
and the polynomial 
\[
  \Phi_{\rho,w} = \det(T\cdot 1-\rho(\frob_w)) \in k_v[T]
\]
is actually in $K[T]$. 

Now let $\{\rho_v:G_K\to \gl_n(k_v)\}$ be a collection of continuous 
homomorphisms, where $v$ ranges over the places of $k$. We say that 
$\{\rho_v\}$ is a \emph{(strictly) compatible system} if there is a finite set 
$S$ of places of $k$ such that 
\begin{enumerate}
  \item if $v\notin S$, then $\rho_v$ is rational at $w$ for all $w\nmid v$
  \item if $u,v\notin S$, then for all $w\nmid u,v$, we have 
    $\Phi_{\rho_u,w} = \Phi_{\rho_v,w}$
\end{enumerate}
We will write $\rho=\rho_\bullet=\{\rho_v\}$ for such a family. We let 
$\scomp$ be the set of all strictly compatible systems. Note that if 
$\rho=\{\rho_v\}\in\scomp$, there is a well-defined positive integer 
$d=\dim\rho$ defined by $d=\dim\rho_v$ for every $v$. 

We would like $\scomp$ to be a neutral Tannakian category, as this would 
allow us to define a group ``$\pi_1(\scomp)$'' which would classify strictly 
compatible families of representations. 

The problem is: \emph{what is the correct notion of a morphism 
$f:\alpha\to\beta$ for $\alpha,\beta\in\scomp$?} Surely it involves a 
collection $\{f_v:\alpha_v\to \beta_v\}$ of $G_K$-linear maps. But we would 
also want some kind of ``compatibility condition.'' If we allow the $f_v$ to 
be arbitrary, then $\scomp$ does not have quotients (just let $f_v=0$ for half 
of the $v$, and $f_v=1$ for the other half). We will attempt to imitate 
Deligne's construction of mixed motives in \cite[1.4]{De}. Let $\dA^f=\dA_k^f$ 
be the ring of finite adeles over $k$. Recall that elements of $\dA^f$ are 
collections $(a_v)\in \prod k_v$ with $a_v\in\fo_v$ for all but finitely many 
$v$. Instead of a family $\{\rho_v:G_K\to \gl_n(k_v)\}$ we will look at a 
single (continuous) representation $\rho:G_K\to \gl_n(\dA^f)$. Of course, this 
means we have to say what the topology on $\dA^f$ is. Essentially, note that 
(as a set) $\dA^f = \varinjlim \dA^f(S)$, where $S$ ranges over all finite 
sets of places of $k$ and 
$\dA^f(S)=\prod_{v\in S} k_v \times \prod_{v\notin S} \fo_v$. We simply 
require that each $\dA^f(S)$ be an open subring of $\dA^f$. We now need to 
decide what it means for $\rho:G_K\to\gl_n(\dA^f)$ to be ``unramified'' or 
``compatible.'' 

First, we introduce a new ring $\dB$. As a set, $\dB$ consists of all 
sequences $(a_v)\in \dA_k^f$ such that there is some finite set $S$ of places 
of $k$, and $a\in K$ such that $a_v=a$ for all $v\notin S$. We assume $K/k$ is 
Galois -- that makes the field ``$K\cap k_v$'' well-defined. We give 
$\dB\subset \dA^f$ the subspace topology. It is a corollary of the strong 
approximation theorem \cite[III.1, ex.1]{Neu} that $\dB$ is dense in $\dA^f$. 
I would like to say that a ``compatible family of representations'' is a 
continuous representation $\rho:G_K\to \gl_n(\dA^f)$ with the characteristic 
polynomial of $\rho(\frob_w)$ an element of $\dB[T]$ for all but finitely 
many $w$. This, however, includes no information about ramification. So, we 
will keep that information separate, i.e. 

A \emph{strictly compatible family} is a continuous representation 
$\rho:G_K\to \gl_n(\dA_k^f)$ such that there exists a finite set $S$ with 
\begin{enumerate}
  \item for $v\notin S$, $\rho_v$ is unramified at $w$ for all $w\nmid v$ 
  \item for $u,v\notin S$ and $w\nmid u,v$, $\Phi_{\rho_u,w}=\Phi_{\rho_v,w}$. 
\end{enumerate}
We may have to stipulate that if $v\in S$ and $u$ has the same residue 
characteristic as $v$, then $u\in S$. 

The ring $\dB$ may still be useful -- this time for defining morphisms 
in $\scomp$. A preliminary definition is follows: a morphism $f:\rho\to\eta$ 
is a $G_k$-linear map such that if $n=\dim \rho$, $m=\dim\eta$, we have 
\[
  f\in M_{m\times n}(\dB)
\]
It is easy to check that if $\rho,\eta\in \scomp$, then 
$\rho\oplus\eta\in \scomp$. Naively, we will define $\rho^*$ in the usual way 
-- the only question is whether the characteristic polynomials of Frobenii 
behave well. Suppose $A$ is an invertible matrix. Then the characteristic 
polynomial of $A$ is $\prod (T-\lambda)$ where $\lambda$ runs over the 
eigenvalues of $A$ (with multiplicity). If $Ax=\lambda x$, then 
$A^{-1} x = \lambda^{-1} x$, so the characteristic polynomial of 
$A^{-1}$ is $\prod (X-\lambda^{-1})$. Essentially by the fundamental theorem 
of symmetric polynomials, this will be expressible in terms of the 
coefficients of the characteristic polynomial of $A$. Thus if 
$\Phi_{\rho_v,w}$ is $K$-rational, so will be $\Phi_{\rho_v^*,w}$. Moreover, 
$\Phi_{\rho_v^*,w}$ only depends on the coefficients of $\Phi_{\rho_v,w}$, so 
$\rho^*\in \scomp$ whenever $\rho\in\scomp$. Finally, similar considerations 
using symmetric polynomials show that if $\rho,\eta\in \scomp$, so is 
$\rho\otimes\eta$. We can define $\hom(\rho,\eta)=\rho^*\otimes\eta$, but I 
think that this forces us to not have quotients in $\scomp$.





\begin{thebibliography}{9}
  \bibitem{De} Deligne, P. \emph{Le groupe fondamental de la droite 
    projective moins trois points}. 
  \bibitem{Neu} Neukirch, J. \emph{Algebraic number theory}. 
\end{thebibliography}





\end{document}
