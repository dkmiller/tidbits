\documentclass{article}

\usepackage[a5paper, total={4.7in,7.25in}]{geometry}
\usepackage{amsmath,amssymb,amsthm,mathrsfs,mathtools,stmaryrd}
\DeclareMathOperator{\pr}{pr}
\newcommand{\bZ}{\mathbf{Z}}
\newcommand{\cO}{\mathcal{O}}
\newcommand{\dd}{\mathrm{d}}
\newcommand{\sC}{\mathscr{C}}
\newcommand{\sD}{\mathscr{D}}
\newcommand{\amice}[2][]{\prescript{#1}{}{\mathrm{A}_{#2}}}
\newcommand{\pow}[1]{\llbracket #1 \rrbracket}
\newtheorem{lemma}[subsection]{Lemma}
\newtheorem{theorem}[subsection]{Theorem}
\theoremstyle{definition}
\newtheorem{definition}[subsection]{Definition}

\title{Distributions and the Amice transform}
\author{Daniel Miller}

\begin{document}
\maketitle





All topological spaces are tacitly assumed to be Hausdorff. Let $X$ be a 
topological space, $\cO$ a topological ring. 





\section{Topological preliminaries}

\begin{definition}
We write $\sC^0(X,\cO)$ for the set of continuous functions $f\colon X\to\cO$. 
We give $\sC^0(X,\cO)$ the compact-open topology, i.e.~the topology generated 
by sets of of the form
\[
  B_{C,I} = \{f\in \sC(X,\cO)\colon f(U)\subset I\} ,
\]
for any compact $C\subset X$, $I\subset \cO$. 
\end{definition}

Note that a \emph{basis} of open sets for $\sC^0(X,\cO)$ is given by finite 
intersections of the $B_{C,I}$. By definition of ``topology generated by,'' to 
show a map $\phi\colon Y\to \sC^0(X,\cO)$ is continuous, it suffices to show 
that $\phi^{-1} B_{C,I}$ is open for all $C,I$. 

\begin{lemma}
The natural map $\cO\times \sC^0(X,\cO)\to \sC^0(X,\cO)$ is continuous. 
\end{lemma}
\begin{proof}
Let $a\in \cO$, $f\in \sC^0(X,\cO)$ such that $a f\in B_{C,I}$. Since 
multiplication $\cO\times \cO\to \cO$ is continuous, for each $c\in C$, there 
exists open neighborhoods $J_c\ni a$, $J_c'\ni f(c)$ such that 
$J_c\cdot J_c'\subset I$. By compactness of $C$, we get open $J\ni a$, 
$J'\supset f(C)$ such that $J\cdot J'\subset I$. It follows that 
\[
  J\cdot B_{C,J'}\subset I ,
\]
and since $f\in B_{C,J'}$, we are done. 
\end{proof}

Thus for any space $X$, the module $\sC^0(X,\cO)$ is a topological 
$\cO$-module. 

\begin{lemma}
Let $\phi\colon X\to Y$ be a continuous map of topological spaces. Then 
$\phi^\ast\colon \sC^0(Y,\cO)\to \sC^0(X,\cO)$, $f\mapsto f\circ\phi$, is 
continuous. 
\end{lemma}
\begin{proof}
Note that for compact $C\subset X$ and open $I\subset \cO$, we have 
\[
  (\phi^\ast)^{-1} B_{C,I} 
    = \{f\in\sC(Y,\cO)\colon f(\phi(C))\subset I\} 
    = B_{\phi(C),I} .
\]
Since $\phi(C)$ is compact, we are done. 
\end{proof}

To sum things up: $\sC^0(-,\cO)$ is a contravariant functor from 
(Hausdorff) topological spaces to topological $\cO$-modules. If $\cO$ is 
linearly topologized, then $\sC^0(-,\cO)$ takes values in linearly topologized 
$\cO$-modules. Clearly the same proofs work for $\sC^0(-,M)$ and any 
topological $\cO$-module $M$. 

\begin{definition}
Write $\sD_0(X,\cO)$ for the continuous dual of $\sC^0(X,\cO)$. That is, an 
element $\mu\in \sD_0(X,\cO)$ is a continuous linear functional 
$\sC^0(X,\cO)\to \cO$. One often writes 
\[
  \int_X f(x)\, \dd \mu(x) = \mu(f) ,
\]
for $f\in \sC^0(X,\cO)$. 
\end{definition}

\begin{lemma}
Let $\phi\colon X\to Y$ be a proper map. Then 
$\phi_\ast\colon \sD_0(X,\cO)\to \sD_0(Y,\cO)$, given by 
$(\phi_\ast\mu)f=\mu(\phi^\ast f)$, is well-defined. 
\end{lemma}
\begin{proof}
Clearly the expression $(\phi_\ast \mu)f=\mu(\phi^\ast f)$ is well-defined. 
What we need to check is that $\phi_\ast\mu$ is also a distribution. Let 
$I\subset \cO$ be open. Since $\mu$ is a continuous, there exists compact 
$C_i\subset X$ and open $J_i\subset \cO$ such that 
$\mu(\bigcap B_{C_i,J_i})\subset I$. 
Note that 
\begin{align*}
  (\phi_\ast \mu)\left(\bigcap B_{\phi(C_i),J_i}\right)
    &= \mu\left(\phi^\ast\left(\bigcap B_{\phi(C_i),J_i}\right)\right) \\
    &\subset \mu\left(\bigcap B_{\phi^{-1}(\phi(C_i)),J_i}\right) \\
    &\subset \mu\left(\bigcap B_{C_i,J_i}\right) .
\end{align*}
We used the properness of $\phi$ in that $\phi^{-1}(\phi(C_i))$ is continuous. 
\end{proof}

We give $\sD_0(X,\cO)$ the \emph{open-open topology}, namely that generated by 
sets of the form 
\[
  B_{C,I,J} = \{\mu\colon \mu(B_{C,I})\subset J\} .
\]

\begin{theorem}
The rule $\sD_0(-,\cO)$ is a (covariant) functor from the category of 
topological spaces with proper maps to topological $\cO$-modules. 
\end{theorem}
\begin{proof}
All we need to do is check that if $\phi\colon X\to Y$ is proper, then 
$\phi_\ast\colon \sD_0(X,\cO)\to \sD_0(Y,\cO)$ is continuous. Fix an open 
$B_{C,I,J}\subset \sD_0(Y,\cO)$. It is easy to check that 
$\phi_\ast(B_{\phi^{-1}(C),I,J})\subset B_{C,I,J}$, so we are done. 
\end{proof}

Often we will have interesting dense subspaces of $\sC^0(X,\cO)$. For example, 
if $X$ is totally disconnected, write $\sC^\infty(X,\cO)$ for the subspace of 
locally constant functions. If $X$ has some kind of analytic structure, we 
write $\sC^\dagger(X,\cO)$ for the space of locally analytic functions. In 
general, if $\sC^\ast(X,\cO)\supset \sC^\infty(X,\cO)$, then write 
$\sD_\ast(X,\cO)$ for the topological dual of $\sC^\ast(X,\cO)$. The inclusions 
$\sC^\infty\hookrightarrow \sC^\ast\hookrightarrow \sC^0$ induce embeddings 
$\sD_0\hookrightarrow \sD_\ast\hookrightarrow \sD_\infty$. So an, e.g. locally 
analytic distribution is just a functional on $\sC^\infty$ that admits a 
continuous extension to $\sC^\dagger$. 





\section{Convolution}

Henceforth, all (abstract) topological spaces are assumed compact. Moreover, we 
assume $\cO$ has a \emph{linear topology}---that is, it has a basis of 
neighborhoods of zero given by additive subgroups. Let $X,Y$ be 
two (compact) topological spaces. Let $\pr_X,\pr_Y$ be the obvious projection 
maps. We have an induced map 
\[
  \pr_X^\ast\otimes \pr_Y^\ast\colon \sC^0(X,\cO)\otimes \sC^0(Y,\cO)\to \sC^0(X\times Y,\cO) .
\]
Namely, it sends $f\otimes g$ to the map $(x,y)\mapsto f(x) g(y)$. We make the 
following assumption:
\begin{equation*}\tag{dense}\label{assumption}
\text{The map $\pr_X^\ast\otimes \pr_Y^\ast$ has dense image.}
\end{equation*}
This is satisfied for example if $X$ and $Y$ are profinite, or if $X$ and $Y$ 
are smooth manifolds. 

\begin{theorem}
Let $X,Y$ satisfy \eqref{assumption}. Then there is a unique map
$\times\colon \sD_0(X,\cO)\otimes \sD_0(Y,\cO)\to \sD_0(X\times Y,\cO)$ such 
that for all $\lambda\in \sD_0(X,\cO)$, $\mu\in \sD_0(Y,\cO)$, 
$f\in \sC^0(X,\cO)$ and $g\in \sC^0(Y,\cO)$, we have 
\[
  \int_{X\times Y} f(x) g(y)\, \dd(\lambda\times \mu)(x,y) = \left(\int_X f(x)\, \dd\lambda(x)\right) \left(\int_Y g(y)\, \dd \mu(y)\right) .
\]
Moreover, we have the \emph{Fubini-Tonelli theorem}: 
\begin{align*}
  \int_{X\times Y} h(x,y)\, \dd(\lambda\times \mu)(x,y)
    &= \int_X \int_Y h(x,y) \, \dd \mu(y) \, \dd \lambda(x) \\
    &= \int_Y \int_X h(x,y)\, \dd \lambda(x)\, \dd \mu(y) ,
\end{align*}
for any $h\in \sC^0(X\times Y,\cO)$. 
\end{theorem}
\begin{proof}
Uniqueness of convolution follows trivially from \eqref{assumption}. By 
continuity, it suffices to show that $\lambda\times\mu$ is continuous on 
$\sC^0(X,\cO)\otimes \sC^0(Y,\cO)$ with respect to the subspace topology. 
Given a neighborhood of zero $J\subset \cO$, we know there exists 
$C_X,J_X$ such that $\mu(B_{C_X,J_X})$\ldots 

[finish later\ldots technicalities.]
\end{proof}

We are especially interested in the case where $G$ is a profinite group. We 
take $m:G\times G\to G$ to be the multiplication map. We often write $\ast$ 
for the composite 
\[
  \sD_0(G,\cO)\otimes \sD_0(G,\cO)\xrightarrow{\times} \sD_0(G\times G,\cO)\xrightarrow{m_\ast} \sD_0(G,\cO) .
\]
That is, 
\[
  \int_G f\, \dd(\lambda\ast \mu) = \int_G \int_G f(x y)\, \dd \lambda(x)\, \dd \mu(y) .
\]
\begin{theorem}
Let $G$ be profinite. Then convolution makes $\sD_0(G,\cO)$ into an 
associative algebra (possibly without unit). 
\end{theorem}
\begin{proof}
This is purely formal. 
\end{proof}





\section{The Amice transform}

Let $X$ be a profinite space. For the remainder of this section, we assume that 
$\cO$ is a profinite ring. Let $M$ be a profinite $\cO$-module. 

\begin{definition}
Fix a continuous map $\psi\colon X\to M$. Symbolically, the \emph{Amice 
transform} induced by $\psi$ is the map 
$\amice[\psi]{}\colon \sD_0(X,\cO)\to M$ given by 
\begin{equation}\label{amice}
  \amice[\psi]{\mu} =  \int_X \psi(x)\, \dd \mu(x) .
\end{equation}
\end{definition}

\begin{theorem}
The equation \eqref{amice} induces a well-defined continuous $\cO$-linear map.
\end{theorem}
\begin{proof}
Since $M=\varprojlim M/I$ for $I\subset \cO$ open, we may assume $M$ itself is 
finite, hence discrete. Thus $\sC^0(X,M) = \sC^0(X,\cO)\otimes M$; the theorem 
essentially follows. Explicitly, $\psi$ is locally constant, so put
\[
  \amice[\psi]{\mu} = \sum_{m\in M} \mu(\chi_{\psi^{-1}(m)}) \cdot m.
\]
Note that 
\begin{align*}
  \amice[\psi]{}^{-1}(m) 
    &= \{\mu\in \sD_0(X,\cO)\colon \amice[\psi]{\mu} = m\} \\
    &\supset \bigcap_{n\in M} B_{\psi^{-1}(n),M,\delta_{m,n} m} .
\end{align*}
It follows that $\amice[\psi]{}$ is continuous. Linearity is trivial. 
\end{proof}

Given the isomorphism $\sC^0(X,M) = \sC^0(X,\cO)\otimes M$ (one has to be 
careful when $M$ is not finite) we see that the Amice transform is essentially 
$\mu\mapsto \mu(\psi)$. The following is a ``non-commutative Fubini-Tonelli.'' 

\begin{lemma}
Let $\langle\cdot,\cdot\rangle\colon M\times M\to M$ be a bilinear pairing. 
Then 
\[
  \amice[\langle\varphi,\psi\rangle]{\lambda\times\mu} = \langle \amice[\varphi]{\lambda},\amice[\psi]{\mu}\rangle .
\]
\end{lemma}
\begin{proof}
To be precise, we are showing that 
\[
  \int_X\int_X \langle\varphi(x),\psi(y)\rangle\, \dd\lambda(x)\, \dd\mu(y) = \left\langle \int_X \varphi(x)\, \dd\lambda(x),\int_X \psi(x)\, \dd\mu(y)\right\rangle .
\]
It suffices to prove the result when $M$ is finite and $\varphi=m \chi_E$, 
$\psi=n \chi_F$. This is a computation:
\begin{align*}
  \amice[\langle\varphi,\psi\rangle]{\lambda\ast\mu} 
    &= \iint \langle m,n\rangle \chi_{E\times F} \, \dd(\lambda\times \mu)(x,y) \\
    &= \langle m,n\rangle (\lambda\times\mu)(\chi_E\otimes \chi_F) \\
    &= \langle \lambda(m\chi_E),\mu(n\chi_F)\rangle ,
\end{align*}
which is exactly $\langle \amice[\varphi]{\lambda},\amice[\psi]{\mu}\rangle$. 
\end{proof}

We are particularly interested in the case where $X=G$ is a profinite group, 
$M=A$ is an associative (but possibly non-commutative) $\cO$-algebra, and 
$\langle a,b\rangle = a b$. 

\begin{theorem}
Let $\psi\colon G\to A^\times$ be a continuous homomorphism. Then 
$\amice[\psi]{}$ respects multiplication.
\end{theorem}
\begin{proof}
This is purely formal:
\begin{align*}
  \amice[\psi]{\lambda\ast\mu} 
    &= \int_G \psi(x)\, \dd(\lambda\ast\mu)(x) \\
    &= \int_G \int_G \psi(x y)\, \dd\lambda(x) \, \dd \mu(y) \\
    &= \int_G \int_G \psi(x)\psi(y)\, \dd \lambda(x) \, \dd\mu(y) \\
    &= \int_G \psi(x)\, \dd\lambda(x) \int_G \psi(y) \, \dd\mu(y) \\
    &= \amice[\psi]{\lambda} \amice[\psi]{\mu} .
\end{align*}
\end{proof}

For any profinite group $G$, we have the profinite group algebra $\cO\pow{G}$. 
There is an obvious (continuous) injection $G\hookrightarrow \cO\pow{G}$. It is 
for this map that the Amice transform becomes really interesting. 

\begin{theorem}
Let $\iota\colon G\hookrightarrow \cO\pow{G}$ the natural map. The Amice 
transform induces an isomorphism 
$\amice[\iota]{}\colon \sD_\infty(G,\cO)\xrightarrow\sim\cO\pow{G}$. 
\end{theorem}
\begin{proof}
This is well-known. 
\end{proof}

As a corollary, we see that $\sD_0(G,\cO)$ and $\sD_\dagger(G,\cO)$ are 
naturally subalgebras of $\cO\pow{G}$. One generally applies this machinery to 
the simplest case---namely $G=\bZ_p$. For that group there is a well-known 
isomorphism $\cO\pow{G}\simeq \cO\pow{t}$, given by 
$x\mapsto (1+t)^x=\sum_{n\geqslant 0} \binom{x}{n} t^n$. In light of this, the 
Amice transform is generally written 
\[
  \amice{\mu} = \int_{\bZ_p} (1+t)^x\, \dd\mu(x) = \sum_{n\geqslant 0}\bigg(\int_{\bZ_p} \binom{x}{n}\, \dd\mu(x)\bigg)t^n .
\]
It realizes various (now commutative) algebras of distributions on $\bZ_p$ as 
more-or-less explicit subalgebras of $\cO\pow{t}$, generally defined by 
conditions on the growth rate of coefficients. 





\end{document}
