\documentclass{article}

\usepackage{amsmath,amssymb,amsthm}
\DeclareMathOperator{\sym}{sym}
\newcommand{\bC}{\mathbf{C}}
\newcommand{\bR}{\mathbf{R}}
\newcommand{\btheta}{{\boldsymbol \theta}}
\newcommand{\bx}{{\boldsymbol x}}
\newcommand{\bZ}{\mathbf{Z}}

\newtheorem{theorem}{Theorem}

\title{A brief summary of thesis results}
\author{Daniel Miller}

\begin{document}
\maketitle





Some quick notation. If $X$ is some space and $f\colon X\to \bC$ a function, 
and $\bx=(x_2,x_3,x_5,\dots)$ a sequence in $X$, write 
\[
	L_f(\bx,s) = \prod_p \frac{1}{1-f(x_p) p^{-s}} .
\]

\begin{theorem}
If, for $\alpha\in [1/2,1]$, we have 
\[
	\left| \sum_{p\leqslant N} f(x_p)\right| \ll N^{\alpha+\epsilon} ,
\]
then $\log L_f(\bx,s)$ has analytic continuation to $\{\Re s>\alpha\}$. 
Conversely, if $L_f(\bx,s)$ has analytic continuation to $\{\Re s>\alpha\}$ and 
moreover, 
\[
	|\log L_f(\bx,\sigma+i t)| \ll |t|^{1-\epsilon},
\]
for all $\sigma>\alpha$, then 
\[
	\left|\sum_{p\leqslant N} f(x_p)\right| \ll \pi(N)^{\alpha+\epsilon} .
\]
\end{theorem}

Roughly, this theorem says that analytic continuation of $\log L_f(\bx,s)$ to 
$\{\Re s>\alpha\}$ is equivalent to the bound 
$|\sum_{p\leqslant N} f(x_p)| \ll N^{\alpha+\epsilon}$. 

\begin{theorem}
Let $d\geqslant 1$. For any $\alpha\in [0,1/2]$, there exists a sequence 
$\bx$ in $(\bR/\bZ)^d$ that is uniformly distributed, such that 
\begin{enumerate}
\item
$D_N = \Omega(N^{-\alpha+\epsilon})$ (aka, big-$O$, but not big-$O$ of anything 
smaller).
\item
For any $f\in C^\infty(\bR/\bZ)^d$ with $\int f=0$, the function 
$\log L_f(\bx,s)$ has analytic continuation to $\{\Re s > 1/2\}$. 
\end{enumerate}
\end{theorem}

This says there are $d$-dimensional sequences whose discrepancy decays 
arbitrarily slowly, but whose $L$-functions are well behaved. 

\begin{theorem}
There exists a sequence $\btheta$ in $[0,\pi]$ such that for each $p$, 
$2\sqrt p \cos(\theta_p)\in \bZ$ and satisfies the Hasse bound, and such that 
\begin{enumerate}
\item
The discrepancy $D_N$ is not $\ll N^{-\epsilon}$ for any $\epsilon$. 
\item
The functions $L(\sym^k\btheta, s)$ satisfy the Riemann Hypothesis (for 
$k$ odd). 
\end{enumerate}
\end{theorem}





\end{document}
