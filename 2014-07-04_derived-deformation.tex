\documentclass{amsart}

\usepackage{amsmath,amssymb}
\DeclareMathOperator{\derive}{\mathsf{D}}
\DeclareMathOperator{\GL}{GL}
\DeclareMathOperator{\localsystem}{\mathsf{LS}}
\DeclareMathOperator{\pseudocompact}{PS}
\DeclareMathOperator{\spec}{Spec}
\newcommand{\cD}{\mathcal{D}}
\newcommand{\cO}{\mathcal{O}}
\newcommand{\dF}{\mathbf{F}}

\newcommand{\etale}{\textnormal{\'et}}

\title{A possible new approach to deformation theory}
\author{Daniel Miller}

\begin{document}
\maketitle





\section{Motivation}

Let $F$ be a number field, $k$ a local field with ring of integers $\cO$ and 
field of fractions $\kappa$. Typically one starts with a continuous 
representation $\bar\rho:G_{F,S} \to \GL_n(\kappa)$ and studies lifts of 
$\bar\rho$ to representations $G_{F,S} \to \GL_n(A)$, where $A$ ranges over 
local pseudocompact $\cO$-algebras with residue field $\kappa$. There is 
generally a pseudocompact $\cO$-algebra $R_{\bar\rho}^\square$ classifying 
such lifts. One proceeds to take quotients of the associated formal scheme, 
and hope that these quotients (and certain subfunctors) are representable. 





\section{Key ideas}

There are two motivating ideas. First, instead of representations 
$G_{F,S}\to \GL_n(\kappa)$, work more categorically. Let 
$U=\spec(\cO_F)\smallsetminus S$ and consider $\kappa$-local systems on 
the \'etale site of $U$. For each artinian $\cO$-algebra $A$, we have the 
category $\localsystem(U,A)$ of $A$-local systems on $U_\etale$. Hopefully 
the functor $A\mapsto\localsystem(U,A)$ is a formal stack (though not 
algebraic) in some sense. In any case, one defines 
$\localsystem(U,\cO) = \varprojlim \localsystem(U,A)$. Write 
$\pi$ for the functor $L\mapsto L\otimes_\cO \kappa$ from 
$\localsystem(U,\cO)$ to $\localsystem(U,\kappa)$. Given 
$L\in \localsystem(U,\kappa)$, we could write $D_L$ for the ``fiber'' 
$\pi^{-1}(L)$, which we again hope to be a formal stack in some reasonable 
sense. 

Everything up to now has merely been a change of language. The next main 
idea is that instead of working with $\cO$-local systems or 
$\kappa$-local systems, we should work with the ``derived category'' 
$\derive_c^b(U,k)$ of bounded constructible $k$-sheaves on 
$U_\etale$. That is, let $\pseudocompact(\cO)$ be the category of 
pseudocompact $\cO$-algebras. For any $A\in \pseudocompact(\cO)$, we have a 
category $\derive_c^b(U,A)$ of bounded constructible $A$-sheaves on $U_\etale$. 
One might hope that $\derive_c^b(U,-)$ is again a formal stack in some sense. 

The analogy of a residual representation $\bar\rho:G_{F,S}\to \GL_n(\kappa)$ is 
the choice of $L\in \derive_c^b(\kappa)$. Hopefully the ``fiber above $L$'' 
gives some kind of formal stack $\cD_L$. But it is not at all obvious that 
$\cD_L$ will be representable, or even a stack. 

If $E$ is an elliptic curve over $F$, we can ``spread out'' $E$ to an elliptic 
scheme over some $U\subset \spec(O_F)$. Let $f:E\to U$ be the structure map; 
then $\mathsf R f_\ast \kappa\in \derive_b^c(U,\kappa)$ is our analogue of the 
residual representation $\bar\rho_{E,\ell}$. The problem is, studying the 
``derived deformation theory'' of $\mathsf Rf_\ast \kappa$ is probably too 
complicated. A better place to start would be with the ``derived deformation 
theory'' of the cyclotomic character. Hopefully this would give us some kind of 
``derived version'' of the Iwasawa algebra. 

The mod-$p$ cyclotomic character should be $\dF_p(1)$ in 
$\derive_b^c(U,\dF_p)$. I need to understand what constructible (as opposed to 
smooth) sheaves on $U$ look like. A major problem is that constructible sheaves 
might have ramification, in which case classifying constructible lifts would be 
the wrong thing. 

So the question is: what (if any) is the concrete interpretation of 
$\derive_c^b(U,A)$? 





\end{document}
