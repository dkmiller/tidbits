\documentclass{article}

\usepackage{amsmath,amssymb,microtype,mathrsfs}
\DeclareMathOperator{\adjoint}{ad}
\DeclareMathOperator{\GL}{GL}
\DeclareMathOperator{\h}{H}
\DeclareMathOperator{\spectrum}{Spec}
\newcommand{\dF}{\mathbf{F}}
\newcommand{\dZ}{\mathbf{Z}}
\newcommand{\sF}{\mathscr{F}}
\newcommand{\sL}{\mathscr{L}}
\newcommand{\abelian}{\mathrm{ab}}
\DeclareFontFamily{U}{wncy}{}
\DeclareFontShape{U}{wncy}{m}{n}{<->wncyr10}{}
\DeclareSymbolFont{mcy}{U}{wncy}{m}{n}
\DeclareMathSymbol{\sha}{\mathord}{mcy}{"58} 

\title{Selmer groups in arithmetic topology}
\author{Daniel Miller}

\begin{document}
\maketitle





\section{Arithmetic setup}

Let $F$ be a number field, $S$ a finite set of places of $F$. Write 
$G_{F,S}=\pi_1(\spectrum(O_F)\smallsetminus S)$ for the Galois group of the 
maximal extension of $F$ unramified outside $S$. Let $M$ be a 
$G_{F,S}$-module. The \emph{$S$-Tate-Shafarevich group} of $M$ is 
\[
  \sha_S^\bullet(M) = \ker\left(\h^\bullet(G_{F,S},M) \to \bigoplus_{v\in S} \h^\bullet(G_v,M)\right) ,
\]
where $G_v=\pi_1(F_v)$ is the decomposition group at $v$. Let's start by giving 
a geometric definition of $\sha$. 

Let $X=\spectrum(O_F)$, and let $S\subset X$ be a closed subscheme. Write 
$i:S\hookrightarrow X$ and $j:U=X\smallsetminus S\hookrightarrow X$ for the 
inclusion maps. We should think of the $G_{F,S}$-module $M$ as being a locally 
constant sheaf $\sF$ on $U$. The question is: how should we think of 
$\bigoplus_{v\in S} \h^\bullet(G_v,M)$? Let $S^+$ be the infinitesimal \'etale 
neighborhood of $S$. Then $S^+=\coprod_{v\in S} \spectrum(O_{F,v})$. It follows 
that 
\[
  \partial S = S^+\smallsetminus S = \coprod_{v\in S} \spectrum(F_v) .
\]
Locally constant sheaves on $\partial S$ are the same thing as a collection of 
$G_v$-modules for $v\in S$. The analogue of ``treating $M$ as a $G_v$-module'' 
is $j_\ast \sF|_{\partial S}$. So our sheaf-theoretic Tate-Shafarevich group is 
\[
  \sha_S^\bullet(\sF) = \ker\left(\h^\bullet(U,\sF) \to \h^\bullet(\partial S,j_\ast \sF|_{\partial S})\right) .
\]

A common place for these groups to arise is in deformation theory. If 
$\bar\rho:G_{F,S}\to \GL_2(\dF_q)$ is a Galois representation, one wants 
$\sha_S^1(\adjoint \bar\rho)$ to vanish. Often, by enlarging $S$ cleverly, one 
can ensure this. 





\section{Topological analogue}

Let $M$ be a three manifold and let $L\subset M$ be a link (not just a knot -- 
this is important). Put $U=M\smallsetminus L$, and let $\sL$ be a local system 
on $U$. Let $j:U\hookrightarrow M$ be the inclusion. Let $V_L$ be a tubular 
neighborhood of $L$, and put $\partial V_L=V_L\smallsetminus L$ (this 
deformation retracts onto a union of tori). The \emph{topological 
Tate-Shafarevich group} is 
\[
  \sha_L^\bullet(\sL) = \ker\left(\h^\bullet(U,\sL) \to \h^\bullet(\partial V_L,j_\ast \sL|_{\partial V_L})\right) .
\]
General question: is $\sha_L^\bullet(\sL)$ an ``already known object''? If so, 
what role does it play?

Let's look at a baby example. Let $K\subset S^3$ be a knot, $\sL$ the constant 
sheaf $\dZ$. Then 
\begin{align*}
  \sha_K^1(\dZ) 
    &= \ker\left(\hom(\pi_1(U),\dZ) \to \hom(\pi_1(\partial V_L),\dZ)\right) \\
    &= \ker\left(\hom(G_K^\abelian,\dZ) \to \dZ^2\right) \\
    &= \left(G_K^\abelian / \dZ^2\right)^\vee ,
\end{align*}
where $\dZ^2\to G_K$ is the peripheral map. Since $G_K^\abelian=\dZ$, this 
``topological Tate-Shafarevich group'' is cyclic. It's not clear to me whether 
we can say much about general $\sha_L^1(\sL)$. 





\end{document}
