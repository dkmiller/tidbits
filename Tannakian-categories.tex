\documentclass{article}

\usepackage{
  amsmath,
  amssymb,
  amsthm,
  bookmark,  % instead of hyperref
  fullpage,
  mathrsfs,  % mathscr
  microtype, % better formatting
  thmtools,  % so that autoref -> 'Theorem *'
  tikz-cd,   % tikzcd
}
% operators
  \DeclareMathOperator{\aut}{Aut}
  \DeclareMathOperator{\bun}{Bun}
  \DeclareMathOperator{\diff}{Diff}
  \DeclareMathOperator{\GL}{GL}
  \DeclareMathOperator{\lie}{Lie}
  \DeclareMathOperator{\measures}{M}
  \DeclareMathOperator{\rep}{Rep}
  \DeclareMathOperator{\spec}{Spec}
  \DeclareMathOperator{\vect}{Vec}
  \DeclareMathOperator{\weil}{R}
% fonts
  \newcommand{\cC}{\mathcal{C}}
  \newcommand{\cU}{\mathcal{U}}
  \newcommand{\dC}{\mathbf{C}}
  \newcommand{\dR}{\mathbf{R}}
  \newcommand{\dS}{\mathbf{S}}
  \newcommand{\dZ}{\mathbf{Z}}
  \newcommand{\fa}{\mathfrak{a}}
  \newcommand{\fg}{\mathfrak{g}}
  \newcommand{\sE}{\mathscr{E}}
  \newcommand{\sH}{\mathscr{H}}
  \newcommand{\sL}{\mathscr{L}}
  \newcommand{\sO}{\mathscr{O}}
% misc
  \newcommand{\dd}{\mathrm{d}}
  \newcommand{\gl}{\mathfrak{gl}}
  \newcommand{\Gm}{\mathbf{G}_\mathrm{m}}
  \newcommand{\hodge}{\mathrm{Hdg}}
  \newcommand{\iso}{\xrightarrow\sim}
% theorems
  \newtheorem{theorem}[subsection]{Theorem}
  
\usepackage[
  hyperref = true,      % links to online documents
  backend  = bibtex,    % use bibtex instead of biber
  sorting  = nyt,       % sorts by (name, year, title)
  style    = alphabetic % citations look like [Har77]
]{biblatex}
\addbibresource{tidbit-sources.bib}
\hypersetup{
  colorlinks = true,
  linkcolor  = blue,
  urlcolor   = cyan
}

\title{Tannakian categories}
\author{Daniel Miller}

\begin{document}
\maketitle





\section{Motivation}\label{sec:examples}

Throughout, $k$ is an arbitrary field of characteristic zero. We will work over 
$k$, so all maps are tacitly assumed to be $k$-linear and all tensor product 
will be over $k$. Consider the following categories. 


\subsection{Representations of an algebraic group}\label{sec:rep-gp}

For $G_{/k}$ an algebraic group, the category $\rep(G)$ has as objects pairs 
$(V,\rho)$, where $V$ is a finite-dimensional $k$-vector space and 
$\rho:G\to \GL(V)$ is a homomorphism of $k$-groups. A morphism 
$(V_1,\rho_1)\to (V_2,\rho_2)$ in $\rep(G)$ is a $k$-linear map 
$f:V_1\to V_2$ such that for all $k$-algebras $A$ and $g\in G(A)$, one has 
$f \rho_1(g) = \rho_2(g)  f$, i.e.~the following diagram commutes:
\[
\begin{tikzcd}
  V_1\otimes A \ar[r, "f"] \ar[d, "\rho_1(g)"] 
    & V_2\otimes A \ar[d, "\rho_2(g)"] \\
  V_1\otimes A \ar[r, "f"]
    & V_2\otimes A .
\end{tikzcd}
\]


\subsection{Representations of a Hopf algebra}\label{sec:rep-alg}

Let $H$ be a co-commutative Hopf algebra. The category $\rep(H)$ has as objects 
$H$-modules that are finite-dimensional over $k$, and morphisms are 
$k$-linear maps. The algebra $H$ acts on a tensor product $U\otimes V$ via 
its comultiplication $\Delta:H\to H\otimes H$. 


\subsection{Representations of a Lie algebra}\label{sec:rep-liealg}

Let $\fg$ be a Lie algebra over $k$. The category $\rep(\fg)$ has as objects 
$\fg$-representations that are finite-dimensional as a $k$-vector space. There 
is a canonical isomorphism $\rep(\fg)=\rep(\cU \fg)$, where $\cU \fg$ is the 
universal enveloping algebra of $\fg$. 


\subsection{Continuous representations of a compact Lie group}\label{sec:rep-cpct}

Let $K$ be a compact Lie group. The category $\rep_\dC(K)$ has as objects 
pairs $(V,\rho)$, where $V$ is a finite-dimensional complex vector space and 
$\rho:K\to \GL(V)$ is a continuous (hence smooth, by Cartan's theorem) 
homomorphism. Morphisms $(V_1,\rho_1)\to (V_2,\rho_2)$ are $K$-equivariant 
$\dC$-linear maps $V_1\to V_2$. 


\subsection{Graded vector spaces}

Consider the category whose objects are finite-dimensional $k$-vector spaces 
$V$ together with a direct sum decomposition $V=\bigoplus_{n\in \dZ} V_n$. 
Morphisms $U\to V$ are $k$-linear maps $f:U\to V$ such that 
$f(U_n)\subset V_n$. 


\subsection{Hodge structures}

Let $V$ be a finite-dimensional $\dR$-vector space. A \emph{Hodge structure} 
on $V$ is a direct sum decomposition $V_\dC=\bigoplus V_{p,q}$ such that 
$\overline{V_{p,q}}=V_{q,p}$. If $U,V$ are vector spaces with Hodge structures, 
a morphism $U\to V$ is a $\dR$-linear map $f:U\to V$ such that 
$f(U_{p,q})\subset V_{p,q}$. Write $\hodge$ for the category of vector spaces 
with Hodge structure. 

Let $\vect(k)$ be the category of finite-dimensional $k$-vector spaces. For 
$\cC$ any of the categories above, there is a faithful functor 
$\omega:\cC\to \vect(k)$. In our examples, it is just the forgetful functor. 
The main theorem will be that for $\pi=\aut(\omega)$, the functor $\omega$ 
induces an equivalence of categories $\cC\iso \rep(\pi)$. We proceed to make 
sense of the undefined terms in this theorem. 





\section{Main definitions}

Our definitions follow \cite{deligne-milne-1982}. As before, $k$ is an 
arbitrary field of characteristic zero. 


\subsection{Tannakian category}

A \emph{$k$-linear category} is an abelian category $\cC$ such that each 
$V_1,V_2$, the group $\hom(V_1,V_2)$ has the structure of a $k$-vector space 
in such a way that the composition map 
$\hom(V_2,V_3)\otimes \hom(V_1,V_2)\to \hom(V_1,V_3)$ is $k$-linear. For us, 
a \emph{rigid $k$-linear tensor category} is a $k$-linear category $\cC$ 
together with the following data:
\begin{enumerate}
\item An exact faithful functor $\omega:\cC\to \vect(k)$. 
\item A bi-additive functor $\otimes:\cC\times \cC\to \cC$. 
\item Natural isomorphisms 
$\omega(V_1\otimes V_2)\iso \omega(V_1)\otimes \omega(V_2)$. 
\item Isomorphisms $V_1\otimes V_2\iso V_2\otimes V_1$ for all $V_i\in \cC$. 
\item Isomorphisms $(V_1\otimes V_2)\otimes V_3\iso V_1\otimes (V_2\otimes V_3)$
\end{enumerate}
These data are required to satisfy the following conditions:
\begin{enumerate}
\item There exists an object $1\in \cC$ such that $\omega(1)$ is 
one-dimensional and such that the natural map $k\to \hom(1,1)$ is an 
isomorphism. 
\item If $\omega(V)$ is one-dimensional, there exists $V^{-1}\in \cC$ such 
that $V\otimes V^{-1}\simeq 1$. 
\item Under $\omega$, the isomorphisms 3 and 4 are the obvious ones. 
\end{enumerate}

By \cite[Pr.~1.20]{deligne-milne-1982}, this is equivalent to the standard 
(more abstract) definition. Note that all our examples in 
\autoref{sec:examples} are rigid $k$-linear tensor categories. One calls the 
functor $\omega$ a \emph{fiber functor}. 


\subsection{Automorphisms of a functor}

Let $(\cC,\otimes)$ be a rigid $k$-linear tensor category. In this setting, 
define a functor $\aut(\omega)$ from $k$-algebras to groups by setting: 
\begin{align*}
  \aut^\otimes(\omega)(A) 
    &= \aut^\otimes\left(\omega:\cC\otimes A\to \rep(A)\right) \\
    &= \left\{(g_V)\in \prod_{V\in \cC} \GL(\omega(V)\otimes A):g_1=1\text{, }g_{V_1\otimes V_2} = g_{V_1}\otimes g_{V_2}\text{, and }f g_{V_1} = g_{V_1} f\text{ for all }f,V_1,V_2\right\} .
\end{align*}
In other words, an element of $\aut(\omega)(A)$ consists of a collection 
$(g_V)$ of $A$-linear automorphisms 
$g_V:\omega(V)\otimes A\iso \omega(V)\otimes A$, where $V$ ranges over objects 
in $\cC$. This collection must satisfy: 
\begin{enumerate}
  \item $g_1 = 1_{\omega(1)}$
  \item $g_{V_1\otimes V_2} = g_{V_1}\otimes g_{V_2}$ for all $V_1,V_2\in \cC$, and 
  \item whenever $f:V_1\to V_2$ is a morphism in $\cC$, the following diagram 
    commutes: 
    \[
    \begin{tikzcd}
      \omega(V_1)_A \ar[r, "f"] \ar[d, "g_{V_1}"] 
        & \omega(V_2)_A \ar[d, "g_{V_2}"] \\
      \omega(V_1)_A \ar[r, "f"] 
        & \omega(V_2)_A .
    \end{tikzcd}
    \]
\end{enumerate}


\subsection{Pro-algebraic group}

Typically one only considers affine group schemes $G_{/k}$ that are 
\emph{algebraic}, i.e.~whose coordinate ring $\sO(G)$ is a finitely generated 
$k$-algebra, or equivalently that admit a finite-dimensional faithful 
representation. Let $G_{/k}$ be an arbitrary affine group scheme, $V$ an 
arbitrary representation of $G$ over $k$. By 
\cite[Cor.~2.4]{deligne-milne-1982}, one has $V=\varinjlim V_i$, where $V_i$ 
ranges over the finite-dimensional subrepresentations of $V$. Applying this to 
the regular representation $G\to \GL(\sO(G))$, we see that 
$\sO(G)=\varinjlim\sO(G_i)$, where $G_i$ ranges over the algebraic quotients of 
$G$. That is, an arbitrary affine group scheme $G_{/k}$ can be written as a 
filtered projective limit $G=\varprojlim G_i$, where each $G_i$ is an affine 
algebraic group over $k$. So we will speak of pro-algebraic groups instead of 
arbitrary affine group schemes. 

If $V$ is a finite-dimensional $k$-vector space and $G=\varprojlim G_i$ is a 
pro-algebraic $k$-group, representations $G\to \GL(V)$ factor through some 
algebraic quotient $G_i$. That is, 
$\hom(G,\GL(V))=\varinjlim \hom(G_i,\GL(V))$. As a basic example of this, 
let $\Gamma$ be a profinite group, i.e.~a projective limit of finite groups. If 
we think of $\Gamma$ as a pro-algebraic group, then algebraic representations 
$\Gamma\to \GL(V)$ are exactly those representations that are continuous when 
$V$ is given the discrete topology. 





\section{Reconstruction theorem}

First, suppose $\cC=\rep(G)$ for a pro-algebraic group $G$, and that 
$\omega:\rep(G)\to \vect(k)$ is the forgetful functor. Then the Tannakian 
fundamental group $\aut^\otimes(\omega)$ carries no new information 
\cite[Pr.~2.8]{deligne-milne-1982}: 

\begin{theorem}\label{thm:reconst}
Let $G_{/k}$ be a pro-algebraic group, $\omega:\rep(G)\to \vect(k)$ the 
forgetful functor. Then $G\iso \aut^\otimes(G)$. 
\end{theorem}

The main theorem is the following, taken essentially verbatum from 
\cite[Th.~2.11]{deligne-milne-1982}. 

\begin{theorem}\label{thm:main}
Let $(\cC,\otimes,\omega)$ be a rigid $k$-linear tensor category. Then 
$\pi=\aut^\otimes(\omega)$ is represented by a pro-algebraic group, and 
$\omega:\cC\to \rep(\pi)$ is an equivalence of categories. 
\end{theorem}

Often, the group $\pi_1(\cC)$ is ``too large'' to handle directly. For 
example, if $\cC$ contains infinitely many simple objects, probably 
$\pi_1(\cC)$ will be infinite-dimensional. For $V\in \cC$, let 
$\cC(V)$ be the Tannakian subcategory of $\cC$ generated by $V$. One 
puts $\pi_1(\cC/V)=\pi_1(\cC(V))$. It turns out that 
$\pi_1(\cC/V)\subset \GL(\omega V)$, so $\pi_1(\cC/V)$ is finite-dimensional. 
One has $\pi_1(\cC)=\varprojlim \pi_1(\cC/V)$. 





\section{Examples}


\subsection{Pro-algebraic groups}

If $G_{/k}$ is a pro-algebraic group, then \autoref{thm:reconst} tells us that if 
$\omega:\rep(G)\to \vect(k)$ is the forgetful functor, then 
$G=\aut^\otimes(G)$. That is, $G=\pi_1(\rep G)$. 


\subsection{Hopf algebras}

Suppose $H$ is a co-commutative Hopf algebra over $k$. Then 
$\pi_1(\rep H)=\spec(H^\circ)$, where $H^\circ$ is the \emph{reduced dual} 
defined in \cite{cartier-2007}. Namely, for any $k$-algebra $A$, $A^\circ$ is 
the set of $k$-linear maps $\lambda:A\to k$ such that $\lambda(\fa)=0$ for some 
two-sided ideal $\fa\subset A$ of finite codimension. The key fact here is that 
$(A\otimes B)^\circ=A^\circ\otimes B^\circ$, so that we can use multiplication 
$m:H\otimes H\to H$ to define comultiplication 
$m^\ast:H^\circ\to (H\otimes H)^\circ=H^\circ\otimes H^\circ$. 
From \cite[II \S 6 1.1]{demazure-gabriel-1980}, if $G$ is a linear algebraic 
group over an algebraically closed field $k$ of characteristic zero, we get an 
isomorphism $\sO(G)^\circ= k[G(k)]\otimes \cU(\fg)$. Here $k[G(k)]$ is the 
usual group algebra of the abstract group $G(k)$, and $\cU(\fg)$ is the 
universal enveloping algebra of $\fg=\lie(G)$, both with their standard Hopf 
structures. 

[Note: one often calls $\sO(G)^\circ$ the ``space of distributions on $G$.'' 
If instead $G$ is a real Lie group, then one often writes $\sH(G)$ for the 
space of distributions on $G$. Let $K\subset G$ be a maximal compact subgroup, 
$\measures(K)$ the space of finite measures on $K$. Then convolution 
$D\otimes \mu\mapsto D\ast\mu$ induces an isomorphism 
$\cU(\fg)\otimes \measures(K)\iso \sH(G)$. In the algebraic setting, 
$k[G(k)]$ is the appropriate replacement for $\measures(K)$.]


\subsection{Lie algebras}

Let $\fg$ be a semisimple Lie algebra over $k$. Then by \cite{milne-2007}, 
$G=\pi_1(\rep\fg)$ is the unique connected, simply connected algebraic group 
with $\lie(G)=\fg$. If $\fg$ is not semisimple, e.g.~$\fg=k$, then things get a 
lot nastier. See the above example. 


\subsection{Compact Lie groups}

By definition, the \emph{complexification} of a real Lie group $K$ is a complex 
Lie group $K_\dC$ such that all morphisms $K\to \GL(V)$ factor uniquely through 
$K_\dC\to\GL(V)$. It turns out that $K_\dC$ is a complex algebraic group, and 
so $\pi_1(\rep K)=K_\dC$. 


\subsection{Graded vector spaces}

To give a grading $V=\bigoplus_{n\in \dZ} V_n$ on a vector space is equivalent 
to giving an action of the split rank-one torus $\Gm$. On each $V_n$, $\Gm$ 
acts via the character $g\mapsto g^n$. Thus 
$\pi_1(\text{graded vector spaces})=\Gm$. 


\subsection{Hodge structures}

Let $\dS=\weil_{\dC/\dR}\Gm$; this is defined by $\dS(A)=(A\otimes\dC)^\times$ 
for $\dR$-algebras $A$. One can check that the category $\hodge$ of Hodge 
structures is equivalent to $\rep_\dR(\dS)$. Thus $\pi_1(\hodge)=\dS$. 





\printbibliography

\end{document}
