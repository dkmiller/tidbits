\documentclass{amsart}

\usepackage{amsmath,amssymb,amsthm,fullpage}
\usepackage[all]{xy}
\usepackage[colorlinks]{hyperref}
\DeclareMathOperator{\adjoint}{ad}
\DeclareMathOperator{\galois}{Gal}
\DeclareMathOperator{\GL}{GL}
\DeclareMathOperator{\h}{H}
\DeclareMathOperator{\trace}{tr}
\newcommand{\cO}{\mathcal{O}}
\newcommand{\dG}{\mathbf{G}}
\newcommand{\dF}{\mathbf{F}}
\newcommand{\dQ}{\mathbf{Q}}
\newcommand{\dT}{\mathbf{T}}
\newcommand{\dZ}{\mathbf{Z}}
\newcommand{\fI}{\mathfrak{I}}
\newcommand{\fm}{\mathfrak{m}}
\newcommand{\sF}{\mathscr{F}}
\newcommand{\arithfrob}{\mathrm{fr}}
\newcommand{\etale}{\textnormal{\'et}}
\newcommand{\iso}{\xrightarrow\sim}
\newcommand{\mmu}{\boldsymbol\mu}
\newcommand{\mult}{\mathrm{m}}
\newtheorem{theorem}[subsection]{Theorem}

\title{Basic results in the deformation theory of Galois representations}
\author{Daniel Miller}

\begin{document}
\maketitle





This is a review of useful results in the study of deformations of (mostly 
two-dimensional) representations of $\pi_1(\dQ)=\galois(\overline\dQ/\dQ)$. 
References to the literature will be given whenever possible. 





\section{Group cohomology}


\subsection{Inflation-restriction}

This is from \cite[1.6.7]{nsw08}. Let $H\subset G$ be a closed normal subgroup 
of a profinite group. If $A$ is a $G$-module, then there is a canonical 
exact sequence 
\[\xymatrix{
  0 \ar[r] 
    & \h^1(G/H,A^H) \ar[r]^-{\mathrm{inf}} 
    & \h^1(G,A) \ar[r]^-{\mathrm{res}} 
    & \h^1(H,A)^{G/H} .
}\]





\section{Duality theorems for Galois cohomology}

Let $l$ be a prime, $X$ a connected noetherian scheme on which $l$ is 
invertible. Let $\dZ_l=\varprojlim\mmu_{l^n}$, considered as a smooth $l$-adic 
sheaf on $X$. For any $l$-adic sheaf $F$ on $X$, put 
$F(n)=F\otimes_{\dZ_l} \dZ_l(1)^{\otimes n}$. 

We call a \emph{$p$-adic field} a nonarchimedean local field of characteristic 
zero with residue characteristic $p$. 

\begin{theorem}[Tate]
Let $k$ be a $p$-adic local field. Let $M$ be a finite $\pi_1(k)$-module. Then 
the cup-product induces an isomorphism 
\[
  \h^\bullet(k,M^\vee(1)) = \h^{2-\bullet}(k,M)^\vee .
\]
\end{theorem}

Let $\pi=\pi_1(k)$, and let $M$ be a $\pi$-module. Suppose we want to compute 
$h^\bullet(M)$. It should be possible to compute 
$h^0(M)$ and $h^2(M)=h^0(M^\vee(1))$. We then use the Euler-Poincar\'e characteristic 
formula of Tate \cite[7.3.1]{nsw08} to do this. 





\section{Galois representations associated to modular forms}

Let $N\geqslant 1$ be an integer and $\varepsilon:(\dZ/N)^\times \to S^1$ a 
character. We write $S_0(N,\varepsilon)$ for the space of cusp forms for 
$\Gamma_1(N)$ with nebentypus $\varepsilon$. We call a form 
$f=\sum_{n\geqslant 0} a_n q^n$ in $S_0(N,\varepsilon)$ \emph{normalized} if 
$a_0=1$. 

\begin{theorem}
Let $N\geqslant 3$ and $k\geqslant 1$ be integers, $l$ an odd prime. Let 
$f_0\in S_0(N,\varepsilon)$ be a normalized eigenfunction for the Hecke 
operators $\{T_p:p\nmid N\}$. Let $K=K_f=\dQ(a_n:n\geqslant 1)$. 
Then there is a continuous irreducible representation 
$\rho_{f,l}:\pi_1\left(\dZ[\frac{1}{l N}]\right)\to \GL_2(K_{f,l})$ such that 
for each prime $p\nmid l N$, 
\begin{align*}
  \trace \rho_{f,l}(\arithfrob_p) &= a_p \\
  \det \rho_{f,l}(\arithfrob_p) &= \varepsilon(p) p^{k-1} .
\end{align*}
This representation is unique up to isomorphism. 
\end{theorem}
\begin{proof}
\textbf{Do this!}
\end{proof}





\section{Two-dimensional representations}

Nice fact if $\phi,\psi$ are characters: 
\[
  \adjoint(\phi\oplus \psi) = \phi^{-1}\psi\oplus \phi\psi^{-1}\oplus 2 .
\]
In particular, 
\[
  h^0(\adjoint(\phi\oplus 1)) = 2+2 h^0(\phi)
\]


\subsection{Peu ramifi\'ee and tr\`es ramif\'ee extensions}

The original source is \cite[2.4.6]{serre-1987}. 
Let $\bar\rho:G_{\dQ_p}\to \GL_2(\dF_q)$ be 
an ordinary representation, i.e. $\bar\rho$ is the extension of an unramified 
character by an unramified twist of the cyclotomic character. Let 
$\dQ_p^\mathrm{ur}(\bar\rho)$ be the extension of $\dQ_p^\mathrm{ur}$ with 
Galois group cut out by $\bar\rho(I)$, where $I\subset G_{\dQ_p}$ is the 
inertia group. It has a subextension $\dQ_p^\mathrm{ur}(\bar\rho|_P)$, where 
$P\subset I$ is wild inertia. Kummer theory tells us that 
\[
  \dQ_p^\mathrm{ur}(\bar\rho) = \dQ_p^\mathrm{ur}\left(\bar\rho|_P)(\sqrt[p]{x_1},\dots,\sqrt[p]{x_r}\right).
\]
We say that $\bar\rho$ is \emph{peu ramifi\'ee} if $v_p(x_i)\equiv 0\pmod p$ 
for each $i$, and $\bar\rho$ is \emph{tr\`es ramifi\'ee} otherwise. 

In \cite[8.2]{edixhoven-1992}, we have an alternative definition. Consider the 
extension $\bar\rho$ as a finite \'etale group scheme $V$ of $\dF_q$-vector 
spaces over $\dQ_p$. Then $\bar\rho$ is peu ramifi\'ee if $V$ can be extended 
to a finite flat group scheme over $\dZ_p$, and tr\`es ramifi\'ee otherwise. 





\section{Modular representations}


\subsection{Hecke operators}

A good (concise) summary of the diamond operators, Atkin-Lehner involution, and 
Hecke operators is \cite[ch.2 \S 5]{mw84}. 


\subsection{New parts of Jacobians}

The following is from \cite[\S 2]{mazur-1978}. 
For $n\geqslant 1$, let $J_0(n)$ be the jacobian of the modular curve $X_0(n)$. 
If $n=n' d$, there is a ``degeneracy map'' $B_d:X_0(n)\to X_0(n')$ that sends a 
pair $(E,C)$ consisting of an elliptic curve and $C\subset E[n]$ of order $n$ 
to the pair $(E/C[d],(C/C[d])[n'])$. There are induced maps 
$B_d^\ast:J_0(n')\to J_0(n)$. Let $J_0(n)_\mathrm{old}\subset J_0(n)$ be the 
abelian subvariety generated by the images of the $B_d$ for $n'<n$, and define 
$J_0(n)^\mathrm{new}$ by the short exact sequence 
\[
  0 \to J_0(n)_\mathrm{old} \to J_0(n) \to J_0(n)^\mathrm{new} \to 0 .
\]
By general theory, there is an isogeny 
$J_0(n)\sim J_0(n)_\mathrm{old}\times J_0(n)^\mathrm{new}$, thus an isomorphism 
of Galois representations 
\[
  V_\ell J_0(n) \simeq V_\ell J_0(n)_\mathrm{old}\oplus V_\ell J_0(n)^\mathrm{new} .
\]
There is an induced action of the Hecke algebra on $J_0(n)^\mathrm{new}$. 


\subsection{Eisenstein ideal}

This definition is from \cite[II.9]{mazur-1977}. Let $\dT=\dT_n$ be the Hecke 
algebra for $\Gamma_0(n)$. So $\dT$ is generated as a $\dZ$-algebra by the 
Hecke operators $T_l$, $l\nmid n$. The \emph{Eisenstein ideal} 
$\fI\subset \dT$ is generated by the $T_l-(l+1)$ for $l\nmid n$, and 
$1+w$. So if $f\in S_k$ is an eigenform annihilated by $\fI$, one has 
$a_p(f) = p+1$. This means $\rho_{f,l}$ should look like 
$\kappa_l\oplus 1$, where $\kappa$ is the cyclotomic character. 





\section{Deformation problems}

Let $\cO$ be a complete dvr with residue field $k$. Our deformation problems 
will be covariant functors on the category $\mathsf{C}_\cO$ of ``test 
objects.'' These are local artinian $\cO$-algebras $A$ such that 
$\cO\to A$ induces an isomorphism $k\iso A/\fm_A$. 


\subsection{Minimal deformations}

Here we follow \cite[\S2.1]{khare-2003}. Let $k$ be a finite field of 
characteristic $p$ and $\bar\rho:G_{\dQ,S}\to \GL_2(k)$ a continuous 
$p$-ordinary representation. One says a lift $\rho:G_{\dQ,S}\to \GL_2(A)$ is 
\emph{minimally ramified} if for $v\in S\smallsetminus p$, 
\[
  \rho|_{I_v}\sim \begin{pmatrix} 1 & \ast \\ & 1 \end{pmatrix} .
\]
\textsf{(This doesn't seem to be the same as 
\cite[p.180]{khare-ramakrishna-2003}. Find out what's wrong.)}


\subsection{New deformation rings}

We follow \cite[df.1]{khare-ramakrishna-2003}. 
Let $\bar\rho:G_\dQ\to \GL_2(\dF_q)$ be a continuous representation unramified 
outside $S$. Suppose $T\supset S$ is a finite set of primes such that 
$\bar\rho$ is nice for $T\smallsetminus S$. Then 
$R_{\bar\rho}^{T\textnormal{-new}}$ represents minimally ramified deformations 
$\rho:G_{\dQ,S} \to \GL_2(A)$ such that for $v\in T\smallsetminus S$, 
$\rho_v$ is a twist of $\begin{pmatrix} \varepsilon & \ast \\ & 1 \end{pmatrix}$. 





\bibliographystyle{alpha}
\bibliography{tidbit-sources}

\end{document}
