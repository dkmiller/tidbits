\documentclass{article}

\usepackage{amsmath,amssymb,amsthm,fullpage}
\usepackage[colorlinks]{hyperref}
\DeclareMathOperator{\adjoint}{ad}
\DeclareMathOperator{\galois}{Gal}
\DeclareMathOperator{\GL}{GL}
\DeclareMathOperator{\h}{H}
\DeclareMathOperator{\trace}{tr}
\newcommand{\dG}{\mathbf{G}}
\newcommand{\dF}{\mathbf{F}}
\newcommand{\dQ}{\mathbf{Q}}
\newcommand{\dZ}{\mathbf{Z}}
\newcommand{\sF}{\mathscr{F}}
\newcommand{\arithfrob}{\mathrm{fr}}
\newcommand{\etale}{\textnormal{\'et}}
\newcommand{\mmu}{\boldsymbol\mu}
\newcommand{\mult}{\mathrm{m}}
\newtheorem{theorem}[subsection]{Theorem}

\title{Basic results in the deformation theory of Galois representations}
\author{Daniel Miller}

\begin{document}
\maketitle





This is a review of useful results in the study of deformations of (mostly 
two-dimensional) representations of $\pi_1(\dQ)=\galois(\overline\dQ/\dQ)$. 
References to the literature will be given whenever possible. 





\section{Duality theorems for Galois cohomology}

Let $l$ be a prime, $X$ a connected noetherian scheme on which $l$ is 
invertible. Let $\dZ_l=\varprojlim\mmu_{l^n}$, considered as a smooth $l$-adic 
sheaf on $X$. For any $l$-adic sheaf $F$ on $X$, put 
$F(n)=F\otimes_{\dZ_l} \dZ_l(1)^{\otimes n}$. 

We call a \emph{$p$-adic field} a nonarchimedean local field of characteristic 
zero with residue characteristic $p$. 

\begin{theorem}[Tate]
Let $k$ be a $p$-adic local field. Let $M$ be a finite $\pi_1(k)$-module. Then 
the cup-product induces an isomorphism 
\[
  \h^\bullet(k,M^\vee(1)) = \h^{2-\bullet}(k,M)^\vee .
\]
\end{theorem}

Let $\pi=\pi_1(k)$, and let $M$ be a $\pi$-module. Suppose we want to compute 
$h^\bullet(M)$. It should be possible to compute 
$h^0(M)$ and $h^2(M)=h^0(M^\vee(1))$. We then use the Euler-Poincar\'e characteristic 
formula of Tate \cite[7.3.1]{nsw08} to do this. 





\section{Galois representations associated to modular forms}

Let $N\geqslant 1$ be an integer and $\varepsilon:(\dZ/N)^\times \to S^1$ a 
character. We write $S_0(N,\varepsilon)$ for the space of cusp forms for 
$\Gamma_1(N)$ with nebentypus $\varepsilon$. We call a form 
$f=\sum_{n\geqslant 0} a_n q^n$ in $S_0(N,\varepsilon)$ \emph{normalized} if 
$a_0=1$. 

\begin{theorem}
Let $N\geqslant 3$ and $k\geqslant 1$ be integers, $l$ an odd prime. Let 
$f_0\in S_0(N,\varepsilon)$ be a normalized eigenfunction for the Hecke 
operators $\{T_p:p\nmid N\}$. Let $K=K_f=\dQ(a_n:n\geqslant 1)$. 
Then there is a continuous irreducible representation 
$\rho_{f,l}:\pi_1\left(\dZ[\frac{1}{l N}]\right)\to \GL_2(K_{f,l})$ such that 
for each prime $p\nmid l N$, 
\begin{align*}
  \trace \rho_{f,l}(\arithfrob_p) &= a_p \\
  \det \rho_{f,l}(\arithfrob_p) &= \varepsilon(p) p^{k-1} .
\end{align*}
This representation is unique up to isomorphism. 
\end{theorem}
\begin{proof}
\textbf{Do this!}
\end{proof}





\section{Two-dimensional representations}

Nice fact if $\phi,\psi$ are characters: 
\[
  \adjoint(\phi\oplus \psi) = \phi^{-1}\psi\oplus \phi\psi^{-1}\oplus 2 .
\]
In particular, 
\[
  h^0(\adjoint(\phi\oplus 1)) = 2+2 h^0(\phi)
\]

Let $\bar\kappa:G_p\to \dF_q^\times$ be the mod-$p$ cyclotomic character. It 
turns out that $\h^1(p,\bar\kappa)=\dQ_p^\times\otimes \dF_q$. The subspace 
corresponding to $\dZ_p^\times\otimes \dF_q$ consists of ``peu ramifi\'ee'' 
characters, and the complement consists of ``tr\`es ramifi\'ee'' characters. 





\bibliographystyle{alpha}
\bibliography{tidbit-sources}

\end{document}
