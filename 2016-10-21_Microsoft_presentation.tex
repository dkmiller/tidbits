\documentclass{beamer}
\usetheme{Boadilla}

\title{Equidistribution, $G$-star discrepancy, and the analytic properties of a strange class of Dirichlet series}
\author{Daniel Miller}
\institute{Cornell University}
\date{27 November 2016}





\begin{document}

\begin{frame}
\titlepage
\end{frame}

\begin{frame}
\frametitle{Outline}
\tableofcontents
\end{frame}

\section{Section 1}
\subsection{sub a}

\begin{frame}
\frametitle{Title}
Associated to any elliptic curve over the rational numbers is a sequence of angles in the interval $[0,\pi]$ indexed by the prime numbers, called the Satake parameters of the curve. The famous Sato--Tate conjecture, proved recently via extremely sophisticated methods, predicts that as one considers increasingly larger finite slices of the set of Satake parameters, these finite slices approach the distribution $\frac{2}{\pi}\sin^2 \theta$. There is a precise measure, called $G$-star discrepancy, of how close a finite data set can be to a continuous distribution. Using techniques from differential geometry, I have created an n-dimensional inverse sampling transform, which allows for fast calculation of the star-discrepancy in 1d, and the isotropic discrepancy in n dimensions. Using this, I numerically verify a conjecture of Akiyama--Tanigawa for a larger set of primes than the previous best. I also define, for any equidistributed sequence, a class of strange Dirichlet series, and prove precise connections between the analytic properties of these Dirichlet series and the distribution of the original sequence. Finally, it is known that the conjecture of Akiyama–Tanigawa on Sato–Tate convergence implies the Riemann hypothesis for the elliptic curve's $L$-function. I show that in fact, their conjecture proves the Riemann Hypothesis for all associated strange Dirichlet series (and thus, for all symmetric power $L$-functions). Moreover, I use Diophantine Approximation to construct a class of sequences whose strange $L$-functions satisfy exceptionally good properties, but whose discrepancy decays very slowly, showing that Akiyama--Tanigawa's conjecture is strictly weaker than the generalized Riemann Hypothesis. 
\end{frame}

\end{document}