\documentclass{article}

\usepackage[a5paper,margin=1.5cm]{geometry}
\usepackage{amsmath,amssymb,amsthm,mathrsfs}
\DeclareMathOperator{\GL}{GL}
\DeclareMathOperator{\lie}{Lie}
\DeclareMathOperator{\M}{M}
\newcommand{\bV}{\mathbf{V}}
\newcommand{\bW}{\mathbf{W}}
\newcommand{\sA}{\mathscr{A}}
\newcommand{\sC}{\mathscr{C}}
\newcommand{\sF}{\mathscr{F}}
\newcommand{\sG}{\mathscr{G}}
\newcommand{\sH}{\mathscr{H}}
\newcommand{\sM}{\mathscr{M}}
\newcommand{\sO}{\mathscr{O}}
\newcommand{\coalg}{\mathsf{cAlg}}
\newcommand{\lf}[1]{{#1}_\mathrm{lf}}
\newcommand{\module}{\mathsf{Mod}}
\newcommand{\qc}[1]{{#1}_\mathrm{qc}}
\newcommand{\ring}{\mathsf{Rin}}
\newcommand{\scheme}{\mathsf{Sch}}
\newcommand{\set}{\mathsf{Set}}
\newtheorem{lemma}[subsection]{Lemma}
\newtheorem{theorem}[subsection]{Theorem}

\title{Noncommutative algebras and algebraic groups}
\author{Daniel Miller}

\begin{document}
\maketitle





Let $k$ be a fixed commutative ring, not necessarily a field. Let $A$ be a 
unital $k$-algebra. The functor $A^\times\colon \scheme_k\to\set$ given by 
\[
	A^\times(X) = \Gamma(X,\sO_X\otimes_k A)^\times 
\]
is, when $A$ is ``reasonable,'' represented by a group scheme which we denote 
$A_{/k}^\times$, or just $A^\times$. The purpose of this note is to relate 
algebraic properties of $A$ with the group $A^\times$. Note for example that 
$\M_n(k)^\times = {\GL_n}_{/k}$.





\section{Foundations}

For the moment, we work in maximal possible generality. Let $S$ be a fixed base 
scheme. If $\sF$ is a sheaf on $S$ and $f\colon X\to S$ is an object in 
$\scheme_S$, we write $\sF_X=f^\ast \sF$ for the pullback of $\sF$ to $X$. 

\begin{theorem}
Let $\sF$ be a quasi-coherent $\sO_S$-module. Then the functor 
$\bV(\sF)\colon \scheme_S\to \set$ given by 
\[
	\bV(\sF)(X) = \hom_{\sO_X}(\sF_X,\sO_X)
\]
is represented by an $S$-scheme, also denoted $\bV(\sF)$. The functor 
$\bV\colon \qc S^\circ \to \scheme_S$ is left-exact. 
\end{theorem}
\begin{proof}
That $\bV(\sF)$ is representable is standard. To check exactness of $\bV$, just 
note that 
\begin{align*}
	\bV\left(\varinjlim \sF_\alpha\right)(X) 
		&= \hom_{\sO_X}\left(\varinjlim \sF_\alpha,\sO_X\right) \\
		&= \varprojlim \hom_{\sO_X}(\sF_\alpha,\sO_X) \\
		&= \left(\varprojlim \bV(\sF_\alpha)\right)(X) .
\end{align*}
\end{proof}

Thus, to give a morphism of schemes $\bV(\sF)\times \bV(\sG)\to \bV(\sH)$, it 
suffices to give a morphism of sheaves $\sH\to \sF\oplus \sG$. Note that the 
functor $\bV$, while trivially faithful, is definitely not full. Since 
$\bV(\sF)(X)=\hom_{\sO_X}(\sF_X,\sO_X)$ is clearly a (commutative) group, 
we see that $\bV(\sF)$ admits a group structure. 

\begin{lemma}
Let $\sC$ be a quasi-coherent $\sO_S$-coalgebra. Then $\bV(\sC)$ is naturally 
an $S$-algebra. 
\end{lemma}
\begin{proof}
In other words, the functor $\bV(\sC)\colon \scheme_S\to \set$ factors through 
the category $\ring$ of associative unital rings. Let 
$\Delta\colon \sC\to\sC\otimes_{\sO_S} \sC$ be the comultiplication map, and 
$\eta\colon\sC\to\sO_S$ the conit. Then $\bV(\sC)(X)$ is given an algebra 
structure via convolution: 
\begin{align*}
	1 &= \eta_X \\
	f\cdot g &= (f\otimes g)\circ\Delta_X .
\end{align*}
The verification that with this structure, $\bV(\sC)$ is an $S$-algebra is 
routine. 
\end{proof}

So $\bV$ gives us a functor $\coalg(\qc S)\to \ring_{/S}$. If $\sC$ is an 
$\sO_S$-coalgebra, then its dual sheaf $\sC^\vee$ is naturally an 
$\sO_S$-algebra. 

\begin{theorem}
Let $\sC$ be a locally free $\sO_S$-coalgebra. Then $\sM\mapsto \sM^\vee$ 
gives an equivalence between the category of locally free (of finite type) 
$\sC$-comodules and the category of locally free (of finite type) 
$\sC^\vee$-modules. 
\end{theorem}

Let $\lf S$ be the category of locally free $\sO_S$-modules of finite type. 

\begin{theorem}
Let $\sM\in \lf S$. Then the functor $\bW(\sM)\colon \scheme_S\to \set$ given 
by 
\[
	\bW(\sM)(X) = \Gamma(X,\sM_X) 
\]
is representable. Moreover, $\sM\mapsto \bW(\sM)$ gives a left-exact 
tensor-functor from $\lf S$ to $\module(\bW(\sO_S))$. 
\end{theorem}
\begin{proof}
We content ourselves with showing that $\bW(\sM)=\bV(\sM^\vee)$. Since 
$\sM$ is locally free of finite type, it is self-dual. Thus 
\[
	\bV(\sM^\vee)(X) = \hom_{\sO_X}(\sM_X^\vee,\sO_X) = \Gamma(X,\sM_X^{\vee\vee}) = \Gamma(X,\sM_X) 
\]
as desired. 
\end{proof}

So we have a functor $\bW\colon \ring(\lf S)\to \ring_{/S}$. 





\section{Representations of groups and algebras}

If $\sA$ is a locally free $\sO_S$-algebra, we write $\GL(\sA)=\sA^\times$ for 
the functor $X\mapsto \Gamma(X,\sA_X)^\times$. This is representable, as it can 
easily be written as a fiber product of schemes. Many well-known algebraic 
groups arise via this construction, or one of its generalizations. 

\begin{theorem}
Let $\sA\in \ring(\lf S)$. Then there is a natural isomorphism 
$\lie(\sA^\times) = (\bW(\sA),[\cdot,\cdot])$. 
\end{theorem}
\begin{proof}
Recall that for $X\in \scheme_S$, we write $X[\epsilon]$ for the scheme whose 
underlying space is the same as $X$, but whose structure sheaf is 
$\sO_X[\epsilon]/\epsilon^2$. Then 
\begin{align*}
	\lie(\sA^\times)(X) 
		&= \ker(\Gamma(X,\sA_X[\epsilon])^\times\to \Gamma(X,\sA_X)^\times) \\
		&= \Gamma(X,1+\epsilon \sA_X) \\
		&\simeq \bW(\sA)(X) .
\end{align*}
Checking that the bracket comes from the commutator on $\sA$ is a simple 
computation. 
\end{proof}





\section{The case of a field}

Let $k$ be a field of characteristic zero. 





\end{document}
