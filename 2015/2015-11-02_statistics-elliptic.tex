\documentclass{article}

\usepackage{amsmath,amssymb,amsthm,hyperref,microtype}
\usepackage[a5paper,margin=1.5cm]{geometry}
\usepackage[
  hyperref = true,      % links to online documents
  backend  = bibtex,    % use bibtex instead of biber
  sorting  = nyt,       % sorts by (name, year, title)
  style    = alphabetic % citations look like [Har77]
]{biblatex}
\addbibresource{tidbit-sources.bib}
\hypersetup{colorlinks=true,linkcolor=green}

\DeclareMathOperator{\GL}{GL}
\DeclareMathOperator{\h}{H}
\DeclareMathOperator{\PGL}{PGL}
\DeclareMathOperator{\RH}{RH}
\DeclareMathOperator{\sign}{sgn}
\DeclareMathOperator{\SL}{SL}
\DeclareMathOperator{\spec}{Spec}
\DeclareMathOperator{\ST}{ST}
\DeclareMathOperator{\trace}{tr}
\newcommand{\bC}{\mathbf{C}}
\newcommand{\bF}{\mathbf{F}}
\newcommand{\bQ}{\mathbf{Q}}
\newcommand{\bR}{\mathbf{R}}
\newcommand{\bZ}{\mathbf{Z}}
\newcommand{\cF}{\mathcal{F}}
\newcommand{\cS}{\mathcal{S}}
\newcommand{\dd}{\mathrm{d}}
\newcommand{\frob}{\mathrm{fr}}
\newtheorem{theorem}{Theorem}
\newtheorem{conjecture}[theorem]{Conjecture}
\newtheorem{lemma}[theorem]{Lemma}
\numberwithin{theorem}{section}

\title{The statistics of the signed trace of Frobenius for elliptic curves}
\author{Daniel Miller}

\begin{document}
\maketitle





\section{Introduction}

Let $E_{/\bQ}$ be an elliptic curve of conductor $N$, $l$ a rational prime and 
$\rho=\rho_{E,l}\colon G_\bQ\to \GL_2(\bZ_l)$ the associated Galois 
representation coming from the Tate module of $E$. We have the sequence 
$\{a_p(E)\}_{p\nmid N l}$, defined by 
\[
	a_p(E) = \trace \rho(\frob_p) .
\]
It is well-known that $a_p(E)\in \bZ$, are independent of $l$, and satisfy 
the \emph{Hasse bound}:
\[
	|a_p(E)| \leqslant 2\sqrt p .
\]
Moreover, Faltings proved that the sequence $a(E) = \{a_p(E)\}$ determines the 
isogeny class of $E$. 

We can define a new sequence, $s(E) = \{s_p=s_p(E)\}_{p\nmid N}$, by 
$s_p = \sign(a_p)$. Write $s_{\leqslant X}(E)$ for 
$\{s_p(E)\colon p\leqslant X\}$. It is known that $s(E)$ determines $E$, but 
the proof depends on very deep results coming from Michael Harris 
and his school. Our goal is to show that, with some elementary input from 
representation theory, the fact that $s(E)$ determines $E$ follows from the 
fact that $a(E)$ determines $E$. 





\section{Motivation from characteristic zero}

Let $\Gamma$ be a finite group, $\rho\colon \Gamma\to \GL(V)$ a 
finite-dimensional absolutely irreducible real representation. Put 
$\chi_\rho=\trace\rho$ and $\sigma_\rho = \sign(\chi_\rho)$. We conjecture that 
$\sigma_\rho$, not just $\chi_\rho$,  determines $\rho$ up to isomorphism. For 
symmetric groups $S_n$, all irreducible representations are real, and the 
conjecture has been checked by the author for $S_n$ up to $n=17$, $D_n$ up 
to $n=20$, $A_n$ up to $n=11$, and $\PGL(2,\bF_p)$ up to $p=31$. Finally, this 
can be proved directly. 

\begin{theorem}
Let $G$ be a compact group, $\rho_1,\rho_2$ two finite-dimensional, 
continuous, irreducbile complex representations with real traces. If 
$\sign(\trace \rho_1) = \sign(\trace \rho_2)$, then $\rho_1\simeq \rho_2$. 
\end{theorem}
\begin{proof}
Representations of a compact group are self-dual, so 
$\hom_\bC(\rho_1,\rho_2) = \rho_1\otimes \rho_2$. For there to be an 
isomorphism between $\rho_1$ and $\rho_2$, we need 
$\h^0(\rho_1\otimes \rho_2) \ne 0$. That is, we need 
\[
	\langle 1, \trace(\rho_1\otimes \rho_2)\rangle = \int_G \trace \rho_1(g) \trace \rho_2(g)\, \dd g
\]
to be nonzero. Since $\sign(\trace \rho_1) = \sign(\trace \rho_2)$, the 
integrand is nonnegative. Moreover, since 
$\rho_1(1) \rho_2(1) = 1$, continuity gives us an open neighborhood 
$U$ of $1$ on which $\rho_1,\rho_2\geqslant 1/2$. We conclude that 
$\langle 1, \trace(\rho_1\otimes \rho_2)\rangle>0$, and the result follows. 
\end{proof}





\section{The main idea}

Let $E_{/\bQ}$ be an elliptic curve of conductor $N$. Since 
$|a_p|\leqslant 2\sqrt p$, if $l>4\sqrt p$, then $a_p$ is determined by its 
reduction modulo $l$. In fact, it is given by the function $\overline\sign$, 
defined by 
\[
	\overline\sign(x) = 
	\begin{cases}
		0 & x=0 \\
		1 & x\equiv 1,\dots,\frac{l-1}{2}\pmod l \\
		-1 & x\equiv \frac{l+1}{2},\dots,l-1\pmod l .
	\end{cases}
\]
If $l$ is not clear from the context, we write $\overline\sign_l$. 

\textbf{Conjecture.} Let $\Gamma$ be a finite group, 
$\rho_1,\rho_2\colon \Gamma\twoheadrightarrow \GL_2(\bF_l)$ two 
representations. If $\det\rho_1=\det\rho_2$ and 
$\overline\sign\trace\rho_1 = \overline\sign\trace\rho_2$, then 
$\rho_1\simeq \rho_2$. 

\textbf{Theorem.} If the conjecture is true, then $E$ is determined by the 
first $O(N\log\log N)$ of the $s_p(E)$. 

\begin{proof}
Let $E_{/\bQ}$ be an elliptic curve with conductor $N$. It is known that the 
first $O(N\log\log N)$ of the $a_p$ determine $E$. Choose a prime $l$ larger 
than $4\sqrt p$ for $p$ the largest of the $O(N\log \log N)$ primes. Then 
we can recover $a_p$ for all $p<\frac{1}{16} l^2$ from $a_p\mod l$. In other 
words, $\bar\rho=\rho\mod l$ determines $E$. If $E_1$ and $E_2$ both have 
conductor $\leqslant N$ and their first $O(N\log\log N)$ of the $s_p$ are 
equal, then 
$\overline\sign\trace \bar\rho_{E_1,l} = \overline\sign\trace \bar\rho_{E_2,l}$. 
By the conjecture, $\bar\rho_{E_1,l}\simeq \bar\rho_{E_2,l}$, hence 
$a_p(E_1)\equiv a_p(E_2)\mod l$ for $p<\frac{1}{16} l^2$. Together with the 
Hasse bound, this implies $a_p(E_1) = a_p(E_2)$ for those $p$, hence 
$E_1$ and $E_2$ are isogenous. 
\end{proof}





\section{Some ideas}

Let $\rho\colon G\twoheadrightarrow \GL_2(\bF_l)$. We'd like to characterize 
$N = \ker\rho$ in terms of $\sigma = \overline\sign(\trace \rho)$. 

For starters, $N$ is a normal subgroup of $G$ with $\sigma(n) = 1$ for all 
$n\in N$. We claim that $N$ is maximal with respect to that property. It comes 
down to: are there any normal subgroups of $\GL_2(\bF_l)$ on which 
$\sigma=1$? The only normal subgroups of $\GL_2(\bF_l)$ lie inside 
$\bF_l^\times$. Since $\bF_l^\times \simeq \bZ/(l-1)$, choose a generator 
$a$. Any subgroup of $\bF_l^\times$ is of the form $\langle a^r\rangle$ for 
some $r$. 

\begin{theorem}
Let $l\geqslant 5$ be prime. If $N\subset \bF_l^\times$ is a subgroup with 
$\overline\sign|_N = 1$, then $N=1$. 
\end{theorem}
\begin{proof}

[Numeric test up to $l\approx 130$.]

Since $\bF_l^\times$ is cyclic, write $N=\langle a\rangle$. 
\end{proof}

\begin{theorem}
Let $\rho\colon G\twoheadrightarrow \GL_2(\bF_l)$ be a representation. Then 
$\ker \rho$ is the largest normal subgroup of $G$ on which 
$\sigma=1$. 
\end{theorem}
\begin{proof}
Let $N$ be such a subgroup; then $N\cdot \ker\rho$ is also such a subgroup, so 
without loss of generality we may assume $\ker\rho\subset N$. Then 
$N/\ker\rho$ is a normal subgroup of $\GL_2(\bF_l)$ on which 
$\overline\sign(\trace) = 1$. Clearly $\SL_2(\bF_l)\not\subset N/\ker\rho$, so 
$N/\ker\rho\subset \bF_l^\times$. Applying the previous theorem, we see that 
$N=\ker\rho$. 
\end{proof}

\begin{theorem}
Let $G$ be a finite group, 
$\rho_1,\rho_2\colon G\twoheadrightarrow \GL_2(\bF_l)$ representations with 
$\det\rho_1 = \det\rho_2$ and $\sigma_1 = \sigma_2$. Then 
$\rho_1\simeq \rho_2$. 
\end{theorem}
\begin{proof}
By the previous result, $\ker\rho_1 = \ker\rho_2$, so we can assume that 
$\rho_1$ and $\rho_2$ are isomorphisms. But an automorphism of $\GL_2(\bF_l)$ 
is determined (up to inner automorphism) by its determinant. Thus 
$\rho_1\simeq \rho_2$. 

% http://groupprops.subwiki.org/wiki/Endomorphism_structure_of_general_linear_group_of_degree_two_over_a_finite_field

[Groupprops wiki: automorphisms of $\GL_2(\bF_l)$ are generated by 
inner automorphisms and twists by a power of the determinant.]
\end{proof}





\section{Problems}

Not quite so simple. We know that $\sign(x)=\sign(y)$ only implies 
$\overline\sign(x\mod l)=\overline\sign(y\mod l)$ if $x$ and $y$ are in the 
interval $(-l/2,l/2)$. 

Suppose $E_1$ and $E_2$ have $s(E_1)=s(E_2)$. Then for $l$ sufficiently large, 
$\bar\rho_i\colon G_\bQ\to \GL_2(\bF_l)$ are surjective, with the same 
determinant and 
$\overline\sign(\trace \rho_1) = \overline\sign(\trace \rho_2)$ on 
$\{\rho_i(\frob_p)\}_{p<l^2/16}$. Since $\pi(l^2/16)\ll l^2$ and 
$\# \GL_2(\bF_l)\sim l^4$, there is no way we know that 
$\overline\sign(\trace \rho_i)$ are equal on all of $\GL_2(\bF_l)$. 

Suppose $s(E_1) = s(E_2)$. 





\section{\texorpdfstring{$p$}{p}-adic Sato--Tate}

Let $E_1,E_2$ be non-isogenous elliptic curves. Let 
$\rho_{E_1\times E_2}\colon G_\bQ\to H(\hat\bZ)$ be the associated adelic 
representation. Its image is an open group we call $\Gamma$. Let 
$\trace\times \trace\colon H(\hat\bZ)\to \hat\bZ\times \hat\bZ$ be the trace 
function and define 
\[
	\mu_{\mathrm{ST}} = (\trace\times\trace)_\ast \mu_{\mathrm{Haar}(\Gamma)} .
\]
Then the pairs $(a_p(E_1),a_p(E_2))$ are equidistributed in $\hat\bZ^2$ with 
respect to $\mu_{\mathrm{ST}}$ in the sense that for any locally constant 
$f\in C(\hat\bZ^2)$, we have 
\[
	\int_{\hat\bZ\times \hat\bZ} f\, \dd \mu_{\mathrm{ST}} = \lim_{x\to \infty} \frac{1}{\pi(x)} \sum_{p\leqslant x} f(a_p(E_1),a_p(E_2)) .
\]

Let $s\in C([-2,2])$. Our question is: does $s(a_p/\sqrt p)$ determine the 
elliptic curve? To answer this, look at $A_s=\bC[s(x),s(y)]\subset C([-2,2]^2)$. 
This is a commutative ring, and $\mu_\mathrm{ST}\in \hom_\bC(A_s,\bC)$. 
\[
	\int s(x) s(y) = \int s(x) \int s(y)
\]

Conjecture: $\mu_{\mathrm{ST}(E_1\times E_2)} = \mu_{\mathrm{ST}(E_1)}\times \mu_{\mathrm{ST}(E_2)}$. 
This implies, if $s\in C(\hat\bZ)$ is idempotent, that $\mu_{\mathrm{ST}(E_1\times E_2)}\in \spec(A_s)$. 


Conjecture: on $H(\bZ_l)$, the haar measure is 
\[
	\int f = \int f(\gamma,\delta)\frac{\dd (\gamma,\delta)}{|\det \gamma|}
\]
Then 
\[
	\int R_{g,h} f = \int f(\gamma g,\delta h) \frac{\dd (\gamma,\delta)}{|\det \gamma|}
\]





\section{Ravi's conjecture via Sato--Tate for pairs}

Harris proved that if $E_1,E_2$ are non-CM, absolutely non-isogenous elliptic 
curves over $\bQ$ then the Sato--Tate conjecture holds for $E_1\times E_2$. 
Namely, the pairs $(\theta_p(E_1), \theta_p(E_2))$ are equidistributed in 
$[0,\pi]^2$ with respect to the measure 
$\frac{4}{\pi^2}\sin^2\theta_1\sin^2\theta_2\dd \theta_1 \dd \theta_2$. 

\begin{theorem}
Let $s\in L^2[0,\pi]$ be piecewise continuous and not constant almost 
everywhere. If $E_{/\bQ}$ is a non-CM 
elliptic curve, then the isogeny class of $E$ is determined by the set 
$s_\theta(E) = \{s(\theta_p)\}_p$. 
\end{theorem}
\begin{proof}
Suppose by way of contradiction that $E_1$ and $E_2$ are non-isogenous and 
$s_\theta(E_1)=s_\theta(E_2)$. Sato--Tate for pairs of elliptic curves tells us 
that 
\begin{align*}
	\int_{[0,\pi]^2} |s(x)-s(y)|\, \dd \mu_{\ST(E_1\times E_2)} 
	&= \lim_{x\to \infty} \frac{1}{\pi(x)} \sum_{p\leqslant x} |s(\theta_p(E_1))-s(\theta_p(E_2))| \\
	&= 0 .
\end{align*}
But the integrand is not zero almost everywhere, and 
$\mu_{\ST(E_1\times E_2)} = \mu_{\ST(E_1)}\times \mu_{\ST(E_2)}$ is a positive 
Borel measure, so the integral is nonzero. This contradiction gives us the 
proof. 
\end{proof}

Now, suppose $s\colon \bZ_l\to \bZ_l$ is continuous and non-constant. We claim 
that $s(E) = \{s(a_p(E))\}_p$ determines $E$ up to isogeny. First, 

$(*)_l$ For all $(E_1,E_2)_{/\bQ}$ non-isogenous and non-CM, the map 
$\trace \rho_{E_1\times E_2,l}\colon G_\bQ\to \bZ_l\times \bZ_l$ is surjective. 

\begin{theorem}[*]
Assume $(\ast)_l$. Then $s(E)$ determines the isogeny class of $E$. 
\end{theorem}
\begin{proof}
Again by contradiction. Let 
\[
	\mu_{\ST(l)} = (\trace\times\trace)_\ast \mu_{\mathrm{Haar}(\rho_{E_1\times E_2}(G_\bQ))} ,
\]
this is a positive Borel measure on $\bZ_l^2$. \v{C}ebotarev and $(\ast)_l$ 
tell us that the $(a_p(E_1),a_p(E_2))$ are equidistributed in 
$\bZ_l^2$ with respect to $\mu_{\ST(l)}$. As above, this yields a 
contradiction. 
\end{proof}





\section{Convergence of \texorpdfstring{$L$}{L}-functions}

Suppose we have an $L$-function of the form 
\[
	L(s) = \prod_p \det(1-\sigma_p p^{-s})^{-1} ,
\]
where the $\sigma_p$ are matrices. One can show directly that 
\[
	-\frac{L'}{L}(s) = (\log L)'(s) = \sum_{p^\nu} \frac{\log(p)\trace(\sigma_p^\nu)}{(p^\nu)^s} ,
\]
which is again a Dirichlet series with the $a_n$ supported on prime powers. 

Now let $E_{/\bQ}$ be an elliptic curve, $\eta\colon [0,\pi]\to \bR$ 
piecewise continuous, and put 
\[
	L_\eta(E,s) = \prod_p \frac{1}{1-\eta(\theta_p) p^{-s}} .
\]

Rearrange:
\begin{align*}
	\sum_p \log(p) \sum_{\nu \geqslant 1} \left(\frac{\eta(\theta_p)}{p^s}\right)^\nu
	&= \sum_p \log(p) \left(\frac{\eta(\theta_p)}{p^s}\right)\frac{1}{1-\frac{\eta(\theta_p)}{p^s}} \\
	&= \sum_p \log(p) \frac{\eta(\theta_p)}{p^s-\eta(\theta_p)}
\end{align*}
Put $s>0$; then $p^s\to \infty$, while $\eta(\theta_p)$ is bounded. For 
$p\gg 0$, $|\eta(\theta_p)| < \frac{1}{2} p^s$, so 
\[
	\frac{2}{3p^s} < \frac{1}{p^s-\eta(\theta_p)} < \frac{2}{p^s} .
\]
So convergence is equivalent to that of 
\[
	\sum_p \frac{\log(p) \eta(\theta_p)}{p^s} = \sum_p \frac{\log p}{p^s} \eta(\theta_p) .
\]
The first idea is to apply Dirichlet's test to $a_n=\eta(\theta_p)$, 
$b_n=\frac{\log p}{p^s}$. Then $b_n\to 0$ from above (for $n\gg 0$) and 
\begin{align*}
	\sum_{p\leqslant X} \eta(\theta_p) &\approx \pi(X) \int_{[0,\pi]} \eta \, \dd \mu_{\ST} \\
	\sum_{p\leqslant X} \frac{\log p}{p^s} &\approx \begin{cases} \log X & s=1 \\ \frac{s}{1-s} X^{1-s} & s<1 \end{cases} .
\end{align*}
So Dirichlet's test isn't very helpful. 

Instead, we'll try Abel's formula:
\[
	\sum_{p\leqslant X} a_p \phi(p) = \phi(X) \sum_{p\leqslant X} a_n - \int_1^X \phi'(x) \left(\sum_{p\leqslant x} a_n \right) \, \dd x .
\]
where $\phi$ is any continuously differentiable function. Choose 
$\phi(x)=\frac{\log x}{x^s}$ and $a_p = \eta(\theta_p)$. Then Abel summation 
tells us that 
\begin{align*}
	\sum_{p\leqslant X} \eta(\theta_p) \frac{\log p}{p^s} 
	&= \frac{\log X}{X^s} \sum_{p\leqslant X} \eta(\theta_p) - \int_1^X \frac{1-s\log x}{x^{s+1}} \sum_{p\leqslant x} \eta(\theta_p) \, \dd x \\
&\approx \mu_{\ST}(\eta)X^{1-s} - \mu_{\ST}(\eta) \int_1^X \frac{1-s\log x}{x^{s+1}} \frac{x}{\log x} \, \dd x \\
	&= \mu_{\ST}(\eta) \left(X^{1-s} - \int_1^X \frac{1-s\log x}{x^s\log x}\, \dd x \right) .
\end{align*}

The term $X^{1-s}$ diverges if $s<1$, and the integrand can be controlled by 
removing the ``$1-$'' term. So we end up studying the convergence of 
\[
	\int_1^\infty \frac{\dd x}{x^s} ,
\]
which isn't helpful either as it only converges if $s>1$. But if 
$\mu_{\ST}(\eta)=0$, everything should vanish. 



\subsection{Some bounds}

Let $\vartheta$ be Chebyshev's second function:
\[
	\vartheta(x) = \sum_{p\leqslant x} \log p .
\]

\begin{theorem}
Assume the RH. Then $|\vartheta(X)-X| = O(\sqrt X\log^2 X)$. 
\end{theorem}
\begin{proof}
This is \cite[Th.~10]{schoenfeld1976}
\end{proof}

\begin{conjecture}[Akiyama--Tanigawa]
Let $f\colon [0,\pi]\to \bR$ be of bounded variation. For every $\epsilon>0$, 
we have 
\[
	\left|\frac{1}{\pi(X)} \sum_{p\leqslant X} f(\theta_p) - \mu_{\mathrm{ST}}(f)\right| < X^{-\frac 1 2 + \epsilon}
\]
for $C\gg 0$. 
\end{conjecture}

This is from \cite{akiyama-tanigawa}? They conjecture something slightly 
different. [Show that their conjecture and this is equivalent.]

[Look at functions of bounded variation: it looks like the sequence 
definition of $\mu_{\ST}(f)$ works if and only if $f$ is of bounded 
variation.]





\section{Riemann Hypothesis}

Let $E_{/\bQ}$ be an elliptic curve, and let $\theta_p=\theta_p(E)$ be the 
normalized frobenius eigenvalues. For $\eta\colon [0,\pi]\to \bR$ of bounded 
variation, \cite{akiyama-tanigawa} defines the function 
\[
	D_X^{(\eta)} = \sup_{x\in [0,\pi]} \left|\frac{\#\{p\leqslant X : \theta_p\in [0,x)\}}{\pi(X)} - \eta(x)\right| ,
\]
and conjecture that $D_X^{(\eta)} = O(X^{-\frac 1 2+\epsilon})$ for all 
$\epsilon>0$. Strictly speaking, \cite{akiyama-tanigawa} only conjecture this 
for $\frac{2}{\pi}\sin^2\theta$. 

We need a generalized Koksma--Hlawka inequality.

For now, we imitate the proof of Theorem 2 in \cite{akiyama-tanigawa}. Note 
that 
\begin{align*}
	\log L_\eta(E,s) &= -\sum_p \log(1-\eta(\theta_p)p^{-s}) \\
		&= \sum_p \sum_{n\geqslant 1} \frac{\eta(\theta_p)^n}{n p^{n s}} \\
		&= \sum_p\left(\frac{\eta(\theta_p)}{p^s} + \sum_{n\geqslant 2} \frac{1}{n} \left(\frac{\eta(\theta_p)}{p^s}\right)^n\right)
\end{align*}

Assume that $\eta$ takes values in $B_1(0) = \{z\in \bC : |z|\leqslant 1\}$. 
If $s\geqslant 1/2$, then $|\eta(\theta_p) / p^s|<1$, so we can sum geometric 
series
\[
	\left| \sum_{n\geqslant 2} \frac{1}{n} \left(\frac{\eta(\theta_p)}{p^s}\right)^n \right|
		\leqslant p^{-2s} \frac{1}{1-p^{-s}} 
		\leqslant 4 p^{-2s}.
\]
Now $\sum_p p^{-2s}$ is holomorphic on $\Re s>1/2$. In other words, we know 
that $\log L_\eta(E,s)$ is holomorphic on $\Re s>1/2$ (and hence RH holds for 
$L_\eta(E,s)$) if and only if $\sum_p \frac{\eta(\theta_p)}{p^s}$ converges 
on that region. Now we apply Abel summation ($1/2<\Re s<1$):
\begin{align*}
	\sum_{p\leqslant X} \frac{\eta(\theta_p)}{p^s}
		&= X^{-s} \sum_{p\leqslant X} \eta(\theta_p) - s\int_1^X \frac{1}{x^{s+1}} \sum_{p\leqslant x} \eta(\theta_p) \, \dd x
\end{align*}
For convergence we need 
$\sum_{p\leqslant X} \eta(\theta_p) = O(X^{\frac 1 2 +\epsilon})$ for any 
$\epsilon>0$. Since $\pi(X) \sim \frac{X}{\log X}$, if $\mu_{\ST}(\eta)=0$, 
then RH for $L_\eta(E,s)$ is equivalent to 
\[
	\frac{1}{\pi(X)}\sum_{p\leqslant X} \eta(\theta_p) = O(X^{-\frac 1 2 + \epsilon}) .
\]
Put $\RH(E,\eta)$ for the associated Riemann Hypothesis for $L_\eta(E,s)$. 
Can we prove $\RH(E,\ST)\Rightarrow \RH(E,\eta)$?




\section{Akiyama--Tanigawa conjecture}

We give the precise statement of Conjecture 1 in \cite{akiyama-tanigawa}. 
Let $E_{/\bQ}$ be an elliptic curve. Define the \emph{discrepancy} 
\[
	D(X) = \sup_{x\in [0,\pi]} \left| \frac{\# \{p\leqslant X : \theta_p\in [0,x)\}}{\pi(X)} - \int_0^x \dd \mu_{\ST} \right| .
\]
Akiyama and Tanigawa conjecture that $D(X)=O(X^{-\frac 1 2+\epsilon})$. 
We can rewrite:
\[
	D(X) = \sup_{x\in [0,\pi]} \left| \frac{1}{\pi(X)} \sum_{p\leqslant X} \chi_{[0,x)}(\theta_p) - \mu_{\ST}(\chi_{[0,x)})\right|
\]
Here's some notation. Put
\[
	\mu_{\ST}^X = \frac{1}{\pi(X)} \sum_{p\leqslant X} \delta_{\theta_p} .
\]
Thus 
$D(X)=\sup_{x\in [0,\pi]} |\mu_{\ST}^X(\chi_{[0,x)}) - \mu_{\ST}(\chi_{[0,x)})|$. 
Let $f$ be of bounded variation. Then for $\epsilon>0$, we can choose 
$a_i,x_i$ such that $\|f-\sum a_i \chi_{[0,x_0)}\|_\infty<\epsilon$. 

******************************

Let $(X,\mu)$ be a probability space, $S\subset X$ a finite set, and 
$\cF$ a collection of functions on $X$. Then 
\[
	D_\cF^\mu(S) = \sup_{f\in \cF} \left|\frac{1}{\# S} \sum_{s\in S} f(s) - \int f \, \dd \mu \right| .
\]


*********************

Let $S\subset \bR$ be finite, let $f$ be a function on $\bR$ of bounded 
variation. Let $\mu=g(t)\, \dd t$ be a probability measure. Write 
$S=\{s_1,\dots,s_n\}$. 

We have 
\begin{align*}
	\frac{1}{\# S} \sum_{s\in S} f(s) 
		&= \frac{1}{n} \sum_{i=1}^n f(s_i) \\
		&= 
\end{align*}

[de Bruijn--Post theorem: $f$ is Riemann integrable iff for all 
equidistributed sequences, ``convergence works for $f$.'']

\begin{theorem}
Let $(X,\mu)$ be a compact space with Radon probability measure. Then for 
$f\in L^1(X,\mu)$, 
\[
	\lim_{X\to \infty} \left| \frac{1}{X} \sum_{n\leqslant X} f(x_n) - \int_X f\, \dd \mu\right| = 0
\]
for all $\mu$-equidistributed $\{x_n\}$ if and only if $f$ is continuous 
almost everywhere. 
\end{theorem}
\begin{proof}
Suppose $f$ is a.e.~continuous and $\{x_n\}$ is $\mu$-equidistributed. 
Then for any $\epsilon>0$, there is open $U\subset X$ such that $f$ is 
continuous on $U$ and $\mu(U)<\epsilon$. 
\end{proof}

****************************

Basic idea: let $\mu,\nu$ be Radon measures on $X$ and $\cF\subset C(X)$. THen 
\[
	d_\cF(\mu,\nu) = \sup_{f\in \cF} \frac{1}{\|f\|_0}\left| \int f\, \dd \mu - \int f \, \dd \nu\right| = \sup_{f\in \cF} \frac{1}{\|f\|_0}\left| \int f\, \dd (\mu-\nu)\right| .
\]
Put $|\mu-\nu| = d(\mu,\nu)=d_{C(X)}(\mu,\nu)$. 

\ldots

Better, for $X$ a compact space, put 
\[
	|\mu| = \sup_{f\in C(X)} \frac{1}{\|f\|_0} \int f\, \dd \mu = \sup_{\|f\|_0\leqslant 1} \int f\, \dd \mu .
\]

\begin{theorem}
Let $E_{/\bQ}$ be an elliptic curve. Then the Akiyama--Tanigawa conjecture for 
$E$ implies 
\[
	\left| \frac{1}{\pi(X)} \sum_{p\leqslant X} \eta(\theta_p) - \int \eta\, \dd \mu_{\ST}\right| = O_\eta(X^{-\frac 1 2+\epsilon})
\]
for all $\eta\colon [0,\pi] \to [-1,1]$ continuous almost everywhere. 
\end{theorem}
\begin{proof}
Write $\mu=\mu_{\ST}$ and 
$\mu^X = \frac{1}{\pi(X)} \sum_{p\leqslant X} \delta_{\theta_p}$. Then we wish 
to show that 
\[
	\sup_{x\in [0,\pi]} |(\mu^X-\mu)(\chi_{[x,\pi]})| = O(X^{-\frac 1 2+\epsilon})
\]
implies 
\[
	|(\mu^X-\mu)(\eta)| = O_\eta(X^{-\frac 1 2+\epsilon}) .
\]
Let $S$ be the vector space spanned by the $\chi_{[x,\pi]}$. We start by 
showing that the A--K conjecture implies that for $s\in S$:
\[
	|(\mu^X-\mu)(s)| = O_s(X^{-\frac 1 2+\epsilon}) .
\]
This follows from an easy computation. Put $s=\sum \lambda_i \chi_{[x_i,\pi]}$. 
Then 
\begin{align*}
	|(\mu^X-\mu)(s)| 
		&= \left| \sum \lambda_i (\mu^X-\mu)(\chi_{[x_i,\pi]})\right| \\
		&\leqslant \sum |\lambda_i| (\mu^X-\mu)(\chi_{[x_i,\pi]}) \\
		&= \left(\sum |\lambda_i|\right) O(X^{-\frac 1 2+\epsilon}) \\
		&= O_s(X^{-\frac 1 2 +\epsilon}) .
\end{align*}
We've proved things for $S$ (the space of all step functions). 
\end{proof}





\section{Zeta functions and distributions}

Let $G$ be a LCA group. If $\chi\colon G\to \bC^\times$ is a character, we 
have the associated representation of $C^\ast$-algebras: 
$\chi\colon L^1(G) \to \bC$,  
\[
	\chi(f) = \int_G \chi(x) f(x)\, \dd x .
\]
Thus we can think of $\chi\in S'(G)$, the space of distributions on $G$. As a 
distribution, $\chi$ transforms in the following way:
\begin{align*}
	(g\cdot \chi)f &= \chi(g^{-1}\cdot \phi) \\
		&= \chi(x\mapsto f(g x)) \\
		&= \int_G \chi(x) f(g x)\, \dd x \\
&= \chi(g)^{-1} \chi(f) .
\end{align*}
In other words, $\chi\in \cS'(G)[\chi^{-1}]$. These are the so-called ``zeta 
distributions'' from Tate's Thesis. 

\begin{lemma}
$\cS'(G)[\chi^{-1}] = \bC\cdot \chi$. 
\end{lemma}
\begin{proof}
We use a tempered distribution $D\in \cS'(G)[\chi^{-1}]$ to construct a 
measure $\mu$ on $G$ that transforms by $\chi$. 
\end{proof}

[
More generally, claim that the $(\mathfrak{g},K)$-module underlying $S'(G)$ is 
isomorphic to $\bigoplus \chi$. 
]

Now let $F$ be a local field. Let $\cS(F)$ be the space of Schwartz functions 
on $F$, $\chi\colon F^\times \to\bC^\times$ a character. Recall that for 
$\Phi\in \cS(F)$, the \emph{zeta function} is 
\[
	\chi(\Phi) = Z(\Phi,\chi) = \int_{F^\times} \chi(x) \Phi(x) \, \dd^\times x .
\]
Let $\cS'(F)$ be the space of continuous linear functionals on $\cS(F)$. 

\begin{theorem}
$\cS'(F)[\chi^{-1}] = \bC\cdot Z(-,\chi)$. 
\end{theorem}

Let $\psi\colon F\to \bC^\times$ be a non-trivial additive character. It gives 
an isomorphism $F = \widehat F$ and thus an automorphism 
$\Phi\mapsto \hat\Phi$. 
\[
	\hat\Phi(x) = \int_F \Phi(y) \psi(xy)\, \dd y .
\]
Moreover, 
\begin{align*}
	Z(\hat\Phi,|\cdot|\chi^{-1}) &= \epsilon_0(\chi,\psi) Z(\Phi,\chi) \\
	L(\chi) &= \frac{1}{1-\chi(p)} \\
	L(\chi,s) &= L(\chi |\cdot|^s) .
\end{align*}






\printbibliography
\end{document}
