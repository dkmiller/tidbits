\documentclass{article}

\usepackage[a5paper]{geometry}
\usepackage[cm]{fullpage}
\usepackage{amsmath,amssymb,microtype,stmaryrd,tikz-cd}
\DeclareMathOperator{\chars}{X}
\DeclareMathOperator{\Hom}{\mathbf{hom}}
\DeclareMathOperator{\spec}{Spec}
\newcommand{\bA}{\mathbf{A}}
\newcommand{\bD}{\mathbf{D}}
\newcommand{\bZ}{\mathbf{Z}}
\newcommand{\Gm}{\mathbf{G}_\mathrm{m}}
\newcommand{\Gr}{\mathrm{Gr}}
\newcommand{\lau}[1]{(\!( #1 )\! )}
\newcommand{\pow}[1]{\llbracket #1 \rrbracket}

\title{The geometric Satake correspondence}
\author{Daniel Miller}

\begin{document}
\maketitle





First, recall that if $X_{/S}$ and $Y_{/S}$ are schemes, there is a 
``hom-scheme'' defined by 
\[
	\Hom(X,Y)(T) = \hom_T(X_{/T},Y_{/T}) .
\]
In general, $\Hom(X,Y)$ will only be an fpqc sheaf. 

Let $\bD = \spec(\bZ\pow{t})$ and $\bD^\ast=\spec(\bZ\lau{t})$. Then we have a 
pullback diagram: 
\[
\begin{tikzcd}
	\bD^\ast \ar[r, hook] \ar[d]
		& \bD \ar[d] \\
	\Gm \ar[r, hook]
		& \bA^1 
\end{tikzcd}
\]
Thus, if $X_{/S}$ is an arbitrary scheme, we get a commutative diagram:
\[
\begin{tikzcd}
	\Hom(\bD^\ast,X)
		& \Hom(\bD,X) \ar[l, hook] \\
	\Hom(\Gm,X) \ar[u] 
		& \Hom(\bA^1,X) \ar[l, hook] \ar[u] 
\end{tikzcd}
\]
In particular, if $T$ is a torus, there is a natural map 
\[
	\chars_\ast(T) = \Hom(\Gm,T)\to \Hom(\bD^\ast,T) = G\lau{t} .
\]
For $\lambda\in \chars_\ast(T)$, write $t^\lambda$ for the image in 
$G\lau{t}$. Let $G$ be a reductive group, and put $\Gr_G = G\lau{t}/G\pow{t}$. 
It turns out that perverse sheaves on $\Gr_G$ are best understood in light of 
orbits $\Gr_G^\lambda = \overline{G\pow{t}\cdot t^\lambda}$ for 
$\lambda\in \chars_\ast(T)$, where $T\subset G$ is a maximal torus. 

If $G=T$ is already a torus, then $\Gr_T = \chars_\ast(T)$, and each 
$\Gr_T^\lambda$ is a single point. 

Note that $\chars_\ast(T)(A) = T(A[t^{\pm 1}])$. 





\end{document}
