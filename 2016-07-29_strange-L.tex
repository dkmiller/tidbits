% !TeX spellcheck = en_US

\documentclass{article}

\usepackage{amsmath,amssymb,amsthm}
\DeclareMathOperator{\disc}{disc}
\DeclareMathOperator{\GL}{GL}
\DeclareMathOperator{\Jac}{Jac}
\DeclareMathOperator{\Lie}{Lie}
\DeclareMathOperator{\N}{N}
\DeclareMathOperator{\R}{R}
\DeclareMathOperator{\sgn}{sgn}
\DeclareMathOperator{\SU}{SU}
\DeclareMathOperator{\sym}{sym}
\DeclareMathOperator{\tr}{tr}
\DeclareMathOperator{\U}{U}
\DeclareMathOperator{\Var}{Var}
\newcommand{\bC}{\mathbf{C}}
\newcommand{\bD}{\mathbf{D}}
\newcommand{\bF}{\mathbf{F}}
\newcommand{\bN}{\mathbf{N}}
\newcommand{\bQ}{\mathbf{Q}}
\newcommand{\bR}{\mathbf{R}}
\newcommand{\bT}{\mathbf{T}}
\newcommand{\btheta}{{\boldsymbol{\theta}}}
\newcommand{\bx}{{\boldsymbol x}}
\newcommand{\by}{{\boldsymbol y}}
\newcommand{\bz}{{\boldsymbol z}}
\newcommand{\bZ}{\mathbf{Z}}
\newcommand{\cF}{\mathcal{F}}
\newcommand{\ft}{\mathfrak{t}}

\newcommand{\alev}{\mathrm{ae}}
\newcommand{\dd}{\mathrm{d}}
\newcommand{\iso}{\mathrm{iso}}
\newcommand{\fr}{\mathrm{fr}}
\newcommand{\Haar}{\mathrm{Haar}}
\newcommand{\simplc}{\mathrm{sc}}
\newcommand{\semis}{\mathrm{ss}}
\newcommand{\ST}{\mathrm{ST}}
\newcommand{\Zar}{\mathrm{Zar}}
\newtheorem{theorem}[subsection]{Theorem}
\newtheorem{lemma}[subsection]{Lemma}
\newtheorem{corollary}[subsection]{Corollary}
\theoremstyle{definition}
\newtheorem{definition}[subsection]{Definition}
\newtheorem{example}[subsection]{Example}

 



\title{Equidistribution and the analytic properties of a strange class of 
$L$-functions}
\author{Daniel Miller}

\begin{document}
\maketitle





\section{Motivation}

Let $E_{/\bQ}$ be an elliptic curve without complex multiplication. By an old 
theorem of Faltings \cite{faltings-1983}, the sequence 
\[
	a_p(E) = p + 1 - \# E(\bF_p) = \tr \rho_{E,l} (\fr_p)
\]
determines $E$ up to isogeny. That is, if $E_1$ and $E_2$ satisfy 
$a_p(E_1)=a_p(E_2)$ for all $E$, then $E_1$ and $E_2$ are isogenous. The 
starting point of this investigation is the 
corollary of a theorem of Harris, that the sequence $(\sgn a_p(E))_p$ in 
fact determines $E$ up to isogeny. Ramakrishna had the insight that this fact 
means the ``strange $L$-function''
\[
	L_{\sgn}(E,s) = \prod_p \frac{1}{1-\sgn a_p(E) p^{-s}} 
\]
determines $E$ up to isogeny. In this note, we define a more general class of 
strange $L$-functions, and show that their analytic properties are closely 
tied to the distribution of the $a_p(E)$. 

Here is a brief discussion of this generalization in the case of a non-CM curve 
$E_{/\bQ}$. It is convenient to repackage the traces of Frobenius as follows:
\[
	\theta_p(E) = \cos^{-1}(a_p(E)/2\sqrt p) .
\]
The Hasse Bound guarantees that the $\theta_p(E)$ are well-defined angles 
laying in the interval $[0,\pi]$. Write 
$\dd\ST = \frac{2}{\pi} \sin^2\theta\, \dd\theta$. Then the Sato--Tate 
conjecture, now a theorem \cite{barnet-lamb-etal-2011}, tells us that for any 
continuous function $f\colon [0,\pi]\to \bC$, we have
\[
	\left| \frac{1}{\pi(N)} \sum_{p\leqslant N} f(\theta_p) - \int_0^\pi f\, \dd\ST\right| = o(1)
\]
as $N\to \infty$. It is well-known that this follows from the analytic 
continuation past $\Re s=1$ and non-vanishing except at $s=1$ of all the 
$L$-functions $L(\sym^k E,s)$ \cite[A.1 Th.1]{serre-1968}. We take as our 
starting point the much stronger conjecture, due to Akiyama--Tanigawa 
\cite{akiyama-tanigawa}, that 
\[
	\left| \frac{1}{\pi(N)} \sum_{p\leqslant N} f(\theta_p) - \int_0^\pi f\, \dd \mu_\ST\right| \ll_f N^{-\frac 1 2+\epsilon}
\]
for all $f$ of bounded variation. (Their conjecture is actually more precise; 
we will discuss their exact statement later.)
They prove that this conjecture implies the Riemann Hypothesis for $E$. We 
prove that not only does their conjecture imply the Riemann Hypothesis for all 
$L(\sym^k E,s)$, it also does for all the strange $L$-functions 
\[
	L_f(E,s) = \prod_p \frac{1}{1-f(\theta_p(E)) p^{-s}}
\]
where $f$ varies over almost-everywhere continuous functions on $[0,\pi]$. 

These results make perfect sense in a much more general context, and we
prove them there. In Section ? I set up this context and 
carefully define strange $L$-functions. In Section ?, I 
prove basic analytic properties of the strange $L$-functions and 
connect their analytic properties with the equidistribution of a sequence. 
In Section ?, I apply these results where ``everything is 
known,'' i.e.~varieties over function fields.
Finally, in Section ?, I apply the general results to the 
following cases: a non-CM elliptic curve $E_{/\bQ}$, the product 
$E_1\times E_2$ of a pair of non-isogenous non-CM elliptic curves over $\bQ$, 
and the Jacobian of a generic genus-$2$ curve $C_{/\bQ}$. 





\section{Strange $L$-functions}\label{sec:strange-L}

Let $\bD=\{z\in \bC : |z|\leqslant 1\}$. Write $\bD^\infty$ for the set of 
sequences in $\bD$ indexed by the primes, i.e.~$\bz\in\bD^\infty$ is 
$(z_2,z_3,\dots)$. The space $\bD^\infty$ is compact, and comes naturally 
equipped with the (product) Lebesgue measure, normalized to have mass $1$. 

\begin{definition}
Let $\bz\in\bD^\infty$. The associated \emph{strange $L$-function} is given by 
\[
	L(\bz,s) = \prod_p \frac{1}{1-z_p p^{-s}} ,
\]
wherever this product converges. 
\end{definition}

Elementary topology tells us that $L\colon \bD^\infty\times \bC^{\Re>1}\to \bC$ 
is continuous. 
We will see that for fixed $\bz\in \bD^\infty$, the analytic properties of 
$L(\bz,s)$ are closely tied to 
estimates for the sums $A_\bz(x) = \sum_{p\leqslant x} z_p$. One 
often gets such estimates in the context of equidistribution, which we consider 
next.

For the remainder of this section, let $X$ be a compact separable metric space with no 
isolated points. We write $X^\infty$ for the space of sequences in $X$ indexed 
by rational primes, i.e.~points $\bx\in X^\infty$ are of the form 
$\bx=(x_2,x_3,\dots)$. By \cite[Cor.2.3.16, Th.4.2.2]{engelking-1989}, the 
compact space $X^\infty$ is metrizable and separable, also with no isolated 
points. 

\begin{definition}
For $\bx\in \bD^\infty$ and $N>0$, write $\bx^N$ for the probability measure on 
$\bD$ given by 
\[
	\int_X f\, \dd \bx^N = \bx^N(f) = \frac{1}{\pi(N)} \sum_{p\leqslant N} f(x_p) .
\]
\end{definition}

Let $\mu$ be a Borel measure on $X$. Recall that $\bx$ is 
\emph{$\mu$-equidistributed} if $\bx^N\to \mu$ weakly, i.e. 
$\mu(f) = \lim_{N\to \infty}\bx^N(f)$ for all $f\in C(X)$. In fact, we can 
extend this to not-necessarily-continuous functions as follows:

\begin{theorem}[Mazzone]
Let $\mu$ be a Borel measure on $X$ and let $f\colon X\to \bC$ be bounded and 
measurable. Then $f$ is continuous almost everywhere if and only if 
$\mu(f)=\lim_{N\to \infty}\bx^N(f)$ for all $\mu$-equidistributed $\bx$. 
\end{theorem}
\begin{proof}
This follows directly from the proof of \cite[Th.1]{mazzone-1995}.
\end{proof}

Fix a Borel measure $\mu$ on $X$, and write $C^\alev(X,\mu)$ for the space of 
bounded, almost-everywhere continuous functions $f\colon X\to \bC$. 


\begin{lemma}
Endowed with the supremum norm $\|f\|_\infty=\sup_{x\in X} |f(x)|$, 
$C^\alev(X,\mu)$ is a Banach space. 
\end{lemma}
\begin{proof}
This is an elementary corollary of the fact that a countable union of 
measure-zero sets has measure zero. 
\end{proof}

\begin{definition}
Let $f\in C^\alev(X,\mu)^{\|\cdot\|_\infty\leqslant 1}$, $\bx\in X^\infty$. The 
associated \emph{strange $L$-function} is defined as 
\[
	L_f(\bx,s) = L(f(\bx),s) = \prod_p \frac{1}{1-f(x_p) p^{-s}} 
\]
for all $s\in \bC$ for which the product converges. 
\end{definition}

Our typical source of a strange $L$-function is as follows. Let $G$ be a 
compact connected Lie group and $X=G^\natural$, the space of conjugacy 
classes of $G$. Then $G^\natural$ inherits the Haar measure from $G$. Given 
any sequence $\bx\in (G^\natural)^\infty = G^{\natural,\infty}$ and 
function $f\in C^\alev(G^\natural)^{\|\cdot\|_\infty\leqslant 1}$, we can 
define $L_f(\bx,s)$. This is 
related to Serre's $L$-functions from \cite[A.2]{serre-1968} as follows. 

\begin{theorem}
Let $G$ be a compact connected Lie group, $\rho\in\widehat G$ an irreducible 
unitary representation of $G$. Then there exist functions 
$\lambda_\rho^1,\dots,\lambda_\rho^{\deg\rho}\colon G^\natural \to S^1$, 
continuous away from the set $\{\det(1-\rho)=0\}$, such that for every 
$x\in G^\natural$, there are angles 
$\theta_1,\dots,\theta_{\deg\rho}\in [0,2\pi)$, satisfying 
$\theta_1\leqslant \cdots \leqslant \theta_{\deg\rho}$, such that 
$\lambda_\rho^j(x) = e^{i \theta_j}$ and moreover
\[
	\det(1-\rho(x) t) = \prod_{j=0}^{\deg \rho} (1-\lambda_\rho^j(x) t) .
\]
\end{theorem}
\begin{proof}
This follows easily from \cite[Lem.1.0.9]{katz-sarnak-1999}. 
\end{proof}

Recall that for $\rho\in \widehat G$, Serre defines 
$L(\rho,s) = \prod_p \det(1-\rho(x_p) p^{-s})^{-1}$. Using his notation, there 
is the identity 
\[
	L(\rho,s) = \prod_{j=1}^{\deg\rho} L_{\lambda_\rho^j}(\bx,s) .
\]





\section{Discrepancy}\label{sec:discrepancy}

The rest of our definitions concern discrepancy, which for now we define only 
in a special context. Let $r\geqslant 1$ be an integer, and consider the space 
$[0,1]^r$. For $x,y\in [0,1]^r$, we write $x<y$ (resp.~$x\leqslant y$) if 
$x=(x_1,\dots,x_r)$, $y=(y_1,\dots,y_r)$ and $x_i<y_i$ 
(resp.~$x_i\leqslant y_i$) for all $i$. Given $x<y\in [0,1]^r$, write 
\begin{align*}
	I_x &= \{z\in [0,1]^r : z<x\} \\
	I_{x,y} &= \{z\in [0,1]^r : x\leqslant z<y\} .
\end{align*}

\begin{definition}
Let $\mu$, $\nu$ be probability measures on $[0,1]^r$. The 
\emph{star-discrepancy} between $\mu$ and $\nu$ is 
\[
	\disc^\star(\mu,\nu) = \sup_{x\in [0,1]^r} |\mu(I_x)-\nu(I_x)| ,
\]
the \emph{discrepancy} between $\mu$ and $\nu$ is 
\[
	\disc(\mu,\nu) = \sup_{x<y\in [0,1]^r} |\mu(I_{x,y})-\nu(I_{x,y})| ,
\]
and the \emph{isotropic discrepancy} between $\mu$ and $\nu$ is 
\[
	\disc^\iso(\mu,\nu) = \sup_{C\subset [0,1]^r} |\mu(C)-\nu(C)| ,
\]
where $C$ ranges over open and closed convex subsets of $[0,1]^r$. 
\end{definition}

Let $\lambda$ be the Lebesgue measure on $[0,1]^r$, $\bx$ a sequence in 
$[0,1]^r$. We write $\disc^\ast(\bx^N)=\disc^\ast(\bx^N,\lambda)$ for 
$\ast\in \{\varnothing,\star,\iso\}$. 

\begin{theorem}\label{thm:discrepancy-related}
Let $\bx$ be a sequence in $[0,1]^r$. Then 
\begin{align*}
	\disc(\bx^N) &\leqslant \disc^\iso(\bx^N) \leqslant (4 r \sqrt r+1) \disc(\bx^N)^{1/r} ,\\
	\disc^\star(\bx^N) &\leqslant \disc(\bx^N) \leqslant 2^r \disc^\star(\bx^N) .
\end{align*}
\end{theorem}
\begin{proof}
The first inequality is Theorem 1.6, and the second is Example 1.2, both from 
\cite[Ch.2]{kuipers-niederreiter-1974}. 
\end{proof}

We can use the above to define discrepancy for sequences in $G^\natural$, the 
space of conjugacy classes in a compact connected semisimple Lie group.

Let $G^\simplc$ be the simply-connected cover of $G$. Choose a maximal torus 
$T\subset G^\simplc$; let $W=\N(T)/T$ be the Weyl group. Let $\ft=\Lie(T)$ and 
recall that the kernel of $\exp\colon \ft\twoheadrightarrow T$ is generated by 
the nodal vectors associated to the root system $\R(G^\simplc,T)$ 
\cite[9.6 Pr.11]{bourbaki-lie-alg-7-9}. Write $\{t_1,\dots,t_r\}\subset \ft$ 
for these vectors. The exponential map $\exp\colon \ft\to T$ induces an 
isomorphism $\ft/(\langle t_i\rangle \rtimes W) \to G^\natural$. In particular, 
we can use the basis $\{t_1,\dots,t_r\}$ to identify $\ft/\langle t_i\rangle$ 
with $[0,1]^r/W$. Let $p_G\colon [0,1]^r \twoheadrightarrow G^\natural$ be the 
surjection $(x_1,\dots,x_r)\mapsto \exp(\sum x_i t_i)$. This is a $\# W$-to-one 
map almost everywhere, so for a measure $\mu$ on $G^\natural$ the ``pullback 
measure'' 
\[
	p_G^\ast \mu(f) = \mu\left(\frac{1}{\# W} \sum_{w\in W} w^\ast f\right)
\]
makes sense. 

\begin{definition}
With the setup as above, let $\mu,\nu$ be probability measures on $G^\natural$. 
The \emph{$\ast$-discrepancy} between $\mu$ and $\nu$ is
\[
	\disc^\ast(\mu,\nu) = \disc^\ast(p_G^\ast \mu,p_G^\ast \nu).
\]
\end{definition}

\begin{example}
Let $G=\SU(2)$ with maximal torus 
\[
	T = \left\{ \begin{pmatrix} e^{2\pi i t} \\ & e^{-2\pi i t} \end{pmatrix} : -1\leqslant t <1\right\} .
\]
Then $W=S_2$, whose nontrivial element acts via $t\mapsto -t$. 
\end{example}

If $\nu$ is the Haar measure on $G^\natural$, we simply write 
$\disc(\mu)$ for $\disc(\mu,\Haar)$. 

The Koksma--Hlawka inequality bounds the difference between the Haar integral 
and weighted average of a function on $G^\natural$ in terms of the discrepancy 
of the sequence and the variation of the function. 

The following result is essential:

\begin{theorem}[Koksma, Hlawka]
Let $G$ be as above. Let $f\colon G^\natural\to \bC$ be such that $f\, \dd x$ 
is a measure with bounded variation. Then 
\[
	\left|\bx^C(f) - \int f\, \dd x\right| \leqslant \Var(f) \disc(\bx^C) .
\]
\end{theorem}
\begin{proof}
This is \cite[Th.~3.2]{okten-1999}. 
\end{proof}

We will often use the soft version of this inequality. Namely, assume 
$\int f\, \dd x=0$. Then $|\bx^C(f)| \ll_f \disc(\bx^C)$ as $C\to \infty$. 
Here is another way of putting it. The sequence $f(\bx)$ has
$|A_{f(\bx)}(C)| \ll_f \pi(C) \disc(\bx^C)$. 

\begin{theorem}
Let $\bx$ be a sequence in $[0,1]^r$. Then 
\[
	\disc^\iso(\bx^N,\mu) = \sup_{P\subset [0,1]^r} \left| \bx^N(P)-\mu(P)\right| ,
\]
where $P$ ranges over all open and closed convex polytopes contained in 
$[0,1]^r$. 
\end{theorem}
\begin{proof}
We follow the proof of \cite[Ch.2~Th.1.5]{kuipers-niederreiter-1974}. Clearly 
the supremum in question is bounded above by isotropic discrepancy, so we only 
need to show the opposite bound. Let $C\subset [0,1]^r$ be a convex 
set. Suppose $C$ contains $x_{i_1},\dots,x_{i_a}$. Then $C$ contains $P$, the 
convex hull of $\{x_{i_1},\dots,x_{i_a}\}$. 

Use the fact that given a convex set, and a point not in the interior of the 
set, the two can be separated by a hyperplane. Intersect half-planes, and get 
$P\subset C\subset Q$, with $P$ and $Q$ polytopes, and 
$\bx^N(P)=\bx^N(C)=\bx^N(Q)$. This yields (via $a\leqslant b\leqslant c$ implies 
$|b|\leqslant\max\{|a|,|c|\}$)
\[
	|\bx^N(C)-\mu(C)| \leqslant \max\{|\bx^N(P)-\mu(P)|,|\bx^N(Q)-\mu(Q)|\} .
\]
\end{proof}





\section{Main results}\label{sec:prelim-result}

\begin{theorem}
Let $\bz\in \bD^\infty$. Then $L(\bz,s)$ defines a holomorphic 
function on the region $\{\Re s>1\}$. Moreover, on that region, 
\[
	\log L(\bz,s) = \sum_{p^n} \frac{z_p^n}{n p^{n s}} .
\]
\end{theorem}
\begin{proof}
Expanding the product for $L(\bz,s)$ formally, we have 
\[
	L(\bz,s) = \sum_{n\geqslant 1} \frac{\prod_{p\mid n} z_p^{v_p(n)}}{n^s} .
\]
An easy comparison with Riemann's zeta function tells us that the series 
expansion is holomorphic on $\{\Re s>1\}$. By \cite[Th.~11.7]{apostol-1976}, 
the product formula holds on the same region. The formula for 
$\log L(\bz,s)$ comes from \cite[11.9 Ex.2]{apostol-1976}.
\end{proof}

\begin{theorem}\label{thm:main-for-sequences}
Assume $A_\bz(x) \ll x^{\alpha +\epsilon}$, $\alpha\in [\frac 1 2,1]$. 
Then $\log L(\bz,s)$ is holomorphic on $\{\Re>\alpha\}$.
\end{theorem}
\begin{proof}
Split the sum for $\log L$ into two pieces:
\[
	\log L(\bz,s) = \sum_p \frac{z_p}{p^s} + \sum_p \sum_{n\geqslant 2} \frac{z_p^n}{n p^{n s}} .
\]
For each $p$, we have 
\[
	\left|\sum_{n\geqslant 2} \frac{z_p^n}{n p^{ns}} \right| \leqslant \sum_{n\geqslant 2} p^{-n \Re s} = p^{-2\Re s} \frac{1}{1-p^{-\Re s}} .
\]
Elementary analysis gives 
\[
	1 \leqslant \frac{1}{1-p^{-\Re s}} \leqslant 2+2\sqrt 2 ,
\]
so the second piece of $\log L(\bz,s)$ converges absolutely when 
$\Re(s)>\frac 1 2$. By \cite[II.1 Th.10]{tenenbaum-1995}, our bound on 
$A_\bz(x)$ yields the holomorphy of $\sum z_p p^{-s}$ on $\{\Re >\alpha\}$. 
\end{proof}

\begin{corollary}\label{cor:ATRH}
Let $G$ be a compact connected semisimple Lie group, 
$\bx\in G^{\natural,\infty}$ satisfy 
$\disc(\bx^C,\dd x)\ll C^{-\frac 1 2+\epsilon}$. Then for every 
$f\in C^\alev(G^\natural)^{\|\cdot\|\leqslant 1}$, 
$L_f(\bx,s)$ has analytic continuation to $\{\Re s>\frac 1 2\}$, and satisfies 
the Riemann Hypothesis, for all $f$ bounded and almost-everywhere continuous 
with $\mu(f)=0$. 
\end{corollary}
\begin{proof}
Koksma--Hlawka tells that if $\mu(f)=0$, then 
$\bx^C(f)\ll C^{-\frac 1 2+\epsilon}$. Thus the sequence 
$f(\bx)$ satisfies $A_{f(\bx)}(x) \ll x^{\frac 1 2+\epsilon}$, and the result 
follows from Theorem \ref{thm:main-for-sequences}. 
\end{proof}





\section{Strange $L$-functions over function fields}\label{sec:function-field}

Let $k$ be a finite field of characteristic $p$ and cardinality $q$. Let 
$C_{/k}$ be a nice curve in the sense of Poonen (i.e., $C$ is smooth, 
projective, and geometrically integral). Write $K=k(C)$ for the function field 
of $C$. Fix a non-empty open subset $U\subset C$ and a geometric point 
$\infty\in U(\bar k)$. Fix a prime $l\ne p$ and an embedding 
$\overline{\bQ_l}\hookrightarrow \bC$. 

\begin{definition}\label{def:good-sheaf}
An $l$-adic sheaf $\cF$ on $U$ is \emph{good} if the following conditions hold. 
\begin{enumerate}
\item 
$\cF$ is pure of weight zero.
\item
Let $G=\overline{\rho_\cF(\pi_1(U_{\overline k},\infty))}^{\Zar}$. 
Assume $\rho_\cF(\pi_1(U,\infty))\subset G(\overline\bQ_l)$. 
\end{enumerate}
\end{definition}

For any good sheaf $\cF$, let $\ST(\cF)$ be a maximal compact subgroup of 
$G(\bC)$.For each $u\in U$, there is a well-defined conjugacy class 
$\theta(u) = \rho(\fr_u)^\semis \in \ST(\cF)^\natural$. For any $C>0$, write 
\[
	\btheta^C_\cF = \frac{1}{\#\{u\in U : q_u \leqslant C\}} \sum_{q_u\leqslant C} \delta_{\theta(u)} .
\]
Katz proves an equidistribution estimate for the $\theta(u)$'s. 

\begin{theorem}
Let $\sigma$ be a non-trivial irreducible representation of $\ST(\cF)$. Then 
\[
	|\btheta^C_\cF(\tr\sigma)| \ll_\cF \dim(\sigma) C^{-\frac 1 2} .
\]
\end{theorem}
\begin{proof}
This is \cite[p.39]{katz-1988}.
\end{proof}

Now let $C^\natural(\ST(\cF))$ be the space of functions 
$f\colon \ST(\cF)^\natural\to \bC$ satisfying:
\[
	\|f\|^\natural = \sum_\sigma \dim(\sigma)|\widehat f(\sigma)| < \infty .
\]
For such functions, we have:
\[
	|\btheta_\cF^C(f) - \mu(f)| \ll_\cF \|f\|^\natural C^{-\frac 1 2} .
\]
Thus for any $f\in C^\natural(\ST(\cF))$, the strange $L$-function 
$L_f(\btheta_\cF,s)$ has analytic continuation to $\{\Re s>\frac 1 2\}$ and 
satisfies the Riemann Hypothesis.

\begin{theorem}
Let $\bz\in \bD^\infty$, and assume $\log L(\bz,s)$ has analytic continuation 
to $\{\Re >\alpha\}$, $\alpha\in [\frac 1 2,1]$, and that for $\sigma>\alpha$, 
we have $|\log L(\bz,\sigma+i t)| \ll |t|^{1-\epsilon}$. Then 
$|A_\bz(x)| \ll x^{\alpha+\epsilon}$. 
\end{theorem}
\begin{proof}
Recall that we can write 
\[
	\log L(\bz,p) = \sum_p \frac{z_p}{p^s} + \sum_p \sum_{n\geqslant 2} \frac{z_p^n}{n p^{n s}} = \sum_p \frac{z_p}{p^s} + O(\zeta(2\Re s)).
\]
Thus, for any $\epsilon>0$, our bound on $|\log L(\bz,\sigma+i t)|$ implies the 
same bound for $\sum \frac{z_p}{p^s}$ on $\{\Re >\alpha+\epsilon\}$. 

Let $\gamma_T=\gamma_{1,T}+\gamma_{2,T}-\gamma_{3,T}-\gamma_{4,T}$ be the 
following contour:
\begin{align*}
	\gamma_{1,T}(t) &= (\alpha+\epsilon)+i t\qquad t\in [-T,T] \\
	\gamma_{2,T}(t) &= t+i T \qquad t\in [\alpha+\epsilon,1+\epsilon] \\
	\gamma_{3,T}(t) &= (1+\epsilon) + i t \qquad t\in [-T,T] \\
	\gamma_{4,T}(t) &= t - i T \qquad t\in [\alpha+\epsilon,1+\epsilon] .
\end{align*}
By \cite[Th.11.18]{apostol-1976}, 
\[
	\lim_{T\to \infty} \frac{1}{2\pi i} \int_{-\gamma_{3,T}} \sum_p \frac{z_p}{p^s} x^z\, \frac{\dd z}{z} =^\ast \sum_{p\leqslant x} z_p .
\]
Let $h(z)$ be the analytic continuation of $\sum z_p p^{-s}$ to 
$\{\Re >\alpha\}$. Since $\int_\gamma h(z) \frac{\dd z}{z} = 0$, we obtain
\[
	\left| \sum_{p\leqslant z} z_p\right| \ll \left|\int_{\gamma_{T,1}}h(z)x^z\frac{\dd z}{z}\right| + \left|\int_{\gamma_{T,2}}h(z)x^z\frac{\dd z}{z}\right| + \left|\int_{\gamma_{T,4}}h(z)x^z\frac{\dd z}{z}\right| .
\]
We know that $|h(\sigma+ i t)| \ll |t|$, so we can bound:
\[
	\left|\int_{\gamma_{T,2}}h(z)\frac{\dd z}{z}\right|
		= \left| \int_{\alpha+\epsilon}^{1+\epsilon} \frac{h(t+i T)x^{t+i T}}{t+i T}\, \dd t\right| 
		\ll (1+\alpha)x^{1+\alpha} T^{-1} ,
\]
and similarly for $\int_{\gamma_{4,T}}$. Finally, we note that 
\[
	\left|\int_{\gamma_{T,1}}h(z)x^z\frac{\dd z}{z}\right| \ll \int_{-T}^T |t|^{1-\epsilon} \frac{x^{\alpha+\epsilon}}{(\alpha+\epsilon)^2+t^2}\, \dd t \ll x^{\alpha+\epsilon} .
\]
Letting $T\to \infty$ we obtain the desired result.
\end{proof}





\section{Applications}\label{sec:application}


Recall, following \cite{bugeaud-2008} that the \emph{irrationality exponent} 
$\mu(\alpha)$ a real irrational number $\alpha$ is the supremum of all real 
numbers $\mu$ such that 
\[
	\left|\alpha-\frac{p}{q}\right| < q^{-\mu}
\]
for infinitely many $p/q\in \bQ$. Bugeaud proves that for any 
$\mu\geqslant 2$, there is an element $\xi_\mu$ of the Cantor set with 
$\mu(\xi_\mu)=\mu$. Moreover, by \cite[?]{kuipers-niederreiter-1974}, for 
every $\epsilon>0$, the sequence $x_n=n\alpha\mod 1$ has discrepancy 
$\disc(\bx^C)=\Omega(C^{-\frac{1}{\mu(\alpha)-1}-\epsilon})$. 

\begin{theorem}
Let $X=S^1$ with the natural Haar measure. For every $\eta\in (0,\frac 1 2)$, 
there is a sequence $\bx=(x_2,x_3,\dots)\in (S^1)^\infty$ such that for all 
$f\in C^\infty(S^1)^{\|\cdot\|_\infty\leqslant 1}$, the function 
$\log L_f(\bx,s)$ has analytic continuation to $\{\Re>\frac 1 2\}$, but for 
all $\epsilon>0$, $|\disc(\bx^C)|=\Omega(C^{-\eta-\epsilon})$. 
\end{theorem}
\begin{proof}
Let $\mu>3$, and let 
$\bx=\{x_2,x_3,\dots\}$ be the sequence $x_{p_n}=e^{2\pi i n \xi_\mu}$. To 
prove that $\log L_f(\bx,s)$ has analytic continuation to $\{\Re >\frac 1 2\}$, 
we need only to prove that $|A_{\exp(2\pi i m \bx)}(t)| \ll t^{1/2}$, uniformly 
for each $m\in \bZ$. This follows easily from:
\[
	\left| \sum_{n=1}^N e^{2\pi i m n \alpha}\right| \leqslant \frac{|-1+e^{2\pi i M n \alpha}|}{|-1+e^{2\pi i a m}|} \leqslant ? \leqslant \frac 1 2 m (\eta-1) \ll_\eta m
\]
\end{proof}

\begin{theorem}
Let $E_{/\bQ}$ be a non-CM elliptic curve, and put $\btheta=\btheta(E)$. 
Assume that $\disc(\btheta^C) \ll C^{-\frac 1 2+\epsilon}$. Then if 
$f\in C^\alev([0,\pi],\ST)^{\|\cdot\|_\infty\leqslant 1}$, the strange 
$L$-function $L_f(\btheta,s)$ has analytic continuation to 
$\{\Re >\frac 1 2\}$ and satisfy the Riemann Hypothesis. In particular, 
this holds for all $L(\sym^k E,s)$. 
\end{theorem}
\begin{proof}
The first conclusion follows from Corollary \ref{cor:ATRH}. The second part follows 
from the fact that any $L(\sym^k E,s)$ can be written as a product of $L_f$'s, 
namely the $L_{\lambda_{\sym^k}^j}$'s in Section \ref{sec:definition}.  
\end{proof}

\begin{theorem}
Fix $f\in C^\alev([0,\pi],\ST)^{\|\cdot\|_\infty\leqslant 1}$ that is not 
almost everywhere constant. 

Let $E_1,E_2$ be two non-isogenous, non-CM elliptic curves over $\bQ$. 
Assume the Akiyama--Tanigawa conjecture for the product $E_1\times E_2$. 
Then for any $f\colon [0,\pi]\to \bC$ that is not almost everywhere 
\end{theorem}






\section{A collection of counterexamples}

In \cite[?]{akiyama-tanigawa}, Akiyama and Tanigawa claim that for $E_{/\bQ}$, 
the ``discrepancy conjecture'' $\disc(\btheta^C)\ll C^{-\frac 1 2+\epsilon}$ is 
equivalent to the Riemann Hypothesis for $L(E,s)$. In this section, I construct 
a collection of examples which show that their conjecture is false for any 
motive with positive-dimensional Sato--Tate group. 

Throughout this section, $|\cdot|_\infty$ is the sup-norm, and $|\cdot|$ can be 
any of the (commensurable) $p$-norms on a finite-dimensional real vector space. 

\begin{definition}\label{def:irrationality-exponent}
Let $x\in \bR^r$ be such that $x_1,\dots,x_r$ are $\bQ$-linearly independent. 
Following \cite{laurent-2009}, we define $r$-dimensional \emph{irrationality 
exponents} as the suprema $\omega_0(x)$ and $\omega_{r-1}(x)$ of the sets of 
$w$ for which there are infinitely many $m=(m_0,\dots,m_r)\in \bZ^{r+1}$ for 
which 
\begin{align*}
	\max \{ |m_0x_i-m_i| \} &\leqslant |m|_\infty^{-w} \\
	|m_0 + m_1 x_1 + \cdots + m_r x_r| &\leqslant |m|_\infty^{-w} 
\end{align*}
respectively. 
\end{definition}

Given $x\in \bR^r$, write $d(x,\bZ^r)=\min_{m\in \bZ^r} |x-m|$. 

\begin{lemma}\label{lem:distance-asymptotic}
Let $x\in \bR^r$ with $|x|_\infty\leqslant 1$ and $\omega_0(x)$
(resp.~$\omega_{r-1}(x)$) is finite. Then 
\begin{align*}
	\frac{1}{d(n x,\bZ^r)} &\ll_{\epsilon,x} n^{\omega_0(x)+\epsilon} \qquad \text{as $n\to \infty$, (resp.)}\\
	\frac{1}{d(\langle m,x\rangle,\bZ)} &\ll_{\epsilon,x} |m|^{\omega_{r-1}(x)+\epsilon}\qquad \text{as $m\to \infty$ in $\bZ^r$ .}
\end{align*}
\end{lemma}
\begin{proof}
Let $\epsilon>0$. Then there are only finitely many $n\in \bN$ 
(resp.~$m\in \bZ^r$) such that the inequalities in 
Definition \ref{def:irrationality-exponent} hold with $\omega_0(x)+\epsilon$ 
(resp.~$\omega_{r-1}(x)+\epsilon$). In other words, there exist 
$C_0,C_{r-1}>0$ such that 
\begin{align*}
	\max\{|m_0 x_i - m_i|\} &\geqslant C_0 |m|_\infty^{-\omega_0(x)-\epsilon} \\
	|m_0 + m_1 x_1 + \cdots + m_r x_r| &\geqslant C_{r-1} |m|_\infty^{-\omega_{r-1}(x)-\epsilon} .
\end{align*}
for all $m\ne 0$. We consider the first inequality, temporarily setting 
$|\cdot|=|\cdot|_\infty$. Then $d(n x,\bZ^r)=\max\{|n x_i-m_i|\}$ for some 
$m_i$ such that $|m_i-n x_i|< 1$. Thus 
$|(n,m_1,\dots,m_r)| \leqslant \max\{|n|,|n x_i|\}\leqslant |n|$. In 
particular, 
\[
	d(nx,\bZ^r) \geqslant C_0 |n|^{-\omega_0(x)-\epsilon} ,
\]
which implies $\frac{1}{d(n x,\bZ^r)} \ll |n|^{\omega_0(x)+\epsilon}$, the 
implied constant depending on both $x$ and $\epsilon$. 

For the second inequality, temporarily set $|\cdot|=|\cdot|_1$, and note that 
$d(m_1 x_1+\cdots+m_r x_r,\bZ)=|m_0+m_1 x_1+\cdots + m_r x_r|$ for 
$|m_0| \leqslant |(m_1,\dots,m_r)|\cdot |x|+1$. Thus 
$|(m_0,\dots,m_r)|_\infty \leqslant 2 |x| |(m_1,\dots,m_r)|$, giving us 
\[
	d(m_1x_1 + \cdots m_r x_r,\bZ) \geqslant C_{r-1}' |(m_1,\dots,m_r)|^{-\omega_{r-1}(x)-\epsilon} ,
\]
which implies $\frac{1}{d(\langle m,x\rangle,\bZ)} \ll |m|^{\omega_{r-1}(x)+\epsilon}$, 
the implied constant again depending on both $x$ and $\epsilon$. 
\end{proof}

Let $\bT^r=(\bR/\bZ)^r$, with Haar measure normalized to have total mass one. 
Given $x\in \bT^r$, we define $\omega_0(x)$ and $\omega_{r-1}(x)$ as in 
Definition \ref{def:irrationality-exponent}, choosing any coset representative of $x$. 
This definition is independent of the choice. Recall that for 
$f\in L^1(\bT^r)$, the \emph{Fourier coefficients} of $f$ are, for $m\in \bZ^r$ 
\[
	\widehat f(m) = \int_{\bT^r} e^{2\pi i \langle m,x\rangle}\, \dd x ,
\]
where $\langle m,x\rangle=m_1 x_1 + \cdots + m_r x_r$ is the usual inner 
product. 


\begin{theorem}[Jarn\'{i}k]\label{thm:existence-exponent}
Let $w\geqslant 1/r$. Then there exists $x\in \bR^r$ such that 
$\omega_0(x)=w$ and $\omega_{r-1}(x)=r w+r-1$. 
\end{theorem}

\begin{theorem}
Fix $x\in \bT^r$ with $\omega_{r-1}(x)$ finite. Then 
\[
	\left| \sum_{n\leqslant N} e^{2\pi i \langle m, nx\rangle}\right| \ll_{\epsilon,x} |m|^{\omega_{r-1}(x)+\epsilon} 
\]
as $m$ ranges over $\bZ^r\smallsetminus 0$. 
\end{theorem}
\begin{proof}
First, note the easy bound:
\[
	\left| \sum_{n\leqslant N} e^{2\pi in\langle m,x\rangle}\right|
		= \left|\frac{e^{2\pi i N \langle m,x\rangle}-1}{e^{2\pi i \langle m,x\rangle}-1}\right| 
		\leqslant \frac{2}{|e^{2\pi i \langle m,x\rangle}-1|} .
\]
Since $|e^{2\pi i \langle m,x\rangle}-1| = \sqrt{2-2\cos(2\pi \langle m,x\rangle)}$ and 
$\cos(2\theta)=1-2\sin^2(\theta)$, we obtain 
$\left|\sum_{n\leqslant N} e^{2\pi i n \langle m,x\rangle}\right| \leqslant \frac{1}{|\sin(\pi \langle m,x\rangle)|}$. 
It is easy to check that $|\sin(\pi t)|\geqslant d(t,\bZ)$, hence 
$\left|\sum_{n\leqslant N} e^{2\pi i n \langle m,x\rangle}\right| \leqslant \frac{1}{d(\langle m,x\rangle,\bZ)}$. 
The final estimate comes from Lemma \ref{lem:distance-asymptotic}. 
\end{proof}

\begin{theorem}\label{thm:hard-sum-bound}
Assume $\omega_{r-1}(x)<\infty$. Let $f\in L^1(\bT^r)$ with 
$\widehat f(0)=0$ and suppose the Fourier coefficients of $f$ satisfy the bound 
$|\widehat f(m)| \ll |m|^{-\frac{1}{r-1}-\omega_{r-1}-\epsilon}$. Then 
\[
	\left|\sum_{n\leqslant N} f(n x)\right| \ll_{f,x} 1 .
\]
\end{theorem}
\begin{proof}
Write $f$ as a Fourier series:
\[
	f(x) = \sum_{m\in \bZ^r} \widehat f(m) e^{2\pi i (m\cdot x)} .
\]
Since $\int f=0$, we have $\widehat f(0)=0$. Thus we can compute 
\begin{align*}
	\left|\sum_{n\leqslant N} f(n x)\right| 
		&= \left| \sum_{n\leqslant N} \sum_{m\in \bZ^r\smallsetminus 0} \widehat f(m) e^{2\pi i n (m\cdot x)} \right| \\
		&\leqslant \sum_{m\in \bZ^r\smallsetminus 0} |\widehat f(m)| \left| \sum_{n\leqslant N} e^{2\pi i n (m\cdot x)}\right| \\
		&\ll_{x,\epsilon} \sum_{m\in \bZ^r\smallsetminus 0} |m|^{-\frac{1}{r-1}-\omega_{r-1}(x)-\epsilon} |m|^{\omega_{r-1}(x)+\epsilon/2} \\
		&\ll_{x,\epsilon} \sum_{m\in \bZ^r\smallsetminus 0} |m|^{-\frac{1}{r-1}-\epsilon/2}.
\end{align*}
The sum converges since the exponent is less than $-\frac{1}{r-1}$, and it 
doesn't depend on $N$, whence the result.
\end{proof}

\begin{corollary}\label{cor:smooth-bounds}
Assume $\omega_{r-1}(x)<\infty$, and let $f\in C^\infty(\bT^r)$ with 
$\widehat f(0)=0$. Then $\left|\sum_{n\leqslant N} f(n x)\right| \ll_{f,x} 1$. 
\end{corollary}
\begin{proof}
This follows from Theorem \ref{thm:hard-sum-bound} and the fact that the Fourier 
coefficients of a smooth function decay faster than $|m|^k$, for any 
$k\in (-\infty,-1]$. 
\end{proof}

\begin{theorem}\label{thm:discrepancy-lower-bound}
If $\omega_0(x)<\infty$, then the sequence $\bx=(n x)_{n\geqslant 1}$ in 
$\bT^r$ has discrepancy 
\[
	\disc(\bx^N) = \Omega\left(2^{-r\left(2+\frac{1}{\omega_0(x)}\right)-\epsilon} N^{-\frac{r}{\omega_0(x)}-\epsilon}\right) .
\]
\end{theorem}
\begin{proof}
We follow the proof of \cite[Ch.2, Th.3.3]{kuipers-niederreiter-1974}. First, 
given $\epsilon>0$, there exists $\delta>0$ such that 
$\frac{r}{\omega_0(x)-\delta} = \frac{r}{\omega_0(x)}+\epsilon$. 

By the definition of $\omega_0(x)$, there exist infinitely many 
$(q,m_1,\dots,m_r)$ with $q>0$ such that 
\[
	|q x-m|_\infty \leqslant (\max \{q,|m|_\infty|\})^{-\omega_0(x)+\delta/2} .	
\]
Since $\max\{q,|m|_\infty\}\geqslant q$, we derive the stronger statement 
that for infinitely many $q\to \infty$, there exists 
$m=(m_1,\dots,m_r)\in \bZ^r$ such that 
$|q x-m|_\infty \leqslant q^{-\omega_0(x)+\delta/2}$, or, equivalently, 
$|x-\frac{m}{q}| \leqslant q^{-1-\omega_0(x)+\delta/2}$. Pick such a $q$, and 
let $N=\lfloor q^{\omega_0(x)-\delta}\rfloor$. Then for $n\leqslant N$, we have 
$|n x-\frac n q m| \leqslant q^{-1-\delta/2}$. Thus, for $n\leqslant N$, each 
$n x$ is within $q^{-1-\delta/2}$ of the grid $\frac{1}{q} \bZ^r\subset \bT^r$. 
Thus, they miss a box with side lengths $q^{-1}-2 q^{-1-\delta/2}$. For 
$q$ sufficiently large, $q^{-1}-2 q^{-1-\delta/2} \geqslant 1/2q$, so the 
(non-star) discrepancy of $\bx^N$ is bounded below by $2^{-r} q^{-r}$. Since 
$q^{\omega_0(x)-\delta}\leqslant 2 N$, the (non-star) discrepancy at $N$ is 
bounded below by 
\[
	2^{-r} \left((2N)^{\frac{1}{\omega_0(x)+\delta}}\right)^{-r} = 2^{-r-\frac{r}{\omega_0(x)+\delta}} N^{-\frac{r}{\omega_0(x)+\delta}} = 2^{-r\left(1+\frac{1}{\omega_0(x)}\right)-\epsilon} N^{-\frac{r}{\omega_0(x)}-\epsilon} .
\]
Since $r$-dimensional star-discrepancy is bounded below by 
$2^{-r}$ times non-star discrepancy, we obtain the final result.
\end{proof}

The key point in the above theorem is that 
\[
	\disc(\bx^N) = \Omega_{x,r,\epsilon}\left(N^{-\frac{r}{\omega_0(x)}-\epsilon}\right) .
\]

\begin{theorem}\label{thm:fast-decay}
Let $\eta\in (0,1)$. Then there exists $x\in \bT^r$ such that for all 
$f\in C^\infty(\bT^r)$ with $\widehat f(0)=0$, the estimate 
\[
	\left| \sum_{n\leqslant N} f(n x)\right|\ll_{f,x} 1
\]
holds, but for which 
\[
	\disc(\bx^N) = \Omega_{\epsilon,r,x}\left(N^{-\eta - \epsilon}\right) .
\]
\end{theorem}
\begin{proof}
Let $w=\frac{r}{\eta} \geqslant \frac 1 r$. By 
Theorem \ref{thm:existence-exponent}, there exists $x\in \bT^r$ with 
$\omega_0(x)=w$ and $\omega_{r-1}(x)=r w+r-1$. The result follows easily 
from Corollary \ref{cor:smooth-bounds} and Theorem \ref{thm:discrepancy-lower-bound}. 
\end{proof}

\begin{lemma}[Moser]\label{lem:factorization}
Let $f$ be a smooth, nonnegative function on $[0,1]^r$ such that $\int f=1$ and 
$f$ vanishes only on the boundary of $[0,1]^r$. Then there is a unique 
factorization 
\[
	f(x_1,\dots,x_r) = f_1(x_1) f_2(x_1,x_2) \dotsm f_r(x_1,\dots,x_r)
\]
of $f$ into smooth functions such that 
\[
	\int_0^1 f_i(x_1,\dots,x_{i-1},t)\, \dd t = 1
\]
for all $1\leqslant i\leqslant r$. 
\end{lemma}
\begin{proof}
We prove this by induction on $r$. For $r=1$, the claim is trivial. Otherwise, 
fix $(x_1,\dots,x_{r-1})$. Then we are trying to solve the following problem. 
Find a factorization $g(t) = \lambda h(t)$, where $\int h=1$. This has the 
obvious (unique) solution $h(t) = g(t)/(\int g)$. Thus, we have:
\begin{align*}
	f_{r-1}(x_1,\dots,x_{r-1}) &= \int_0^1 f(x_1,\dots,x_{r-1},t)\, \dd t \\
	f_r(x_1,\dots,x_r) &= f(x_1,\dots,x_r) / f_{r-1}(x_1,\dots,x_{r-1}) .
\end{align*}
\end{proof}

\begin{lemma}\label{lem:translate-measure}
Let $\lambda$ be the Lebesgue measure on $[0,1]^r$, and $\mu=f\lambda$ 
where $f\geqslant 0$ is smooth, and $f\ne 0$ on the interior of $[0,1]^r$. Then 
there is a diffeomorphism $u\colon [0,1]^r\to [0,1]^r$, identity on the 
boundary, such that $u_\ast \lambda = \mu$. 
\end{lemma}
\begin{proof}
We follow \cite{moser-1965}. First, use Lemma \ref{lem:factorization} to 
factor $f$ as a product 
\[
	f(x_1,\dots,x_r) = f_1(x_1) f_2(x_1,x_2) \dotsm f_r(x_1,\dots,x_r) .
\]
Let 
\[
	u_i(x_1,\dots,x_i) = \int_0^{x_i} f_i(x_1,\dots,x_{i-1},t)\, \dd t .
\]
Then each $u_i$ is a strictly increasing function, and 
$u=(u_1,\dots,u_r)$ is a diffeomorphism of the unit square, which is the 
identity on the boundary. Moreover, 
\[
	\det (\Jac u) = \prod \frac{\dd u_i}{\dd x_i} = \prod f_i = f .
\]
Now, by the change of variables formula, 
\[
	\int \phi\, \dd u_\ast^{-1} \lambda = \int \phi\circ u^{-1}\, \dd\lambda = \int \phi \det(\Jac u)\, \dd \lambda = \int \phi\, \dd\mu ,
\]
i.e.~$\mu=u_\ast^{-1} \lambda$. 
\end{proof}

\begin{theorem}
Let $\mu$, $f$ be as above. Then there exists a sequence 
$\bx$ in $[0,1]^r$ such that $\disc(\bx^N,\mu)=\Omega(N^{-r\eta-\epsilon})$, 
but for which 
$|\sum g(x_n)| \ll_g 1$ for all smooth $g$ with $\mu(g)=0$. 
\end{theorem}
\begin{proof}
By Lemma \ref{lem:translate-measure}, there exists a boundary-preserving 
diffeomorphism $u\colon [0,1]^r \to [0,1]^r$, such that $u_\ast \lambda=\mu$, 
where $\lambda$ is the Lebesgue measure as above. 

Start with a sequence $\by_n=n y$, where $y$ is as in Theorem 
\ref{thm:fast-decay}. Let $\bx=u_\ast \by$, i.e.~$x_n = u(y_n)$. Then, if 
$\phi\in C^\infty([0,1]^r)$, the composite $\phi\circ u$ is also smooth, so 
\[
	\left| \sum_{n\leqslant N} \phi(u_\ast y_n)\right| = \left| \sum_{n\leqslant N} (\phi\circ u)(y_n)\right| \ll_{\phi\circ u, y} 1 .
\]
Thus, all we need is a lower bound on the discrepancy. The proof of 
Theorem \ref{thm:discrepancy-lower-bound} tells us that for infinitely 
many $N\to \infty$, there is an $r$-ball $B_N$ with volume 
$C N^{-\eta-\epsilon}$ ($C$ not depending on $N$) that does not contain any of 
$y_1,\dots,y_N$. By \cite[Th.2.1]{polyak-2001}, for $N$ sufficiently large, the 
set $u(B_N)$ is convex, and moreover $\mu(u(B_N)) = \lambda(B_N)$. Thus, since 
$\bx^N(u(B_N))=\varnothing$, we have 
\[
	\disc^\iso(\bx^N,\mu) \geqslant C N^{-\eta-\epsilon} ,
\]
and thus 
\[
	\disc(\bx^N,\mu) = \Omega( N^{-r\eta- \epsilon})
\]
as desired.

[Need: bounds relating discrepancy and isotropic discrepancy to hold for 
non-Lebesgue measure.]
\end{proof}

\begin{theorem}
Let $[0,1]^r,\mu$ be as above. Then there exists $\bx$ such that for all smooth 
$f$, $L_f(\bx,s)$ satisfies the Riemann Hypothesis (analytic continuation and 
no zeros on $\{\Re>\frac 1 2\}$, but for which 
$\disc(\bx^N)=\Omega(N^?)$. 
\end{theorem}






\bibliographystyle{alpha}
\bibliography{tidbit-sources}
\end{document}
