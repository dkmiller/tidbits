\documentclass{article}

\usepackage{amsmath,amssymb,amsthm,bookmark,thmtools}
\DeclareMathOperator{\disc}{disc}
\DeclareMathOperator{\GL}{GL}
\DeclareMathOperator{\Lie}{Lie}
\DeclareMathOperator{\N}{N}
\DeclareMathOperator{\R}{R}
\DeclareMathOperator{\sgn}{sgn}
\DeclareMathOperator{\sym}{sym}
\DeclareMathOperator{\tr}{tr}
\DeclareMathOperator{\U}{U}
\DeclareMathOperator{\Var}{Var}
\newcommand{\bC}{\mathbf{C}}
\newcommand{\bD}{\mathbf{D}}
\newcommand{\bF}{\mathbf{F}}
\newcommand{\bQ}{\mathbf{Q}}
\newcommand{\btheta}{{\boldsymbol{\theta}}}
\newcommand{\bx}{\boldsymbol{x}}
\newcommand{\bz}{{\boldsymbol z}}
\newcommand{\bZ}{\mathbf{Z}}
\newcommand{\cF}{\mathcal{F}}
\newcommand{\ft}{\mathfrak{t}}

\newcommand{\alev}{\mathrm{ae}}
\newcommand{\dd}{\mathrm{d}}
\newcommand{\fr}{\mathrm{fr}}
\newcommand{\simplc}{\mathrm{sc}}
\newcommand{\semis}{\mathrm{ss}}
\newcommand{\ST}{\mathrm{ST}}
\newcommand{\Zar}{\mathrm{Zar}}
\newtheorem{theorem}[subsection]{Theorem}
\newtheorem{lemma}[subsection]{Lemma}
\newtheorem{corollary}[subsection]{Corollary}
\theoremstyle{definition}
\newtheorem{definition}[subsection]{Definition}

\usepackage[
  hyperref = true,
  backend  = bibtex,
  sorting  = nyt,
  style    = alphabetic
]{biblatex}
\addbibresource{tidbit-sources.bib}
\hypersetup{colorlinks=true,linkcolor=green}



\title{Equidistribution and the analytic properties of a strange class of 
$L$-functions}
\author{Daniel Miller}

\begin{document}
\maketitle





\section{Motivation}

Let $E_{/\bQ}$ be an elliptic curve without complex multiplication. By an old 
theorem of Faltings \cite{faltings-1983}, the quantities 
\[
	a_p(E) = p + 1 - \# E(\bF_p) = \tr \rho_{E,l} (\fr_p)
\]
determine $E$ up to isogeny. That is, if $E_1$ and $E_2$ satisfy 
$a_p(E_1)=a_p(E_2)$ for all $E$, then $E_1$ and $E_2$ are isogenous. The 
starting point of this investigation is the 
corollary of a theorem of Harris, that the collection $\{\sgn a_p(E)\}_p$ in 
fact determines $E$ up to isogeny. Ramakrishna had the insight that this fact 
means the ``strange $L$-function''
\[
	L_{\sgn}(E,s) = \prod_p \frac{1}{1-\sgn a_p(E) p^{-s}} 
\]
determines $E$ up to isogeny. In this note, I define a more general class of 
strange $L$-functions, and show that their analytic properties are closely 
tied to the equidistribution of the $a_p(E)$. 

Here is a brief discussion of this generalization in the case of a non-CM curve 
$E_{/\bQ}$. It is convenient to repackage the traces of Frobenius as follows:
\[
	\theta_p(E) = \cos^{-1}(a_p(E)/2\sqrt p) .
\]
The Hasse Bound guarantees that the $\theta_p(E)$ are well-defined angles 
laying in the interval $[0,\pi]$. Write 
$\dd\ST = \frac{2}{\pi} \sin^2\theta\, \dd\theta$. Then the Sato--Tate 
conjecture (now a theorem \cite{barnet-lamb-etal-2011}) tells us that for any 
continuous function $f\colon [0,\pi]\to \bC$, we have
\[
	\left| \frac{1}{\pi(C)} \sum_{p\leqslant C} f(\theta_p) - \int_0^\pi f\, \dd\ST\right| = o(1)
\]
as $C\to \infty$. It is well-known that this follows from the analytic 
continuation (past $\Re s=1$) and non-vanishing except at $s=1$ of all the 
$L$-functions $L(\sym^k E,s)$ \cite[A.1, Th.1]{serre-1968}. We take as our 
starting point the much stronger conjecture, due to Akiyama--Tanigawa 
\cite{akiyama-tanigawa}, that 
\[
	\left| \frac{1}{\pi(C)} \sum_{p\leqslant C} f(\theta_p) - \int_0^\pi f\, \dd \mu_\ST\right| = O_f(C^{-\frac 1 2+\epsilon}) 
\]
for all continuous $f$. (Their conjecture is actually more general; we will 
discuss the precise statement later.)
They prove that this conjecture implies the Riemann Hypothesis for $E$. I 
prove that not only does their conjecture imply the Riemann Hypothesis for all 
$L(\sym^k E,s)$, it also does for all the strange $L$-functions 
\[
	L_f(E,s) = \prod_p \frac{1}{1-f(\theta_p(E)) p^{-s}}
\]

These results make perfect sense in a much more general context, and I will 
prove them there. In \autoref{sec:definition} I set up this context and 
carefully define strange $L$-functions. In \autoref{sec:prelim-result}, I 
prove basic analytic properties of the strange $L$-functions and 
connect their analytic properties with the equidistribution of a sequence. 
In \autoref{sec:function-field}, I apply these results where ``everything is 
known,'' i.e.~varieties over function fields.
Finally, in \autoref{sec:application}, I apply the general results to the 
following cases: a non-CM elliptic curve $E_{/\bQ}$, the product 
$E_1\times E_2$ of a pair of non-isogenous non-CM elliptic curves over $\bQ$, 
and the Jacobian of a generic genus-$2$ curve $C_{/\bQ}$. 





\section{Definitions}\label{sec:definition}

Let $\bD=\{z\in \bC : |z|\leqslant 1\}$. Write $\bD^\infty$ for the set of 
sequences in $\bD$ indexed by the primes, i.e.~$\bz\in\bD^\infty$ is 
$(z_2,z_3,\dots)$. The space $\bD^\infty$ is compact, and comes naturally 
equipped with the (product) Lebesgue measure, normalized to have mass $1$. 

\begin{definition}
Let $\bz\in\bD^\infty$. The associated \emph{strange $L$-function} is given by 
\[
	L(\bz,s) = \prod_p \frac{1}{1-z_p p^{-s}} ,
\]
wherever this product converges. 
\end{definition}

Elementary topology tells us that $L\colon \bD^\infty\times \bC^{\Re>1}\to \bC$ 
is continuous. 
We will see that for fixed $\bz\in \bD^\infty$, the analytic properties of 
$L(\bz,s)$ are closely tied to 
estimates for the sums $A_\bz(x) = \sum_{p\leqslant x} z_p$. One 
often gets such estimates in the context of equidistribution, which we consider 
next.

For the remainder of this section, let $X$ be a compact separable metric space with no 
isolated points. We write $X^\infty$ for the space of sequences in $X$ indexed 
by rational primes, i.e.~points $\bx\in X^\infty$ are of the form 
$\bx=(x_2,x_3,\dots)$. By \cite[Cor.2.3.16, Th.4.2.2]{engelking-1989}, the 
compact space $X^\infty$ is metrizable and separable, also with no isolated 
points. 

\begin{definition}
For $\bx\in X^\infty$ and $C>0$, write $\bx^C$ for the probability measure 
given by 
\[
	\int_X f\, \dd \bx^C = \bx^C(f) = \frac{1}{\pi(C)} \sum_{p\leqslant C} f(x_p) .
\]
\end{definition}

Let $\mu$ be a Borel measure on $X$. Recall that $\bx$ is 
\emph{$\mu$-equidistributed} if $\bx^C\to \mu$ weakly, i.e. 
$\bx^C(f) \to \mu(f)$ for all $f\in C(X)$. In fact, we can extend this to 
not-necessarily-continuous functions as follows:

\begin{theorem}[Mazzone]
Let $\mu$ be a Borel measure on $X$ and let $f\colon X\to \bC$ be bounded and 
measurable. Then $f$ is continuous almost everywhere if and only if 
$\bx^C(f) \to \mu(f)$ for all $\mu$-equidistributed $\bx$. 
\end{theorem}
\begin{proof}
This follows directly from the proof of \cite[Th.1]{mazzone-1995}.
\end{proof}

Fix a Borel measure $\mu$ on $X$, and write $C^\alev(X,\mu)$ for the space of 
bounded, almost-everywhere continuous functions $f\colon X\to \bC$. 


\begin{theorem}
Endowed with the supremum norm $\|f\|_\infty=\sup_{x\in X} |f(x)|$, 
$C^\alev(X,\mu)$ is a Banach space. 
\end{theorem}
\begin{proof}
This is an elementary corollary of the fact that a countable union of 
measure-zero sets has measure zero. 
\end{proof}

\begin{definition}
Let $f\in C^\alev(X,\mu)^{\|\cdot\|_\infty\leqslant 1}$, $\bx\in X^\infty$. The 
associated \emph{strange $L$-function} is defined as 
\[
	L_f(\bx,s) = L(f(\bx),s) = \prod_p \frac{1}{1-f(x_p) p^{-s}} 
\]
for all $s\in \bC$ for which the product converges. 
\end{definition}

Our typical source of a strange $L$-function is as follows. Let $G$ be a 
compact connected Lie group and $X=G^\natural$, the space of conjugacy 
classes of $G$. Then $G^\natural$ inherits the Haar measure from $G$. Given 
any sequence $\bx\in (G^\natural)^\infty = G^{\natural,\infty}$ and 
function $f\in C^\alev(G^\natural)^{\|\cdot\|_\infty\leqslant 1}$, we can 
define $L_f(\bx,s)$. This is 
related to Serre's $L$-functions from \cite[A.2]{serre-1968} as follows. 

\begin{theorem}
Let $G$ be a compact connected Lie group, $\rho\in\widehat G$ an irreducible 
unitary representation of $G$. Then there exist functions 
$\lambda_\rho^1,\dots,\lambda_\rho^{\deg\rho}\colon G^\natural \to S^1$, 
continuous away from the set $\{\det(1-\rho)=0\}$, such that for every 
$x\in G^\natural$, there are angles 
$\theta_1,\dots,\theta_{\deg\rho}\in [0,2\pi)$, satisfying 
$\theta_1\leqslant \cdots \leqslant \theta_{\deg\rho}$, such that 
$\lambda_\rho^j(x) = e^{i \theta_j}$ and moreover
\[
	\det(1-\rho(x) t) = \prod_{j=0}^{\deg \rho} (1-\lambda_\rho^j(x) t) .
\]
\end{theorem}
\begin{proof}
This follows easily from \cite[Lem.1.0.9]{katz-sarnak-1999}. 
\end{proof}

Recall that for $\rho\in \widehat G$, Serre defines 
$L(\rho,s) = \prod_p \det(1-\rho(x_p) p^{-s})^{-1}$. Using his notation, there 
is the identity 
\[
	L(\rho,s) = \prod_{j=1}^{\deg\rho} L_{\lambda_\rho^j}(\bx,s) .
\]



The rest of our definitions concern discrepancy, which for now we define only 
in a special context. Let $G$ be a compact connected semisimple Lie group. We 
will define discrepancy for sequences in $G^\natural$. 

Let $G^\simplc$ be the simply-connected cover of $G$. Choose a maximal torus 
$T\subset G^\simplc$; let $W=\N(T)/T$ be the Weyl group. Let $\ft=\Lie(T)$ and 
recall that the kernel of $\exp\colon \ft\twoheadrightarrow T$ is generated by 
the nodal vectors associated to the root system $\R(G^\simplc,T)$ 
\cite[9.6 Pr.11]{bourbaki-lie-alg-7-9}. Write $\{t_1,\dots,t_r\}\subset \ft$ 
for these vectors. The exponential map $\exp\colon \ft\to T$ induces an 
isomorphism $\ft/(\langle t_i\rangle \rtimes W) \to G^\natural$.  Given 
$x=(x_1,\dots,x_r)\in [0,1]^r$, write 
\[
	I_x = \left\{\sum_{i=1}^r a_i t_i : a_i \in [0,x_i] \right\} \subset \ft .
\]

\begin{definition}
With the setup as above, let $\mu,\nu$ be probability measures on $G^\natural$. 
The \emph{discrepancy} between $\mu$ and $\nu$ is
\[
	\disc(\mu,\nu) = \sup_{x\in [0,1]^r} \left|\mu(\exp I_x) - \nu(\exp I_x) \right| .
\]
\end{definition}

If $\nu=\dd x$, the Haar measure on $G^\natural$, we simply write 
$\disc(\mu)$ for $\disc(\mu,\dd x)$. 

The Koksma--Hlawka inequality bounds the difference between the Haar integral 
and weighted average of a function on $G^\natural$ in terms of the discrepancy 
of the sequence and the variation of the function. 

The following result is essential:

\begin{theorem}[Koksma, Hlawka]
Let $G$ be as above. Let $f\colon G^\natural\to \bC$ be such that $f\, \dd x$ 
is a measure with bounded variation. Then 
\[
	\left|\bx^C(f) - \int f\, \dd x\right| \leqslant \Var(f) \disc(\bx^C) .
\]
\end{theorem}
\begin{proof}
This is \cite[Th.~3.2]{okten-1999}. 
\end{proof}

We will often use the soft version of this inequality. Namely, assume 
$\int f\, \dd x=0$. Then $|\bx^C(f)| \ll_f \disc(\bx^C)$ as $C\to \infty$. 
Here is another way of putting it. The sequence $f(\bx)$ has
$|A_{f(\bx)}(C)| \ll_f \pi(C) \disc(\bx^C)$. 





\section{Main results}\label{sec:prelim-result}

\begin{theorem}
Let $\bz\in \bD^\infty$. Then $L(\bz,s)$ defines a holomorphic 
function on the region $\{\Re s>1\}$. Moreover, on that region, 
\[
	\log L(\bz,s) = \sum_{p^n} \frac{z_p^n}{n p^{n s}} .
\]
\end{theorem}
\begin{proof}
Expanding the product for $L(\bz,s)$ formally, we have 
\[
	L(\bz,s) = \sum_{n\geqslant 1} \frac{\prod_{p\mid n} z_p^{v_p(n)}}{n^s} .
\]
An easy comparison with Riemann's zeta function tells us that the series 
expansion is holomorphic on $\{\Re s>1\}$. By \cite[Th.~11.7]{apostol-1976}, 
the product formula holds on the same region. The formula for 
$\log L(\bz,s)$ comes from \cite[11.9 Ex.2]{apostol-1976}.
\end{proof}

\begin{theorem}\label{thm:main-for-sequences}
Assume $A_\bz(x) \ll x^{\alpha +\epsilon}$, $\alpha\in [\frac 1 2,1]$. 
Then $\log L(\bz,s)$ is holomorphic on $\{\Re>\alpha\}$.
\end{theorem}
\begin{proof}
Split the sum for $\log L$ into two pieces:
\[
	\log L(\bz,s) = \sum_p \frac{z_p}{p^s} + \sum_p \sum_{n\geqslant 2} \frac{z_p^n}{n p^{n s}} .
\]
For each $p$, we have 
\[
	\left|\sum_{n\geqslant 2} \frac{z_p^n}{n p^{ns}} \right| \leqslant \sum_{n\geqslant 2} p^{-n \Re s} = p^{-2\Re s} \frac{1}{1-p^{-\Re s}} .
\]
Elementary analysis gives 
\[
	1 \leqslant \frac{1}{1-p^{-\Re s}} \leqslant 2+2\sqrt 2 ,
\]
so the second piece of $\log L(\bz,s)$ converges absolutely when 
$\Re(s)>\frac 1 2$. By \cite[II.1 Th.10]{tenenbaum-1995}, our bound on 
$A_\bz(x)$ yields the holomorphy of $\sum z_p p^{-s}$ on $\{\Re >\alpha\}$. 
\end{proof}

\begin{corollary}\label{cor:ATRH}
Let $G$ be a compact connected semisimple Lie group, 
$\bx\in G^{\natural,\infty}$ satisfy 
$\disc(\bx^C,\dd x)\ll C^{-\frac 1 2+\epsilon}$. Then for every 
$f\in C^\alev(G^\natural)^{\|\cdot\|\leqslant 1}$, 
$L_f(\bx,s)$ has analytic continuation to $\{\Re s>\frac 1 2\}$, and satisfies 
the Riemann Hypothesis, for all $f$ bounded and almost-everywhere continuous 
with $\mu(f)=0$. 
\end{corollary}
\begin{proof}
Koksma--Hlawka tells that if $\mu(f)=0$, then 
$\bx^C(f)\ll C^{-\frac 1 2+\epsilon}$. Thus the sequence 
$f(\bx)$ satisfies $A_{f(\bx)}(x) \ll x^{\frac 1 2+\epsilon}$, and the result 
follows from \autoref{thm:main-for-sequences}. 
\end{proof}





\section{Strange \texorpdfstring{$L$}{L}-functions over function fields}\label{sec:function-field}

Let $k$ be a finite field of characteristic $p$ and cardinality $q$. Let 
$C_{/k}$ be a nice curve in the sense of Poonen (i.e., $C$ is smooth, 
projective, and geometrically integral). Write $K=k(C)$ for the function field 
of $C$. Fix a non-empty open subset $U\subset C$ and a geometric point 
$\infty\in U(\bar k)$. Fix a prime $l\ne p$ and an embedding 
$\overline{\bQ_l}\hookrightarrow \bC$. 

\begin{definition}\label{def:good-sheaf}
An $l$-adic sheaf $\cF$ on $U$ is \emph{good} if the following conditions hold. 
\begin{enumerate}
\item 
$\cF$ is pure of weight zero.
\item
Let $G=\overline{\rho_\cF(\pi_1(U_{\overline k},\infty))}^{\Zar}$. 
Assume $\rho_\cF(\pi_1(U,\infty))\subset G(\overline\bQ_l)$. 
\end{enumerate}
\end{definition}

For any good sheaf $\cF$, let $\ST(\cF)$ be a maximal compact subgroup of 
$G(\bC)$.For each $u\in U$, there is a well-defined conjugacy class 
$\theta(u) = \rho(\fr_u)^\semis \in \ST(\cF)^\natural$. For any $C>0$, write 
\[
	\btheta^C_\cF = \frac{1}{\#\{u\in U : q_u \leqslant C\}} \sum_{q_u\leqslant C} \delta_{\theta(u)} .
\]
Katz proves an equidistribution estimate for the $\theta(u)$'s. 

\begin{theorem}
Let $\sigma$ be a non-trivial irreducible representation of $\ST(\cF)$. Then 
\[
	|\btheta^C_\cF(\tr\sigma)| \ll_\cF \dim(\sigma) C^{-\frac 1 2} .
\]
\end{theorem}
\begin{proof}
This is \cite[p.39]{katz-1988}.
\end{proof}

Now let $C^\natural(\ST(\cF))$ be the space of functions 
$f\colon \ST(\cF)^\natural\to \bC$ satisfying:
\[
	\|f\|^\natural = \sum_\sigma \dim(\sigma)|\widehat f(\sigma)| < \infty .
\]
For such functions, we have:
\[
	|\btheta_\cF^C(f) - \mu(f)| \ll_\cF \|f\|^\natural C^{-\frac 1 2} .
\]
Thus for any $f\in C^\natural(\ST(\cF))$, the strange $L$-function 
$L_f(\btheta_\cF,s)$ has analytic continuation to $\{\Re s>\frac 1 2\}$ and 
satisfies the Riemann Hypothesis.

\begin{theorem}
Let $\bz\in \bD^\infty$, and assume $\log L(\bz,s)$ has analytic continuation 
to $\{\Re >\alpha\}$, $\alpha\in [\frac 1 2,1]$, and that for $\sigma>\alpha$, 
we have $|\log L(\bz,\sigma+i t)| \ll |t|^{1-\epsilon}$. Then 
$|A_\bz(x)| \ll x^{\alpha+\epsilon}$. 
\end{theorem}
\begin{proof}
Recall that we can write 
\[
	\log L(\bz,p) = \sum_p \frac{z_p}{p^s} + \sum_p \sum_{n\geqslant 2} \frac{z_p^n}{n p^{n s}} = \sum_p \frac{z_p}{p^s} + O(\zeta(2\Re s)).
\]
Thus, for any $\epsilon>0$, our bound on $|\log L(\bz,\sigma+i t)|$ implies the 
same bound for $\sum \frac{z_p}{p^s}$ on $\{\Re >\alpha+\epsilon\}$. 

Let $\gamma_T=\gamma_{1,T}+\gamma_{2,T}-\gamma_{3,T}-\gamma_{4,T}$ be the 
following contour:
\begin{align*}
	\gamma_{1,T}(t) &= (\alpha+\epsilon)+i t\qquad t\in [-T,T] \\
	\gamma_{2,T}(t) &= t+i T \qquad t\in [\alpha+\epsilon,1+\epsilon] \\
	\gamma_{3,T}(t) &= (1+\epsilon) + i t \qquad t\in [-T,T] \\
	\gamma_{4,T}(t) &= t - i T \qquad t\in [\alpha+\epsilon,1+\epsilon] .
\end{align*}
By \cite[?]{?}, 
\[
	\lim_{T\to \infty} \frac{1}{2\pi i} \int_{-\gamma_{3,T}} \sum_p \frac{z_p}{p^s} x^z\, \frac{\dd z}{z} =^\ast \sum_{p\leqslant x} z_p .
\]
Let $h(z)$ be the analytic continuation of $\sum z_p p^{-s}$ to 
$\{\Re >\alpha\}$. Since $\int_\gamma h(z) \frac{\dd z}{z} = 0$, we obtain
\[
	\left| \sum_{p\leqslant z} z_p\right| \ll \left|\int_{\gamma_{T,1}}h(z)x^z\frac{\dd z}{z}\right| + \left|\int_{\gamma_{T,2}}h(z)x^z\frac{\dd z}{z}\right| + \left|\int_{\gamma_{T,4}}h(z)x^z\frac{\dd z}{z}\right| .
\]
We know that $|h(\sigma+ i t)| \ll |t|$, so we can bound:
\[
	\left|\int_{\gamma_{T,2}}h(z)\frac{\dd z}{z}\right|
		= \left| \int_{\alpha+\epsilon}^{1+\epsilon} \frac{h(t+i T)x^{t+i T}}{t+i T}\, \dd t\right| 
		\ll (1+\alpha)x^{1+\alpha} T^{-1} ,
\]
and similarly for $\int_{\gamma_{4,T}}$. Finally, we note that 
\[
	\left|\int_{\gamma_{T,1}}h(z)x^z\frac{\dd z}{z}\right| \ll \int_{-T}^T |t|^{1-\epsilon} \frac{x^{\alpha+\epsilon}}{(\alpha+\epsilon)^2+t^2}\, \dd t \ll x^{\alpha+\epsilon} .
\]
Letting $T\to \infty$ we obtain the desired result.
\end{proof}





\section{Applications}\label{sec:application}

\begin{theorem}
Let $E_{/\bQ}$ be a non-CM elliptic curve, and put $\btheta=\btheta(E)$. 
Assume that $\disc(\btheta^C) \ll C^{-\frac 1 2+\epsilon}$. Then if 
$f\in C^\alev([0,\pi],\ST)^{\|\cdot\|_\infty\leqslant 1}$, the strange 
$L$-function $L_f(\btheta,s)$ has analytic continuation to 
$\{\Re >\frac 1 2\}$ and satisfy the Riemann Hypothesis. In particular, 
this holds for all $L(\sym^k E,s)$. 
\end{theorem}
\begin{proof}
The first conclusion follows from \autoref{cor:ATRH}. The second part follows 
from the fact that any $L(\sym^k E,s)$ can be written as a product of $L_f$'s, 
namely the $L_{\lambda_{\sym^k}^j}$'s in \autoref{sec:definition}.  
\end{proof}

\begin{theorem}
Fix $f\in C^\alev([0,\pi],\ST)^{\|\cdot\|_\infty\leqslant 1}$ that is not 
almost everywhere constant. 

Let $E_1,E_2$ be two non-isogenous, non-CM elliptic curves over $\bQ$. 
Assume the Akiyama--Tanigawa conjecture for the product $E_1\times E_2$. 
Then for any $f\colon [0,\pi]\to \bC$ that is not almost everywhere 
\end{theorem}





\printbibliography
\end{document}
