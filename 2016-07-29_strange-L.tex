\documentclass{article}

\usepackage{amsmath,amssymb,amsthm,bookmark,thmtools}
\DeclareMathOperator{\sgn}{sgn}
\DeclareMathOperator{\sym}{sym}
\DeclareMathOperator{\tr}{tr}
\newcommand{\bC}{\mathbf{C}}
\newcommand{\bF}{\mathbf{F}}
\newcommand{\blambda}{{\boldsymbol{\lambda}}}
\newcommand{\bQ}{\mathbf{Q}}
\newcommand{\btheta}{{\boldsymbol{\theta}}}
\newcommand{\bx}{\boldsymbol{x}}
\newcommand{\bZ}{\mathbf{Z}}
\newcommand{\alev}{\mathrm{ae}}
\newcommand{\dd}{\mathrm{d}}
\newcommand{\fr}{\mathrm{fr}}
\newcommand{\ST}{\mathrm{ST}}
\newtheorem{theorem}[subsection]{Theorem}
\newtheorem{lemma}[subsection]{Lemma}
\newtheorem{corollary}[subsection]{Corollary}
\theoremstyle{definition}
\newtheorem{definition}[subsection]{Definition}

\usepackage[
  hyperref = true,
  backend  = bibtex,
  sorting  = nyt,
  style    = alphabetic
]{biblatex}
\addbibresource{tidbit-sources.bib}
\hypersetup{colorlinks=true,linkcolor=green}

\title{Equidistributed sequences and the analytic properties of a strange class of $L$-functions}
\author{Daniel Miller}

\begin{document}
\maketitle





\section{Motivation}

Let $E_{/\bQ}$ be an elliptic curve without complex multiplication. By an old 
theorem of Faltings, the quantities 
\[
	a_p(E) = p + 1 - \# E(\bF_p) = \tr \rho_{E,l} (\fr_p)
\]
determine $E$ up to isogeny. The starting point of this investigation is the 
corollary of a theorem of Harris, that the collection $\{\sgn a_p(E)\}_p$ in 
fact determines $E$ up to isogeny. Ramakrishna had the insight that this fact 
means the ``strange $L$-function''
\[
	L_{\sgn}(E,s) = \prod_p \frac{1}{1-\sgn a_p(E) p^{-s}} 
\]
determines $E$ up to isogeny. In this note, I define a more general class of 
strange $L$-functions, and show that their analytic properties are closely 
tied to the equidistribution of the $a_p(E)$. 

Here is a brief discussion of this generalization in the case of a non-CM curve 
$E_{/\bQ}$. It is convenient to repackage these traces of 
Frobenius as follows:
\[
	\theta_p(E) = \cos^{-1}(a_p(E)/2\sqrt p) .
\]
The Hasse Bound guarantees that the $\theta_p(E)$ are well-defined angles 
laying in the interval $[0,\pi]$. Write 
$\mu_\ST = \frac{2}{\pi} \sin^2\theta\, \dd\theta$. Then the Sato--Tate 
conjecture (now a theorem) tells us that for any continuous function 
$f\colon [0,\pi]\to \bC$, we have:
\[
	\left| \frac{1}{\pi(C)} \sum_{p\leqslant C} f(\theta_p) - \int_0^\pi f\, \dd \mu_\ST\right| = o(1)
\]
as $C\to \infty$. It is well-known that this is equivalent to the analytic 
continuation of all the $L$-functions $L(\sym^k E,s)$. We take as our starting 
point the stronger conjecture, due to Akiyama--Tanigawa 
\cite{akiyama-tanigawa}, that 
\[
	\left| \frac{1}{\pi(C)} \sum_{p\leqslant C} f(\theta_p) - \int_0^\pi f\, \dd \mu_\ST\right| = O_f(C^{-\frac 1 2+\epsilon}) .
\]
They prove that this conjecture implies the Riemann Hypothesis for $E$. I 
prove that not only does their conjecture imply the Riemann Hypothesis for all 
$L(\sym^k E,s)$, it also does for all the strange $L$-functions 
\[
	L_f(E,s) = \prod_p \frac{1}{1-f(\theta_p(E)) p^{-s}}
\]

These results make perfect sense in a much more general context, and I will 
prove them there. In \autoref{sec:definition} I set up this context and 
carefully define strange $L$-functions there. In \autoref{sec:prelim-result}, I 
prove basic analytic properties of the strange $L$-functions, and in 
\autoref{sec:main-result}, I prove the main results connecting the analytic 
properties of strange $L$-functions with the equidistribution of a sequence. 
Finally, in \autoref{sec:application}, I apply the general results to the 
following cases: a non-CM elliptic curve $E_{/\bQ}$, the product 
$E_1\times E_2$ of a pair of non-isogenous non-CM elliptic curves over $\bQ$, 
and the Jacobian of a generic genus-$2$ curve $C_{/\bQ}$. 





\section{Definitions}\label{sec:definition}

Throughout this section, let $X$ be a compact separable metric space with no 
isolated points. We write $X^\infty$ for the space of sequences in $X$ indexed 
by rational primes, i.e.~points $\bx\in X^\infty$ are of the form 
$\bx=(x_2,x_3,\dots)$. By \cite[Cor.~2.3.16 \& Th.~4.2.2]{engelking-1989}, the 
compact space $X^\infty$ is metrizable and separable, also with no isolated 
points. 

\begin{definition}
For $\bx\in X^\infty$ and $C>0$, write $\bx^C$ for the probability measure 
given by 
\[
	\int_X f\, \dd \bx^C = \bx^C(f) = \frac{1}{\pi(C)} \sum_{p\leqslant C} f(x_p) .
\]
\end{definition}

Let $\mu$ be a Borel measure on $X$. Recall that $\bx$ is 
\emph{$\mu$-equidistributed} if $\bx^C\to \mu$ weakly, i.e. 
$\bx^C(f) \to \mu(f)$ for all $f\in C(X)$. In fact, we can extend this to 
not-necessarily-continuous functions as follows:

\begin{theorem}[Mazzone]
Let $\mu$ be a Borel measure on $X$ and let $f\colon X\to \bC$ be bounded and 
measurable. Then $f$ is continuous almost everywhere if and only if 
$\bx^C(f) \to \mu(f)$ for all $\mu$-equidistributed $\bx$. 
\end{theorem}
\begin{proof}
This follows directly from the proof of \cite[Th.~1]{mazzone-1995}.
\end{proof}

Fix a Borel measure $\mu$ on $X$, and write $C^\alev(X,\mu)$ for the space of 
bounded, almost-everywhere continuous functions $f\colon X\to \bC$. 


\begin{theorem}
Endowed with the supremum norm $\|f\|_\infty=\sup_{x\in X} |f(x)|$, 
$C^\alev(X,\mu)$ is a Banach space. 
\end{theorem}
\begin{proof}
This is an elementary corollary of the fact that a countable union of 
measure-zero sets has measure zero. 
\end{proof}

\begin{definition}
Let $f\in C^\alev(X,\mu)$, $\bx\in X^\infty$. The associated \emph{strange 
$L$-function} is defined as 
\[
	L_f(\bx,s) = \prod_p \frac{1}{1-f(x_p) p^{-s}} 
\]
for all $s\in \bC$ for which the product converges. 
\end{definition}

The rest of our definitions concern discrepancy, which for now we define only 
in a special context. Let $G$ be a compact connected Lie group. 





\section{Preliminary results}\label{sec:prelim-result}

Here we make a yet more general definition. Given 
$\blambda=(\lambda_2,\lambda_3,\dots)$, with 
$\|\blambda\|=\sup_p |\lambda_p|\leqslant 1$, define 
\[
	L(\blambda,s) = \prod_p \frac{1}{1-\lambda_p p^{-s}} .
\]
Write 
$A_\blambda(x) = \sum_{p\leqslant x} \lambda_p$. We make the following 
assumption: $A_\blambda(x) = O(x^{\frac 1 2+\epsilon})$. 

\begin{theorem}
Assume $A_\blambda(x) = O(x^{\frac 1 2 +\epsilon})$. Then $L(\blambda,s)$ 
converges on $\{\Re>\frac 1 2\}$, and $\log L(\blambda,s)$ has no poles on that 
region. 
\end{theorem}
\begin{proof}
Standard reductions reduce this to showing that 
\[
	\sum_p \frac{\lambda_p}{p^s} \qquad \textnormal{and}\qquad \sum_p \frac{\log(p) \lambda_p}{p^s} 
\]
converge on that region. We deal with $\sum \log(p)\lambda_p p^{-s}$; the 
other is similar. Use Abel summation:
\[
	\sum_{p\leqslant x} \frac{\lambda_p}{p^s}
		= \frac{\log x}{x^s} A_\blambda(x) - \int_2^x \frac{1-s\log t}{t^{s+1}} A_\blambda(t)\, \dd t .
\]
We show that the first term approaches zero and that the integral converges 
absolutely. We have: 
\[
	\left| \frac{\log x}{x^s} A_\blambda(x)\right| \ll \frac{\log x}{x^{\Re s}} x^{\frac 1 2+\epsilon}  .
\]
Since $\epsilon$ is arbitrary, the exponent of $x$ is negative. Moreover, 
\begin{align*}
	\int_2^x \frac{1}{t^{s+1}} | A_\lambda(t)|\, \dd t
		& \ll \int_2^x \frac{1}{t^{\Re s+1}} t^{\frac 1 2+\epsilon} \, \dd t \\
	\int_2^x \frac{\log t}{t^{s+1}} |A_\blambda(t)|\, \dd t
		& \ll \int_2^x \frac{\log t}{t^{\Re s+1}} t^{\frac 1 2+\epsilon}\, \dd t .
\end{align*}
Both these integrals converge because $\epsilon$ is arbitrary. 
\end{proof}





\section{Main results}\label{sec:main-result}

Let $E_{/\bQ}$ be an elliptic curve, or more generally, let 
$M$ be a motive. The associated analytic $L$-function $L(M,s)$ is of the form 
\[
	L(M,s) = \prod_p P_p(M,p^{-s})^{-1} ,
\]
where the $P_p(M,t)\in \bZ[t]$ have absolute value $1$. In the case of 
$E_{/\bQ}$, we have $p t^2-a_p t + 1$, which are normalized to 
\[
	(t-e^{i\theta_p})(t-e^{-i\theta_p}) = t^2 - 2\cos(\theta_p) t + 1 = t^2 - \frac{a_p}{\sqrt p} t + 1 .
\]
Let $d=\deg P_p(M,t)$. Then we can write 
\[
	P_p(M,t) = (t-e^{i\theta_p^{(1)}}) \dots (t-e^{-i\theta_p^{(d)}}) ,
\]
where $\theta^{(1)} < \cdots < \theta^{(d)}$ in $[0,2\pi]$. Then 
\[
	L(M,s) = L(\btheta^{(1)},s) \dotsm L(\btheta^{(d)},s)
\]

More general example:
\[
	L(\sym^k E,s) = L(\btheta^k, s) L(\btheta^{k-1}
\]





\section{Connection to Serre's perspective}

Let $G$ be a compact connected Lie group, $G^\natural$ the space of conjugacy 
classes in $G$, and $\bx$ a sequence in $G^\natural$. Given 
$\rho\in \widehat G$, Serre defines an $L$-function 
\[
	L(\rho,s) =\prod_p \det(1-\rho(x_p)p^{-s})^{-1} .
\]
Given $x\in G^\natural$, the matrix $\rho(x)$ has eigenvalues 
$\lambda_p^{(1),\rho},\dots,\lambda_p^{(\deg\rho),\rho}$ whose angles form a 
nondecreasing sequence in $[0,2\pi]$. The functions 
$\lambda_p^{(j),\rho}\colon G^\natural\to \bC$ are almost-everywhere 
continuous, and 
\[
	L(\rho,s) = \prod_{j=0}^{\deg\rho} L(\lambda_p^{(j),\rho},s) = \prod_{j=0}^{\deg\rho} L_{\lambda^{(j),\rho}}(\bx,s).
\]





\section{Applications}\label{sec:application}





\printbibliography
\end{document}
