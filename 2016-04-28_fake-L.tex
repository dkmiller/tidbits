\documentclass{article}

\usepackage{
	amsmath,
	amssymb,
	amsthm,
	hyperref,
	microtype
}
\usepackage[a5paper,margin=1.5cm]{geometry}
\usepackage[
  hyperref = true,
  backend  = bibtex,
  sorting  = nyt,
  style    = alphabetic
]{biblatex}
\addbibresource{tidbit-sources.bib}
\hypersetup{colorlinks=true,linkcolor=green}

\DeclareMathOperator{\GL}{GL}
\DeclareMathOperator{\tr}{tr}
\DeclareMathOperator{\Var}{Var}
\newcommand{\bE}{\mathbf{E}}
\newcommand{\bF}{\mathbf{F}}
\newcommand{\bQ}{\mathbf{Q}}
\newcommand{\bZ}{\mathbf{Z}}
\newcommand{\dd}{\mathrm{d}}
\newcommand{\fr}{\mathrm{fr}}
\newcommand{\ST}{\mathrm{ST}}

\newtheorem{theorem}{Theorem}
\newtheorem{conjecture}[theorem]{Conjecture}
\newtheorem{lemma}[theorem]{Lemma}
\numberwithin{theorem}{section}

\title{Riemann Hypothesis for ``fake $L$-functions'' coming from elliptic curves}
\author{Daniel Miller}

\begin{document}
\maketitle





\section{Introduction}

Let $E_{/\bQ}$ be an elliptic curve, $l$ an odd prime, and 
$\rho_{E,l}:G_\bQ\to \GL_2(\bZ_l)$ the associated Galois representation. 
For $p$ not dividing the conductor of $E$, we define 
\begin{align*}
	a_p &= p+1-\# E(\bF_p) \\
		&= \tr \rho_{E,l}(\fr_p) .
\end{align*}
Then the Hasse bound $|a_p|\leqslant 2\sqrt p$ allows us to define an angle 
$\theta_p=\theta_p(E)\in [0,\pi]$ by the equation 
\[
	2\cos \theta_p = \frac{a_p}{\sqrt p} .
\]
Let $S([0,\pi])$ be the space of step functions on $[0,\pi]$. For 
$\eta\in S([0,\pi])$ with $\|\eta\|_\infty\leqslant 1$, we define a ``fake 
$L$-function'' $L_\eta(E,s)$ by an infinite product 
\[
	L_\eta(E,s) = \prod_p \left(1-\eta(\theta_p) p^{-s}\right)^{-1}
\]
where here and in the remainder of the document we take the product over all 
unramified primes of $E$. 

Our goal is to show that, assuming a kind of ``strong Sato--Tate conjecture'' 
for $E$, the $L_\eta(E,s)$ satisfy the appropriate analogue of the Riemann 
Hypothesis. 





\section{Heuristics}

Just for this section, let $\{\theta_p\}$ be a collection of i.i.d.~random 
variables with joint distribution $\mu_\ST$ and $f$ a function on $[0,\pi]$. 
Then $\{f(\theta_p)\}$ is a collection of i.i.d.~random variables with 
expected value $\bE[f(\theta_p)] = \mu_\ST(f)$. The strong law of large 
numbers tells us that 
\[
	\frac{1}{\pi(X)} \sum_{p\leqslant X} f(\theta_p) \to \mu_\ST(f) .
\]
We want quantitative bound on the convergence. A basic but involved computation 
yields 
\[
	\bE\left[\left(\frac{1}{\pi(X)} \sum_{p\leqslant X} f(\theta_p) - \mu_\ST(f)\right)^2\right] 
	= \frac{1}{\pi(X)} \Var(f)
\]
So heuristically, if the $\theta_p$ come from an elliptic curve, we expect 
\[
	\left| \frac{1}{\pi(X)} \sum_{p\leqslant X} f(\theta_p) - \int f\, \dd \mu_\ST\right| = O\left(\Var(f) X^{-\frac 1 2+\epsilon}\right) .
\]





\section{First properties}

Throughout this section, $E_{/\bQ}$ is an elliptic curve and all notation is as 
above. 

\begin{lemma}
The product for $L_\eta(E,s)$ converges absolutely on the region 
$\{\Re s>1\}$. 
\end{lemma}
\begin{proof}
By \cite[\S3.7, Th.~5]{knopp-1956}, it suffices to prove that the sum 
$\sum_p \frac{|\eta(\theta_p)|}{p^s}$ converges when $\Re s>1$. This is 
simple: 
\[
	\sum_p \frac{|\eta(\theta_p)|}{p^s} \leqslant \sum_n n^{-s} = \zeta(s)
\]
which is already known to converge. 
\end{proof}

Let $\mu_\ST$ be the Sato--Tate measure 
$\frac{2}{\pi}\sin^2\theta\, \dd \theta$ and, for each $X>0$, define a measure 
$\mu^X=\frac{1}{\pi(X)}\sum_{p\leqslant X} \delta_{\theta_p}$. Then the 
Sato--Tate conjecture states that $\mu^X\to^\ast \mu_\ST$, in the sense that 
for each $f\in C([0,\pi])$, 
\[
	\lim_{X\to \infty} \frac{1}{\pi(X)} \sum_{p\leqslant X} f(\theta_p) = \int_{[0,\pi]} f\, \dd \mu_\ST .
\]
To simplify notation, for any measure $\mu$, write $\mu(f)=\int f\, \dd \mu$. 

\begin{conjecture}[Akiyama--Tanigawa]
\[
	\sup_{x\in [0,\pi]} \left| (\mu^X - \mu_\ST)(\chi_{[x,\pi]})\right| = O(X^{-\frac 1 2+\epsilon}) .
\]
\end{conjecture}
This is a trivial rewriting of the statement in 
\cite[Conj.~1]{akiyama-tanigawa}. 

\begin{theorem}
Assume the Akiyama--Tanigawa conjecture. Then for any step function 
$\eta\colon [0,\pi] \to [-1,1]$, we have 
\[
	\left| (\mu^X-\mu_\ST)(\eta)\right| = O_\eta(X^{-\frac 1 2+\epsilon}) .
\]
\end{theorem}
\begin{proof}
We may write $\eta = \sum \lambda_i \chi_{[x_i,\pi]}$. Then 
\begin{align*}
	\left| (\mu^X-\mu_\ST)(\eta)\right|
		&= \left| \sum \lambda_i (\mu^X-\mu_\ST)(\chi_{[x_i,\pi]}) \right| \\
		&\leqslant \sum |\lambda_i| \left| (\mu^X-\mu_\ast)(\chi_{[x_i,\pi]})\right| \\
		&= \left(\sum |\lambda_i|\right) O(X^{-\frac 1 2+\epsilon}) .
\end{align*}
\end{proof}





\printbibliography
\end{document}
