\documentclass{article}

\usepackage[a5paper]{geometry}
\usepackage[cm]{fullpage}
\usepackage{amsmath,amssymb,amsthm,bm,fullpage,mathrsfs,microtype,stmaryrd,tikz-cd}
\DeclareMathOperator{\ext}{Ext}
\DeclareMathOperator{\sheaves}{Sh}
\DeclareMathOperator{\spf}{Spf}
\DeclareMathOperator{\witt}{W}
\newcommand{\bZ}{\mathbf{Z}}
\newcommand{\cS}{\mathcal{S}}
\newcommand{\cX}{\mathcal{X}}
\newcommand{\cY}{\mathcal{Y}}
\newcommand{\fa}{\mathfrak{a}}
\newcommand{\sO}{\mathscr{O}}
\newcommand{\dd}{\mathrm{d}}
\newcommand{\kk}{\mathsf{k}} % finite field
\newcommand{\LL}{\mathrm{L}} % cotangent complex
\newcommand{\vaf}{\mathsf{Vaf}}
\newcommand{\pow}[1]{\llbracket #1 \rrbracket}
\newtheorem{definition}[subsection]{Definition}
\newtheorem{theorem}[subsection]{Theorem}

\title{Obstruction theory via the cotangent complex}
\author{Daniel Miller}

\begin{document}
\maketitle





Our main example is as follows. Let $k$ be a finite field, $\witt(k)$ its ring 
of Witt vectors. Consider the category $\vaf_{\witt(k)}$ of ``formal varieties 
over $\witt(k)$.'' It is the opposite category of the full category of 
topological $\witt(k)$-algebras that are filtered projective limits of finite 
$\witt(k)$-algebras. We will give $\vaf_{\witt(k)}$ a suitable subcanonical 
Grothendieck topology, and consider sheaves on it. Note that 
$\sheaves(\vaf_{\witt(k)})$ comes with a commutative ring object -- namely the 
forgetful functor $\spf(A)\mapsto A$. We will denote this functor by $\sO$. 
If $\cX$ is a formal scheme, or just a sheaf on $\vaf_{\witt(k)}$, we will 
consider $\cX$ as the topos $\sheaves\left(\vaf_{\witt(k)}\right)_{/\cX}$. 
This has an obvious commutative ring object $\sO_\cX=\sO\times \cX$. So for the 
rest of this note we will work with an arbitrary ringed topos $(\cX,\sO)$, but 
the reader should keep in mind this specific example. 

Our main reference is \cite{illusie-1971}. Also be aware that we will sometimes 
work with sheaves on the category of connected, pointed $\witt(k)$-formal 
varieties -- that is the opposite category of the category of local profinite 
$\witt(k)$-algebras with residue field $k$. We will do this in the context of 
specific deformation problems. 

Brief justification that this generalization works. Let $\cX$ be a topos, 
$\mathfrak{Top}$ the category of all topoi and geometric morphisms. Then the 
``slice functor'' $x\mapsto \cX_{/x}$ from $\cX\to\mathfrak{Top}_{/\cX}$ is 
a fully faithful embedding by \cite[4.38]{johnstone-1977}. So there is no 
loss replacing a formal scheme over $\witt(k)$ with the topos of sheaves over 
this scheme, regarded as a topos over the topos of sheaves over $\witt(k)$. 





\section{Cotangent complex for morphisms of topoi}

If $\cX,\cY$ are topoi, we call a \emph{morphism} $f:\cX\to \cY$ an adjoint 
pair $(f^{-1},f_\ast)$, where is a morphism $f_\ast:\cX\to \cY$ and 
$f^{-1}:\cY\to \cX$ is exact. If $\cX$ and $\cY$ are ringed topoi, then a 
morphism $f:\cX\to \cY$ must come with a morphism of ring objects 
$\sO_\cY\to f_\ast \sO_\cX$, or equivalently $f^{-1} \sO_\cY\to \sO_\cX$. 

\begin{definition}[{\cite[II 1.2.7]{illusie-1971}}]
Let $f:\cX\to \cY$ be a morphism of ringed topoi. The \emph{cotangent complex} 
of $\cX$ over $\cY$ is the simplicial $\sO_\cX$-module given by 
$L_{\cX/\cY} = L_{\sO_\cX/f^{-1} \sO_\cY}$. 
\end{definition}

Here $L_{B/A}$ is defined as in \cite[II 1.2]{illusie-1971}. Our main example 
of interest is when $\cX$ is a some deformation functor for a residual Galois 
representation $\bar\rho$. The representation $\bar\rho$ will correspond to 
$\bar\rho:\spf(k)\to\cX$, and we will be concerned with 
$L_{\bar\rho/\cX} = L_{\spf(k)/_{\bar\rho}\cX}$. This is a simplicial 
$k$-vector space. 





\section{Obstruction theory}

Our goal is as follows. Work over a base topos $\cS$. Suppose 
$x_0:\cX_0\to \cY$ is a morphism and $I$ is an $\sO_{\cX_0}$-module. We are 
interested in extensions of $x_0$ to $x:\cX\to \cY$, where $\cX$ has the same 
underlying topos as $\cX_0$, but for which $\sO_\cX$ is a square-zero extension 
of $\sO_{\cX_0}$ by the ideal $I$. 

\begin{theorem}
Let $\cX_0\xrightarrow{x_0}\cY$ be a morphism over $\cY$, and $I$ be an 
$\sO_{\cX_0}$-module. Then there is a canonical \emph{obstruction class} 
\[
  o(x_0)\in \ext_{\cX_0}^2(L_{\cX_0/\cY},I) 
\]
which is $0$ if and only if an extension of $x_0$ to $\cX\to \cY$ exists. If 
such an extension exists, then the extensions are a 
$\ext_{\cX_0}^1(L_{\cX_0/\cY},I)$-torsor, and each extension has automorphism 
group $\ext_{\cX_0}^0(L_{\cX_0/\cY},I)$. 
\end{theorem}
\begin{proof}
This is \cite[III 2.1.7]{illusie-1971}, where $\cY_0=\cY$ and the base topos is 
hidden from notation.  
\end{proof}





\section{One-dimensional representations}

Let $\Gamma$ be a finitely generated $\bZ_p$-module. Write $\cX_\Gamma$ for the 
deformation space parameterizing lifts of $1\colon \Gamma\to \kk^\times$. So 
$\cX_\Gamma$ is a (formal) scheme over $\witt(\kk)$. One way to understand the 
cotangent complex $\LL_{\cX_\Gamma/\witt(\kk)}$ is by embedding $\cX_\Gamma$ 
into a smooth scheme. 

Let $\Gamma_\bullet\twoheadrightarrow \Gamma$ be a minimal free resolution of 
$\Gamma$ as a $\bZ_p$-module. So 
$\Gamma_\bullet=[\Gamma_1\hookrightarrow \Gamma_0]$. Then we have a closed embedding 
$\cX_\Gamma\hookrightarrow\cX_{\Gamma_0} = \spf(\witt(\kk)\pow{\Gamma_0})$. 
Then \cite[III 3.3.6]{illusie-1971} tells us that 
\[
	\LL_{\cX_\Gamma/\witt(\kk)} = \left[ \fa/\fa^2 \to \Omega^1_{\cX_{\Gamma_0} / \witt(\kk)}\otimes_{\witt(\kk)\pow{\Gamma_0}} \witt(\kk)\pow\Gamma] \right] ,
\]
where $\fa=\ker(\witt(\kk)\pow{\Gamma_0} \twoheadrightarrow \witt(\kk)[\Gamma])$. 





\section{Take two}

Suppose $\Gamma=\bZ_p^{\oplus r} \times \bigoplus_i \bZ/p^{n_i}$. Then 
\[
	R = \Lambda\pow{\Gamma}\simeq \Lambda\pow{s_1,\dots,s_r,t_i} / \langle 1-(1-t_i)^{p^{n_i}}\rangle .
\]
This gives us an obvious surjection 
$\Lambda\pow{\bm s,\bm t}\twoheadrightarrow R$. Let $\fa$ be its kernel. Then 
\[
	\LL_{R/\Lambda} \simeq \left[\fa/\fa^2 \xrightarrow\dd R\, \dd(\bm s,\bm t) \right] .
\]
Now, more or less by definition, $\fa=\langle 1-(1-t_i)^{p^{n_i}}\rangle$. 





\bibliographystyle{alpha}
\bibliography{tidbit-sources}

\end{document}