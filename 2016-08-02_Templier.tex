\documentclass{article}

\usepackage{amsmath,amssymb}
\DeclareMathOperator{\sym}{sym}
\newcommand{\blambda}{{\boldsymbol\lambda}}
\newcommand{\bQ}{\mathbf{Q}}
\newcommand{\btheta}{{\boldsymbol\theta}}

\title{A strange class of $L$-functions}
\author{Daniel Miller}

\begin{document}
\maketitle





Let $E_{/\bQ}$ be an elliptic curve, and write 
$\theta_p = \cos^{-1}\left(\frac{a_p(E)}{2\sqrt p}\right)$. Then the 
$k$-th symmetric power (analytic) $L$-function of $E$ has product formula: 
\[
	L(\sym^k E,s) = \prod_p \prod_{j=0}^k \frac{1}{1-e^{i(n-2j)\theta_p} p^{-s}} .
\]
Rewrite this as 
\[
	L(\sym^k E,s) = \prod_{j=0}^k \prod_p \frac{1}{1-e^{i(n-2j)\theta_p} p^{-s}} .
\]
This writes $L(\sym^k E,s)$ as a product of ``strange $L$-functions.''


Write $\blambda=(\lambda_2,\lambda_3,\lambda_5,\dots)$ for a sequence of 
complex numbers indexed by the primes. Given such a $\blambda$, we have 
\[
	L(\blambda,s) = \prod_p \frac{1}{1-\lambda_p p^{-s}} .
\]
Given this definition, we have 
\[
	L(\sym^k E,s) = \prod_{j=0}^k L(e^{i(n-2j)\btheta},s)
\]
with $e^{i(n-2j)\btheta} = (e^{i(n-2j)\theta_2},e^{i(n-2j)\theta_3},\dots)$. 

It is now known that $L(\sym^k E,s)$ has analytic continuation and functional 
equation. I'm wondering: does this imply anything about analytic continuation 
of the $L(e^{i(n-2j)\btheta},s)$? 





\end{document}
