\documentclass{article}

\usepackage{amsmath,amssymb,extarrows,fullpage,mathpazo,microtype,stmaryrd}
\usepackage[hidelinks]{hyperref}
\DeclareMathOperator{\class}{Cl}
\DeclareMathOperator{\galois}{Gal}
\DeclareMathOperator{\h}{H}
\newcommand{\dF}{\mathbf{F}}
\newcommand{\dQ}{\mathbf{Q}}
\newcommand{\dZ}{\mathbf{Z}}
\newcommand{\fo}{\mathfrak{o}}
\newcommand{\isomorphism}{\xlongrightarrow\sim}
\newcommand{\power}[1]{\llbracket #1 \rrbracket}

\title{$p$-adic $L$-functions and the Iwasawa main conjecture}
\author{Daniel Miller}

\begin{document}
\maketitle





An impossible-to-find, but conceptual source seems to be \cite{fk06}. A more 
accessible source may be \cite{pr00}. 

Things more or less got started in \cite{ri76}, and really got going in 
\cite{mw84}. A more recent result is \cite{su14}. 





\section{The start of the field}

A lot of mathematical activity was kicked off by Ribet's influential paper 
\cite{ri76}. Let $p$ be an odd prime, and let 
$C=\class\left(\dQ(\mu_p)\right)\otimes \dF_p$. Then $G_\dQ$ 
acts on this through its quotient $\galois(\dQ(\mu_p)/\dQ)$, and if 
$\chi:G_\dQ\to \dF_p^\times$ is the corresponding cyclotomic character, we 
have a decomposition $C=\bigoplus_{i\in \dZ/(p-1)} C(\chi^i)$, where 
\[
  C(\chi^i)=\{c\in C:\sigma \cdot c = \chi^i(\sigma) c\text{ for all }\sigma\in G_\dQ\} .
\]

Ribet proved that if for $2\leqslant k\leqslant p-3$ even, 
$C(\chi^k)\ne 0$ if and only if $v_p(B_k)>0$. One direction had already been 
proved by Herbrand, so all Ribet has to do is prove that $v_p(B_k)>0$ implies 
$C(\chi^k)\ne 0$. He does this by constructing the appropriate unramified 
extension of $\dQ(\mu_p)$ using modular forms. 





\section{The Main Conjecture for modular forms}

We follow \cite[III]{ka04}. Let $\dQ_\infty$ be the $\dZ_p$-extension of $\dQ$, 
with Galois group $\Gamma\isomorphism \dZ_p^\times$ via the cyclotomic 
character $\kappa$. We are interested in modules over the Iwasawa algebra  
$\Lambda = \dZ_p\power{\Gamma}$, which is (non-canonically) isomorphic to the 
power-series ring $\dZ_p\power{t}$. If $M$ is a $\Gamma$-module over $\dZ_p$, 
we can interpret it as a representation of $G_\dQ$ unramified outside $p$, 
hence a sheaf on the \'etale site of $\dZ[\frac 1 p]$. Let 
$\fo_n = \dZ[\zeta_{p^{n+1}}]$ and $\fo_{n,\{p\}}=\fo_n[\frac 1 p]$. We put 
\[
  \boldsymbol\h^i(M) = \varprojlim_n \h^i(\fo_{n,\{p\}},M)
\]
It turns out that $\boldsymbol\h^i(M)=0$ unless $i\in \{1,2\}$. If 
$M$ is a $\dQ_p$-representation of $\Gamma$, we choose a 
$\Gamma$-stable lattice $M_0\subset M$, and put 
\[
  \boldsymbol\h^i(M) =\boldsymbol\h^i(M_0)\otimes \dQ .
\]

Let $p$ be an odd prime, 
$k\geqslant 2$ and $N\geqslant 1$. We start with a normalized newform 
$f\in S_k(X_1(N))$. Choose a prime $\ell\nmid N p$, let $F_f=\dQ(f)$, and let 
$V_f$ be the associated two-dimensional $F_{f,\lambda}$-vector space with 
continuous $G_\dQ$-action. Kato defines a canonical map 
$z:V_f \to \boldsymbol\h^1(V_f)$. 





\bibliographystyle{alpha}
\bibliography{tidbit-sources}

\end{document}
