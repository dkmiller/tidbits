\documentclass{article}

\usepackage{amsmath,amssymb,fullpage,mathpazo,microtype}
\usepackage[hidelinks]{hyperref}
\DeclareMathOperator{\class}{Cl}
\DeclareMathOperator{\galois}{Gal}
\newcommand{\dF}{\mathbf{F}}
\newcommand{\dQ}{\mathbf{Q}}
\newcommand{\dZ}{\mathbf{Z}}

\title{$p$-adic $L$-functions and the Iwasawa main conjecture}
\author{Daniel Miller}

\begin{document}
\maketitle





An impossible-to-find, but conceptual source seems to be \cite{fk06}. A more 
accessible source may be \cite{pr00}. 

Things more or less got started in \cite{ri76}, and really got going in 
\cite{mw84}. A more recent result is \cite{su14}. 





\section{The start of the field}

A lot of mathematical activity was kicked off by Ribet's influential paper 
\cite{ri76}. Let $p$ be an odd prime, and let 
$C=\class\left(\dQ(\mu_p)\right)\otimes \dF_p$. Then $G_\dQ$ 
acts on this through its quotient $\galois(\dQ(\mu_p)/\dQ)$, and if 
$\chi:G_\dQ\to \dF_p^\times$ is the corresponding cyclotomic character, we 
have a decomposition $C=\bigoplus_{i\in \dZ/(p-1)} C(\chi^i)$, where 
\[
  C(\chi^i)=\{c\in C:\sigma \cdot c = \chi^i(\sigma) c\text{ for all }\sigma\in G_\dQ\} .
\]

Ribet proved that if for $2\leqslant k\leqslant p-3$ even, 
$C(\chi^k)\ne 0$ if and only if $v_p(B_k)>0$. One direction had already been 
proved by Herbrand, so all Ribet has to do is prove that $v_p(B_k)>0$ implies 
$C(\chi^k)\ne 0$. He does this by constructing the appropriate unramified 
extension of $\dQ(\mu_p)$ using modular forms. 





\bibliographystyle{alpha}
\bibliography{tidbit-sources}

\end{document}
