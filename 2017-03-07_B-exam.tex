\documentclass[handout]{beamer}
% Remove [handout] for pause to work.
\usetheme{metropolis}

\DeclareMathOperator{\cdf}{cdf}
\DeclareMathOperator{\D}{D}
\DeclareMathOperator{\N}{N}
\DeclareMathOperator{\R}{R}
\DeclareMathOperator{\ST}{ST}
\DeclareMathOperator{\SU}{SU}
\DeclareMathOperator{\X}{X}
\newcommand{\bC}{\mathbf{C}}
\newcommand{\bF}{\mathbf{F}}
\newcommand{\bQ}{\mathbf{Q}}
\newcommand{\bZ}{\mathbf{Z}}
\newcommand{\dd}{\mathrm{d}}
\newcommand{\Gm}{\mathbf{G}_\mathrm{m}}

\title{Counterexamples related to the Sato--Tate conjecture}
\author{Daniel Miller}
\institute{Cornell University}
\date{14 April 2017}





\begin{document}

\begin{frame}
\titlepage
\end{frame}

\begin{frame}
\frametitle{Outline}
\tableofcontents
\end{frame}





\section{Motivation}

\begin{frame}{Sato--Tate Conjecture}
$E_{/\bQ}$ non-CM elliptic curve, $\theta_p = \cos^{-1}\left( \frac{a_p}{2\sqrt p}\right)$. 
\pause

Sato--Tate measure $\ST = \frac{2}{\pi} \sin^2 \theta\, \dd \theta$ (Haar 
measure on $\SU(2)$). 
\pause

\begin{theorem}[Taylor et.~al.]
The $\theta_p$ are equidistributed with respect to $\ST$. 
\end{theorem}
\pause

Serre generalized conjecture to arbitrary motives. 
\pause

Stick to elliptic curves and CM abelian varieties.
\pause

Quantify rate of convergence of 
$\frac{1}{\pi(N)} \sum_{p\leqslant N} \delta_{\theta_p}$ to $\ST$. 
\pause

Use discrepancy (Kolmogorov--Smirnov statistic). 
\end{frame}


\begin{frame}{Akiyama--Tanigawa Conjecture}
Let 
\[
	D_N = \sup_{x\in [0,\pi]}\left| \frac{1}{\pi(N)} \sum_{p\leqslant N} 1_{[0,x)}(\theta_p) - \int_0^x \, \dd\ST\right| .
\]
\pause

Conjecture (A--T): $D_N \ll N^{-\frac 1 2 + \epsilon}$. 
\pause

Akiyama--Tanigawa Conjecture makes sense for CM elliptic curve.

\begin{theorem}[A--T]
A--T implies RH($E$).
\end{theorem}
\end{frame}





\begin{frame}
\begin{center}
\Huge Questions?
\end{center}
\end{frame}





\end{document}
