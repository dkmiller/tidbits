\documentclass[handout]{beamer}
% Remove [handout] for pause to work.
\usetheme{metropolis}

\DeclareMathOperator{\cdf}{cdf}
\DeclareMathOperator{\D}{D}
\DeclareMathOperator{\N}{N}
\DeclareMathOperator{\R}{R}
\DeclareMathOperator{\ST}{ST}
\DeclareMathOperator{\SU}{SU}
\DeclareMathOperator{\sym}{sym}
\DeclareMathOperator{\X}{X}
\newcommand{\bC}{\mathbf{C}}
\newcommand{\bF}{\mathbf{F}}
\newcommand{\bQ}{\mathbf{Q}}
\newcommand{\bZ}{\mathbf{Z}}
\newcommand{\dd}{\mathrm{d}}
\newcommand{\Gm}{\mathbf{G}_\mathrm{m}}

\newtheorem{conjecture}{Conjecture}

\title{Counterexamples related to the Sato--Tate conjecture}
\author{Daniel Miller}
\institute{Cornell University}
\date{14 April 2017}





% Outline:
% Motivation: 
%  - What is the problem I'm trying to solve?
%    2 problems: does RH => AT? is there a Galois proof of ST?
%  - Why is this problem interesting or important?
%    * AT=>RH: so far proofs of ST have proved things about L-functions. 
%      How far can knowledge of L-functions take you?
%    * Galois ST: does ST conjecture follow from Galois theory alone? 
%  - Why is my solution interesting?
%    * Two "No!" answers. 
%    * Proving AT will take more than just L-functions.
%    * ST fundamentally involves automorphy!
% Work
%  - What theorems have I proved? 
%    * "Master Galois theorem"
%    * "dioph approx counterexample"
%  - What systems have I built? 
%    * see above. 
% Related work
%  - Who else has attacked this problem or similar problems? 
%    * Pande, KLR, Thorner, Mazur
%  - In what ways is my solution better? 
%    * Better than Pande: stronger results
%    * convenient "packaging" of KLR
%    * shows that Thorner doesn't generalize to CM
%  - In what ways is it limited? 
%    * only addresses CM varieties - may not work for SU(2)
% So what?
%  - need new directions to prove AT, even for CM case. 





\begin{document}

\begin{frame}
\titlepage
\end{frame}

\begin{frame}
\frametitle{Outline}
\tableofcontents
\end{frame}





\section{Motivation}

\begin{frame}{Sato--Tate Conjecture}
$E_{/\bQ}$ non-CM elliptic curve, $\theta_p = \cos^{-1}\left( \frac{a_p}{2\sqrt p}\right)$. 
\pause

Sato--Tate measure: $\ST = \frac{2}{\pi} \sin^2 \theta\, \dd \theta$ (Haar 
measure on $\SU(2)$). 
\pause

\begin{theorem}[Taylor et.~al.]
The $\theta_p$ are equidistributed with respect to $\ST$. 
\end{theorem}
\pause

Generalized by Serre: conjecture for arbitrary motives. 
\pause

Stick to elliptic curves and CM abelian varieties.
\pause

Quantify rate of convergence of 
$\frac{1}{\pi(N)} \sum_{p\leqslant N} \delta_{\theta_p}$ to $\ST$. 
\pause

Use discrepancy (Kolmogorov--Smirnov statistic). 
\end{frame}


\begin{frame}{Akiyama--Tanigawa Conjecture}
\[
	D_N = \sup_{x\in [0,\pi]}\left| \frac{1}{\pi(N)} \sum_{p\leqslant N} 1_{[0,x)}(\theta_p) - \int_0^x \, \dd\ST\right| .
\]
\pause

\begin{conjecture}[Akiyama--Tanigawa]
$D_N \ll N^{-\frac 1 2 + \epsilon}$. 
\end{conjecture}
\pause

There is a variant of this conjecture for CM elliptic curve.
\pause

\begin{theorem}[Akiyama--Tanigawa]
Akiyama--Tanigawa conjecture $\Rightarrow$ Riemann Hypothesis for $E$.
\end{theorem}
\pause

\begin{theorem}[Mazur]
Akiyama--Tanigawa conjecture $\Rightarrow$ Riemann Hypothesis for $\sym^k E$
\end{theorem}
\end{frame}


\begin{frame}{Questions}
1. Does Riemann Hypothesis for all $\sym^k E$ imply Akiyama--Tanigawa conjecture? 
\pause

2. Does functoriality imply the Akiyama--Tanigawa conjecture? \pause 
For CM elliptic curves?
\pause

3. Is the Sato--Tate conjecture a Galois-theoretic result?
\pause

4. Does every Galois representation have equidistribution coming from a 
Sato--Tate group?
\pause

1. Yes! (sort of)
\pause

2--4. No!
\end{frame}





\begin{frame}
\begin{center}
\Huge Questions?
\end{center}
\end{frame}





\end{document}
