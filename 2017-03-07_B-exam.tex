\documentclass{beamer}
\usetheme{metropolis}

\DeclareMathOperator{\cdf}{cdf}
\DeclareMathOperator{\D}{D}
\DeclareMathOperator{\N}{N}
\DeclareMathOperator{\R}{R}
\DeclareMathOperator{\ST}{ST}
\DeclareMathOperator{\X}{X}
\newcommand{\bC}{\mathbf{C}}
\newcommand{\bF}{\mathbf{F}}
\newcommand{\bZ}{\mathbf{Z}}
\newcommand{\dd}{\mathrm{d}}
\newcommand{\Gm}{\mathbf{G}_\mathrm{m}}

\title{Title here}
\author{Daniel Miller}
\institute{Cornell University}
\date{28 April 2017}





\begin{document}

\begin{frame}
\titlepage
\end{frame}

\begin{frame}
\frametitle{Outline}
\tableofcontents
\end{frame}





\section{Motivation}

\begin{frame}{Sato--Tate Conjecture}
Equation of the form $E:y^2=x^3+ax+b$.
\pause

Simplify: assume $a,b\in \bZ$.
\pause

Non-singular: $4a^3+27b^2\ne 0$. 
\pause

Count points modulo $p$: 
\[
	\# E(\bF_p) = \#\{(x,y)\in (\bF_p)^2 : x^2=y^3+ax+b\} + 1 .
\]
\pause

$+1=$ ``point at infinity.''
\pause

Geometric structure of $E(\bC)$
\end{frame}





\end{document}
