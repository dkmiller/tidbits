\documentclass{article}

\usepackage{amsmath,amssymb,amsthm}
% Operators
	\DeclareMathOperator{\Ad}{Ad}
	\DeclareMathOperator{\D}{D}
	\DeclareMathOperator{\Gal}{Gal}
	\DeclareMathOperator{\GL}{GL}
	\DeclareMathOperator{\h}{H}
	\DeclareMathOperator{\SU}{SU}
	\DeclareMathOperator{\sym}{sym}
	\DeclareMathOperator{\tr}{tr}
% Symbols
	\newcommand{\bF}{\mathbf{F}}
	\newcommand{\bQ}{\mathbf{Q}}
	\newcommand{\bR}{\mathbf{R}}
	\newcommand{\bx}{{\boldsymbol x}}
	\newcommand{\by}{{\boldsymbol y}}
	\newcommand{\bZ}{\mathbf{Z}}
	\newcommand{\dd}{\mathrm{d}}
	\newcommand{\frob}{\mathrm{fr}}
	\newcommand{\nr}{\mathrm{nr}}
	\newcommand{\ram}{\mathrm{ram}}
	\newcommand{\ST}{\mathrm{ST}}
% Sha (cyrillic character)
	\DeclareFontFamily{U}{wncy}{}
	\DeclareFontShape{U}{wncy}{m}{n}{<->wncyr10}{}
	\DeclareSymbolFont{mcy}{U}{wncy}{m}{n}
	\DeclareMathSymbol{\Sha}{\mathord}{mcy}{"58} 
% Environments
	\newtheorem{theorem}{Theorem}

\title{Galois representations with specified Sato--Tate distributions}
\author{Daniel Miller}

\begin{document}
\maketitle





\section{Introduction}

% TODO: Introduction and motivation. 
(Summary of \cite{pande})





\section{Notation and preliminary results}

Now we loosely summarize the results of \cite{klr}, adapting them as needed for 
our context. For a field $F$, write $G_F = \Gal(\overline F / F)$ for the 
absolute Galois group of $F$. If $M$ is a $G_F$-module,  write 
$\h^\bullet(F,M)$ in place of $\h^\bullet(G_F,M)$. All Galois representations 
will be to $\GL_2(\bZ/l^n)$ or $\GL_2(\bZ_l)$ for $l$ a (fixed) rational prime, 
and all deformations will have fixed determinant, so we only consider the 
cohomology of $\Ad^0\bar\rho$, the induced representation on trace-zero 
matrices by conjugation. 

If $S$ is a set of rational primes, $\bQ_S$ denotes the largest extension of 
$\bQ$ unramified outside $S$. So $\h^i(\bQ_S,-)$ is what is usually written as 
$\h^1(G_{\bQ,S},-)$. If $M$ is a $G_\bQ$-module and $S$ a finite set of primes, 
write 
\[
	\Sha^i_S(M) = \ker\left( \h^i(\bQ_S,M) \to \prod_{p\in S} \h^i(\bQ_p,M)\right) .
\]
If $l$ is a rational prime and $S$ a finite set of primes containing $l$, then 
for any $\bF_l[G_{\bQ_S}]$-module $M$, write $M^\vee=\hom_{\bF_l}(M,\bF_l)$ 
with the obvious $G_{\bQ_S}$-action, and write $M^\ast = M^\vee(1)$ for the 
Cartier dual. By \cite[Th.~8.6.7]{nsw}, there is an 
isomorphism $\Sha^1_S(M^\ast) = \Sha_S^2(M)^\vee$. 


A \emph{good (residual) representation} is an odd, absolutely irreducible, 
weight-$2$ representation $\bar\rho\colon G_\bQ \to \GL_2(\bF_l)$, where 
$l\geqslant 7$ is a rational prime. Roughly, good residual representations are 
well-behaved enough that we can prove a lot about them directly, without 
assume the modularity results of Khare--Wingenberger. 

\begin{theorem}[{\cite[Th.~1]{ravi-FM}}]\label{thm:always-can-lift}
Let $\bar\rho\colon G_{\bQ_S} \to \GL_2(\bF_l)$ be a good residual 
representation. Then there exists a weight-$2$ lift of $\bar\rho$ to $\bZ_l$. 
\end{theorem}


Let $\bar\rho\colon G_\bQ\to \GL_2(\bF_l)$ be a good representation. An 
unramified prime $p\not\equiv \pm 1\pmod l$ is \emph{nice} if 
$\Ad^0\bar\rho\simeq \bF_l \oplus \bF_l(1)\oplus \bF_l(-1)$, i.e.~if the 
eigenvalues of $\bar\rho(\frob_p)$ have ratio $p$. If $p$ is nice, then all 
unramified torsion lifts of $\left.\bar\rho\right|_{G_{\bQ_p}}$ have lifts to 
characteristic zero. 

Now we introduce some new terminology and notation to condense the lifting 
profess used in \cite{klr}. 

Fix a good residual representation $\bar\rho$. We will consider weight-$2$ 
deformations of $\bar\rho$ to $\bZ/l^n$ and $\bZ_l$. Call such a deformation a 
``lift of $\bar\rho$ to $\bZ/l^n$ (resp.~$\bZ_l$).'' We will often restrict the 
local behavior of such lifts, i.e.~the restrictions of a lift to $G_{\bQ_p}$ 
for $p$ in some set of primes. The necessary constraints are captured in the 
following definition. 


Let $\bar\rho$ be a good representation, $h\colon \bR^+ \to \bR^+$. An 
\emph{$h$-bounded lifting datum} is a tuple 
$(\rho_n,R,U,\{\rho_p\}_{p\in R\cup U})$, where 
\begin{enumerate}
\item
$\rho_n\colon G_{\bQ_R} \to \GL_2(\bZ/l^n)$ is a lift of $\bar\rho$.

\item
$R$ and $U$ are finite sets of primes, $R$ containing $l$ and all primes at 
which $\rho_n$ ramifies. 

\item
$\pi_R(x)\leqslant h(x)\pi(x)$ for all $x$. 

\item
$\Sha_R^1(\Ad^0\bar\rho) = \Sha_R^2(\Ad^0\bar\rho) = 0$. 

\item
For all $p\in R\cup U$, 
$\rho_p\equiv \left. \rho_n\right|_{G_{\bQ_p}}\pmod{l^n}$. 

\item
For all $p\in R$, $\rho_p$ is ramified. 

\item
$\rho_n$ admits a lift to $\bZ/l^{n+1}$. 
\end{enumerate}

If $(\rho_n,R,U,\{\rho_p\})$ is an $h$-bounded lifting datum, we call 
another $h$-bounded lifting datum $(\rho_{n+1},R',U',\{\rho_p\})$ a \emph{lift 
of $(\rho_n,R,U,\{\rho_p\})$} if $U\subset U'$, $R\subset R'$, and for all 
$p\in R\cup U$, the two possible ``$\rho_p$'' agree. 

\begin{theorem}\label{thm:lifting-datum}
Let $\bar\rho$ be a good residual representation, $h\colon \bR^+ \to \bR^+$ 
decreasing to zero. If $(\rho_n,R,U,\{\rho_p\})$ is an $h$-bounded lifting 
datum, $U'\supset U$ is a finite set of primes disjoint from $R$, and 
$\{\rho_p\}_{p\in U'}$ extends $\{\rho_p\}_{p\in U}$, then there exists an 
$h$-bounded lift $(\rho_{n+1},R',U',\{\rho_p\})$ of 
$(\rho_n,R,U,\{\rho_p\})$. 
\end{theorem}
\begin{proof}


By \cite[Lem.~8]{klr}, there exists a finite set $N$ of nice primes, such that 
the map 
\begin{equation}\label{eq:h1-isom}
	\h^1(\bQ_{R\cup N},\Ad^0\bar\rho) \to \prod_{p\in R} \h^1(\bQ_p,\Ad^0\bar\rho) \times \prod_{p\in U'} \h_\nr^1(\bQ_p,\Ad^0\bar\rho) 
\end{equation}
is an isomorphism. In fact, $\# N = \dim\h^1(\bQ_{R\cup N},\Ad^0\bar\rho^\ast)$, 
and the primes in $N$ are chosen, one at a time, from Chebotarev sets. This means 
we can force them to be large enough to ensure that the bound 
$\pi_{R\cup N}(x) \leqslant h(x) \pi(x)$ continues to hold. 

By our hypothesis, $\rho_n$ admits a lift to $\bZ/l^{n+1}$; call one such lift 
$\rho^\ast$. For each $p\in R\cup U'$, $\h^1(\bQ_p,\Ad^0\bar\rho)$ acts simply 
transitively on lifts of $\left.\rho_n\right|_{G_{\bQ_p}}$ to $\bZ/l^{n+1}$. In 
particular, there are cohomology classes $f_p\in \h^1(\bQ_p,\Ad^0\bar\rho)$ 
such that $f_p\cdot \rho^\ast \equiv \rho_p\pmod{l^{n+1}}$ for all 
$p\in R\cup U'$. Moreover, for all $p\in U'$, the class $f_p$ is unramified. 
Since the map \eqref{eq:h1-isom} is an isomorphism, there exists 
$f\in \h^1(\bQ_{R\cup N},\Ad^0\bar\rho)$ such that 
$\left.f\cdot \rho^\ast\right|_{G_{\bQ_p}}\equiv \rho_p\pmod{l^{n+1}}$ for all 
$p\in R\cup U'$. 

Clearly $\left. f\cdot \rho^\ast\right|_{G_{\bQ_p}}$ admits a lift to $\bZ_l$ 
for all $p\in R\cup U'$, but it does not necessarily admit such a lift for 
$p\in N$. By repeated applications of \cite[Prop.~3.10]{pande}, there 
exists a set $N'\supset N$, with $\# N'\leqslant 2\# N$, of nice primes and 
$g\in \h^1(\bQ_{R\cup N'},\Ad^0\bar\rho)$ such that 
$(g+f)\cdot \rho^\ast$ still agrees with $\rho_p$ for $p\in R\cup U'$, and 
$(g+f)\cdot \rho^\ast$ is nice for all $p\in N'$. As above, the primes in $N'$ 
are chosen one at a time from Chebotarev sets, so we can continue to ensure the 
bound $\pi_{R\cup N'}(x)\leqslant h(x) \pi(x)$. Let 
$\rho_{n+1} = (g+f) \cdot \rho^\ast$. Let $R' = R\cup N'$. For each 
$p\in R'\smallsetminus R$, choose a ramified lift $\rho_p$ of 
$\left. \rho_{n+1}\right|_{G_{\bQ_p}}$ to $\bZ_l$. 

Since $\left.\rho_{n+1}\right|_{G_{\bQ_p}}$ admits a lift to $\bZ/l^{n+2}$ (in 
fact, it admits a lift to $\bZ_l$) for each $p$, and 
$\Sha_{R'}^2(\Ad^0\bar\rho) = 0$, the deformation $\rho_{n+1}$ admits a lift to 
$\bZ/l^{n+2}$. Thus $(\rho_{n+1},R',U',\{\rho_p\})$ is the desired lift of 
$(\rho_n,R,U,\{\rho_p\})$. 
\end{proof}





\section{Master theorem}

Fix a good residual representation $\bar\rho$. We 
consider weight-$2$ deformations of $\bar\rho$. The final deformation, 
$\rho\colon G_\bQ \to \GL_2(\bZ_l)$, will be constructed as the inverse limit 
of a compatible collection of lifts $\rho_n\colon G_\bQ \to \GL_2(\bZ/l^n)$. At 
any given stage, we will be concerned with making sure that there exists a 
lift to the next stage, that such a lift can be forced to have the necessary 
properties. Fix a sequence $(x_1,x_2,\dots)$ in $[-1,1]$. The set of unramified 
primes of $\rho$ is not determined at the beginning, but at each stage there 
will be a large finite set $U$ of primes which we know will remain unramified. 
Re-indexing $(x_n)$ by these unramified primes, we will construct $\rho$ so that 
for all unramified primes $p$, $\tr\rho(\frob_p)\in \bZ$, satisfies the Hasse 
bound, and has $\tr\rho(\frob_p) \approx x_p$. Moreover, we can ensure that the 
set of ramified primes has density zero in a very strong sense (controlled by a 
parameter function $h$) and that our trace of Frobenii are very close to 
specified values, the ``closeness'' again controlled by a parameter function.
Write $\pi_{\ram(\rho)}$ for the function which counts $\rho_n$-ramified 
primes. 

\begin{theorem}\label{thm:master-Galois}
Let $l$, $\bar\rho$, $(x_n)$ be as above. Fix functions 
$h\colon \bR^+\to \bR^+$ (resp.~$b\colon \bR^+ \to \bR_{\geqslant 1}$) which 
decrease to zero (resp.~increase to infinity). Then there exists a weight-$2$ 
deformation $\rho$ of $\bar\rho$, such that 
\begin{enumerate}
\item
$\pi_{\ram(\rho)}(x) \ll h(x) \pi(x)$. 

\item
For each unramified prime $p$, $a_p=\tr\rho(\frob_p)\in \bZ$ and satisfies the 
Hasse bound. 

\item
For each unramified prime $p$, 
$\left| \frac{a_p}{2\sqrt p} - x_p\right| \leqslant \frac{l b(p)}{2\sqrt p}$. 
\end{enumerate}
\end{theorem}
\begin{proof}
Begin with $\rho_1= \bar\rho$. By \cite[Lem.~6]{klr}, 
there exists a finite set $R$, containing the set of primes at which $\bar\rho$ 
ramifies, such that $\Sha_R^1(\Ad^0\bar\rho) = \Sha_R^2(\Ad^0\bar\rho) = 0$. 
Let $R_2$ be the union of $R$ and all primes $p$ with 
$\frac{l}{2\sqrt p} > 2$. For all $p\notin R_2$ and any $a\in \bF_l$, there 
exists $a_p\in \bZ$ satisfying the Hasse bound with $a_p\equiv a\pmod l$. In 
fact, given any $x_p\in [-1,1]$, there exists $a_p\in \bZ$ satisfying the Hasse 
bound such that 
$\left| \frac{a_p}{2\sqrt p} - x_p\right| \leqslant \frac{l}{2\sqrt p}$.
Choose, for all primes $p\in R_2$, a ramified 
lift $\rho_p$ of $\left. \rho_1\right|_{G_{\bQ_p}}$. Let $U_2$ be the set of 
primes not in $R_2$ such that 
$\frac{l^2}{2\sqrt p} > \min\left(2, \frac{l b(p)}{2\sqrt p}\right)$. 
For each $p\in U_2$, there exists $a_p\in \bZ$, satisfying the 
Hasse bound, such that 
\[
	\left| \frac{a_p}{2\sqrt p} - x_p\right| \leqslant \frac{l}{2\sqrt p} \leqslant \frac{l b(p)}{2\sqrt p} ,
\]
and moreover $a_p\equiv \tr\bar\rho(\frob_p)\pmod l$. For each $p\in U_2$, let 
$\rho_p$ be an unramified lift of $\left.\bar\rho\right|_{G_{\bQ_p}}$ with 
$a_p\equiv\tr\rho_p(\frob_p)\pmod l$. It may not be that 
$\pi_{R_2}(x) \leqslant h(x) \pi(x)$ for all $x$, but there is a scalar 
multiple $h^\ast$ of $h$ so that $\pi_{R_2}(x) \leqslant h^\ast(x) \pi(x)$ for 
all $x$. 

We have constructed our first $h^\ast$-bounded lifting datum 
$(\rho_1,R_2,U_2,\{\rho_p\})$. We proceed to construct 
$\rho = \varprojlim \rho_n$ inductively, by constructing a new $h^\ast$-bounded 
lifting datum for each $n$. We ensure that $U_n$ contains all primes for which 
$\frac{l^n}{2\sqrt p} > \min\left(2, \frac{l b(p)}{2\sqrt p}\right)$, so there 
are always integral $a_p$ satisfying the Hasse bound which satisfy any 
mod-$l^n$ constraint, and that can always choose these $a_p$ so as to preserve 
statement 2 in the theorem. 

The base case is already complete, so suppose we are given 
$(\rho_n,R_n,U_n,\{\rho_p\})$. We may assume that $U_n$ contains all primes for 
which $\frac{l^n}{2\sqrt p} > \min\left(2, \frac{l b(p)}{2\sqrt p}\right)$. Let 
$U_{n+1}$ be the set of all primes not in $R_n$ such that 
$\frac{l^{n+1}}{2\sqrt p} > \min\left(2, \frac{l b(p)}{2\sqrt p}\right)$. For 
each $p\in U_{n+1}\smallsetminus U_n$, there is an integer $a_p$, satisfying 
the Hasse bound, such that $a_p\equiv \rho_n(\frob_p)\pmod{l^n}$, and moreover 
$\left|\frac{a_p}{2\sqrt p} - x_p\right| \leqslant \frac{l b(p)}{2\sqrt p}$. 
For such $p$, let $\rho_p$ be an unramified lift of 
$\left. \rho_n\right|_{G_{\bQ_p}}$ such that 
$a_p\equiv\tr\rho_n(\frob_p)\pmod{l^n}$. By Theorem \ref{thm:lifting-datum}, 
there exists an $h^\ast$-bounded lifting datum 
$(\rho_{n+1},R_{n+1},U_{n+1},\{\rho_p\})$ extending and lifting 
$(\rho_n,R_n,U_n,\{\rho_p\})$. This completes the inductive step.  
\end{proof}








\section{Main construction}

For $k\geqslant 1$, let 
$U_k(\theta) = \frac{\sin((k+1)\theta)}{\sin\theta}$, the trace of the $k$-th 
symmetric power under the identification of $[0,\pi]$ with conjugacy classes 
in $\SU(2)$. Recall that $U_k(\cos^{-1} t)$ is a polynomial in $t$. 

Let $\mu = f(t)\, \dd t$ be a probability measure on $[0,\pi]$. We assume 
$f$ is bounded, that $f(t) \ll \sin(t)$, and that moreover 
$f(\pi/2-\theta) = f(\theta)$. Call such $\mu$ \emph{nice}. 

The key facts about Sato--Tate compatible measures are that $\cos_\ast\mu$ 
satisfies the hypotheses of Theorem \ref{thm:discrepancy-arbitrary}, so 
there are ``$N^{-\alpha}$-decaying van der Corput sequences'' for 
$\cos_\ast\mu$, and also that since $\cos\colon [0,\pi] \to [-1,1]$ is an 
order anti-isomorphism, we know that for any sequence $(x_n)$ on $[-1,1]$, there 
is equality $\D(\{x_n\}^N,\cos_\ast\mu) = \D(\cos^{-1}(x_n)^N,\mu)$. 


\begin{theorem}\label{thm:integral-a_p-alpha}
Let $\mu$ be a Sato--Tate compatible measure, and fix $\alpha\in (0,1/2)$. 
Then there exists a sequence of integers $a_p$ satisfying the Hasse bound, 
such that if we set $\theta_p = \cos^{-1}\left(\frac{a_p}{2\sqrt p}\right)$, 
then $\D^\star(\{\theta\}^N,\mu) = \Theta(\pi(N)^{-\alpha})$. 
\end{theorem}
\begin{proof}
Apply Theorem \ref{thm:discrepancy-arbitrary} to find a sequence $(x_n)$ such 
that $\D(\{x_n\}^N,\cos_\ast \mu) = \Theta(\pi(N)^{-\alpha})$. For each prime 
$p$, there exists an integer $a_p$ such that $|a_p|\leqslant 2\sqrt p$ and 
$\left| \frac{a_p}{2\sqrt p} - x_p\right| \leqslant p^{-1/2}$. Let 
$y_p = \frac{a_p}{2\sqrt p}$. Now apply 
Lemma \ref{lem:disc-of-two-seq} with $\epsilon = N^{-1/2}$. We obtain 
\[
	\left| \D(\{x\}^N,\cos_\ast \mu) - \D(\{y\}^N, \cos_\ast \mu)\right| \ll  N^{-1/2} + \frac{\pi(N^{1/2})}{\pi(N)} ,
\]
which tells us that $\D(\{y\}^N,\cos_\ast\mu) = \Theta(\pi(N)^{-\alpha})$. 
Now let $\{\theta\} = \cos^{-1}(\{y\})$. Apply Lemma \ref{lem:pushforward-reverse} to 
$\{\theta\} = \cos^{-1}(\{y\})$, and we see that 
$\D(\{\theta\}^N,\mu) = \Theta(\pi(N)^{-\alpha})$. 
\end{proof}

We can improve this example by controlling the behavior of sums of the form 
$\sum_{p\leqslant N} U_k(\theta_p)$ for odd $k$. Let $\sigma$ be 
the involution of $[0,\pi]$ given by $\sigma(\theta) = \pi-\theta$. Note that 
$\sigma_\ast \ST = \ST$. Moreover, note that for any odd $k$, 
$U_k\circ\sigma = - U_k$, so $\int U_k\, \dd\ST = 0$. (Of course, 
$\int U_k = 0$ for the reason that $U_k$ is the trace of a non-trivial unitary 
representation, but we will directly exploit the ``oddness'' of $U_k$ in what 
follows.)

\begin{theorem}\label{thm:int-flip-seq}
Let $\mu$ be a $\sigma$-invariant Sato--Tate compatible measure. Fix 
$\alpha\in (0,1/2)$. Then there is a sequence of integers $a_p$, satisfying 
the Hasse bound, such that for 
$\theta_p =\cos^{-1}\left( \frac{a_p}{2\sqrt p}\right)$, we have
\begin{enumerate}
\item
$\D(\{\theta\}^N,\mu) = \Theta(\pi(N)^{-\alpha})$. 

\item
For all odd $k$, 
$\left| \sum_{k\leqslant N} U_k(\theta_p)\right| \ll \pi(N)^{1/2}$. 
\end{enumerate}
\end{theorem}
\begin{proof}
The basic ideas is as follows. Enumerate the primes 
\[
	p_1 = 2, q_1 = 3, p_2 = 5, q_2 = 7, p_3 = 11, q_3 = 13, \dots .
\]
Consider the measure $\left.\mu\right|_{[0,\pi/2)}$. An argument 
nearly identical to the proof of Theorem \ref{thm:integral-a_p-alpha} shows 
that we can choose $a_{p_i}$ satisfying the Hasse bound so that 
\[
	\D\left( \left\{\theta_{p_i}\right\}_{i\leqslant N},\left.\mu\right|_{[0,\pi/2)}\right) = \Theta(N^{-\alpha}) .
\]
We can also choose the $a_{q_i}\in [\pi/2,\pi]$ so that 
\[
	\left| \frac{a_{p_i}}{2\sqrt{p_i}} + \frac{a_{q_i}}{2\sqrt{q_i}}\right| \ll \frac{1}{\sqrt{p_i}} .
\]
If $\{x\}$ is the sequence of the $\frac{a_{p_i}}{2\sqrt{p_i}}$ and $\{y\}$ is 
the similar sequence with the $q_i$-s, then Lemma \ref{lem:flip-discrepancy}, 
Lemma \ref{lem:disc-of-two-seq}, and Theorem \ref{thm:wreath-seq} tell us 
that $\D((\{x\}\wr\{y\})^N,\mu) = \Theta(N^{-\alpha})$. 

Moreover, $U_k(\cos^{-1} t)$ is an odd polynomial in $t$, so if 
$|x_i - (-y_i)| \ll p_i^{-1/2}$, then 
$|U_k(\theta_{p_i}) + U_k(\theta_{q_i})| \ll p_i^{-1/2}$. We can then bound 
\[
	\left| \sum_{i\leqslant N} U_k(\theta_{p_i}) + U_k(\theta_{q_i})\right| \ll \sum_{p\leqslant N} p^{-1/2} \ll \pi(N)^{1/2} .
\]
\end{proof}




Now we combine the results of the last section and Chapter 
\ref{ch:construct-Galois} to obtain a ``beefed-up'' version of Theorem 
\ref{thm:int-flip-seq}. 

\begin{theorem}\label{thm:bad-Galois}
Let $\mu$ be a Sato--Tate compatible $\sigma$-invariant measure on $[0,\pi]$. 
Fix $\alpha\in (0,1/2)$ and a good residual representation 
$\rho\colon G_\bQ \to \GL_2(\bF_l)$. Then there exists a weight-$2$ lift 
$\rho\colon G_\bQ \to \GL_2(\bZ_l)$ of $\bar\rho$ such that 
\begin{enumerate}
\item
$\pi_{\ram(\rho)}(x) \ll e^{-x}\pi(x)$. 

\item
For each unramified prime $p$, $a_p = \tr\rho(\frob_p)\in \bZ$ and satisfies 
the Hasse bound. 

\item
If, for unramified $p$ we set 
$\theta_p = \cos^{-1}\left(\frac{a_p}{2\sqrt p}\right)$, then 
$\D(\{\theta\}^N,\mu) = \Theta(\pi(N)^{-\alpha})$. 

\item
For each odd $k$, the function $L(\sym^k \rho,s)$ satisfies the Riemann 
Hypothesis. 
\end{enumerate}
\end{theorem}
\begin{proof}
Let $\{x\}$ be an $N^{-\alpha}$-decay van der Corput sequence for 
$\cos_\ast \left.\mu\right|_{[0,\pi/2)}$. Let $\by = -\bx$. Then 
$\D((\{x\}\wr\{y\})^N,\cos_\ast\mu) = \Theta(N^{-\alpha})$. Set $h(x) = e^{-x}$ 
and $b(x) = \log(x)$. By Theorem \ref{thm:master-Galois}, there is a 
$\rho\colon G_\bQ \to \GL_2(\bZ_l)$ lifting $\bar\rho$ such that parts 
1 and 2 of the theorem hold. The discrepancy estimate comes from Lemma 
\ref{lem:flip-discrepancy}, Lemma \ref{lem:disc-of-two-seq}, and Theorem 
\ref{thm:wreath-seq} as above, while the Riemann Hypothesis for odd symmetric 
powers follows from the proof of Theorem \ref{thm:int-flip-seq}. 
\end{proof}







\begin{thebibliography}{9}
\bibitem{klr}
C.~Khare, M.~Larsen and R.~Ramakrishna. Constructing semisimple $p$-adic Galois 
representations with prescribed properties, in \emph{Amer.~J.~Math.} 
\textbf{127}(4) (2005), 709--734. 

\bibitem{nsw}
J.~Neukirch, A.~Schmidt and K.~Wingberg, \emph{Cohomology of number fields (2nd edition)}, (Springer--Verlag, 2008). 

\bibitem{pande}
A.~Pande. Deformations of Galois representations and the theorems of Sato--Tate 
and Lang--Trotter, in \emph{Int.~J.~Number Theory} \textbf{7}(8) (2011), 
2065--2079. 

\bibitem{ravi-FM}
R.~Ramakrishna. Deforming Galois representations and the conjectures of Serre 
and Fontaine--Mazur, in \emph{Ann.~of Math (2)} \textbf{156}(1) (2002), 
115--154. 
\end{thebibliography}
\end{document}
