\documentclass{article}

\usepackage{amsmath,amssymb,microtype}
\usepackage[a5paper]{geometry}
\DeclareMathOperator{\GL}{GL}
\DeclareMathOperator{\sign}{sgn}
\DeclareMathOperator{\trace}{tr}
\newcommand{\bQ}{\mathbf{Q}}
\newcommand{\bZ}{\mathbf{Z}}
\newcommand{\frob}{\mathrm{fr}}

\title{The statistics of the signed trace of Frobenius for elliptic curves}
\author{Daniel Miller}

\begin{document}
\maketitle





Let $E_{/\bQ}$ be an elliptic curve of conductor $N$, $l$ a rational prime and 
$\rho=\rho_{E,l}\colon G_\bQ\to \GL_2(\bZ_l)$ the associated Galois 
representation coming from the Tate module of $E$. We have the sequence 
$\{a_p(E)\}_{p\nmid N}$, defined by 
\[
	a_p(E) = \trace \rho(\frob_p) .
\]
It is well-known that $a_p(E)\in \bZ$, are independent of $l$, and satisfy 
the \emph{Hasse bound}:
\[
	|a_p(E)| \leqslant 2\sqrt p .
\]
Moreover, Faltings proved that the sequence $a(E) = \{a_p(E)\}$ determines the 
isogeny class of $E$. 

We can define a new sequence, $s(E) = \{s_p=s_p(E)\}_{p\nmid N}$, by 
$s_p = \sign(a_p)$. Write $s_{\leqslant X}(E)$ for 
$\{s_p(E)\colon p\leqslant X\}$. It is known [but prove this] that $s(E)$ 
determines $E$. Consider the following function: 
\[
	C_\mathrm{prelim}(N) = \min\{X\textnormal{ s.t. }\{s_{\leqslant X}(E)\}_{\mathrm{cond}(E)\leqslant N}\textnormal{ are all distinct}\} ;
\]
here we have $E$ range over isogeny classes of elliptic curves. Experimental 
data suggests that to tell elliptic curves with conductor $\leqslant X$ apart, 
you need $s_p(E)$ for the first $O(\log X)$ primes. Since the $X$th prime is 
$\approx X\log X$, this gives us the conjecture 
\[
	C_\mathrm{prelim}(N) = O(\log N) .
\]

The basic approach is as follows. For the symmetric groups $S_n$, it seems 
[and I should prove] that the sign of the trace of a (complex) representation 
determines the representation. Given $E_{/\bQ}$, pick the ``right'' $n$ and 
look at the torsion representation $\rho_{E,n}\colon G_\bQ\to \GL_2(\bZ/n)$. 
For $n$ sufficiently large ($\geqslant 4\sqrt p$ to be precise) this ``knows'' 
$s_p(E)$. Also, if we choose $n$ right, $\rho_{E,n}$ will be surjective. If 
automorphisms of $\GL_2(\bZ/n)$ are determined by ``signs of traces,'' then 
the ``signs from $\GL_2(\bZ/n)$'' will ``know'' $\rho_{E,n}$. We then use 
explicit Faltings to see how large $n$ needs to be. 





\end{document}
