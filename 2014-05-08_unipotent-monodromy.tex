\documentclass{article}

\usepackage{amsmath,amssymb,fullpage,mathpazo}
\usepackage[hidelinks]{hyperref}
\DeclareMathOperator{\h}{H}
\newcommand{\dC}{\mathbf{C}}
\newcommand{\dZ}{\mathbf{Z}}

\title{Vanishing cycles, cyclotomic polynomials, and monodromy}
\author{Daniel Miller}

\begin{document}
\maketitle





\section{Motivation}

This is inspired by the discussion at the beginning of \cite[\S 9]{mi68}. Let 
$f$ be a regular function on $\dC^{n+1}$. Let $V=V(f)$; this is an 
$n$-dimensional complex variety. Fix an isolated critical point $x\in V$. For 
$\varepsilon>0$, let $S_\varepsilon$ be a (real) sphere of radius $\varepsilon$ 
centered at $x$, and put $K_\varepsilon = V\cap S_\varepsilon$. Define 
$\phi_\varepsilon:S_\varepsilon\smallsetminus K_\varepsilon \to S^1$ by 
\[
  \phi_\varepsilon(x) = \frac{f(x)}{|f(x)|} .
\]
The ``fibration theorem'' tells us that $\phi_\varepsilon$ is a fibration; we 
put 
$F_s = \phi_\varepsilon^{-1}(s)\subset S_\varepsilon\smallsetminus K_\varepsilon$ 
for $s\in S^1$. It is known that $F_s$ is a smooth $2n$-dimensional real 
manifold. Let $h$ be the canonical generator of $\pi_1(S^1)$. There is an 
endomorphism $h_\ast$ of $\h_n(F_s,\dZ)$. Let $\Delta$ be its characteristic 
polynomial. Milnor claims that Grothendieck has proved $\Delta$ is a product of 
cyclotomic polynomials. 





\section{Reinterpretation}

Let $f$, $V$, \ldots be as above. Instead of considering the homology of 
individual fibers $F_s$ of the map $\phi_\varepsilon$, we consider the  
pushforward $\mathsf R^n \phi_{\varepsilon,\ast} \dZ$ as a sheaf on $S^1$. This 
is a local system, so we can consider it as a representation of the group 
$\pi_1(S^1)$. Hopefully, there is a theorem that 
$\mathsf R^n \phi_{\varepsilon,\ast} \dZ$ is quasi-unipotent. 

Consider \cite{sga7-i} and \cite{sga7-ii}





\bibliographystyle{alpha}
\bibliography{tidbit-sources}

\end{document}
